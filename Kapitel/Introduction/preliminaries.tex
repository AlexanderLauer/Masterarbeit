\section{Preliminaries}\label{preliminaries}
Our graph repair process is based on the concept of the double-pushout approach \cite{hartmut2006fundamentals}. 
In this chapter, we introduce some formal prerequisites such as graphs, graph morphisms, nested graph conditions and constraints, and graph transformations. 
  
\subsection{Graphs and Graph morphisms}
We start by introducing graphs and graph morphisms according to \cite{hartmut2006fundamentals}.

\begin{definition}[\textbf{graph}, \cite{hartmut2006fundamentals}]
	A \emph{graph} $G = (V,E,\src, \tar)$ consists of a set of
	 vertices (or nodes)
	$V$, a set of edges $E$ and two mappings $\src, \tar: E \to 
	V$ that assigns the source and target vertices to an edge.
	The edge $e \in E$ connects the vertices $\tar(e)$ and $\src(e)$. 
	
	If no tuple as above is given, $V_G$, $E_G$, $\tar_G$ and
	$\src_G$ denote the sets of vertices, edges and target and source mappings, 
	respectively.

\end{definition}

For the rest of this thesis, we will assume that all graphs are finite, i.e.
given a graph $G$, the sets $V_G$ and $E_G$ are finite.
	
\begin{definition}[\textbf{graph morphism}, \cite{hartmut2006fundamentals}]
	Let graphs $G$ and $H$ be given. A \emph{graph morphism}
	$f : G \to H$ consists of two mappings $f_V: V_G \to V_H$
	and $f_E: E_G \to E_H$ such that the source and target 
	functions are preserved. This means
	\begin{equation*}
		\begin{split}
			&f_V \circ \src_G = \src_H \circ f_E \text{ and} \\
			&f_V \circ \tar_G = \tar_H \circ f_E
		\end{split}
	\end{equation*}
	holds.
	A graph morphism $f$ is called injective (surjective) if 
	$f_E$ and $f_V$ are injective (surjective) mappings. An injective morphism $f: G \to H$ is called \emph{inclusion} if 
	$f_E(e) = e$ and $f_V(v) = v$ for all edges $e \in E_G$ and all nodes $v \in V_G$.
	If $f$ is injective, it is denoted with  $f: G \inj H$.
	Two morphisms $f_1 :G_1 \to H$ and $f_2: G_2 \to H$ 
	are called \emph{jointly surjective} if for each element
	$e$ of $H$ either an element $e' \in G_1$ with $f_1(e') =e$
	or an element $e' \in G_2$ with $f_2(e') = e$ exists. 

\end{definition}

For our newly introduced notions of consistency increase and 
maintainment, we also need to consider \emph{subgraphs}, \emph{overlaps} of graphs, and so-called \emph{intermediate graphs}. Intuitively, intermediate graphs are graphs $G'$ which lie between two given graphs $G$ and $H$. That is, $G$ is a subgraph of $G'$ and $G'$ is a subgraph of $H$.


\begin{definition}[\textbf{subgraph}]
	Let the graphs $G$ and $H$ be given. Then $G$ is called a 
	\emph{subgraph} of $H$ if an inclusion $p: G \inj H$
	exists. $G$ is called a \emph{proper subgraph of $H$} if the morphism $p$ is not bijective. 
\end{definition}
Note that since the inclusion can also be surjective, by this definition every graph $G$ is a subgraph of itself.

\begin{definition}[\textbf{intermediate graph}]
	Let $G$ and $H$ be graphs such that $G$ is a subgraph of $H$. 
	A graph $C$ is called an \emph{intermediate graph} of $G$ 
	and $H$, if $G$ is a proper subgraph of $C$ and $C$ is a subgraph of $H$. 
	The set of intermediate graphs of $G$ and 
	$H$ is denoted by $\ig{G}{H}$.
\end{definition}

\begin{definition}[\textbf{overlap}]
	Let the graphs $G_1$ and $G_2$ be given. An \emph{overlap} $P = (H, 
	i_{G_1}, i_{G_2})$ consists of a graph $H$ and a jointly surjective pair of  
	injective morphisms $i_{G_1}: G_1 \inj H$ and $i_{G_2}: G_2 \inj H$ with 
	$i_{G_1}(G_1) \cap i_{G_2}(G_2) \neq \emptyset$. 
	The set of all overlaps of $G_1$ and $G_2$ is denoted by 
	$\overlay(G_1,G_2)$.
	If a tuple as above is not given, then $G_P$, $i_{G_1}^P$ and  $i_{G_2}^P$ 
	denote the graph and morphisms of a given overlap $P \in \overlay(G_1,G_2)$. 
\end{definition}

Note that $(H, i_{G_1}, i_{G_2})$ where $i_{G_1}$ and $i_{G_2}$ are jointly surjective and  $i_{G_1}(G_1) \cap i_{G_2}(G_2)= \emptyset$ could also be considered as an overlap of $G_1$ and $G_2$.
In this thesis we only need to consider overlaps with $i_{G_1}(G_1) \cap i_{G_2}(G_2)\neq \emptyset$.
So we have embedded this property directly into the definition.
 
As mentioned above, our approach also considers intermediate graphs. Therefore a notion of restricted graph morphisms is needed. For this, we introduce the notion of \emph{restrictions of morphisms}, which intuitively is the restriction of the domain and co-domain of a morphism $p: G \inj H$ with subgraphs of $G$ and $H$ respectively.

\begin{definition}[\textbf{restriction of a morphism}]
	Let the graphs $G$, $H$ and a morphism $f : G \to H$ be given. 
	Then, a morphism $f' : G' \to H'$ is called a 
	\emph{restriction} of $p$ if inclusions $i: G' \inj G$
	and $i': H' \inj H$ exist , i.e. $G'$ is a subgraph of $G$ and $H'$ is a subgraph of $H$, such that 
	\begin{equation*}
		\begin{split}
			&i'_E \circ f'_E = f_E \circ i_E \text{ and} \\
			&i'_V \circ f'_V = f_V \circ i_V.		
		\end{split}
	\end{equation*}
	A restriction of $p$ is denoted by $p^r$.
\end{definition}
Note that given a morphism $p: G \to H$ a restriction 
$p^r: G' \to H'$ of $p$ is uniquely determined by $G'$ and $H'$.
Assume two restrictions $p^r: G' \to H'$ and $q^r: G' \to H'$ of $p$ are given. 
It holds that $i' \circ p^r = p \circ i =  i' \circ q^r$ and $p^r = q^r$ follows with the injectivity of $i'$.  

\subsection{Nested Graph Conditions and Constraints}
\emph{Nested graph constraints} are useful for specifying graph 
properties. The more general notion of \emph{nested graph conditions} allows the specification of properties for graph morphisms and the definition of graph conditions and constraints in a recursive manner.  Within these conditions, only quantifiers and Boolean operators are used \cite{habel2009correctness}.
 


\begin{definition}[\textbf{nested graph condition},\cite{habel2009correctness}]
A \emph{nested graph condition} over a graph $C_0$ is defined recursively as

\begin{enumerate}
	\item \textsf{true} is a  graph condition over every graph.
	\item $\exists(a_0:C_0 \inj C_1,d)$ is a graph condition over $C_0$ if $a_0$ is an inclusion and $d$ is a graph condition over $C_1$. 
	\item $\neg d$, $d_1 \wedge d_2$ and $d_1 \vee d_2$ are 
		  graph conditions over $C_0$ if $d$, $d_1$ and $d_2$
		  are graph conditions over $C_0$.
	
\end{enumerate}
	Conditions over the empty graph $\emptyset$ are called \emph{constraints}.
We use the abbreviations $\forall(a_0:C_0 \xhookrightarrow{} C_1,d) := \neg \exists(a_0:C_0 \xhookrightarrow{} C_1,\neg d)$ and $\false = \neg \true$.



Conditions of the form $\exists(a_0:C_0 \inj C_1, d)$ are called 
\emph{existential} and the graph $C_1$ is called
\emph{existentially bound}. 
Conditions of the form $\forall(a_0:C_0 \inj C_1, d)$ are called 
\emph{universal} and the graph $C_1$ is called
\emph{universally bound}. 

\end{definition}

Since these are the only types of conditions that will be used in this paper, we will refer to them only as \emph{conditions} and \emph{constraints}. We will use the more compact notations $\exists(C_1,d)$ for $\exists(a_0: C_0 \inj C_1, d)$ and $\forall(C_1,d)$ for $\forall(a_0: C_0 \inj C_1, d)$ if $C_0$ and $a_0$ are clear from the context.


\begin{example}
	Given a condition $c = \forall(a_0 :\emptyset \inj C_1, \exists(a_2: C_1 \inj C_2, \true))$.
	The compact notation of this condition is given by $\forall(C_1, \exists(C_2, \true))$.
\end{example}

\begin{definition}[\textbf{semantic of graph conditions},\cite{habel2009correctness}]
	Given a graph $G$, a condition $c$ over $C_0$ and a graph 
	morphism $p: C_0 \inj G$. Then $p$ satisfies $c$,
	denoted by $p \models c$, if
	\begin{enumerate}
		\item	
			$c = \true$.
		\item 
			$c = \exists(a_0:C_0 \inj C_1,d)$ and there exists an injective 
			morphism $q: C_1 \inj G$ with $p = q \circ a_0$ and $q \models d$.
		\item 
			$c = \neg d$ and  $p \not \models d$. 
		\item 
			$c = d_1 \wedge d_2$ and  
			$p \models d_1$ and $p \models d_2$.
		\item 
			$c = d_1 \vee d_2$ and $p \models d_1$ or $p \models d_2$.			
	\end{enumerate}
	A graph $G$ satisfies a constraint $c$, denoted by $G \models 
	c$, if the empty morphism $p: \emptyset \inj G$ satisfies $c$.
\end{definition}

Our approach is designed to repair a specific type of constraint, constraints  without any boolean operators. Each of these conditions can be transformed into an equivalent condition in so-called \emph{alternating quantifier normal form} \cite{sandmann2019rule}. As the name suggests, these are conditions with alternating quantifiers and without any Boolean operators.


\begin{definition}[\textbf{alternating quantifier normal form (ANF)},\cite{sandmann2019rule}]
	Conditions in \emph{alternating quantifier normal form} (ANF) 
	are defined recursively as 
	
	\begin{enumerate}
		\item $\true$ and $\false$ are conditions in ANF.
		\item \label{c_2} 
			$\exists(a_0: C_0 \inj C_1,d)$ is a condition in ANF
			if either $d$ is a universal condition over $C_1$ in ANF 
			or $d =\true$.
		\item\label{c_3}
			$\forall(a_0: C_0 \inj C_1,d)$ is a condition in ANF
			if either $d$ is an existential condition over $C_1$ in ANF  
			or $d = \false$.
	\end{enumerate}
	Every condition is a \emph{subcondition} of itself.
	In cases \ref{c_2} and \ref{c_3}, $d$ is called a \emph{subcondition} of $\exists(a: C_0 \inj 
	C_1,d)$ or $\forall(a: C_0 \inj C_1,d)$ 
	respectively. All subcondition of $d$ are also subconditions of $\exists(a: 
	C_0 \inj C_1,d)$ or $\forall(a: C_0 \inj C_1,d)$ respectively. 
	The \emph{nesting level} $\nlvl(c)$ of a condition $c$ is 
	recursively defined as $\nlvl(\true)= \nlvl(\false) = 0$ and  $\nlvl
	(\exists(a: P \inj Q, d)) = \nlvl(\forall(a:P \inj Q,d)) := \nlvl(d) +1$.
\end{definition}


In the literature, conditions in ANF also allow conditions that end with conditions of the form $\exists( C_1, \false)$ or $\forall(C_1, \true)$. We exclude these cases so that conditions in ANF can only end with conditions of the form $\exists(C_1, \true)$ or $\forall( C_1, \false)$, since it is easily seen that every morphism $p: C_0 \inj G$ satisfies $\forall( C_1, \true)$ and does not satisfy $\exists(C_1, \false)$. Therefore, these conditions can be replaced by $\true$ and $\false$ respectively.

In the following, we assume that all graphs and the nesting level of a condition are finite. 

Using the \emph{shift over morphism} construction, we are able to transform a 
nested condition over $C$ into a nested condition over $C'$ via an injective 
morphism $i : C \inj C'$ \cite{habel2009correctness}.

\begin{definition}[\textbf{shift over morphism},\cite{habel2009correctness}]
	Let a condition $c$ over $C_0$ and a morphism $i:C_0 \inj C_0'$ be given.
	The \emph{shift of $c$ over $i$}, denoted by $\shiftm(c,i)$, is given by 
	\begin{enumerate}
		\item 
			If $c = \true$, $\shiftm(c,i) = \true$.
		\item 
			If $c = \exists(a_1:C_0 \inj C_1, d)$, $\shiftm(c,i) = 
			\bigvee_{(a',i') \in \mathcal{F}} \exists(a', \shiftm(d,i'))$
			with $\mathcal{F}$ being the set of all pairs $(a',i')$ of injective 
			morphisms that are jointly surjective and $i' \circ a = a' \circ i$, 
			i.e., the diagram shown in Figure \ref{fig_shift} commutates.
		\item 
			If $c = \neg d$, $\shiftm(c, i) = \neg \shiftm(d, i)$
		\item
			If $c = d_1 \wedge d_2$, $\shiftm(c,i) = \shiftm(d_1,i) \wedge 
			\shiftm(d_2,i)$
		\item
			If $c = d_1 \vee d_2$, $\shiftm(c,i) = \shiftm(d_1,i) \vee 
			\shiftm(d_2,i)$ 
	\end{enumerate}
	
\end{definition} 

\begin{lemma}[\cite{habel2009correctness}]
	Let a condition $c$ over $C_0$ and a morphism $i: C_0 \inj C_0'$ be given.
	Then, for each morphism $m: C_0' \inj G$, 
	$$m \models \shiftm(c,i) \iff m \circ i \models c$$
\end{lemma}
\begin{figure}
\center
	\begin{tikzpicture}
		\node (L) at (0,2) {$C_0$};
		\node (K) at (2,2) {$C_0'$};
		
		\node (G) at (0,0) {$C_1$};
		
		\node (H) at (2,0) {$C_1'$};
		
		\draw[right hook-stealth] (L) edge node [above] {$i$}  (K);
		\draw[left hook-stealth] (L) edge node [left] {$a_0$}  (G);
		\draw[right hook-stealth] (G) edge node [above] {$i'$}  (H);
		\draw[right hook-stealth] (K) edge node [right] {$a_0'$}  (H);
		
	\end{tikzpicture}
	\caption{Diagram for the $\shiftm$ operator.}\label{fig_shift}
\end{figure}
\subsection{Rules and Graph Transformations}
Via \emph{rules} and \emph{graph transformation} graphs can be modified by inserting or deleting nodes and edges.
We will use the concept of the double-pushout approach for rules and transformations, which is based on category theory \cite{hartmut2006fundamentals}. A rule consists of the three graphs $L$, called the \emph{left-hand side}, $K$, called \emph{context},  and $R$, called \emph{right-hand side}, where $K$ is a subgraph of $L$ and $R$. During a transformation, denoted by $G 
\Longrightarrow H$, elements of $L \setminus K$
are removed and elements of $R \setminus K$ are inserted 
so that a new morphism $p: R \inj H$ is created.
In addition, the so-called \emph{dangling edge condition} must be satisfied. Intuitively, for every edge $e \in E_H$ there are vertices $u,v \in V_H$ such that $\tar(e) = u$ and 
$\src(e) = v$ or vice versa.
We also define application conditions. These are nested conditions over $L$ and $R$ that  prevent the transformation if they are not satisfied. 
Later, we will use application conditions to ensure that transformations cannot reduce consistency. 
For example, application conditions that prevent a transformation 
if $G \models c$ and $H\not \models c$.

\begin{definition}[\textbf{rules and application conditions},\cite{hartmut2006fundamentals}]
	A \emph{plain rule} $\rho = \rle{L}{l}{K}{r}{R}$ consists of 
	graphs $L,K,R$ and inclusions $l: K \inj L$ 
	and $r:  K \inj R$. The rule $\rho^{-1} = \rle{R}{r}{K}{l}{L}$
	is called the \emph{inverse rule of $\rho'$}. 
	
	An \emph{application condition} is a nested condition over 
	$L$ or $R$ respectively. A \emph{rule} $(\ap_L, \rho, \ap_R)$ 
	consists of a plain rule $\rho$ and application conditions 
	$\ap_L$ over $L$, called \emph{left application condition}, 
	and $\ap_R$	over $R$, called \emph{right application 
	condition} respectively. 
\end{definition}



\begin{definition}[\textbf{graph transformation},\cite{hartmut2006fundamentals}]
	Let a rule $\rho = (\ap_L,\rho', \ap_R)$, a graph $G$ 
	and a morphism $m: L \inj G$,
	called the \emph{match}, be given. Then, a \emph{graph 
	transformation}, denoted by $t: G \Longrightarrow_{\rho,m} H$, is given
	in Figure \ref{fig_dpo} if the squares $(1)$ and $(2)$ are
	pushouts in the sense of category theory, $m\models \ap_L$
	and the morphism $n:L \inj H$, called the co-match of $t$, 
	satisfies $\ap_R$.
	The morphisms $g: D \inj G $ and $h: D \inj H$ are called the \emph{transformations morphisms} of $t$. 
\end{definition}

\begin{figure}
\center
	\begin{tikzpicture}
		\node (L) at (0,2) {L};
		\node (K) at (2,2) {K};
		\node (R) at (4,2) {R};
		\node (G) at (0,0) {G};
		\node (D) at (2,0) {D};
		\node (H) at (4,0) {H};
		\node (1) at (1,1) {(1)};
		\node (2) at (3,1) {(2)};
		
		\draw[left hook-stealth] (K) edge node [above] {$l$}  (L); 
		\draw[right hook-stealth] (K) edge node [above] {$r$}  (R); 
		\draw[->] (D)   edge node[above]{$g$}(G); 
		\draw[->] (D)  edge node [above]{$h$}(H);
		\draw[->] (K) edge node[fill = white] {$k$} (D);  
		\draw[->] (L) -- node[left]{$m$} (G);
		\draw[->] (R) edge node [right] {$n$} (H);    
	\end{tikzpicture}
	\caption{Diagram of a transformation in the double-pushout approach.}\label{fig_dpo}
\end{figure}

The presence of right application conditions leads to unpleasant side effects.
The satisfaction of a right application condition can only be checked after 
the transformation. The transformation must therefore be reversed if the co-match does not satisfy this condition. To avoid this, we introduce the \emph{shift over rule} operation, which is capable of transforming a right into an equivalent
left application condition \cite{habel2009correctness}. 
\begin{definition}[\textbf{shift over rule},\cite{habel2009correctness}]\label{def:shift}
	Let a rule $\rho = \rle{L}{l}{K}{r}{R}$ with the right
	application condition $\ap$  be given. The \emph{shift of $\ap$ over 
	$\rho$}, denoted with $\shift(\ap, \rho)$, is defined as
	\begin{enumerate}
		\item If $\ap = \true$, $\shift(\ap, \rho) := \true$.
		\item 
			If $\ap = \neg d$, $\shift(\ap, \rho) := \neg \shift(d, \rho)$.
		\item 
			If $\ap = d_1 \wedge d_2$, $\shift(\ap, \rho) := \shift(d_1, \rho) 
			\wedge \shift(d_2, \rho)$.
		\item 
			If $\ap = d_1 \vee d_2$, $\shift(\ap, \rho) := \shift(d_1, \rho) 
			\vee \shift(d_2, \rho)$. 
		\item 
			If $\ap = \exists(a_0: R \inj C_1,d)$, $\shift(\ap, \rho) := 
			\exists(a_0': L \inj C_0', \shift(d, \rho')) $ where 
			$\rho' = \rle{C_0}{g}{D}{h}{C_0'} $ is the rule shown in Figure \ref{fig_left} which is derived by applying $\rho^{-1}$ at match $a_0$. 
			If this transformation does not exist, we set 
			$\shift(\ap, \rho) := \false$.
	
	\end{enumerate}
	 
\end{definition}

Shift over rule produces an equivalent left application condition, 
meaning that, given a right application condition $\ap$ and a plain rule $\rho$ , a match of a transformation satisfies $\shift(\ap,
\rho)$ if and only if the co-match satisfies $\ap$ \cite{habel2009correctness}. 

\begin{lemma}[\cite{habel2009correctness}]
	Let a plain rule $\rho = \rle{L}{l}{K}{r}{R}$, a right application condition 
	$\ap$ for $\rho$ and a transformation $t: G \Longrightarrow_{\rho,m}H$ be 
	given. Then, 
	$$m \models \shift(\ap,\rho) \iff n \models \ap.$$
\end{lemma}
\begin{figure}
\center
	\begin{tikzpicture}
		\node (L) at (0,2) {R};
		\node (K) at (2,2) {K};
		\node (R) at (4,2) {L};
		\node (G) at (0,0) {$C_0$};
		\node (D) at (2,0) {D};
		\node (H) at (4,0) {$C_0'$};
		\node (1) at (1,1) {(1)};
		\node (2) at (3,1) {(2)};
		
		\draw[left hook-stealth] (K) edge node [above] {$r$}  (L); 
		\draw[right hook-stealth] (K) edge node [above] {$l$}  (R); 
		\draw[left hook-stealth] (D)   edge node[above]{$g$}(G); 
		\draw[right hook-stealth] (D)  edge node [above]{$h$}(H);
		\draw[left hook-stealth] (K) edge node[fill = white] {$k$} (D);  
		\draw[left hook-stealth] (L) -- node[left]{$a_0$} (G);
		\draw[right hook-stealth] (R) edge node [right] {$a_0'$} (H);    
	\end{tikzpicture}
	\caption{Transformation for the shift over rule operator.}\label{fig_left}
\end{figure}


Since every right application condition can be transformed into an equivalent left application condition, we will assume from now on that each rule contains only left application conditions. These rules are denoted by $(\ap, \rho)$.
The following Theorem shows, under which conditions a rule $\rho$ is applicable at a match $m$, i.e. there is transformation via $\rho$ at match $m$.

\begin{figure}
\center
	\begin{tikzpicture}
		\node (L) at (0,2) {L};
		\node (K) at (2,2) {K};
		\node (R) at (4,2) {R};
		\node (G) at (0,0) {G};
		\node (D) at (2,0) {D};

		\node (1) at (1,1) {(1)};

		
		\draw[left hook-stealth] (K) edge node [above] {$l$}  (L); 
		\draw[right hook-stealth] (K) edge node [above] {$r$}  (R); 
		\draw[left hook-stealth] (D)   edge node[above]{$g$}(G); 
		
		\draw[left hook-stealth] (K) edge node[fill = white] {$k$} (D);  
		\draw[left hook-stealth] (L) -- node[left]{$m$} (G);
		    
	\end{tikzpicture}
	\caption{Applicability of a rule $\rle{L}{l}{K}{r}{R}$ at match $m$.}\label{fig_appl}
\end{figure}
\begin{theorem}[\cite{hartmut2006fundamentals}]
Given a rule $\rho = (\ap, \rle{L}{l}{K}{r}{R})$, a graph $G$ and a match $m: L \inj G$ such that $m \models \ap$. The rule $\rho$ is applicable at match $m$, i.e. there is a transformation $t: G \Longrightarrow_{\rho, m} H$ if and only if there is a graph $D$ such that the square $(1)$ in Figure \ref{fig_appl} is a pushout. 
This is equivalent to $\rho$ and $m$ satisfying the \emph{dangling-edge condition}, i.e. $m \models \ap$ and
$$IP_{\rho,m} \cup DP_{\rho,m} \subseteq GP_{\rho,m}$$ 
where $GP$ is the set of all nodes and edges in $L$ that are not deleted by $\rho$, i.e.
$$GP_{\rho,m} = l(K),$$ 
$IP$ is the set of nodes and edges in $L$ that are identified by $m$, i.e. 
\begin{equation*}
	\begin{split}
		IP_{\rho,m} = &\{v \in V_L \mid \exists v' \in V_L (v' \neq v \text{ and } m_v(v) = m_V(v'))\} \cup \\
		&\{e \in E_L \mid \exists e' \in E_L(e \neq e' \text{ and } m_v(e) = m_V(e'))\},
	\end{split}
\end{equation*}
and DP is the set of nodes in $L$ whose images under $m$ are the source or target of an edge in $G$ that does not belong to $m(L)$, i.e. 
\begin{equation*}
	DP_{\rho,m} = \{v\in V_L \mid \exists e \in E_G \setminus m_E(E_L)(\src(e) = m_V(v) \text{ or } \tar(e) = m_V(v))\}.
\end{equation*}

\end{theorem}
Since we will only consider injective matches, the set $IP_{\rho,m}$ is always empty, and therefore $DP_{\rho,m} \subseteq GP_{\rho,m}$ is sufficient to state that the \emph{dangling-edge condition} is satisfied.

Via the \emph{track morphism} it is possible to track elements across a transformation \cite{plump2005confluence}.

\begin{definition}[\textbf{track morphism},\cite{plump2005confluence}]
	Consider the transformation $t$ shown in figure \ref{fig_dpo}.
	The \emph{track morphism} of $t$, denoted by  $\track_t: G \dashrightarrow H$, is 
	a partial morphism defined as
	$$\track_t = \begin{cases}
					h (g^{-1}(e)) & \text{if $ e \in g(D)$} \\
					\text{undefined} & \text{otherwise.}
				 \end{cases}$$

\end{definition}

For example, given a transformation $t: G \Longrightarrow H$, the track morphism can be used to check whether a morphism $p:C \inj G$ extends to the derived graph $H$ by checking whether $\track_t \circ p$ is total, or whether a new morphism $q:C \inj H$ has been inserted by checking that no morphism $p : C \inj H$ with $q = \track_t \circ p$ exists \cite{kosiol2022sustaining}. We will use these results later on.

\begin{lemma}[\cite{kosiol2022sustaining}]
	Let a transformation $t: G \Longrightarrow H$ with transformation morphisms $g: D \inj G$, $h: D \inj H$ and an occurrence $p: C \inj G$ of a graph $C$ be given. Then,
	\begin{enumerate}
		\item The track morphism $\track_t$ of $t$ is total when restricted to $p(C)$, i.e. $\track_t \circ p$ is total, if and only if there is an injective morphism $p': C \inj D$ such that $p = g \circ p'$. 
		\item 
			Given an injective morphism $p: C \inj H$, $p(C)$ is contained in $\track_t(G)$ if and only if there is an injective morphism $p': C \inj D$ such that $p = h \circ p'$.
	\end{enumerate}
\end{lemma}

\begin{lemma}[\cite{kosiol2022sustaining}]
	Given a transformation $t:G \Longrightarrow H$ with the transformation morphisms $g: D \inj G$, $h:D \inj H$ and a constraint $c$ in ANF that contains the morphism $a_i: C_{i-1} \inj C_i$. Given injective morphisms $p_{i-1}: C_{i-1} \inj G$ and $p_i:C_i \inj G$ such that $p_{i-1} = p_i \circ a_i$ and the track morphism $\track_t : G \dashrightarrow H$ is total when restricted to $p_i(C_i)$. 
	Then, $p'_{i-1} = p'_i \circ a_i$ where $p'_{i-1} = \track_t \circ p_{i-1}$  and $ p'_{i} = \track_t \circ p_{i}$.
	Also, given injective morphisms $p'_{i-1}:C_{i-1} \inj H$ and $p'_{i}:C_{i} \inj H$ such that $p'_{i-1} = p'_i \circ a_i$ and 
	$p_i(C_i)$ is contained in $\track_t(G)$, then $p_{i-1} = p_i \circ a_i$ where $p_{i-1} = \track_t^{-1} \circ p'_{i-1}$ and $p_i = \track_t^{-1} \circ p'_{i}$. 
\end{lemma}



Given a sequence of transformations, the notion of \emph{concurrent rules}
can be used to describe this sequence by a rule. In other words, any sequence of transformations can be replaced by a transformation via its concurrent rule \cite{ehrig2010parallelism}.
\begin{definition}[\textbf{concurrent rule},\cite{ehrig2010parallelism}]
	Let the rules $\rho_1 = \rle{L_1}{l_1}{K_1}{r_1}{R_1}$, $\rho_2 = \rle{L_2}
	{l_2}{K_2}{r_2}{R_2}$ and a sequence of transformations 
	$$G_1 \Longrightarrow_{\rho_1,m_1} G_2 \Longrightarrow_{\rho_2,m_2}G_3$$ 
	be given. Then, $\rho' = \rle{G_1}{l'}{K}{r'}{G_3}$ is called the 
	\emph{concurrent rule of the transformation sequence} if the 
	square $(5)$ in Figure \ref{fig_concurrent_rule} is a pullback.
	
\end{definition}
A transformation sequence $G_1 \Longrightarrow_{\rho_1,m_1} G_2 \Longrightarrow_{\rho_2,m_2}G_3$ can be replaced by a transformation
$G_1 \Longrightarrow_{\rho', \id} G_3$ via its concurrent rule. By inductive application, a concurrent rule for a transformation sequence $G_1 \Longrightarrow_{\rho_0} \ldots \Longrightarrow_{\rho_n} G_n$ of arbitrary finite length can be derived.

\begin{figure}
\center
	\begin{tikzpicture}
		\node (L1) at (0,2) {$L_1$};
		\node (K1) at (2,2) {$K_1$};
		\node (R1) at (4,2) {$R_1$};
		\node (G1) at (0,0) {$G_1$};
		\node (D1) at (2,0) {$D_1$};
		\node (G2) at (5,0) {$G_2$};
		
		\node(L2) at (6,2) {$L_2$};
		\node(K2) at (8,2) {$K_2$};
		\node(R2) at (10,2) {$R_2$};
		\node(D2) at (8,0) {$D_2$};
		\node(G3) at (10,0) {$G_3$};
		\node(K) at (5, -2) {$K$};
		\node (1) at (1,1) {(1)};
		\node (2) at (3,1) {(2)};
		\node (3) at (7,1) {(3)};
		\node (4) at (9,1) {(4)};
		\node (5) at (5,-1) {(5)};
		
		\draw[left hook-stealth] (K1) edge node [above] {$l_1$}  (L1); 
		\draw[right hook-stealth] (K1) edge node [above] {$r_1$}  (R1); 
		\draw[left hook-stealth] (D1)   edge node[above]{$g_1$}(G1); 
		\draw[right hook-stealth] (D1)  edge node [above]{$h_1$}(G2);
		\draw[left hook-stealth] (K1) edge node[fill = white] {$k_1$} (D1);  
		\draw[left hook-stealth] (L1) -- node[left]{$m_1$} (G1);
		\draw[right hook-stealth] (R1) edge node [left] {$n_1$} (G2); 
		\draw[left hook-stealth] (L2) edge node[right]{$m_2$}(G2);
		\draw[left hook-stealth] (K2) edge node[above]{$l_2$}(L2);
		\draw[right hook-stealth] (K2) edge node[above]{$r_2$}(R2);
		\draw[left hook-stealth] (D2) edge node[above]{$g_2$}(G2);
		\draw[right hook-stealth] (D2) edge node[above]{$h_2$}(G3); 
		\draw[left hook-stealth] (K2) edge node[fill = white]{$k_2$}(D2);
		\draw[right hook-stealth] (R2) edge node[right]{$n_2$}(G3); 
		\draw[left hook-stealth] (K) edge node[below]{$l'$}(G1);
		\draw[right hook-stealth] (K) edge node[below]{$r'$}(G3);
		\draw[left hook-stealth] (K) edge node[below]{}(D1);
		\draw[right hook-stealth] (K) edge node[below]{}(D2);  
	\end{tikzpicture}
	\caption{Pushout diagram of the transformation sequence 
	$G_1 \Longrightarrow_{\rho_1,m_1} G_2 \Longrightarrow_{\rho_2, m_2}G_3$ using 
	the rules $\rho_1  = \rle{L_1}{l_1}{K_1}{r_1}{R_1}$ and $\rho_2  = \rle{L_2}{l_2}{K_2}{r_2}{R_2}$.}\label{fig_concurrent_rule}
\end{figure}
\subsection{Concepts of Consistency}\label{sec_consistency}

Now, we will introduce familiar consistency concepts. 
Namely, the notions of consistency preserving and guaranteeing transformations 
\cite{habel2009correctness} and the notions of (direct) consistency sustaining and improving transformations \cite{kosiol2022sustaining}.
Later, we will examine how these concepts differ from and correlate to our newly introduced concept of consistency.

\begin{definition}[\textbf{consistency preserving and guaranteeing transformations},\cite{habel2009correctness}]
	Let a constraint $c$ and a transformation $t: G \Longrightarrow H$ be given. 
	Then, $t$ is called \emph{$c$-preserving} if 
	$$ G \models c \implies H \models c.$$
	The transformation $t$ is called \emph{$c$-guaranteeing} if $H \models c$.

\end{definition}

While consistency preserving and guaranteeing transformations are defined for nested conditions, the finer-grained notions of (direct) consistency sustaining and improving transformations are defined only for conditions in ANF.



\begin{definition}[\textbf{consistency sustaining and improving transformations},\cite{kosiol2022sustaining}]
	Let a constraint $c$ in ANF and a transformation $t: G 
	\Longrightarrow_{\rho} H$ be given.
	If $c$ is existentially bound, $t$ is called \emph{consistency sustaining 
	w.r.t. $c$} if it is $c$-preserving and $t$ is called \emph{consistency 
	improving w.r.t. $c$} if it is $c$-guaranteeing.
	If $c = \forall(a_0: \emptyset \inj C_1,d)$ is universal, $t$ is 
	called \emph{consistency sustaining w.r.t. $c$} if
	$$ |\{p:C_1 \inj G \mid p \not \models d\}| \geq |\{p:C_1 \inj H \mid p \not 
	\models d\}|$$
	and $t$ is called \emph{consistency improving w.r.t. $c$} if 
	$$ |\{p:C_1 \inj G \mid p \not \models d\}| > |\{p:C_1 \inj H \mid p \not 
	\models d\}|.$$
	The number of elements of these sets is called the \emph{number of 
	violations in $G$} and \emph{number of violations in $H$} respectively.
\end{definition}

The even stricter notion of direct sustaining and improving transformations prohibits the insertion of new violations altogether.
\begin{definition}[\textbf{direct sustaining and improving transformations},\cite{kosiol2022sustaining}]
	Let a constraint $c$ in ANF and a transformation $t: G \Longrightarrow_{\rho} 
	H$ be given. If $c$ is existential, $t$ is called \emph{direct 
	consistency sustaining w.r.t. $c$} if $t$ is $c$-preserving and 
	$t$ is called \emph{direct consistency improving w.r.t. $c$} if $t$ is 
	$c$-guaranteeing. 
	
	If $c = \forall(a_0:\emptyset \inj C_1,d)$, $t$ is called 
	\emph{consistency sustaining w.r.t. $c$} if 
	\begin{equation*}
		\begin{split}
			&\forall p:C_0 \inj G((p \models d \wedge \track_t \circ p \text{ is total}) 
	\implies \track_t \circ p \models d) \text{ and} \\& \forall p':C_0 \inj H(\neg 
	\exists q:C_0 \inj G(p' = \track_t \circ q) \implies p' \models d))
		\end{split}
	\end{equation*} 
	and $t$ is called \emph{consistency improving w.r.t. $c$} if additionally 
	\begin{equation*}
		\begin{split}
			&\exists p:C_0 \in G(p \not \models d \text{ and} \track_t \circ p \text{ 
			is total} \wedge \track_t \circ p \models d) \vee \\
			&\exists p:C \inj G (p \not \models d \wedge \track_t \circ p \text{ 
			is not total}).
		\end{split}
	\end{equation*}
\end{definition}

