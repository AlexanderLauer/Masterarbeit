\section{Preliminaries}
Our graph repair process is based on the concept of the double-pushout approach \cite{hartmut2006fundamentals}. 
In this chapter we introduce some formal prerequisites such as graphs, graph morphisms, nested graph conditions and constraints, and graph transformations. 
  
\subsection{Graphs and Graph morphisms}
We start by introducing graphs and graph morphisms according to \cite{hartmut2006fundamentals}.

\begin{definition}[\textbf{graph}]
	A \emph{graph} $G = (V,E,\src, \tar)$ consists of a set of
	 vertices (or nodes)
	$V$, a set of edges $E$ and two mappings $\src, \tar: E \to 
	V$ that assign the source and target vertices to an edge.
	The edge $e \in E$ connects the vertices $\tar(e)$ and $\src(e)$. 
	
	If no tuple as above is given, $V_G$, $E_G$, $\tar_G$ and
	$\src_G$ denote the sets of vertices, edges and target and source mappings, 
	respectively.

\end{definition}

For the rest of this paper we will assume that all graphs are finite, i.e.
given a graph $G$, the sets $V_G$ and $E_G$ are finite.
	
\begin{definition}[\textbf{graph morphism}]
	Let the graphs $G$ and $H$ be given. A \emph{graph morphism}
	$f : G \to H$ consists of two mappings $f_V: V_G \to V_H$
	and $f_E: E_G \to E_H$ such that the source and target 
	functions are preserved. This means
	\begin{equation*}
		\begin{split}
			&f_V \circ \src_G = \src_H \circ f_E \wedge \\
			&f_V \circ \tar_G = \tar_G \circ f_E
		\end{split}
	\end{equation*}
	holds.
	A graph morphism $f$ is called injective (surjective) if 
	$f_E$ and $f_V$ are injective (surjective) mappings.
	If $f$ is injective, it is denoted with  $f: G \inj H$.
	Two morphisms $f_1 :G_1 \to H$ and $f_2: G_2 \to H$ 
	are called \emph{jointly surjective} if for each element
	$e$ of $H$ either an element $e' \in G_1$ with $f_1(e') =e$
	or an element $e' \in G_2$ with $f_2(e') = e$ exists. 

\end{definition}

For our newly introduced notions of consistency increase and 
maintainment, we also need to consider \emph{subgraphs}, \emph{overlaps} of graphs, and so-called \emph{intermediate graphs}. Intuitively, intermediate graphs are graphs $G'$ which lie between two given graphs $G$ and $H$. That is, $G$ is a subgraph of $G'$ and $G'$ is a subgraph of $H$.


\begin{definition}[\textbf{subgraph}]
	Let the graphs $G$ and $H$ be given. Then $G$ is called a 
	\emph{subgraph} of $H$ if an injective morphism $p: G \inj H$
	exists.
\end{definition}
Note that since the injective morphism can also be surjective, by this definition every graph $G$ is a subgraph of itself.

\begin{definition}[\textbf{intermediate-graph}]
	Let $G$ and $H$ be graphs such that $G$ is a subgraph of $H$. 
	A graph $C$ is called an \emph{intermediate-graph} of $G$ 
	and $H$, if $G$ is a subgraph of $C$ and $C$ is a subgraph of $H$. 
	The set of intermediate-graphs of $G$ and 
	$H$ is denoted by $\ig{G}{H}$.
\end{definition}

\begin{definition}[\textbf{overlap}]
	Let the graphs $G_1$ and be $G_2$ be given. An \emph{overlap} $P = (H, 
	i_{G_1}, i_{G_2})$ consists of a graph $H$ and a jointly surjective pair of  
	injective morphisms $i_{G_1}: G_1 \inj H$ and $i_{G_2}: G_2 \inj H$ with 
	$i_{G_1}(G_1) \cap i_{G_2}(G_2) \neq \emptyset$. 
	The set of all overlaps of $G_1$ and $G_2$ is denoted by 
	$\overlay(G_1,G_2)$.
	If a tuple as above is not given, then $G_P$, $i_{G_1}^P$ and  $i_{G_2}^P$ 
	denote the graph and morphisms of a given overlap $P \in \overlay(G_1,G_2)$. 
\end{definition}

Note that $(H, i_{G_1}, i_{G_2})$ with $i_{G_1}$ and $i_{G_2}$ being jointly surjective and  $i(G_1) \cap i'(G_2)= \emptyset$ could also be considered as an overlap of $G_1$ and $G_2$.
In this paper we only need to consider overlaps with $i_{G_1}(G_1) \cap i_{G_2}(G_2)\neq \emptyset$.
So we have embedded this property directly into the definition.
 
As mentioned above, our approach also considers intermediate graphs. Therefore a notion of restricted graph morphisms is needed. For this, we introduce the notion of \emph{restricted morphisms}, which intuitively is the restriction of the domain and co-domain of a morphism $p: G \inj H$ with subgraphs of $G$ and $H$ respectively.

\begin{definition}[\textbf{restricted morphism}]
	Let the graphs $G$, $H$ and a morphism $f : G \to H$ be given. 
	Then, a morphism $f' : G' \to H'$ is called a 
	\emph{restricted morphism} of $p$ if morphisms $i: G' \inj G$
	and $i': H' \inj H$ exist ($G'$ is a subgraph of $G$ and $H'$ 
	is a subgraph of $H$) such that 
	\begin{equation*}
		\begin{split}
			&i'_E \circ f'_E = f_E \circ i_E \wedge \\
			&i'_V \circ f'_V = f_V \circ i_V.		
		\end{split}
	\end{equation*}
	A restricted morphism of $p$ is denoted by $p^r$.
\end{definition}
Note that given a morphism $p: G \to H$ a restriction 
$p^r: G' \to H'$ of $p$ is uniquely determined by $G'$ and $H'$.

\subsection{Nested Graph Conditions and Constraints}
\emph{Nested graph constraints} are useful for specifying graph 
properties. The more general notion of \emph{nested graph conditions} allows the specification of properties for graph morphisms and the definition of graph conditions and constraints in a recursive manner.  Within these conditions, only quantifiers and Boolean operators are used \cite{habel2009correctness}.
 


\begin{definition}[\textbf{nested graph condition}]
A \emph{nested graph condition} over a graph $C_0$ is defined recursively as

\begin{enumerate}
	\item \textsf{true} is a  graph condition over every graph.
	\item $\exists(a_0:C_0 \inj C_1,d)$ is a graph condition over $C_0$ if $a_0$ is an injective graph morphism and $d$ is a graph condition over $C_1$. 
	\item $\neg d$, $d_1 \wedge d_2$ and $d_1 \vee d_2$ are 
		  graph conditions over $C_0$ if $d$, $d_1$ and $d_2$
		  are graph conditions over $C_0$.
	
\end{enumerate}
	Conditions over the empty graph $\emptyset$ are called \emph{constraints}.
We use the abbreviations $\forall(a_0:C_0 \xhookrightarrow{} C_1,d) := \neg \exists(a_0:C_0 \xhookrightarrow{} C_1,\neg d)$ and $\false = \neg \true$.



Conditions of the form $\exists(a_0:C_0 \inj C_1, d)$ are called 
\emph{existentially bound}, the graph $C_1$ is also called
existentially bound. 
Conditions of the form $\forall(a_0:C_0 \inj C_1, d)$ are called 
\emph{universally bound}, the graph $C_1$ is also called
universally bound. 

\end{definition}

Since these are the only types of conditions that will be used in this paper, we will refer to them only as \emph{conditions} and \emph{constraints}. We will use the more compact notations $\exists(C_1,d)$ for $\exists(a_0: C_0 \inj C_1, d)$ and $\forall(C_1,d)$ for $\forall(a_0: C_0 \inj C_1, d)$ if $C_0$ and $a_0$ are clear from the context.

\begin{definition}[\textbf{semantic of graph conditions}]
	Given a graph $G$, a condition $c$ over $C_0$ and a graph 
	morphism $p: C_0 \inj G$. Then $p$ satisfies $c$,
	denoted by $p \models c$, if
	\begin{enumerate}
		\item	
			If $c = \true$.
		\item 
			If $c = \exists(a_0:C_0 \inj C_1,d)$: There does exists an injective 
			morphism $q: C_1 \inj G$ with $p = q \circ a_0$ and $q \models d$.
		\item 
			If $c = \neg d$:  $p \not \models d$. 
		\item 
			If $c = d_1 \wedge d_2$:  
			$p \models d_1$ and $p \models d_2$.
		\item 
			If $c = d_1 \vee d_2$: $p \models d_1$ or $p \models d_2$.			
	\end{enumerate}
	A graph $G$ satisfies a constraint $c$, denoted by $G \models 
	c$, if the morphism $p: \emptyset \inj G$ satisfies $c$.
\end{definition}

Our approach is designed to repair a specific type of constraint, constraints  without any boolean operators. Each of these conditions can be transformed into an equivalent condition in so-called \emph{alternating quantifier normal form} \cite{sandmann2019rule}. As the name suggests, these are conditions with alternating quantifiers and without any boolean operators.

\begin{definition}[\textbf{alternating quantifier normal form (ANF)}]
	Conditions in \emph{alternating quantifier normal form} (ANF) 
	are defined recursively as 
	
	\begin{enumerate}
		\item $\true$ and $\false$ are conditions in ANF.
		\item 
			$\exists(a_0: C_0 \inj C_1,d)$ is a condition in ANF
			if either $d$ is an universally bound condition over $C_1$ in ANF 
			or $d =\true$.
		\item
			$\forall(a_0: C_0 \inj C_1,d)$ is a condition in ANF
			if either $d$ is an existentially bound condition over $C_1$ in ANF  
			or $d = \false$.
	\end{enumerate}
	In both cases, $d$ is called a \emph{subcondition} of $\exists(a: C_0 \inj 
	C_1,d)$ or $\forall(a: C_0 \inj C_1,d)$ 
	respectively. All subcondition of $d$ are also subconditions of $\exists(a: 
	C_0 \inj C_1,d)$ or $\forall(a: C_0 \inj C_1,d)$ respectively. 
	The \emph{nesting level} $\nlvl(c)$ of a condition $c$ is 
	recursively defined as $\nlvl(\true)= \nlvl(\false) = 0$ and  $\nlvl
	(\exists(a: P \inj Q, d)) = \nlvl(\forall(a:P \inj Q,d)) := \nlvl(d) +1$.
\end{definition}


In the literature, conditions in ANF also allow conditions that end with conditions of the form $\exists( C_1, \false)$ or $\forall(C_1, \true)$. We exclude these cases so that conditions in ANF can only end with conditions of the form $\exists(C_1, \true)$ or $\forall( C_1, \false)$, since it is easily seen that every morphism $p: C_0 \inj G$ satisfies $\forall( C_1, \true)$ and does not satisfy $\exists(C_1, \false)$. Therefore, these conditions can be replaced by $\true$ and $\false$ respectively.

In the following, we assume that all conditions are finite. As a direct 
consequence, the nesting level is also finite.

\subsection{Rules and Graph Transformations}
Via \emph{rules} and \emph{graph transformation} graphs can be modified by inserting or deleting nodes and edges.
We will use the concept of the double-pushout approach for rules and transformations, which is based on category theory \cite{hartmut2006fundamentals}. A rule consists of the three graphs $L$, called the \emph{left-hand side}, $K$, called \emph{context},  and $R$, called \emph{right-hand side}, where $K$ is a subgraph of $L$ and $R$. During a transformation, denoted by $G 
\Longrightarrow H$, elements of $L \setminus K$
are removed and elements of $R \setminus K$ are inserted 
so that a new morphism $p: R \inj H$ is created.
In addition, the so-called \emph{dangling edge condition} must be satisfied, which means that for every edge $e \in E_H$ there are vertices $u,v \in V_H$ such that $\tar(e) = u$ and 
$\src(e) = v$ or vice versa.
We also define application conditions. These are nested conditions over $L$ and $R$ that  prevent the transformation if they are not satisfied. 
Later, we will use application conditions to ensure that transformations cannot reduce consistency. 
For example, application conditions that prevent a transformation 
if $G \models c$ and $H\not \models c$.

\begin{definition}[\textbf{rules and application conditions}]
	A \emph{plain rule} $\rho = \rle{L}{l}{K}{r}{R}$ consists of 
	graphs $L,K,R$ and injective graph morphisms $l: K \inj L$ 
	and $r:  K \inj R$. The rule $\rho^{-1} = \rle{R}{r}{K}{l}{L}$
	is called the \emph{inverse rule of $\rho'$}. 
	
	An \emph{application condition} is a nested condition over 
	$L$ or $R$ respectively. A \emph{rule} $(\ap_L, \rho, \ap_R)$ 
	consists of a plain rule $\rho$ and application conditions 
	$\ap_L$ over $L$, called \emph{left application condition}, 
	and $\ap_R$	over $R$, called \emph{right application 
	condition} respectively. 
\end{definition}



\begin{definition}[\textbf{graph transformation}]
	Let a rule $\rho = (\ap_L,\rho', \ap_R)$, a graph $G$ 
	and a morphism $m: L \inj G$,
	called the \emph{match}, be given. Then, a \emph{graph 
	transformation} $t: G \Longrightarrow_{\rho,m} H$ is given
	in Figure \ref{fig_dpo} if the squares $(1)$ and $(2)$ are
	pushouts in the sense of category theory, $m\models \ap_L$
	and the morphism $n:L \inj H$, called the co-match of $t$, 
	satisfies $\ap_R$.
\end{definition}

\begin{figure}
\center
	\begin{tikzpicture}
		\node (L) at (0,2) {L};
		\node (K) at (2,2) {K};
		\node (R) at (4,2) {R};
		\node (G) at (0,0) {G};
		\node (D) at (2,0) {D};
		\node (H) at (4,0) {H};
		\node (1) at (1,1) {(1)};
		\node (2) at (3,1) {(2)};
		
		\draw[left hook-stealth] (K) edge node [above] {$l$}  (L); 
		\draw[right hook-stealth] (K) edge node [above] {$r$}  (R); 
		\draw[->] (D)   edge node[above]{$g$}(G); 
		\draw[->] (D)  edge node [above]{$h$}(H);
		\draw[->] (K) edge node[fill = white] {$k$} (D);  
		\draw[->] (L) -- node[left]{$m$} (G);
		\draw[->] (R) edge node [right] {$n$} (H);    
	\end{tikzpicture}
	\caption{Diagram of a transformation in the double-pushout approach.}\label{fig_dpo}
\end{figure}

The presence of right applications conditions leads to unpleasant side effects.
The satisfaction of a right application condition can only be checked after 
the transformation. The transformation must therefore be reversed if the co-match does not satisfy this condition. To avoid this, we introduce the \emph{shift over rule} operation, which is capable of transforming a right into an equivalent
left application condition \cite{habel2009correctness}. 
\begin{definition}[\textbf{shift over rule}]\label{def:shift}
	Let a plain rule $\rho = \rle{L}{l}{K}{r}{R}$ and a right 
	application condition $\ap$ of the form 
	$\neg \exists(m:R \inj C, \true)$ be given. Then, $\ap$
	can be shifted into a  left application condition with the 
	transformation $t: C \Longrightarrow_{\rho^{-1},m} C'$.
	The shifted condition over $\rho$, $\shift(\ap, \rho)$, is
	given by $\neg \exists(n: L \inj C', \true)$ with $n$ being the 
	co-match of $t$. 
\end{definition}

Shift over rule produces an equivalent left application condition, 
meaning that, given a right application condition $\ap$ and a plain rule $\rho$ , a match of a transformation satisfies $\shift(\ap,
\rho)$ if and only if the co-match satisfies $\ap$ \cite{habel2009correctness}. In general, every right application
condition can be shifted into an equivalent left application condition and vice versa. Since we only need to shift conditions
of the form $\neg \exists(a:R \inj C, \true)$ we introduced
the shift only for these kind of application conditions and assume that each rule only contains left application conditions, denoted by $(\ap, \rho)$.

Via the \emph{track morphism} it is possible to track elements across a transformation \cite{plump2005confluence}. 

\begin{definition}[\textbf{track morphism}]
	Consider the transformation $t$ shown in figure \ref{fig_dpo}.
	The \emph{track morphism}, $\track_t: G \to H$, of $t$ is 
	defined as
	$$\track_t = \begin{cases}
					h (g^{-1}(e)) & \text{if $ e \in g(D)$} \\
					\text{undefined} & \text{otherwise.}
				 \end{cases}$$

\end{definition}

For example, given a transformation $t: G \Longrightarrow H$, via the track morphism it can be checked whether a morphism $p:C \inj G$ is still present in the derived graph $H$ by checking whether $\track_t \circ p$ is total, or that a 
new morphism $q:C \inj H$ has been inserted by checking that no morphism
$p : C \inj H$ with $q = \track_t \circ p$ exists. 






