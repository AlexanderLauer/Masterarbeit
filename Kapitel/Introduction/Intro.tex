\section{Introduction}

Graphs and graph transformations are an appropriate framework for model-driven software engineering. 
The models (graphs) must be consistent with respect to a set of constraints. The concept of \emph{nested graph conditions} is suitable to describe these constraints based on second order logic \cite{habel2005nested}.
During development, inconsistencies may arise and a method for resolving them, called \emph{model repair}, is necessary. 
Due to the versatility of graphs and graph transformations, the concept of model repair can be used to resolve inconsistencies for all kinds of graph-like structures. For example, in traffic light control or road networks.

For model repair, a notion is needed to assess the degree of inconsistency of a graph with respect to a constraint.
There are already concepts for this, such as consistency preserving and guaranteeing transformations, and rules \cite{habel2009correctness} that assess the change in consistency due to a graph transformation. 
These notions only look at the first graph of a constraint or whether the derived graph is consistent. 
They are not able to detect small increases or decreases in consistency caused by the insertion or deletion of single elements. 
In this paper we introduce the finer grained notions of \emph{consistency-maintaining} and \emph{consistency-increasing} transformations and rules that are able to detect these small increases or decreases in consistency. 
In addition, we introduce the more restricted notions of \emph{direct consistency-maintaining} and \emph{direct consistency-increasing} transformations and rules, which completely prohibit the insertion of new inconsistencies and have the advantage that second-order formulas can be used to specify these notions. 
We will also introduce direct consistency-maintaining and direct consistency-increasing application conditions for general rules, and direct consistency-increasing application conditions for a specific type or rules, called \emph{basic rules}. 
It is known that the application conditions for these concepts are either very restrictive or large. For basic rules, we are able to construct smaller and less restrictive application conditions.
We also present a rule-based graph repair approach capable of repairing specific constraints, namely \emph{circular conflict free constraints}, and we extend this repair approach to repair specific sets of constraints, namely \emph{circular conflict free sets of constraints}.

This paper is structured as follows:
Formal prerequisites are introduced in section \ref{preliminaries}. The notions of (direct) consistency-maintaining and (direct) consistency-increasing transformations are given in section \ref{increase_maintainment} and the construction of application conditions and characterisation of basic rules is given in section \ref{appl_conds}. 
The repair process is presented in section \ref{repair}, and we summarise related graph repair approaches in section \ref{rel_work}, before concluding the paper in section \ref{conclusion}.
 