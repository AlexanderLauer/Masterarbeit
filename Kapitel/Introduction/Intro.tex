\section{Introduction}

Graphs and graph transformations are a suitable framework for model-driven software engineering. 
The models (graphs) need to be consistent with respect to a set of constraints. The concept of \emph{nested graph conditions} is suitable to describe these constraints on the basis of second order logic \cite{habel2005nested}.
During development, inconsistencies may occur and a method to resolve these, called \emph{model repair}, is necessary. 
Due to the versatility of graphs and graph transformations, the concept of model repair can be used to resolve inconsistencies for all kinds of graph-like structure. For example in traffic light control or street networks. 

For model repair, a notion to assess the level of  inconsistency of a graph with respect to a constraint is needed.
For this, there are already concepts, for example consistency preserving and guaranteeing transformations and rules \cite{habel2009correctness}, which asses the change of consistency through a graph transformation. 
These notions only look at the first graph of a constraint or whether the derived graph is consistent. 
They are not able to detect small increases or decreases of consistency that arise by the insertion or deletion of single elements. 
In this paper, we will introduce the finer grained notions of \emph{consistency-maintaining} and \emph{consistency-increasing} transformations and rules, that are able to detect these small  increase or decreasements of consistency. 
In addition, we introduce the more restricted notions of \emph{direct consistency-maintaining} and \emph{direct consistency-increasing} transformations and rules, which prohibit the insertion of new inconsistencies completely and yield the advantage that second-order formulas can be used to decide whether a transformation is (direct) consistency-maintaining or increasing. 
We will also introduce direct consistency maintaining and direct consistency increasing application conditions for general rules and direct consistency increasing application conditions for a specific type or rules, called \emph{basic rules}. 
It is known that application conditions for these notions are either very restrictive or large in size. For basic conditions, we are able to construct smaller and less restrictive application conditions. 
In addition, we present a rule-based graph repair approach, which is capable of repairing specific constraints, namely \emph{circular conflict free constraints} and we extend this repair approach to repair specific sets of constraints, namely  \emph{circular conflict free sets of constraints}.

This paper is structured in the following way:
Formal prerequisites are introduced in section \ref{preliminaries}. The notions of (direct) consistency-maintaining and (direct) consistency-increasing transformations are given in section \ref{increase_maintainment} and the construction of application conditions and characterisation of basic rules is given in section \ref{appl_conds}. 
The repair process is presented in section \ref{repair} and we summarize related graph repair approaches in section \ref{rel_work} before section \ref{conclusion} concludes the paper.
 