\section{Introduction}

Model-driven software engineering is a suitable tool to deal with the increasing complexity of software development processes. 
Graphs and graph transformations have emerged as a suitable framework for model-driven software engineering, where models are represented by graphs and changes in models are represented by graph transformations. 
These models need to be consistent with respect to certain constraints, e.g. multiplicities if the model represents an object diagram. 
The concept of nested graph constraints has proven to be suitable for expressing these constraints \cite{habel2005nested}.
As models are changed during development, the new model may become inconsistent and the consistency must at some point be restored.

The problem of restoring this consistency is called \emph{graph repair}: Given a constraint $c$ and an inconsistent graph $G$, transform $G$ into a graph that satisfies $c$.
In particular, we will consider the problem of \emph{rule-based graph repair}, which is defined as follows: Given a graph $G$, a constraint $c$ and a set of rules $\mathcal{R}$, use the rules of $\mathcal{R}$ to transform $G$ into a graph that satisfies $c$.
Because of the versatility of graphs and graph transformations, the concept of graph repair can be used to resolve inconsistencies for all kinds of graph-like structures.  

There are several rule-based graph repair approaches and we will now briefly discuss some of them.
Habel and Pennemann have introduced the notions of \emph{consistency-preserving} and \emph{consistency-guaranteeing} transformations \cite{habel2009correctness}. As the name suggests, these binary notions allow to decide whether a transformation guarantees or preserves the consistency of the derived graph. In addition, they presented application conditions which guarantee that a transformation via rules equipped with them is consistency-preserving or consistency-guaranteeing respectively. 
Kosiol et al. have presented the graduated notions of \emph{consistency-sustaining} and \emph{consistency-improving} transformations and rules \cite{kosiol2022sustaining}. In contrast to the notions of consistency-preserving and consistency-guaranteeing, these notions allow an evaluation  of the magnitude of the inconsistency of a graph. They have also presented consistency-sustaining application conditions. 
Nassar et al. have optimised the construction of the application conditions mentioned above \cite{nassar2020constructing}.
There is a rule-based graph repair approach for so-called proper constraints introduced by Sandmann and Habel \cite{sandmann2019rule}.
Nassar et al. have presented a repair approach to repair multiplicities for EMF models \cite{nassar2017rule,nassar2017rule1}. 

None of these approaches is able to repair multiple constraints with a nesting level higher than $2$, and the application conditions constructed are very complex. 

In this thesis, we will introduce a rule-based graph repair process for certain constraints in alternating normal-form (ANF), so-called \emph{circular conflict-free constraints}, and for sets of these constraints. 

The main idea of our approach is to increase the consistency of a graph level by level. That is, given a constraint $c = \forall(C_1, \exists(C_2, \forall(C_3, \exists(C_4, \ldots)))$ and an inconsistent graph $G$, in the first step we repair the graph such that the derived graph satisfies $\forall(C_1, \exists(C_2, \true))$. 

In the next step, we repair the graph until it satisfies $\forall(C_1, \exists(C_2, \forall(C_3, \exists(C_4, \true)))$ and so on until finally, the derived graph satisfies $c$.
Note that we always repair two nesting levels at a time, since $\forall(C_1, \true)$ is always satisfied, and the satisfaction of $\forall(C_1, \false)$ immediately implies the satisfaction of $c$. Therefore, a process that repairs only one nesting level would return a consistent graph after at most two iterations, with the drawback that many occurrences of graphs are simply deleted. 

To do this, we will introduce new notions of consistency, called \emph{consistency-main\-tain\-ing} and \emph{con\-sis\-ten\-cy-increasing} transformations and rules. As the name suggests, \emph{consistency-maintaining} transformations do not decrease consistency, and  \emph{con\-sis\-ten\-cy-increasing} transformations increase the consistency. 
These notions are based on the first two levels of a constraint that are not satisfied. That is, given a graph $G$ that satisfies $\forall(C_1, \exists(C_2, \true))$ but not $\forall(C_1, \exists(C_2, \forall(C_3, \exists(C_4, \true)))$  only occurrences of $C_3$ are considered. This is because only occurrences of $C_3$ need to be repaired in order to derive a graph from $G$ that satisfies  $\forall(C_1, \exists(C_2, \forall(C_3, \exists(C_4, \true)))$. In particular, our notions are also able to detect the smallest changes in consistency, namely, the insertion or deletion of single elements, which will lead to a more consistent graph. 
Therefore, our notions are more fine-grained than the notions of consistency-preserving, consistency-guaranteeing, consistency-sustaining and consistency-improving transformations and rules. 

In addition, we will refine these notions into the notions of \emph{direct consistency-main\-tain\-ing} and \emph{direct consistency-in\-crea\-sing} transformations and rules, which can be expressed via second-order logic formulas. These notions completely prohibit the insertion of new violations. We will formally compare our newly introduced notions with those described above to ensure that they are indeed new notions of consistency, and to point out similarities and differences. 

We will introduce weaker notions of (direct) consistency-maintaining and (direct) consistency-increasing rules, called (direct) consistency-maintaining rules at layer and (direct) consistency-increasing rules at layer. 
Intuitively, a rule is (direct) consistency-maintaining at layer or (direct) consistency-increasing at layer if all its applications to graphs satisfying the constraint up to that layer are (direct) consistency-maintaining or (direct) consistency-increasing with respect to that constraint. With these notions, we will be able to construct less complex application conditions.
We will present direct consistency-maintaining and two types of direct consistency-increasing application conditions at layer. One for general rules and one for a specific set of rules called \emph{basic increasing rules}. For basic increasing rules, we are able to construct direct consistency-increasing application conditions at layer that are less restrictive and less complex compared to the general ones. 
We will show that each of the constructed application conditions produces consistency-maintaining rules or consistency-increasing rules at layer.

We introduce the notion of \emph{circular conflict-free constraints} and \emph{circular conflict-free sets of constraints}.
Intuitively, a constraint $c$ is \emph{circular conflict free}, if there is an ordering $C_0, \ldots, C_n$ of all graphs of $c$ such that there is no $j<i$ such that a repair of $C_i$ leads to the insertion of a new violation of $C_j$. Analogously, a set of constraint $\mathcal{C}$ is \emph{circular conflict free}, if there is an ordering $c_1, \ldots, c_n$ of all constraints of $\mathcal{C}$ such that there is no $j <i$ such that a repair of $c_i$ at all graphs satisfying $c_j$ leads to a graph  not satisfying $c_j$.

Finally, we present two rule-based graph repair approaches. The first is for a \emph{circular conflict-free} constraint and the second for a \emph{circular conflict-free set of constraints}, which uses the repair approach for one constraint. 
Both processes make use of a given set of rules $\mathcal{R}$ and we present a characterisation of when such a set of rules is able to repair the constraint or the set of constraints, respectively. We will use the notions of \emph{consistency-maintainment} and \emph{consistency-increasement} to characterise sequences of transformations, that are able to repair an occurrence of a graph $C_k$ without introducing new violations of certain other graphs of the constraint, and in particular of $C_k$ itself.
Of course, this is only a sufficient criterion, since it depends strongly on the input graph whether a consistent graph can be derived by applying the rules of $\mathcal{R}$. In addition, we show the correctness and termination of our approach. 



This thesis is structured as follows:
Formal prerequisites are introduced in Section \ref{preliminaries}. The notions of (direct) consistency-maintaining and (direct) consistency-increasing transformations and rules are given in Section \ref{increase_maintainment} and the construction of application conditions and characterisation of basic rules is given in Section \ref{appl_conds}. 
The repair process is presented in Section \ref{repair}. We summarise related graph repair approaches in Section \ref{rel_work}, before concluding the paper with Section \ref{conclusion}.
 