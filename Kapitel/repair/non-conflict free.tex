\subsection{Rule-based Graph Repair for multiple Constraints}

We will now present our rule-based repair approach for a set of constraints in UANF. 


\begin{definition}[\textbf{satisfaction of constraint sets}]
	Let $\mathcal{C}$ be a set of constraints. A graph $G$ satisfies $\mathcal{C}$, denoted by $G \models \mathcal{C}$, if $G \models \bigwedge_{c \in \mathcal{C}} c$. The set $\mathcal{C}$ is called \emph{satisfiable } if there exists a graph $G$ with $G \models \mathcal{C}$.
\end{definition}

To guarantee that a set of constraints can be repaired by a set of rules, we need to extend the notion of repairing sets such that a set of rules is called a \emph{repairing set} for a set of constraints if it is a repairing set for every constraint in the constraint set.
\begin{definition}[\textbf{repairing set for a set of constraints}]
	Given a set of constraints $\mathcal{C}$ and a set of rules $\mathcal{R}$. 
Then $\mathcal{R}$ is called a \emph{repairing set for $\mathcal{C}$} if $\mathcal{R}$ is a repairing set for all constraints $c \in \mathcal{C}$.

\end{definition}

We also extend the notion of conflicts to \emph{conflicts between constraints}. 
Intuitively, a constraint $c$ causes a conflict for another constraint $c'$ if one of its graphs causes a conflict for a graph of $c'$.

\begin{definition}[\textbf{conflict between constraints}]
	Let the constraints $c$, $c'$ in UANF and a set of rules $\mathcal{R}$ be given. 
Then $c$ \emph{causes a conflict} for $c'$ w.r.t. $\mathcal{R}$ if a repairing sequence 
$$C_k = G_0 \Longrightarrow_{\rho_1,m_1} \dots \Longrightarrow_{\rho_n,m_n}
	G_n$$
	for a graph $C_k$ of $c$ exists such that the concurrent rule of that 
	sequence is not a basic consistency maintaining rule w.r.t. $\forall(C_j, 
	\false)$ or $\exists(C_j,\true)$ for any universally or existentially bounded 
	graph $C_j$ of $c'$.

\end{definition}

The following lemma is a useful statement for proving the correctness of our repair approach. 
It states that applying a repairing sequence to a constraint $c$ cannot destroy the satisfaction of $c'$ if $c$ causes no conflict for $c'$.

\begin{lemma}\label{lemma_preserving}
	Given a set of rules $\mathcal{R}$ and constraints $c$ and $c'$ in UANF such that $c$ causes no conflict for $c'$ w.r.t. $\mathcal{R}$.
Then, the concurrent rule $\rho$ of any graph of $c$ is a $c'$-preserving rule.
\end{lemma}
\begin{proof}
	Suppose $\rho$ is not a $c'$-preserving rule.  Then there exists a transformation $t : G \Longrightarrow_{\rho,m} H$ such that $G \models c'$ and $H \not \models c'$. 
	Therefore, either a universally bound graph of $c'$ has been inserted or an existentially bound graph of $c'$ has been removed.
	It follows that $\rho$ is not a basic maintaining rule w.r.t. $\forall(C_j, \false)$ for all universally bound graphs $C_j$ of $c'$ or $\rho$ is not a basic maintaining rule w.r.t. $\exists(C_j, \true)$ for all existentially bound graphs $C_j$ of $c'$, which is a contradiction.
\end{proof}

\begin{figure}
\centering
\begin{algorithm}[H]
	%\DontPrintSemicolon
	\KwData{A graph $G$, circular constraint-conflict free set of constraints $
	\mathcal{C}$ and a repairing set $\mathcal{R}$ for $\mathcal{C}$.}
	\KwResult{A graph $H$ with $H \models \bigwedge_{c \in \mathcal{C}}c$.}
	$(c_1, \ldots ,c_n) \gets $ topological ordering of $\mathcal{C}$ \;
	\For{$i \gets 1$ \KwTo $n$}{
		
		Repair $c_i$ in $G$ with Algorithm \ref{Algo_conflict_free}, let $H$ be the 		returned graph \;
		$G \gets H$ \;
	
	}
	\KwRet{G}\;
	

\caption{Repair for a circular constraint-conflict free set of constraints }\label{Algo_non-conflict_free}

\end{algorithm}
\end{figure}

The \emph{conflict graph} for a set of constraints and \emph{circular conflicts of a set of constraints} are defined in a similar way to the conflict graph and circular conflicts for one constraint. 
A set of constraints is called \emph{circular conflict free} if each of its constraints is circular conflict-free and there is no sequence $c = c_0, \ldots, c_n = c$ such that $c_i$ has a conflict with $c_{i+1}$ for all $0\leq i < n$. In other words, the conflict graph of this set is acyclic.

\begin{definition}[\textbf{conflict graphs, circular conflicts}]
	Given a set of rules $\mathcal{R}$ and a set of constraints $\mathcal{C}$ in UANF. The \emph{conflict graph} of $\mathcal{C}$  w.r.t. $\mathcal{R}$ is constructed in the following way. For each constraint $c \in \mathcal{C}$ there is a node. If $c$ causes a conflict for $c'$ w.r.t. $\mathcal{R}$, there is an edge $e$ with $\src(e) = c$ and $\tar(e) = c'$.
	
	A constraint $c$ has a \emph{transitive conflict} with $c'$ if the 
conflict graph of $\mathcal{C}$ contains a path from $c$ to $c'$. 
A constraint $c$ has a \emph{circular conflict} if $c$ has a transitive conflict with itself. 
A set of constraints $\mathcal{C}$ is called \emph{circular conflict free w.r.t. $\mathcal{R}$} if every constraint in $\mathcal{C}$ is circular conflict-free and $\mathcal{C}$ contains no circular conflicts.
\end{definition}

\begin{example}
	Consider the rules \emph{\texttt{resolve, resolve2, createFeatures}} and constraints $c_1$ and $c_5$ given in Figures \ref{fig:constraint_conflict} and \ref{fig:constraints}.
The constraint set $\mathcal{C} = \{c_1, c_5\}$ is a multiplicity which states that \enquote{Each node of type \emph{\texttt{Class}} is connected to exactly two nodes of type \emph{\texttt{Feature}}}.
	With the rule set $\mathcal{R}_1 = \{\emph{\texttt{resolve}}, \emph{\texttt{createFeatures}}\}$, there is only one conflict in $\mathcal{C}$; $c_1$ causes a conflict for $c_5$, since applying \emph{\texttt{createFeatures}} could lead to inserting the universally bound graph of $c_5$. With the rule set $\mathcal{R}_2 = \{\emph{\texttt{resolve2}}, \emph{\texttt{createFeatures}}\}$ there are two conflicts. Again, $c_1$ causes a conflict for $c_5$ and  $c_5$ causes a conflict for $c_1$, since applying \emph{\texttt{resolve}} can destroy an occurrence of the existentially bound graph of $c_1$.
	
	Therefore, our approach will terminate with the rule set $\mathcal{R}_1$ 
	but not with  $\mathcal{R}_2$ because $\mathcal{C}$ is 
	not circular conflict free w.r.t. $\mathcal{R}_2$.
	
\end{example}
\begin{figure}
	\centering
	\includegraphics[scale=0.8]{figures/images/example_conflicts_graph_constraints}

	\caption{Constraints $c_5$ and conflicts graphs of the constraint set 
	$\{c_1,c_5\}$ with the rule sets $\{\texttt{resolve},\texttt{createFeatures}\}$ and $\{\texttt{resolve2},\texttt{createFeatures}\}$. }\label{fig:constraint_conflict}
\end{figure}

Our repair process exploits the fact that the conflict graph of a circular conflict-free set of constraints in UANF is acyclic. 
In particular, our approach uses the \emph{topological ordering} of this 
conflict graph.

\begin{definition}[\textbf{topological ordering of a graph}]
	Given is a graph $G$. A sequence $(v_1, \ldots, v_n)$ of nodes of $G$ is called a \emph{topological ordering} of $G$ if no edge $e \in E_G$ exists with $\src(e) = v_i$, $\tar(e) = v_j$ and $i \geq j$. The topological ordering of a circular conflict-free set of constraints $\mathcal{C}$ is the topological order of its conflict graph.
\end{definition}
It is well known that every directed acyclic graph has a topological ordering that can be computed in $\Theta(|V| + |E|)$ where $V$ and $E$ are the set of nodes and edges of the respective graph \cite{cormen2022introduction}. Therefore every conflict graph of a circular conflict-free set of constraints also has a topological ordering.

The repair process is given in Algorithm \ref{Algo_non-conflict_free} and proceeds as follows. 
First, the topological ordering of the constraint set is determined (line 1). Then Algorithm \ref{Algo_conflict_free} is used to repair each constraint of $\mathcal{C}$ in the order of the topological ordering (lines 2 -- 4). 
This ensures that the satisfaction of a constraint that has already been repaired is not destroyed by the repair of another constraint. 
\begin{theorem}
	Given rule set $\mathcal{R}$, a graph $G$ and a satisfiable of constraints in UANF $\mathcal{C}$ which is circular conflict free w.r.t. $\mathcal{R}$.
Then Algorithm \ref{Algo_non-conflict_free} terminates and returns a graph $H$ with $H \models \mathcal{C}$.
\end{theorem}
\begin{proof}
	Since $\mathcal{C}$ is finite and every $c \in \mathcal{C}$ is circular conflict-free,  Algorithm \ref{Algo_conflict_free} terminates for every $c \in \mathcal{C}$. Therefore,  Algorithm \ref{Algo_non-conflict_free} will also terminate.
	It remains to show that the returned graph satisfies $\mathcal{C}$.
	Let $(c_1, \ldots, c_n)$ be a topological ordering of $\mathcal{C}$ w.r.t. $\mathcal{R}$. Then, no constraint $c_j$ with $j \neq 1$ causes a conflict for $c_1$, and  Lemma \ref{lemma_preserving} implies that the concurrent rule of every repairing sequence for every graph of $c_i$ with $2 \leq i \leq n$ is a $c_1$-preserving rule.
	In general, the concurrent rule of each repairing sequence for graphs of $c_j$ is a
	$c_i$-preserving rule if $i < j$.
	After one iteration it holds that $G \models c_1$. Suppose that after $m$ iterations it holds that $G \models c_i$ for all $1 \leq i \leq m$.
	In iteration $m+1$, $c_{m+1}$ is repaired by Algorithm 
 \ref{Algo_non-conflict_free}. Since each concurrent rule of each repairing sequence for graphs of  $c_{m+1}$ is a $c_i$-preserving rule for all $1 \leq i \leq m$ and the application of repairing sequence can be replaced by an application of its concurrent rule, it follows that  $ H \models c_i$ for all $1 \leq i \leq m+1$.  
	Therefore, after $n$ iterations, $H \models c_i$ for all $1 \leq i \leq n$ 
	and the returned graph $G$ satisfies 
	$\mathcal{C}$.
\end{proof}