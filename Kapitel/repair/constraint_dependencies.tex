\subsection{Conflicts within Conditions}
During a repair process, inserting elements of an existentially bound constraint $C_j$ could also insert new occurrences of universally bound graphs $C_i$.
This insertion is unproblematic if $i > \kmax+2$, but if $i \leq \kmax+2$ it could lead either to the insertion of new violations or to a reduction of the largest satisfied layer.
Additionally, removing elements of a universally bound graph $C_j$ may destroy occurrences of an existentially bound graph $C_i$.
Again, this can lead to the insertion of new violations or a reduction of the largest satisfied layer.

We will now introduce the notion of \emph{conflicts within conditions}, which states that $C_j$ has a conflict with $C_i$ if and only if one of the cases described above can occur. 
Note that conflicts can only occur between existentially and universally bound graphs, and vice versa. There cannot be a conflict between two existentially bound or two universally bound graphs, since the insertion of elements cannot destroy occurrences of existentially bound graphs, and the removal of elements cannot insert new occurrences of universally bound graphs.


\begin{definition}[\textbf{conflicts within conditions}]\label{def_conflicts}
	Given a condition $c$ in UANF.
	A existentially bound graph $C_k$ \emph{causes a conflict for} an universally bound 
	graph $C_j$ if there is a transformation $t: G \Longrightarrow_{\rho} H$ with 
	$\rho = \rle{C_{k-1}}{\id}{C_{k-1}}{a_{k-1}}{C_k}$ such that
	$$\exists p:C_j \inj H(\neg \exists 	q:C_j \inj G(\track_t \circ q = p)).$$
	A universally bound graph $C_j$ \emph{causes a conflict for} an existentially bound 
	graph $C_k$ if there is a transformation $t: G \Longrightarrow_{\rho} H$ with 
	$\rho = \rle{C_j}{a_{j-1}^r}{C}{\id}{C}$ for any $C \in \ig{C_{j-1}}{C_j}$ 
	such that
	$$\exists p: C_k \inj G(\track_t \circ p \text{ is not total}).$$

\end{definition}


Additionally, we introduce \emph{conflict graphs}, which represent the conflicts within a condition via a graph.
With these we are able to define \emph{transitive conflicts}, \emph{circular conflicts} and their absence, which will be a necessary property for the termination of our repair process.
Intuitively, as the name suggests, a condition $c$ contains a circular conflict if a graph $C_k$ has a conflict with itself or if there exists a sequence $C_k= C_{j_1}, \ldots, C_{j_n}= C_k$ of graphs such that $C_{j_i}$ has a conflict with $C_{j_{i+1}}$. We can check this property by checking whether the 
conflict graph contains cycles.
Note that conflict graphs contain additional edges that do not correspond to the conflicts within the constraint. 
These edges ensure that during repair it can be chosen, whether a violation will be removed by deletion or insertion. Otherwise, this needs to be alternating. That means, after a violation has been removed by deletion, all violations introduced by this deletion must be removed by insertion and vice versa. Therefore the definition of repairing set also becomes more restrictive.


\begin{definition}[\textbf{conflict graph, circular conflicts}]\label{conflicts_within}
	Let a condition $c$ in UANF be given. The \emph{conflict graph of $c$} is 
	constructed in the following way. 
	For every $0 \leq k < \nlvl(c)$ there is a node labelled $k$.	If  $C_k$ causes a conflict for $C_j$, there is an edge $e$ with $\src(e) = k'$ and $\tar(e) = j'$ if either $k = k'$ or $k = k'+1$, either $j = j'$ or $j = j'+1$ and $j' \neq k'$.
	
	A graph $C_k$ causes a \emph{transitive conflict} with $C_j$ if there exists a path from $k$ to $j$ in the conflict graph of $c$.
	A graph $C_k$ has a \emph{circular conflict} if $C_k$ has a transitive conflict with itself.
A condition $c$ is called \emph{circular conflict free} if $c$ does not contain a circular conflict.
\end{definition}
In other words, a condition $c$ is \emph{circular conflict free} if its conflict graph is acyclic.
\begin{example}
	Consider constraint $c_3$ and the transformations $t_1$ and $t_2$ shown in 
	Figure \ref{fig:conflict_example}.
	Transformation $t_1$ shows that $C_1$ has a conflict with $C_2$ because the
	rule $\rho = \rle{C_1}{\id}{C_1}{a_1}{C_2}$ has been applied and there is a newly inserted occurrence of $C_1$.
	Transformation $t_2$ shows that $C_2$ has a conflict with $C_1$, since the 
	rule $\rle{C_2}{a_1}{C_1}{\id}{C_1}$ has been applied and one occurrence 
	of $C_1$ has been destroyed. 
	So $c_3$ contains a circular conflict, the conflict graph 
	of $c_3$ is shown in Figure \ref{fig:conflict_graph}.
	
	In general, the statement \enquote{ $C_j$ has a conflict with $C_k$} 
	does not imply that \enquote{ $C_k$ has a conflict with $C_j$} as shown by 
	constraint $c_4$ given in Figure \ref{fig:conflict_example}.
	The conflict graph of $c_4$ is also shown in Figure \ref{fig:conflict_graph}. 
	It can be seen that $c_4$ is a circular conflict free constraint.
	
\end{example}

\begin{figure}
\centering
\includegraphics[scale=0.9]{figures/sources/counterexample_circles}

\caption{Constraint $c_3$ and the transformation that show the existence of 
conflicts between $C_1$ and $C_2$ and $C_2$ and $C_1$.}\label{fig:conflict_example}
\end{figure}
\begin{figure}
\centering
\includegraphics[scale=1]{figures/sources/example_conflict_graph}

\caption{Conflict graphs of $c_3$ and $c_4$.}\label{fig:conflict_graph}
\end{figure}
 
We will now present a characterisations of conflicts.
For $C_k$, which is existentially bound and $C_j$, which is universally bound, the characterisation checks whether for each overlap of $C_k$ and $C_j$, such that the overlap morphisms restricted to $C_k \setminus C_{k+1}$ and $C_j$ overlap, the rule that only deletes $C_k\setminus C_{k-1}$ is applicable.
If this is not possible, there does not exist a transformation as described in Definition \ref{conflicts_within}.
If $C_k$ is universally bound and $C_j$ is existentially bound, the characterisation checks whether for each overlap of $C_k$ and $C_j$ such that 
the elements of $C_k \setminus C_{k-1}$ and $C_j \setminus C_{j-1}$ overlap, a rule is applicable that only removes elements of $C_k \setminus C_{k-1}$. 
Again, if this is not possible, there is no transformation as described in Definition \ref{conflicts_within}.

\begin{lemma}\label{charact_conflict}
Given a constraint $c$ in UANF.
\begin{enumerate}
	\item 
		Let $C_k$ be an existentially and $C_j$ a universally bound graph of 
		$c$. Then, $C_k$ causes a conflict for $C_j$, if and only if there is an overlap $P 
		\in \overlay(C_k,C_j)$ with 
		$$ i_{C_k}^P(C_k\setminus C_{k-1}) 
		\cap i_{C_j}^P(C_j) \neq \emptyset$$ 
		and the rule $\rho = \rle{C_k}{a_{k-1}}{C_{k-1}}{\id}{C_{k-1}}$ is 
		applicable at match $i_{C_k}^P$.

	\item 
		Let $C_j$ be a universally and $C_k$ be an existentially bound graph of 
		$c$. Then, $C_j$ causes a conflict with $C_k$ if an only if there is an overlap $P \in \ig{C_j}{C_k}$ with 
		$$i_{C_j}^P(C_j \setminus C_{j-1}) \cap i_{C_k}^P(C_k) 
		\neq \emptyset$$ 
		and a rule $\rho = \rle{C_j}{a_{j-1}^r}{C}{\id}{C}$ with $C \in 
		\ig{C_{j-1}}{C_j}$ and $i_{C_j}^P(C_k \setminus C) \cap i_{C_k}^P(C_k 
		)\neq \emptyset $ is applicable at match $i_{C_j}^P$.

\end{enumerate}
\end{lemma}

\begin{proof}
Given a condition $c$ in UANF.
\begin{enumerate}
\item 
\enquote{$\Longrightarrow$}: Let $C_k$ be an existentially bound graph that causes a conflict for an universally bound graph $C_j$. 
Then, there is a transformation $t: G \Longrightarrow_{\rho} H$ with $\rho = \rle{C_{k-1}}{\id}{C_{k-1}}{a_{k-1}}{C_k}$ such that a new occurrence $p$ of $C_j$ is inserted. 
Since only elements of $C_k \setminus C_{k-1}$ are inserted, it holds that $p(C_j) \cap n(C_k\setminus C_{k-1}) \neq \emptyset$, with $n$ being the co-match of $t$. 
The graph $P = p(C_j) \cup n(C_k)$ is the overlap we are looking for, and the rule $\rho^{-1} = \rle{C_{k}}{a_{k-1}}{C_{k-1}}{\id}{C_{k-1}}$ must be applicable at the match $i_{C_k} ^P$. 
\\
\enquote{$\Longleftarrow$}:
Let $C_k$ be an existentially and $C_j$ an universally bound graph such that there exists an overlap $P \in \overlay(C_k,C_j)$ with $i_{C_k}^P(C_k\setminus C_{k-1}) \cap i_{C_j }^P(C_j\setminus C_{j-1}) \neq \emptyset$ so that the rule $\rho = \rle{C_{k}}{a_{k-1}}{C_{k-1}}{\id}{C_{k-1}}$ is applicable at match $i_{C_k}^P$.
Then the inverse transformation of $t: P \Longrightarrow_{\rho,i_{C_k}^P} H$ is the transformation we are looking for and $C_k$ causes a conflict for $C_j$. 

\item
\enquote{$\Longrightarrow$}: Let $C_j$ be a universally bound graph that causes a conflict for an existentially bound graph $C_k$. 
Then, there is a transformation $t: G \Longrightarrow_{\rho} H$ with $\rho = \rle{C_j}{a_{j-1}^r}{C}{\id}{C}$ and $C \in \ig{C_{j-1}}{C_j}$ such that $\track_t \circ p$ is no total for an occurrence $p : C_k \inj G$. 
The graph $p(C_k) \cup m(C_j)$ is the overlap we are looking for and  $i_{C_j}^P(C_j \setminus C) \cap i_{C_k}^P(C_k) \neq \emptyset$ must hold since $\rho$ only deletes elements of $C_j\setminus C$.
\\
\enquote{$\Longleftarrow$}:
Let $C_j$ be universally and $C_k$ existentially bound such that there is an  overlap $P \in \overlay(C_j,C_k)$ with $i_{C_j}^P(C_j \setminus C_{j-1}) \cap i_{C_k}^P(C_k) \neq \emptyset$  so that a rule $\rho = 	\rle{C_j}{a_{j-1}^r}{C}{\id}{C}$ with $C \in \ig{C_{j-1}}{C_j}$ is applicable at match $i_{C_j}^P$.
Then, the transformation of $t: P \Longrightarrow_{\rho, i_{C_j}^P} H$ is the transformation we are looking for and $C_j$ causes a conflict for $C_k$.
\end{enumerate}
\end{proof}

Note that this characterization can also be expressed via the notions of \emph{conflicts between rules } and  \emph{parallel independency} \cite{lambers2006conflict}.
Using these notions, the first part of Lemma \ref{charact_conflict} can be expressed as: An existentially bound graph $C_k$ causes a conflict for a universally bound graph $C_j$ if and only if the rules $\rho = \rle{C_k}{a_{k-1}}{C_{k-1}}{\id}{C_{k-1}}$ and $\rle{C_j}{\id}{C_{j}}{\id}{C_{j}}$ are parallel independent. 
The second part can be expressed as: A universally bound graph $C_j$ causes a conflict for an existentially bound graph $C_j$ if and only if the rules $\rle{C_j}{a_{j-1}^r}{C}{\id}{C}$ and $\rle{C_k}{\id}{C_{k}}{\id}{C_{k}}$ are parallel independent for all $C \in \ig{C_{j-1}}{C_j}$.

%Our second characterisation of conflicts is based on the notion of basic maintaining rules. 
%\begin{lemma}\label{basic_conflict}
%	Let a condition $c$ in UANF be given. 
%	\begin{enumerate}
%		\item 
%			Let $C_k$ be an existentially and $C_j$ be a universally bound graph 
%			of $c$. Then, $C_k$ has a  conflict with $C_j$ if and only if  
%			the rule $\rho = \rle{C_{k-1}}{\id}{C_{k-1}}{a_{k-1}}{C_k}$
%			is not a basic consistency maintaining rule up to layer $-1$ w.r.t. 
%			$\forall(a_{j-1} \circ \ldots 
%			\circ a_0: C_0 \inj C_j, \false)$.
%		\item
%			Let $C_k$ be a universally and $C_j$ be an existentially bound graph 
%			of $c$. Then, $C_k$ has a conflict with $C_j$ if and only if each 
%			rule
%			$\rho = \rle{C_k}{a_{k-1}^r}{C}{\id}{C}$ with $C \in 
%			\ig{C_{k-1}}{C_k}$ is not a basic consistency  maintaining rule up to 
%			layer $-1$ w.r.t. 
%			$\exists(a_{j-1}\circ \ldots \circ a_0: C_0 \inj C_j, \true)$.
%	\end{enumerate}
%\end{lemma}
%\begin{proof}
%	\begin{enumerate}
%		\item 
%			Let $C_k$ be an existentially and $C_j$ an universally bound graph of 
%			$c$.
%			
%			\enquote{$\Longrightarrow$}: Assume that $C_k$ has a conflict with 
%			$C_j$. Therefore, there does exist a transformation $t: G 
%			\Longrightarrow_{\rho} H$ with $\rho = \rle{C_{k-1}}{\id}{C_{k-1}}
%			{a_{k-1}}{C_k}$ such that a new occurrence $p: C_j \inj H$ has been 
%			inserted. 
%			Then, $t$ does not satisfy the universally condition and
%			$\rho$ is not a basic maintaining rule up to layer $-1$.
%			
%			\enquote{$\Longleftarrow$}: Assume that $\rho = \rle{C_{k-1}}{\id}
%			{C_{k-1}}{a_{k-1}}{C_k}$ is not a basic maintaining rule up to layer 
%			$-1$ w.r.t. $\forall(a_{j-1} \circ \ldots \circ a_0: C_0 \inj C_j, 
%			\false)$. Since this constraint only contains universally 
%			bound graphs, there must exist a transformation $t: G 
%			\Longrightarrow_{\rho}$ that does not satisfy 
%			the universally condition. Therefore, a new occurrence of $C_j$ 
%			has been inserted by $t$ and with Definition \ref{def_conflicts} 
%			follows that $C_k$ has a conflict with $C_j$.
%			
%		\item 
%			Let $C_k$ be an universally and $C_j$ be an existentially bound graph 
%			of $c$ and $c' = \exists(a_{j-1}\circ \ldots \circ a_0: C_0 \inj C_j, 
%			\true)$ .
%			
%			\enquote{$\Longrightarrow$}: Assume that $C_k$ has a conflict with 
%			$C_j$. Therefore, there is a transformation $t: G 
%			\Longrightarrow_{\rho} H$ with $\rho = \rle{C_k}{a_{k-1}^r}{C}{\id} 
%			{C}$, for a $C \in \ig{C_{k-1}}{C_k}$ such that an occurrence of 
%			$C_j$ has been destroyed. Then, $t$ does not satisfy 
%			the existentially condition.
%			Therefore, $\rho$ is not a basic consistency maintaining rule 
%			w.r.t. $c$ up to layer $-1$. 
%			
%			\enquote{$\Longleftarrow$}: Assume that $\rho = \rle{C_k}{a_{k-1}^r}
%			{C}{\id} {C}$ is a not a basic increasing rule w.r.t. $c$ up to 
%			layer $-1$.  
%			Therefore, there is transformation $t: G \Longrightarrow_{
%			\rho} H$ that does not satisfy the existentially condition and an occurrence of $C_j$ has been removed by $t$.
%			It follows that  $C_k$ has a conflict with $C_j$.
%
%			
%	\end{enumerate}
%\end{proof}
%\begin{definition}[\textbf{Conflicts between conditions}]
%	Let conditions $c_1$ and $c_2$ be given. Then, $c_1$ has a conflict with $c_2$ if 
%	graphs $C_k$ and $C_j$ of $c_1$ and $c_2$exist such that $C_k$ has
%	a conflict with $C_j$ or vice versa. 
%	A set of conditions is called \emph{conflict free} if no conditions $c_1$ and $c_2$ 
%	exist such that $c_1$ has a conflict with $c_2$ or vice versa.
%	A condition $c_1$ has a \emph{transitive conflict with} $c_2$ if a condition $c'$
%	exists such that $c_1$ has a conflict with $c'$ and $c'$ has a (transitive) conflict
%	with $c_2$.
%	A set of conditions does contain a \emph{circular conflict} if a condition $c$ has 
%	a transitive conflict with itself. 
%	Otherwise, the set is called \emph{circular conflict free}.
%\end{definition}

%\begin{definition}[\textbf{conflict free graphs}]
%	Let a graph $G$ and a constraint $c$ in UANF be given.
%	Then, $G$ is called \emph{conflict free} w.r.t $c$ if for
%	each transformation
%	$t: G \Longrightarrow_{\rho,m} H$ with $$\rho =  C_{\kmax}
%	\overset{\id}{\hookleftarrow} C_{\kmax} 
%	\overset{i_{C_{\kmax}}}{\hookrightarrow} C'$$
%	and $C' \in \ig{C_{\kmax+1}}{C_{\kmax+2}}$ or $$\rho = 
%	 C_{\kmax+1}
%	\overset{\i_{C'}}{\hookleftarrow} C' 
%	\overset{\id}{\hookrightarrow} C'$$  and $C' \in 
%	\ig{C_{\kmax}}{C_{\kmax+1 }}$ and (\ref{direct_improving_3})
%	 and (\ref{direct_improving_4}) hold. 
%\end{definition}
%
%\begin{lemma}
%	Let a graph $G$ and a constraint $c$ in UANF be given.
%	Let $P$ be the set of all 
%	Then, $G$ is conflict free w.r.t $c$ if and only if either 
%	$c$ is
%	conflict free or
%	$$G \models \bigwedge_{P \in P} \forall(a: \emptyset \inj P, \false)$$
%\end{lemma}
%\begin{proof}
%
%\end{proof}

