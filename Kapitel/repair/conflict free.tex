\subsection{Rule-based Graph Repair for one Constraint}

In the following, we present our graph repair process for one circular conflict 
free constraint in UANF. 
The process is shown in Algorithm \ref{Algo_conflict_free} and proceeds in 
the following way. 
The algorithm starts by finding all occurrences of $C_{\kmax +2}$ that do 
not satisfy $\cut{0}{\scond{\kmax+2}{c}}$ (line 2). This condition is equal to $\false$ if 
$\kmax +2 = \nlvl(c)-2$ and equal to $\exists(C_{\kmax +2}, \true)$ otherwise. 
If $P$ is empty, it must follow that $G \models_{\kmax+2} c$ and therefore, 
we will apply repairing sequences at these occurrences. 
It might be enough to only repair some of these occurrences. 
Since it is unknown which of these are able to increase the 
satisfaction at layer, we choose one uniformly at random (line 3). 
For example, for existentially bound constraints $d$, that means, its equivalent 
constraint in UANF is equal to $\forall(\emptyset,d)$, there might exist 
occurrences of  $C_{\kmax +2}$, whose repair will not  lead to an increase
of the satisfaction at layer. 

There are two possible ways to repair the chosen occurrence, either 
by destroying it, or by inserting elements such that the occurrence 
satisfies   $\cut{0}{\scond{\kmax+2}{c}}$. The Algorithm chooses one of these options (line 4) and applies the corresponding repairing sequence 
(line 5--11). Note that a repairing sequence for $C_{\kmax+2}$ might not exist, since this graph is universally bound. If this is the case we use the repairing sequence for $C_{\kmax+3}$. This must exist since $C_{\kmax+3}$ is existentially bound. 

If the repairing sequence for  $C_{\kmax+2}$ has been applied, occurrences 
of existentially bound graphs might have been destroyed. 
Note that this can only be occurrences of graphs $C_i$ such that $C_{\kmax+2}$
has a conflict with $C_i$.
This might lead to a decrease of satisfaction of layer. Therefore, the algorithm finds all of these
destroyed occurrences, in particular it finds all occurrences $p$ of universally bound 
graphs $C_i$ such that an occurrence $q$ of $C_{i+1}$ with $p = q \circ a_j$ has been removed (line 7). 

If the repairing sequence for $C_{\kmax+3}$ has been applied, occurrences 
of universally bound graphs might have been inserted. 
Again, this can only be occurrences of graphs $C_i$ such that 
$C_{\kmax+2}$ has a conflict with $C_i$ and this might lead to a decrease of 
satisfaction at layer. Again, the algorithm finds all inserted occurrences of 
universally bound graphs (line 10).
If the satisfaction at layer has not been decreased, the algorithm chooses 
the next occurrence in $P$.

Otherwise, the satisfaction at layer needs to be restored. For this, the occurrences contained in $M$ need to be repaired. 
The repair of these occurrences might again lead to an insertion of existentially bound graphs or the removal of universally bound ones. 
These  occurrences are added to $H$ and this process repeats until 
the satisfaction at layer is restored, i.e. $H \models_{\kmax}c$ (line 12 -- 25).
This whole process will repeat until a graph satisfying $c$ is derived. 

Form this, it becomes clear, why $c$ has to be circular conflict free. 
For a constraint with circular conflicts, during the restore phase, new occurrence 
of $C_{\kmax +2}$ can be inserted an occurrence of $C_{\kmax +3}$ can be removed.
In particular cases, this could lead to an infinite loop and therefore, there is no guarantee that this algorithm will terminate. 
For example, consider constraint $c_3$ given in Figure \ref{fig:conflict_example}.
The set of rules that are used for the transformations $t_1$ and $t_2$ in Figure 
\ref{fig:conflict_example} forms a repairing set. 
During a repair process using Algorithm \ref{Algo_conflict_free} with the starting 
graph being the first graph of $t_1$ it might be possible that Algorithm \ref{Algo_conflict_free} runs into an infinite loop, by alternately applying $t_1$ and $t_2$.   

A optimization of the repair algorithm in terms of the number of inserted or deleted elements can be performed by using partial repairing sequences if possible.
For example, consider the repairing sequence 
$$ C_k \Longrightarrow C_1 \Longrightarrow \ldots \Longrightarrow C_{k+1}$$
with $C_1 \in \ig{C_k}{C_{k+1}}$. 
For an occurrence $p$ of $C_k$ that already satisfies $\exists(C_1, \true)$
it might be sufficient to only apply the sequence 
$$C_1 \Longrightarrow \ldots \Longrightarrow C_{k+1}$$ at $p$. 
But, after this, it needs to be checked that no occurrences of existentially 
bound graphs have been destroyed and that no occurrence of universally bound graphs 
$C_i$ such that $C_k$ has no conflict with $C_i$ have been inserted. 
If this is the case, the transformations need to be undone and another (partial) 
repairing sequence needs to be used. 
Even if this would lead to an optimization in terms of the number of inserted and 
deleted elements, due to the reversion of transformations, this will lead to an increase of runtime.

For each circular conflict free constraint, Algorithm \ref{Algo_conflict_free} will always terminate as shown by the following Theorem.


\begin{figure}
\centering
\begin{algorithm}[H]
	%\DontPrintSemicolon
	\KwData{A graph $G$, a conflict free constraint in UANF and a repairing set $\mathcal{R}$ for each layer $\kmax  < k \nlvl(c)$}
	\KwResult{A graph $H$ with $H \models c$}
	\For{$i = \kmax+1$ \KwTo $\nlvl(c)$}{
		\While{$G \not \models_i c$}{
			Choose one occurrence $p: C_i \inj G$ uniformly 
			at random \;
			apply a repairing sequence
			$G \Longrightarrow \ldots \Longrightarrow H$\;
					
		}
	}


\caption{Repair for conflict free constraints and repairing set}\label{Algo_conflict_free}

\end{algorithm}
\end{figure}



\begin{theorem}
	Let a graph $G$, a circular conflict free condition in UANF and 
	a repairing set $\mathcal{R}$ be given. 
	Then, Algorithm \ref{Algo_conflict_free} with input $G$,$c$ and $\mathcal{R}$ 
	terminates and returns a graph $H$ with $H \models c$.
\end{theorem}
\begin{proof}
	If Algorithm \ref{Algo_conflict_free} terminates, it returns a graph 
	satisfying $c$. Therefore, it is sufficient to show that Algorithm 
	\ref{Algo_conflict_free} will terminate.
	Since $G$ is finite, the set $P$ must also be finite.
	If a repairing sequence has been applied, the set 
	$M$ only contains occurrences of graphs $C_j$ such that 
	$C_{\kmax +2}$ has a (transitive) conflict with $C_j$ since the 
	repairing sequence is not able to destroy or insert occurrences 
	of $C_i$ such that $C_{\kmax +2}$ has no (transitive) conflict with $C_i$.
	Because $G$ is finite, $|M|$ must also be finite.
	
	If the derived graph does not satisfy $\cut{\maxk{c}{G}}{c}$, we need 
	to restore the satisfaction at layer. 
	Because the satisfaction at layer only decreases if an occurrence of an 
	existentially bound graph has been destroyed or an occurrence of 
	universally bound graphs has been inserted and $M$ does contain all these 
	occurrences, we only need to consider the occurrences contained in $M$. 
	The application of repairing sequences at occurrences $p :C_j \inj H \in M$
	could again lead to an insertion of universally bound or an removal 
	of existentially bound graphs. The set $M'$ contains all these 
	occurrences 
	and again, this are only occurrences of $C_i$ such that $C_j$ has a 
	(transitive) conflict with $C_i$.
	Since $c$ is circular conflict free, $M'$ cannot contain any occurrences 
	of $C_{\kmax +2}$, otherwise, $C_j$ would have a (transitive) conflict with 
	$C_{\kmax +2}$ and therefore $C_{\kmax +2}$ has circular conflict. Therefore 
	no occurrences of $C_{\kmax +3}$ will be destroyed an no occurrences 
	of $C_{\kmax+2}$ will be inserted.
	Additionally, $C_{\kmax +2}$ has a (transitive) conflict with $C_i$ and 
	the repair of any $p \in M'$ will not lead to an insertion of an occurrence 
	of 	$C_{\kmax +2}$ or the removal of an occurrence of $C_{\kmax +3}$.
	
	Since $c$ is circular conflict free, there must exist graphs $C_i$, such that 
	$C_i$ has no conflict with any other graph 
	$C_{i'}$ and $C_{\kmax+2}$ has a (transifitve= conflict with $C_i$. 
	Therefore, the application of repairing sequences at occurrences 
	of these graphs will not lead to the insertion or removal of any universally 
	or existentially bound graph, respectively.
	Since $c$ is finite, the number of graphs $C_i$ 
	such that $C_{\kmax+2}$ has a (transitive) conflict with $C_i$ is finite.
	Since $|M'|$ is also finite, after a finite number of repairing sequence 
	applications, $M'$ only contains occurrences of graphs 
	that do not have any conflicts. After a repairing sequence has been 
	applied at all those occurrences, $M'$ is empty and
	$H \models_{\maxk{c}{G}} c$, since all occurrence $p$ of $C_j$, that either 
	have 
	been inserted or an occurrence $q$ of $C_{j+1}$ with $p = a_j \circ 
	q$ has been removed, satisfy $\exists (C_{j+1},\true)$. 
	Additionally holds that $\nv{\kmax +1}{H} < \nv{\kmax +1}{G}$.
	
	Therefore, after a finite number of iterations, the satisfaction at layer 
	has been increased by at least $1$. 
	It follows that after a finite number of iterations $G \models c$. 
	Then, Algorithm \ref{Algo_conflict_free} terminates and returns $G$.
\end{proof}


\begin{example}
	Consider constraint $c =\forall(C_2^2, \exists(C_2^1, \true))$ composed of
	the graphs shown in Figure \ref{fig:constraints}.
	This constraint is circular conflict free and a repairing set for $c$ is 
	given in Figure \ref{fig:rep}.
	There does exist a repairing sequence for $C_2^2$ via the rule 		
	\emph{\texttt{remove}} and a repairing sequence for $C_2^1$ via the rule 
	\emph{\texttt{insert}}.
	Using the rule set $\{\emph{\texttt{remove}},\emph{\texttt{insert}}\}$,
	Algorithm \ref{Algo_conflict_free} could return one of the graphs 
	$G_1, G_2$ or $G_3$ given in Figure \ref{fig:rep}, depending on the 
	repairing sequences that have been used. 
	
\end{example}
\begin{figure}

	\includegraphics[scale=0.8]{figures/images/example_repair}

	\caption{Possible outputs of the repairing process for $G$ and $\forall(C_2^2, \exists(C_2^1, \true))$ using the rule set $\{\texttt{remove},\texttt{insert}\}$.}\label{fig:rep}
\end{figure}