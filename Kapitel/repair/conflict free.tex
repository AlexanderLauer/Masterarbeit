\subsection{Rule-based Graph Repair for one Constraint}

In the following, we present our graph repair process for a circular conflict-free constraint in UANF. 
The process is shown in Algorithm \ref{Algo_conflict_free} and proceeds as follows.
The algorithm starts by finding all occurrences of $C_{\kmax +2}$ that do 
not satisfy $\cut{0}{\scond{\kmax+2}{c}}$ (line 2). This condition is equal to $\false$ if 
$\kmax +2 = \nlvl(c)-2$ and equal to $\exists(C_{\kmax +2}, \true)$ otherwise. 
If $P$ is empty, it must follow that $G \models_{\kmax+2} c$, and so we will apply repairing sequences at occurrences contained in this set.
It may be sufficient to repair only some of these occurrences. 
Since we do not know which of these are likely to increase the 
consistency, we choose one at random (line 3). 
For example, for existentially bound constraints $d$, i.e. their equivalent constraint in UANF is $\forall(\emptyset,d)$, there may exist occurrences of $C_{\kmax +2}$ whose repair will never lead to an increase of the largest satisfied layer. 

There are two ways to repair the selected occurrence, either by destroying it, or by inserting elements such that the occurrence satisfies $\cut{0}{\scond{\kmax+2}{c}}$. The algorithm chooses one of these options (line 4) and applies the appropriate repair sequence 
(lines 5--11). Note that there may be no repair sequence for $C_{\kmax+2}$, since this graph is universally bound. If this is the case, we use the repairing sequence for $C_{\kmax+3}$. This must exist because $C_{\kmax+3}$ is existentially bound.

If the repairing sequence for  $C_{\kmax+2}$ was applied, occurrences of existentially bound graphs may have been destroyed.
Note that these can only be occurrences of graphs $C_i$ such that $C_{\kmax+2}$ has a conflict with $C_i$.
This could lead to a reduction of the largest satisfied layer. Therefore the algorithm finds all these destroyed occurrences, in particular it finds all occurrences $p$ of universally bound graphs $C_i$ such that an occurrence $q$ of $C_{i+1}$ with $p = q \circ a_j$ has been removed (line 7).
If the repairing sequence for $C_{\kmax+3}$ has been applied, occurrences of universally bound graphs may have been inserted. 
Again, these can only be occurrences of graphs $C_i$ such that 
$C_{\kmax+2}$ has a conflict with $C_i$ and this could lead to a decrease of the largest satisfied layer. Again, the algorithm finds all inserted occurrences of universally bound graphs (line 10).
If the largest satisfied layer has not been reduced, the algorithm chooses the next occurrence in $P$.

Otherwise, the largest satisfied layer must be restored. To do this, the occurrences contained in $M$ must be repaired. 
Repairing these occurrences may again result in the insertion of existentially bound graphs or the removal of universally bound graphs. 
These occurrences are added to $H$, and this process is repeated until 
the largest satisfied layer is restored, i.e. $H \models_{\kmax}c$ (line 12 -- 25).
The whole process is repeated until a graph satisfying $c$ is derived. 

From this it is clear why $c$ must be circular conflict free. 
For a constraint with circular conflicts, during the restoring phase a new occurrence of $C_{\kmax +2}$ can be inserted and an occurrence of $C_{\kmax +3}$ can be removed.
In certain cases this could lead to an infinite loop, so there is no guarantee that this algorithm will terminate.
For example, consider the constraint $c_3$ given in Figure \ref{fig:conflict_example}.
The set of rules used for the transformations $t_1$ and $t_2$ in figure \ref{fig:conflict_example} forms a repairing set. 
During a repair process using Algorithm \ref{Algo_conflict_free}, where the starting graph is the first graph of $t_1$, it is possible for the Algorithm \ref{Algo_conflict_free} to enter an infinite loop by alternately applying $t_1$ and $t_2$.

Optimisation of the repair algorithm in terms of the number of elements inserted or deleted can be achieved by using partial repairing sequences where possible.
For example, consider the repairing sequence
$$ C_k \Longrightarrow C_1 \Longrightarrow \ldots \Longrightarrow C_{k+1}$$
with $C_1 \in \ig{C_k}{C_{k+1}}$. 
For an occurrence $p$ of $C_k$, which already satisfies the condition $\exists(C_1, \true)$, it may be sufficient to apply only the sequence
$$C_1 \Longrightarrow \ldots \Longrightarrow C_{k+1}$$ at $p$. 
But then you have to check that no occurrences of existentially bound graphs have been destroyed, and that no occurrences of universally bound graphs $C_i$ such that $C_k$ has no conflict with $C_i$ have been inserted.
If this is the case, the transformations must be reversed and another (partial) repairing sequence must be used.
Even though this would lead to an optimisation in terms of the number of elements inserted and deleted, it would lead to an increase in runtime due to the reversal of transformations.

For any circular conflict-free constraint, Algorithm \ref{Algo_conflict_free} will always terminate according to the following Theorem.


\begin{figure}
\centering
\begin{algorithm}[H]
	%\DontPrintSemicolon
	\KwData{A graph $G$, a conflict free constraint in UANF and a repairing set $\mathcal{R}$ for each layer $\kmax  < k \nlvl(c)$}
	\KwResult{A graph $H$ with $H \models c$}
	\For{$i = \kmax+1$ \KwTo $\nlvl(c)$}{
		\While{$G \not \models_i c$}{
			Choose one occurrence $p: C_i \inj G$ uniformly 
			at random \;
			apply a repairing sequence
			$G \Longrightarrow \ldots \Longrightarrow H$\;
					
		}
	}


\caption{Repair for conflict free constraints and repairing set}\label{Algo_conflict_free}

\end{algorithm}
\end{figure}



\begin{theorem}
Given n graph $G$,  a circular conflict free condition $c$ in UANF and 
	a repairing set $\mathcal{R}$ of $c$. 
	Then, Algorithm \ref{Algo_conflict_free} with input $G$,$c$ and $\mathcal{R}$ 
	terminates and returns a graph $H$ with $H \models c$.
\end{theorem}
\begin{proof}
	If Algorithm \ref{Algo_conflict_free} terminates, it returns a graph 
	that satisfies $c$. Therefore, it is sufficient to show that Algorithm 
	\ref{Algo_conflict_free} terminates.
	Since $G$ is finite, the set $P$ must also be finite. If a repairing sequence has been applied, the set $M$ contains only occurrences of graphs $C_j$ such that $C_{\kmax +2}$ has a (transitive) conflict with $C_j$, since the repairing sequence is not able to destroy or insert occurrences of $C_i$ such that $C_{\kmax +2}$ has no (transitive) conflict with $C_i$. Since $G$ is finite, $|M|$ must also be finite.
	
	If the derived graph does not satisfy $\cut{\maxk{c}{G}}{c}$, we need to restore the largest satisfied layer.
	Since the largest satisfied layer only decreases if an occurrence of an existentially bound graph is destroyed or an occurrence of universally bound graphs is inserted, and $M$ contains all these occurrences, we only need to consider the occurrences contained in $M$.
	Applying repairing sequences to occurrences $p :C_j \inj H \in M$ could again lead to the insertion of universally bound graphs or the removal of existentially bound graphs. The set $M'$ contains all these occurrences, and again these are only occurrences of $C_i$ such that $C_j$ has a (transitive) conflict with $C_i$.
	Since $c$ is free of circular conflict, $M'$ cannot contain any occurrences of $C_{\kmax +2}$, otherwise $C_j$ would have a (transitive) conflict with $C_{\kmax +2}$ and therefore $C_{\kmax +2}$ has a circular conflict. Therefore no occurrences of $C_{\kmax +3}$ are destroyed and no occurrences of $C_{\kmax+2}$ are inserted.
	In addition, $C_{\kmax +2}$ has a (transitive) conflict with $C_i$, and repairing any $p \ in M'$ will not lead to insertion of an occurrence of $C_{\kmax +2}$ or removal of an occurrence of $C_{\kmax +3}$.
	
	Since $c$ is circular conflict free, there must exist graphs $C_i$, such that $C_i$ has no conflict with any other graph $C_{i'}$ and $C_{\kmax+2}$ has a (transitive) conflict with $C_i$. 
Therefore, the application of repairing sequences at occurrences of these graphs will not lead to the insertion or removal of any universally or existentially bound graph, respectively.
	Since $c$ is finite, the number of graphs $C_i$ 
	such that $C_{\kmax+2}$ has a (transitive) conflict with $C_i$ is finite.
	Since $|M'|$ is also finite, after a finite number of applications of  repairing sequences, $M'$ contains only occurrences of graphs that have no conflicts. After a repairing sequence has been applied to all these occurrences, $M'$ is empty and $H \models_{\maxk{c}{G}} c$, since all occurrences $p$ of $C_j$ which have either been inserted or an occurrence $q$ of $C_{j+1}$ with $p = a_j \circ q$ has been removed satisfy $\exists (C_{j+1},\true)$. 
	
	Therefore, after a finite number of iterations, the set $P$ is empty and the largest satisfied layer has been increased by at least $1$.
It follows that after a finite number of iterations $G \models c$. 
Then Algorithm \ref{Algo_conflict_free} terminates and returns $G$.
\end{proof}


\begin{example}
	Consider constraint $c =\forall(C_2^2, \exists(C_2^1, \true))$ which is composed of
	the graphs shown in Figure \ref{fig:constraints}.
	This constraint is circular conflict free and a repairing set for $c$ is 
	given in Figure \ref{fig:rep}.
	There does exist a repairing sequence for $C_2^2$ via the rule 		
	\emph{\texttt{remove}} and a repairing sequence for $C_2^1$ via the rule 
	\emph{\texttt{insert}}.
	Using the rule set $\{\emph{\texttt{remove}},\emph{\texttt{insert}}\}$,
	Algorithm \ref{Algo_conflict_free} could return one of the graphs 
	$G_1, G_2$ or $G_3$ given in Figure \ref{fig:rep}, depending on the repairing sequences used.
\end{example}

If $\mathcal{R}$ is not a repairing set for a constraint $c$ in UANF, it is unclear whether a graph $G$ can be repaired with this set of rules.
The only approach is a brute-force approach using consistency-maintaining or consistency-increasing rules up to layer $\kmax$. This can be done by equipping the rules of $\mathcal{R}$ with maintaining, increasing or basic application conditions. 
In this case, we can repair $G$ by brute-force if there is a sequence $$G \Longrightarrow_{\rho_1} \ldots \Longrightarrow_{\rho_n} H$$ of transformations with $\rho_1, \ldots, \rho_n \in \mathcal{R}$ such that each transformation is consistency-maintaining or consistency-increasing w.r.t. $c$ and $H \models c$. 
If no such sequence exists, this approach cannot repair $G$, even if $G$ can be repaired with rules of $\mathcal{R}$.

\begin{figure}

	\includegraphics[scale=0.8]{figures/images/example_repair}

	\caption{Possible outputs of the repairing process for $G$ and $\forall(C_2^2, \exists(C_2^1, \true))$ using the rule set $\{\texttt{remove},\texttt{insert}\}$.}\label{fig:rep}
\end{figure}