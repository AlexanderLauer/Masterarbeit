\subsection{Rule-based Graph Repair for one Constraint}


In the following, we present our graph repair process for a circular conflict-free constraint in UANF.
We start with an algorithm that computes $\kmax$ given a graph $G$ and a constraint $c$ in UANF, as shown in Algorithm \ref{Algo_kmax}.  
\SetKw{KwBy}{by}
\begin{figure}
\centering
\begin{algorithm}[H]
	%\DontPrintSemicolon
	\KwData{A graph $G$, a constraint $c$ in UANF.}
	\KwResult{$\kmax$.}
	
	\If{$G \models c$}{
		\KwRet{$\nlvl(c) -1$};	
	}

	\For{$i\gets -1$ \KwTo $\nlvl(c)-1$ \KwBy $2$ }{
		\If{$G \not \models \cut{i}{c}$}{
			\KwRet{$i-2$}\;
		}
	}



\caption{Determine $\kmax$.}\label{Algo_kmax}

\end{algorithm}
\end{figure}


\begin{theorem}
	Given a graph $G$ and a constraint $c$ in UANF, Algorithm \ref{Algo_kmax} returns $\maxk{c}{G}$.  
\end{theorem}
\begin{proof}
	If $G \models c$, then $\kmax = \nlvl(c)-1$ which will be returned by the algorithm. 
	Otherwise, if $G \not \models c$, Corollary \ref{corol:satisfaction} implies that $G \not \models_k c$ for all even $-1 \leq k < \nlvl(c)$. Therefore, $\kmax$ must be odd. 
	If the algorithm returns $k$, which is odd,  it holds that $G \not \models_{k+2} c$ and Lemma \ref{lem_ex_lower} implies that $k$ must be equal to $\kmax$. 

\end{proof}
The repair process is shown in Algorithm \ref{Algo_conflict_free} and proceeds as follows.
The algorithm starts by finding all potentially increasing occurrences  $p$ of $C_{\kmax +2}$ at layer $\kmax$ w.r.t. $c$. Recall, these occurrences satisfy the following: 
\begin{enumerate}
	\item $p \not \models d$.
	\item $p = a_{\kmax+1} \circ \ldots \circ a_0 \circ q$ where  
	$a_i \circ \ldots \circ q \models \scond{i+1}{\cut{\kmax}{c}}$ for all  $0 \leq i \leq \kmax$ and $q: \emptyset \inj G$ is the empty morphism.
\end{enumerate}
The condition $d$ is equal to $\false$ if 
$\kmax +2 = \nlvl(c)-2$ and equal to $\exists(C_{\kmax +2}, \true)$ otherwise.
All these occurrences are contained in the set $P$ (line 3).  
If $P$ is empty,  Lemma \ref{num_violations} implies $G \models_{\kmax+2} c$, and so we only will apply repairing sequences at occurrences contained in this set.
It may be sufficient to repair only some of these occurrences. 
Since we do not know which of these are likely to increase the 
consistency, we choose one at random (line 4). 
For example, for an existential constraint $c$, i.e. their equivalent constraint in UANF is $\forall(\emptyset,c)$, there may exist occurrences of $C_{\kmax +2}$ whose repair will never lead to an increase of the largest satisfied layer. 

There are two ways to repair the selected occurrence, either by destroying it or by inserting elements such that the occurrence satisfies $\cut{0}{\scond{\kmax+2}{c}}$. The algorithm chooses one of these options (line 5) and applies the appropriate repair sequence 
(lines 6--12). Note that there may be no repair sequence for $C_{\kmax+2}$ since this graph is universally bound. If this is the case, we use the repairing sequence for $C_{\kmax+3}$. This must exist because $C_{\kmax+3}$ is existentially bound and $\mathcal{R}$ is a repairing set for $c$.

If the repairing sequence for  $C_{\kmax+2}$ was applied, occurrences of existentially bound graphs may have been destroyed.
Note that these can only be occurrences of graphs $C_i$ such that $C_{\kmax+2}$ has a conflict with $C_i$.
This could lead to a reduction of the largest satisfied layer. Therefore the algorithm finds all these destroyed occurrences, in particular, it finds all occurrences $p$  of universally bound graphs $C_i$ such that an occurrence $q$ of $C_{i+1}$ with $p = q \circ a_j$ has been removed (line 8).
If the repairing sequence for $C_{\kmax+3}$ has been applied, occurrences of universally bound graphs may have been inserted. 
Again, these can only be occurrences of graphs $C_i$ such that 
$C_{\kmax+2}$ has a conflict with $C_i$. This could lead to a reduction of the largest satisfied layer and the algorithm finds all new occurrences of these universally bound graphs (line 11). In both cases, the algorithm only finds potentially increasing occurrences at the respective layer. 
If the largest satisfied layer has not been reduced, the algorithm chooses the next occurrence in $P$.

Otherwise, the largest satisfied layer must be restored. To do this, all newly inserted potentially increasing occurrences of universally bound graphs and potentially increasing occurrences $p$ of   universally bound graphs $C_j$ such that $p \models \exists(C_{j+1}, \true)$ and  $\track \circ p \not\models \exists(C_{j+1}, \true)$ where $\track$ is the morphism of the repairing sequence application are collected in the set $M$ and must be repaired. Again, it might be sufficient to only repair some of these occurrences in order to restore the largest satisfied layer. 
Since it is unclear for which  of these occurrences a deletion or extension leads to an increase of the largest satisfied layer, we select one at random. 
Repairing these occurrences may again result in the insertion of universally bound graphs or the removal of existentially bound graphs. 
These occurrences are added to $M$, and this process is repeated until 
the largest satisfied layer is restored, i.e. $H \models_{\kmax}c$ (line 13 -- 26).
The whole process is repeated until a graph satisfying $c$ is derived. 

This shows why $c$ must be free of circular conflicts. 
For a constraint with circular conflicts, a new occurrence of $C_{\kmax +2}$ can be inserted and an occurrence of $C_{\kmax +3}$ can be removed during the recovery phase.
In certain cases, this could lead to an infinite loop, so there is no guarantee that this algorithm will terminate.
For example, consider the constraint $c_3$ given in Figure \ref{fig:conflict_example}.
The set of rules used for the transformations $t_1$ and $t_2$ in figure \ref{fig:conflict_example} forms a repairing set. 
During a repair process using Algorithm \ref{Algo_conflict_free}, where the starting graph is the first graph of $t_1$, Algorithm \ref{Algo_conflict_free} can enter an infinite loop by alternately applying $t_1$ and $t_2$.

Optimisation of the repair algorithm in terms of the number of elements inserted or deleted can be achieved by using partial repairing sequences where possible.
For example, consider the repairing sequence
$$ C_k \Longrightarrow C' \Longrightarrow \ldots \Longrightarrow C_{k+1}$$
with $C' \in \ig{C_k}{C_{k+1}}$. 
For an occurrence $p$ of $C_k$, which already satisfies the condition $\exists(C_1, \true)$, it may be sufficient to apply only the sequence
$$C_1 \Longrightarrow \ldots \Longrightarrow C_{k+1}$$ at $p$. 
Then, we need to check that no occurrences of existentially bound graphs have been destroyed, and that no occurrences of universally bound graphs $C_i$ such that $C_k$ does not cause a conflict for $C_i$ have been inserted. This can be achieved by replacing the application of the repairing sequence by an application of its concurrent rule, and by equipping this concurrent rule with its (basic) consistency-increasing application condition of $c$ at layer $k$ with $C_{k+1}$. 
If the application condition is not satisfied, another (partial) repairing sequence must be used.
Although this would lead to an optimisation in terms of the number of elements inserted and deleted, it would lead to an increase in runtime due to the construction of application conditions.

For any circular conflict-free constraint, Algorithm \ref{Algo_conflict_free} is correct and will always terminate according to the following Theorem.


\begin{figure}
\centering
\begin{algorithm}[H]
	%\DontPrintSemicolon
	\KwData{A graph $G$, a conflict free constraint in UANF and a repairing set $\mathcal{R}$ for each layer $\kmax  < k \nlvl(c)$}
	\KwResult{A graph $H$ with $H \models c$}
	\For{$i = \kmax+1$ \KwTo $\nlvl(c)$}{
		\While{$G \not \models_i c$}{
			Choose one occurrence $p: C_i \inj G$ uniformly 
			at random \;
			apply a repairing sequence
			$G \Longrightarrow \ldots \Longrightarrow H$\;
					
		}
	}


\caption{Repair for conflict free constraints and repairing set}\label{Algo_conflict_free}

\end{algorithm}
\end{figure}



\begin{theorem}
Given a graph $G$,  a circular conflict free constraint $c$ in UANF and 
	a repairing set $\mathcal{R}$ of $c$. 
	Then, Algorithm \ref{Algo_conflict_free} with input $G$, $c$ and $\mathcal{R}$ 
	terminates and returns a graph $H$ with $H \models c$.
\end{theorem}
\begin{proof}
	If Algorithm \ref{Algo_conflict_free} terminates, it returns a graph 
	that satisfies $c$. Therefore, it is sufficient to show that Algorithm 
	\ref{Algo_conflict_free} terminates.
	Since $G$ is finite, the set $P$ must also be finite. If a repairing sequence has been applied, the set $M$ contains only occurrences of graphs $C_j$ such that $C_{\kmax +2}$ causes a (transitive) conflict with $C_j$, since the repairing sequence is not able to destroy or insert occurrences of $C_i$ such that $C_{\kmax +2}$ does not cause a conflict with $C_i$. 
	Note that this is also the case for repairing sequence for universally bound graphs. Since $G$ is finite, $|M|$ must also be finite.
	
	If the derived graph does not satisfy $\cut{\maxk{c}{G}}{c}$, we need to restore the largest satisfied layer.
	Since the largest satisfied layer only decreases if an occurrence of an existentially bound graph $C_j$ is destroyed such that an occurrence $p$ of $C_{j-1}$ satisfies $\exists(C_j, \true)$  and $\track \circ p$ does not satisfy $\exists(C_j, \true)$ or an occurrence of universally bound graphs is inserted, and $M$ contains all these occurrences that can also improve the satisfaction at layer (see Lemma \ref{num_violations}). So we only need to consider the occurrences contained in $M$.
	Applying repairing sequences to occurrences $p :C_j \inj H \in M$ could again lead to the insertion of universally bound graphs or the removal of existentially bound graphs. The set $M'$ contains all these occurrences, and again these are only occurrences of $C_i$ such that $C_j$ causes a (transitive) conflict for $C_i$.
	Since $c$ is free of circular conflict, $M'$ cannot contain any occurrences of $C_{\kmax +2}$, otherwise, $C_j$ would have caused a (transitive) conflict for $C_{\kmax +2}$ and therefore $C_{\kmax +2}$ has a circular conflict. Therefore no occurrences of $C_{\kmax +3}$ are destroyed and no occurrences of $C_{\kmax+2}$ are inserted.
	In addition, $C_{\kmax +2}$ causes a (transitive) conflict for $C_i$, and repairing any $p \in M'$ will not lead to the insertion of an occurrence of $C_{\kmax +2}$ or removal of an occurrence of $C_{\kmax +3}$.
	
	Since $c$ is circular conflict free, there must exist graphs $C_i$, such that $C_i$ does not cause a conflict with any other graph $C_{i'}$ and $C_{\kmax+2}$ causes a (transitive) conflict for $C_i$. 
Therefore, the application of repairing sequences at occurrences of these graphs will not lead to the insertion or removal of any universally or existentially bound graph, respectively.
	Since $c$ is finite, the number of graphs $C_i$ 
	such that $C_{\kmax+2}$ causes (transitive) conflict with $C_i$ is finite.
	Since $|M'|$ is also finite, after a finite number of applications of  repairing sequences, $M'$ contains only occurrences of graphs that do not cause any conflicts. After a repairing sequence has been applied to all these occurrences, $M'$ is empty and $H \models_{\maxk{c}{G}} c$, since all occurrences $p$ of $C_j$ which have either been inserted or an occurrence $q$ of $C_{j+1}$ with $p = a_j \circ q$ has been removed satisfy $\exists (C_{j+1},\true)$ or have been removed. 
	
	Therefore, after a finite number of iterations, the set $P$ is empty and Lemma \ref{num_violations} implies that the largest satisfied layer has been increased by at least $1$.
It follows that after a finite number of iterations $G \models c$. 
Then Algorithm \ref{Algo_conflict_free} terminates and returns $G$.
\end{proof}


\begin{example}
	Consider constraint $c =\forall(C_2^2, \exists(C_2^1, \true))$ which is composed of
	the graphs shown in Figure \ref{fig:constraints}.
	This constraint is circular conflict-free and a repairing set for $c$ is 
	given in Figure \ref{fig:rep}.
	There is a repairing sequence for $C_2^2$ via the rule 		
	\emph{\texttt{remove}} and a repairing sequence for $C_2^1$ via the rule 
	\emph{\texttt{insert}}.
	Using the rule set $\{\emph{\texttt{remove}},\emph{\texttt{insert}}\}$,
	Algorithm \ref{Algo_conflict_free} could return one of the graphs 
	$G_1, G_2$ or $G_3$ given in Figure \ref{fig:rep}, depending on the repairing sequences used.
\end{example}

%If $\mathcal{R}$ is not a repairing set for a constraint $c$ in UANF, it is unclear whether a graph $G$ can be repaired with this set of rules.
%The only approach is a brute-force approach using consistency-maintaining or consistency-increasing rules up to layer $\kmax$. This can be done by equipping the rules of $\mathcal{R}$ with maintaining, increasing or basic application conditions. 
%In this case, we can repair $G$ by brute-force if there is a sequence $$G \Longrightarrow_{\rho_1} \ldots \Longrightarrow_{\rho_n} H$$ of transformations with $\rho_1, \ldots, \rho_n \in \mathcal{R}$ such that each transformation is consistency-maintaining or consistency-increasing w.r.t. $c$ and $H \models c$. 
%If no such sequence exists, this approach cannot repair $G$, even if $G$ can be repaired with rules of $\mathcal{R}$.

\begin{figure}

	\includegraphics[scale=0.8]{figures/images/example_repair}

	\caption{Possible outputs of the repairing process for $G$ and $\forall(C_2^2, \exists(C_2^1, \true))$ using the rule set $\{\texttt{remove},\texttt{insert}\}$.}\label{fig:rep}
\end{figure}