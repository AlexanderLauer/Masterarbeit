\subsection{Conditions up to Layer}










\begin{figure}
\centering
\includegraphics[scale=1]{figures/sources/constraints}


\caption{constraints}\label{fig:constraints}
\end{figure}
\begin{figure}

\includegraphics[scale=0.8]{figures/images/graph}

\caption{graph}\label{fig:graph}
\end{figure}

As already mentioned, the goal of our approach is to increase the consistency of 
a constraint layer by layer. 
For this, we introduce a notion of partial consistency, called \emph{satisfaction 
at layer} which enables to check whether a constraint is satisfied at a 
certain layer by checking whether the so-called \emph{truncated condition} at 
this layer is satisfied.
 
\begin{definition}[\textbf{sub-condition at layer}]
Let $c$ be a condition in ANF. 
The \emph{sub-condition at layer $0 \leq k \leq \nlvl(c)$}, denoted by $\scond{k}{c}$ , is the sub-condition $d$ of $c$ with $\lay(d) = k$.
\end{definition}

\begin{example}
	Consider the condition $ c =\forall(a_0:C_0 \inj C_1, \exists(a_1: C_1
	\inj C_2, \forall(a_2:C_2 \inj C_3, \false)))$.
	Then, $\scond{1}{c} = \exists(a_1:C_1 \inj C_2, \forall(a_2: C_2 \inj C_3,
	\false))$.
\end{example}

First, we introduce an operator which allows to replace a sub-condition
$\scond{k}{c}$ by an arbitrary condition over $C_k$, called \emph{replacement 
at layer}.

\begin{definition}[\textbf{replacement at layer}]
Let a condition $c = Q(a_0: C_0 \inj C_1, d)$, with $Q \in \{\forall, \exists\}$ in ANF and a condition $e$ over $C_k$ in ANF be given. 
The \emph{replacement in $c$  at layer $k$ with $e$}, denoted by $\repl{k}{c}{e}$, is recursively defined as:
\begin{equation*}
	\repl{k}{c}{e} := 
		\begin{cases}
			e & \text{if $k = 0$} \\
			Q(a_0:C_0 \inj C_1, \repl{k-1}{d}{e}) &\text{otherwise}	
		\end{cases}
\end{equation*}
\end{definition}

\begin{example}
Let the conditions $c := \forall(a_0: C_0 \inj C_1, \exists(a_1: C_1 \inj C_2, \true))$ and $d = \exists(a'_1: C_1 \inj C_3, e)$ be given. 
Then, 
$$\repl{1}{c}{d} = \forall(a_0: C_0 \inj C_1, \exists(a'_1: C_1 \inj C_3, e)).$$ 
\end{example}

Using replacement at layer we now define \emph{truncated conditions}.
Intuitively, a condition is-cut off at a certain layer, by replacing the sub-condition at this layer by $\true$ or $\false$, depending on the quantifier, the replaced sub-condition is bound by.


\begin{definition}[\textbf{truncated condition}]
	Let $c$ be a condition in UANF and $d = \scond{k}{c}$ with $ 0 \leq k 
	\leq 	\nlvl(c)-1$. The \emph{truncated condition at layer} $k$ of $c$, 
	denoted by $\cut{k}{c}$, is defined as

$$ \cut{k}{c} := \begin{cases}
					\repl{k +1}{c}{\true}&\text{if $d$ is existentially bound,
					 i.e. $k$ is odd} \\
					\repl{k+1}{c}{\false}&\text{if $d$ is universally bound, 
					i.e. $k$ is even.} \\
				\end{cases}$$

\end{definition}

\begin{example}
Consider constraint $c_2$ given in Figure \ref{fig:constraints}.
Then, $\cut{1}{c_2} = \forall C_1^1 \exists C_2^2$.
\end{example}

With these prerequisites we are now able to introduce \emph{satisfaction 
at layer} which enables to check whether a condition is satisfied at a certain
layer. A morphism or graph satisfies a condition or constraint, respectively, 
if it satisfies the truncated condition at this layer.


\begin{definition}[\textbf{satisfaction at layer}]
Let a graph $G$ and a condition $c$ in UANF be given.
A morphism $p: C_0 \inj G$ \emph{satisfies $c$ at layer $0 \leq k \leq \nlvl(c)-1$}, denoted by $p \models_k c$, if $$p\models \cut{k}{c}.$$

A graph $G$ satisfies a constraint $c$ at layer $0 \leq k \leq \nlvl(c)$, denoted by $G \models_k c$, if $q: \emptyset \inj G$ satisfies $\cut{k}{c}$.
The biggest $0 \leq k \leq \nlvl(c)$ such that $G \models_k c$ and no $k < j \leq \nlvl(c)$ with $G \models_j c$ exists is denoted by $\maxk{c}{G}$. 
If no such $k$ exists, we set $\maxk{c}{k} = -1$.
We use the abbreviation $\kmax$ when $c$ and $G$ are clear from the context.
\end{definition}

Note that, given a graph $G$ and a constraint $c$, by definition and since 
$\nlvl(c)$ is finite, $\maxk{c}{k}$ always exists.
Also, if $p \models_{\nlvl(c)-1} c$ it follows immediately that $p \models c$. 

\begin{example}
Consider the graph $G$ given in Figure \ref{fig:graph} and the constraint $c_2$ given in Figure \ref{fig:constraints}. 
This graph does not satisfy $c_2$, since the second occurrence of \emph{\texttt{Class}} does not satisfy $\exists C_2^2 \forall C_3^2 \exists C_4^2$, but it satisfies $\cut{1}{c_2}$ and therefore $$G \models_1 c_2 \text{ and } \kmax = 1$$
\end{example}


\begin{table}
\centering
\begin{tabular}{c|c|c|c|c|c}



$p \models_k c$ & \multicolumn{2}{c|}{$p \models_{j < k} c$} & \multicolumn{2}{c|}{$p \models_{j > k} c$} &$ p \models c $ \\
%\hline
 & $j$ even & $j$ odd & $j$ even & $j$ odd & \\

\hline
$k$ even & ? & \cmark &\cmark & \cmark & \cmark \\
$k$ odd & ? & \cmark & ? & ? & ? \\


\end{tabular}

\caption{Overview of the inferences made about satisfaction up to layer, where \enquote{\cmark} indicates that $p \models_j c$ or $p \models c$ if $p \models_k c$ and \enquote{?} indicates that it cannot be inferred from $p \models_k c$ whether $p \models_j c$ or $p \not \models_j c$.}\label{table_satisfaction_at_layer}

\end{table}

Let a graph $G$, a condition $c$ and a morphism $p : C_0 \inj G$ be given.
Assume that $p \models_k c$ for any $0 \leq k < \nlvl(c)$. 
Then, we are able to conclude results for the satisfaction at other layers.
If $k$ is even, that means $\scond{k}{c}$ is universally bound, we can conclude
that $p \models_j c$ for all $ k < j < \nlvl(c)$ and in particular $p \models c$.
For any $0 \leq k < \nlvl(c)$ with $p \models_k c$ we can conclude that 
$p \models j$ for all odd $0 \leq j < k$. 
An overview of these conclusion is shown in Table \ref{table_satisfaction_at_layer}. We show these results within the following 
lemmas. 

We start by investigating the conclusions for satisfaction at layer $j > k$ if 
$p \models_k c$. Our first result shows that  the replacement of the sub-condition $\scond{k+1}{c}$ by
any arbitrary condition over $C_{k+1}$ leads to a condition that is satisfied
by $p$ if $k$ is even.

\begin{lemma}\label{lemma_help_lay_sat}
Let a graph $G$, a condition $c$ in UANF and a morphism $p : C_0 \inj G$ with $p \models_k c$, such that $0 \leq k < \nlvl(c)$ is even, be given.
Then, for any condition $f$ over $C_{k+1}$ it holds that
$$p \models \repl{k + 1}{c}{f}.$$
\end{lemma}

\begin{proof}
Let $0 \leq j \leq \nlvl(c)$ be the smallest number with  $\scond{j}{c} = \forall(a_j:C_{j} \inj C_{j+1}, d)$ being universally bound and $p \models_j c$. 
This must exist, since at least one of these exists due to the assumption.
Let $q: C_{j} \inj G$ be a morphism such that $q \models \forall(a_j:C_{j} \inj C_{j+1}, \false)$. 
This must exist, since $p \models_j c$ and $j$ is the smallest even number such that $p \models_j c$.
Therefore, there does not exist a morphism $q': C_{j+1} \inj G$ with $q = q' \circ a_j$.
Hence, for every condition $f$ over $C_{j+1}$ a morphism $q': C_{j+1} \inj G$ with $q \not \models f$ and $q = q' \circ a_j$ cannot exist. 
It follows immediately that $q \models \forall(a_j:C_{j} \inj C_{j+1}, f)$ and with that $p \models \repl{j + 1}{c}{f}$.

We can now conclude that for every even $j < k \leq \nlvl(c)$, such that $p \models_k c$, and every condition $d$ over $C_{k + 1}$ it holds that $p \models \repl{j + 1}{c}{f}$ with $f = \scond{j+1}{\repl{k+1}{c}{d}}$.
Since $\repl{j+1}{c}{f} = \repl{k+1}{c}{d}$ it follows that $p \models \repl{k + 1}{c}{d}$.
\end{proof}

As a direct consequence of the previous lemma, a morphism satisfying a condition
at layer $k$ with $k$ being even also satisfies the condition at layer
$j$ for all $j > k$.


\begin{lemma}\label{lemma_lay_sat}

Let a graph $G$, a morphism $p:C_0 \inj G$ and a condition $c$ in UANF be given.
If $0 \leq k < \nlvl(c)$ is even, i.e. $\scond{k}{c}$ is universally bound, then for all $k < j < \nlvl(c)$  it holds that
	$$p \models_k c \implies p \models_j c.$$

\end{lemma}
\begin{proof}
Follows immediately by using lemma \ref{lemma_help_lay_sat} and setting $f$ equal to $\scond{k+1}{\cut{j}{c}}$.
\end{proof}

Since a morphism $p$ satisfies a condition $c$ in UANF if and only if $p$ satisfies $c$ at layer $\nlvl(c)-1$, because $\cut{\nlvl(c)-1}{c} = c$,
we can conclude the following.


\begin{corollary}\label{corol:satisfaction}
	Let a graph $G$, a morphism $p:C_0 \inj G$ and a condition $c$ in UANF be
	given. If $0 \leq k < \nlvl(c)$ is even it holds that 
	$$p \models_k c \implies p \models c.$$
\end{corollary}

Furthermore, this allows us to make statements about the satisfaction of other 
conditions. Let a graph $G$, a morphism $p : C_0 \inj G$ and a condition $c$ be given such that $p \models_k c$ for an even $0 \leq k < \nlvl(c)$.
Then, $p \models c$ and for every condition $c'$ with $\cut{k}{c} = \cut{k}{c'}$. With Lemma \ref{lemma_help_lay_sat}, it follows that $p \models c'$.

Let us now investigate the satisfaction at layer $j$ with $j < \kmax$.
If $j$ is odd, i.e. $\scond{j}{c}$ is existentially bound, we can conclude
that $p \models_j c$ as shown in Lemma \ref{lem_ex_lower}.
If $j$ is even, i.e. $\scond{j}{c}$ is universally bound, we are only able to make
statements depending on $\kmax$. If $\kmax < \nlvl(c) -1$, it follows that
$p \not \models_j c$. Otherwise $p \models c$ and therefore $\kmax = \nlvl(c)-1$ would follow immediately with Corollary \ref{corol:satisfaction}.
If $\kmax = \nlvl(c)-1$ we can only state that at least one even $j \leq \kmax$
with $p \models_j c$ exists if $c$ ends with a condition of the form
$\forall(a_k: C_k \inj C_{k+1}, \false)$.


\begin{lemma}\label{lem_ex_lower}
Let a graph $G$, a morphism $p: C_0 \inj G$ and a constraint $c$ in UANF be given.
Then,  for all odd $0 \leq j < \kmax$, i.e. $\scond{j}{c}$ is existentially bound, it holds that  
$$p \models_j c.$$

\end{lemma}
\begin{proof}

If an even $0 \leq j < \kmax$ with $p \models_j c$ exists such that $\scond{j}{c}$ is universally bound, let $j'$ be the smallest of these. 
With lemma \ref{lemma_help_lay_sat} follows that $p \models_{\ell} c$ for all $j' \leq \ell < \nlvl(c)$. 
Otherwise, if no such $j'$ exists we set $j' = \kmax$.

Let $\ell < j'$, such that $\scond{\ell}{c}$ is existentially bound and let $d = \scond{\ell}{\cut{j'}{c}} = \exists(a_{\ell}: C_{\ell} \inj C_{\ell+1}, e)$ be the sub-condition at layer $\ell $ of the condition up to layer $j'$ of $c$. 
	Since $\ell < j'$, a morphism $q:  C_{\ell} \inj G$ with $q \models d$ must exists and therefore a morphism $q': C_{\ell +1}\inj G$ with $q = q' \circ a_{\ell}$ must exists. 
	It follows that $q \models \exists(a_{\ell}: C_{\ell} \inj C_{\ell+1}, \true)$ and with that $p \models_{\ell} c$.
\end{proof}

Through satisfaction at layer, an increase of consistency can be detected in the following way: 
Let $t: G \Longrightarrow H$ be a transformation. 
If $\maxk{c}{G} < \maxk{c}{H}$, we consider the transformation as consistency increasing, since $H$ satisfies more layers of the constraint than $G$. 
But, the notion of consistency increasement should also be able to detect the smallest changes, performed by a transformation, that lead to an increase of consistency, namely the inserting of a single edge or node of an existentially bound graph.
To remedy this issue, we introduce \emph{intermediate conditions}, which will
be used to recognize these types of increasements by checking whether an
intermediate condition not satisfied by $G$ is satisfied by $H$.
Obviously, an decrease of consistency can be detected in a similar manner, by 
checking whether an intermediate condition satisfied by $G$ is not satisfied by $H$. 
Intuitively, given a constraint $c$ in UANF with $\scond{k}{c} =\exists(a_k :C_k \inj C_{k+1}, d)$ and $0 \leq k < \nlvl(c)$, the condition $\scond{k}{c}$ is replaced by $\exists(a^r_k: C_k \inj C', \true)$ with $C' \in \ig{C_k}{C_{k+1}}$.

The construction of intermediate conditions is designed to only replace graphs in existentially bound layers, since the replacement in an universally bound layer would lead to a more restrictive constraint than the original condition up to layer. 
That means, given the condition $c = \forall(a_0: C_0 \inj C_1, \false)$, let $C' \in \ig{C_0}{C_1}$. 
If the condition $c'= \forall(a^r_0: C_0 \inj C', \false)$ is satisfied the satisfaction of $c$ is implied but the backwards implication does not hold.

\begin{definition}[\textbf{intermediate condition}]
	Let a condition $c$ in UANF be given. 
	Let $0 \leq k < \nlvl(c)$ such that $k$ is odd, i.e. $\scond{k}{c}$ is 
	existentially bound. The \emph{intermediate condition}, denoted by $\ic{k}{c}
	{C'}$, of $c$ at layer $k$ with $C' \in \ig{C_k} {C_{k+1}}$ is defined as
	$$\ic{k}{c}{C'} := \repl{k}{c}{\exists(a_k^r:C_k \inj C', \true)}.$$
\end{definition}

%\begin{definition}[\textbf{partial condition}]
%Let $c$ be a condition in EANF. 
%
%The \emph{partial condition of $c$ at layer $k < \nlvl(c)$ with} with 
%$$ C' \in \begin{cases}
%			\mathcal{U}(C_k, C_{k+1}) &\text{if $k$ is even} \\
%			\mathcal{U}(C_{k+1}, C_{k+2}) &\text{if $k$ is odd},
%		   \end{cases}
%$$
%$\parcond(k,c,C')$, is defined as:
%\begin{enumerate}
%\item If $k$ is odd, let $\scond{c}{k+1} = \exists(a: C_{k+1} \inj C_{k+2},f)$:
%$$\parcond(k,c,C') := \sub\bigl(k+1,c,\exists(a^p:C_{k+1} \inj C', \true)\bigr)$$
%\item If $k$ is even, let $\scond{c}{k+1} = \exists(a:C_k\inj C_{k+1},f)$:
%$$\parcond(k,c,C') := \sub\bigl(k,c,\exists(a^p:C_{k+1} \inj C', \true)\bigr)$$
%\end{enumerate}
%\end{definition}

\begin{example}
Consider constraint $c_1$ given in figure \ref{fig:constraints}. 
Since $C_2^2 \in \ig{C_1^1}{C_2^1}$, we can construct a intermediate condition of $c_1$ at layer $1$ with $C_2^2$ as $\ic{1}{c_1}{C_2^2} = \forall C_1^1 \exists C_2^2$. 
Whereas $c_1$ checks whether each node of type \emph{\texttt{Class}} is connected to at least two nodes of  type \emph{\texttt{Feature}}, the intermediate condition checks whether each node of type \emph{\texttt{Class}} is connected to at least one node of type \emph{\texttt{Feature}} which is trivially satisfied if $c_1$ is satisfied.
\end{example}

Given a graph $G$ and a constraint $c$ in UANF with $\kmax < \nlvl(c)-3$, note that in this case $\scond{\kmax}{c}$ has to be existentially bound, it holds that $G \not \models_{\kmax + 2} c$ and there does exist at least one graph $C' \in \ig{C_{\kmax+2}}{C_{\kmax+3}}$ such that $G \models \ic{\kmax+2}{c}{C'}$ since 
$G$ always satisfies $\ic{\kmax+2}{c}{C_{\kmax + 2}}$.

