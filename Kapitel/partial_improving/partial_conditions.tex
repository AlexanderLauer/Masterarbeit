\subsection{Satisfaction up to Layer}










\begin{figure}
\centering
\includegraphics[scale=1]{figures/sources/constraints}


\caption{constraints}\label{fig:constraints}
\end{figure}
\begin{figure}

\includegraphics[scale=0.8]{figures/images/graph}

\caption{graph}\label{fig:graph}
\end{figure}

The goal of our approach is to increase the consistency of a constraint layer by layer, as we have already mentioned. To do this, we introduce a notion of partial consistency, called \emph{satisfaction up to layer}, which allows us to check whether a constraint is satisfied at a particular layer by checking whether the so-called \emph{truncated condition after layer} is satisfied at that layer.

We first define the \emph{subcondition at layer} $-1 \leq k \leq \nlvl(c)$ of a condition $c$. As the name suggests, the subcondition at layer $0 \leq k \leq \nlvl(c)$ denotes the subcondition of $c$ with layer $k$. 
We also define the subcondition at layer $-1$, which is $\true$. This will be useful for evaluating the satisfaction up to layer when a graph does not satisfy any layer of a constraint.
 
\begin{definition}[\textbf{subcondition at layer}]
Let a condition $c$ in ANF be given. The \emph{subcondition at layer $-1 \leq k \leq \nlvl(c)$}, denoted by $\scond{k}{c}$, is the subcondition $d$ of $c$ with $\lay(d) = k$ if $0 \leq k \leq \nlvl(c)$ and $\true$ if $k = -1$. 
\end{definition}

Note that by definition the subcondition at layer $k$ is always a condition over the graph $C_k$ and the morphism is denoted by $a_k$.
\begin{example}
	Consider the condition $ c =\forall(a_0:C_0 \inj C_1, \exists(a_1: C_1
	\inj C_2, \forall(a_2:C_2 \inj C_3, \false)))$.
	Then, 
	\begin{enumerate}
		\item $\scond{-1}{c} = \true$.
		\item $\scond{0}{c} =  \forall(a_0:C_0 \inj C_1, \exists(a_1: C_1
	\inj C_2, \forall(a_2:C_2 \inj C_3, \false))) = c$.
		\item $\scond{1}{c} =  \exists(a_1: C_1
	\inj C_2, \forall(a_2:C_2 \inj C_3, \false))$.
	\item $\scond{2}{c} =  \forall(a_2:C_2 \inj C_3, \false)$.
	\item $\scond{3}{c} =   \false$.
	\end{enumerate}
\end{example}

Let us first introduce an operator which allows to replace a subcondition $\scond{k}{c}$ by an arbitrary condition over $C_k$, called \emph{replacement starting from layer}.

\begin{definition}[\textbf{replacement starting from layer}]
Given a condition $c = Q(a_0: C_0 \inj C_1, d)$ in ANF  with $Q \in \{\forall, \exists\}$ and a condition $e$ over $C_k$ in ANF.  
The \emph{replacement starting from layer $k$ in $c$ by $e$}, denoted by $\repl{k}{c}{e}$, is defined recursively as
\begin{equation*}
	\repl{k}{c}{e} := 
		\begin{cases}
			e & \text{if $k = 0$} \\
			Q(a_0:C_0 \inj C_1, \repl{k-1}{d}{e}) &\text{otherwise.}	
		\end{cases}
\end{equation*}
\end{definition}

\begin{example}
Consider the conditions $c := \forall(a_0: C_0 \inj C_1, \exists(a_1: C_1 \inj C_2, \true))$ and $e = \exists(a'_1: C_1 \inj C_3, d)$. 
The replacement starting from layer $1$ in $c$ by $e$ is given by
$$\repl{1}{c}{e} = \forall(a_0: C_0 \inj C_1, \exists(a'_1: C_1 \inj C_3, d)).$$ 
\end{example}

We now define \emph{truncated conditions after layer} using the concept of replacement starting from layer.
Intuitively, a condition is truncated after a particular layer by replacing the subcondition at the next layer with $\true$ or $\false$, depending on which quantifier the replaced subcondition is bound by.


\begin{definition}[\textbf{truncated condition after layer}]
	Let a condition $c$ in UANF be given. 
	The \emph{truncated condition of $c$ after layer $-1 \leq k < \nlvl(c)$}, 
	denoted by $\cut{k}{c}$, is defined as

$$ \cut{k}{c} := \begin{cases}
					\true &\text{if $k = -1$} \\
					\repl{k +1}{c}{\true}&\text{if $\scond{k}{c}$ is a existential condition,
					 i.e. $k$ is odd} \\
					\repl{k+1}{c}{\false}&\text{if $\scond{k}{c}$ is a universal condition, 
					i.e. $k$ is even.} \\
				\end{cases}$$

\end{definition}

\begin{example}
Consider constraint $c_2$ given in Figure \ref{fig:constraints}.
The truncated conditions of $c_2$ after layer $-1 \leq k < 3$ are given by 
\begin{enumerate}
	\item $\cut{-1}{c_2} = \true$.
	\item $\cut{0}{c_2} = \forall(C_1^1, \false)$.
	\item $\cut{1}{c_2} = \forall (C_1^1, \exists (C_2^2,\true))$.
	\item $\cut{2}{c_2} = \forall (C_1^1, \exists (C_2^2,(\forall C_3^2, \false)))$.
	\item $\cut{3}{c_2} = \forall (C_1^1, \exists (C_2^2,(\forall C_3^2, \exists(C_4^2, \true)))) = c_2$.
\end{enumerate}
\end{example}


Note that the truncated condition of a condition $c$ at layer $\nlvl(c) -1$ is $c$ itself.
With these prerequisites, we can now introduce \emph{satisfaction up to layer}, which allows us to check whether a condition is satisfied up to a given layer. A morphism or graph satisfies a condition or constraint up to a layer if it satisfies the truncated condition after that layer.


\begin{definition}[\textbf{satisfaction up to layer}]
Let a graph $G$ and a condition $c$ in UANF be given.
A morphism $p: C_0 \inj G$ \emph{satisfies $c$ up to layer  $-1 \leq k < \nlvl(c)$}, denoted by $p \models_k c$, if $$p\models \cut{k}{c}.$$

A graph $G$ satisfies a constraint $c$ up to layer $-1 \leq k < \nlvl(c)$, denoted by $G \models_k c$, if $q: \emptyset \inj G$ satisfies $\cut{k}{c}$.
The largest $-1 \leq k < \nlvl(c)$ such that $G \models_k c$ and there is no $k < j < \nlvl(c)$ with $G \models_j c$, called the \emph{largest satisfied layer}, is denoted by $\maxk{c}{G}$. 
When $c$ and $G$ are clear from the context, we use the abbreviation $\kmax$.
\end{definition}

Note that given a graph $G$ and a constraint $c$, $\maxk{c}{G}$ always exists, since $\cut{-1}{c} = \true$ and every graph satisfies $\true$.
Moreover, if $p \models_{\nlvl(c)-1} c$, i.e. $\kmax = \nlvl(c)-1$, it immediately follows that $p \models c$. 
\begin{example}\label{ex_graph}
Consider the graph $G$ given in Figure \ref{fig:graph} and the constraint $c_2$ given in Figure \ref{fig:constraints}. 
This graph does not satisfy $c_2$ because the occurrence of \emph{\texttt{Class}} denoted with \emph{\texttt{gc2}}  does not satisfy $\scond{1}{c_2} = \exists(C_2^2,(\forall C_3^2,(\exists C_4^2,\true)))$, but it satisfies $\cut{1}{c_2} = \forall(C_1^1, \exists(C_2^2, \true))$ and therefore $$G \models_1 c_2 \text{ and } \kmax = 1.$$
\end{example}


\begin{table}
\centering
\begin{tabular}{c|c|c|c|c|c}



$p \models_k c$ & \multicolumn{2}{c|}{$p \models_{j < k} c$} & \multicolumn{2}{c|}{$p \models_{j > k} c$} &$ p \models c $ \\
%\hline
 & $j$ even & $j$ odd & $j$ even & $j$ odd & \\

\hline
$k$ even & ? & \cmark &\cmark & \cmark & \cmark \\
$k$ odd & ? & \cmark & ? & ? & ? \\


\end{tabular}

\caption{Overview of the inferences made about satisfaction up to layer, where \enquote{\cmark} indicates that $p \models_j c$ or $p \models c$ if $p \models_k c$ and \enquote{?} indicates that it cannot be inferred from $p \models_k c$ whether $p \models_j c$ or $p \not \models_j c$.}\label{table_satisfaction_at_layer}

\end{table}



Given a graph $G$, a condition $c$ and a morphism $p : C_0 \inj G$.
Suppose that $p \models_k c$ with $0 \leq k < \nlvl(c)$. 
Then we can infer results for the satisfaction until up to other layers.
If $k$ is even, i.e. $\scond{k}{c}$ is a universal condition,  we can conclude that $p \models_j c$ for all $k < j < \nlvl(c)$ and especially $p \models c$.
It also follows that $p \models_j c$ for all odd $0 \leq j < k$, i.e. $\scond{j}{c}$ is an existential condition. 
We present these results in the following lemmas,  an overview is given in Table \ref{table_satisfaction_at_layer}.

We start by examining the consequences for the satisfaction up to layer $\nlvl(c) > j > k$. Our first lemma shows that replacing the subcondition $\scond{k+1}{c}$ by any condition over $C_{k+1}$ leads to a condition that is satisfied by $p$ if $k$ is even.

\begin{lemma}\label{lemma_help_lay_sat}
Given a graph $G$, a condition $c$ in UANF and a morphism $p : C_0 \inj G$ with $p \models_k c$ where $-1 \leq k < \nlvl(c)$ is even.
Then, for any condition $f$ over $C_{k+1}$ it holds that
$$p \models \repl{k + 1}{c}{f}.$$
\end{lemma}

\begin{proof}
We start by showing the statement for the smallest $-1 \leq j < \nlvl(c)$ such that $\scond{j}{c}$ is universally bound and $p \models_j c$. 
After this, we can conclude that this statement holds for all $-1 \leq i < \nlvl(c)$ such that $\scond{i}{c}$ is universally bound and $p \models_i c$.

Let $q: C_{j} \inj G$ be a morphism such that $q \models \forall(a_j:C_{j} \inj C_{j+1}, \false)$. 
This morphism must exist, since $j$ is the smallest even number with $p \models_j c$.
Therefore, there is no morphism $q': C_{j+1} \inj G$ with $q = q' \circ a_j$.
Hence, for every condition $f$ over $C_{j+1}$ a morphism $q': C_{j+1} \inj G$ with $q \not \models f$ and $q = q' \circ a_j$ cannot exist. 
It follows immediately that $q \models \forall(a_j:C_{j} \inj C_{j+1}, f)$ and with that $p \models \repl{j + 1}{c}{f}$.

We can now conclude that for every even $j < k \leq \nlvl(c)$, such that $p \models_k c$, and every condition $d$ over $C_{k + 1}$ it holds that $p \models \repl{k + 1}{c}{d}$ because $\repl{k + 1}{c}{d} = \repl{j+1}{c}{\scond{j+1}{\repl{k+1}{c}{d}}}$.
\end{proof}

As a direct consequence of the previous lemma, a morphism which satisfies a condition up to layer $k$, where $k$ is even, also satisfies the condition at layer
$j$ for all $j > k$.


\begin{lemma}\label{lemma_lay_sat}

Given a graph $G$, a morphism $p:C_0 \inj G$ and a condition $c$ in UANF.
If $0 \leq k < \nlvl(c)$ is even, i.e. $\scond{k}{c}$ is a universal condition, then for all $k < j < \nlvl(c)$ it holds that
	$$p \models_k c \implies p \models_j c.$$

\end{lemma}
\begin{proof}
Follows immediately by using Lemma \ref{lemma_help_lay_sat} and setting $f$ equal to $\scond{k+1}{\cut{j}{c}}$.
\end{proof}

Since a morphism $p$ satisfies a condition $c$ in UANF if and only if $p$ satisfies $c$ up to layer $\nlvl(c)-1$, we can conclude the following.


\begin{corollary}\label{corol:satisfaction}
	Given a graph $G$, a morphism $p:C_0 \inj G$ and a condition $c$ in UANF. If 
	$0 \leq k < \nlvl(c)$ is even, it holds that 
	$$p \models_k c \implies p \models c.$$
\end{corollary}
In the following, we will mostly assume that $\kmax$ is odd. This is because an even $\kmax$ implies that the condition is already satisfied.
Furthermore, this allows us to make statements about the satisfaction of other conditions. Given a graph $G$, a morphism $p : C_0 \inj G$ and a condition $c$ such that $p \models_k c$ for an even $-1 \leq k < \nlvl(c)$. It follows that $p \models c$ and in particular $p \models c'$ for each condition $c'$ with  $\cut{k}{c} = \cut{k}{c'}$.

Let us now examine the satisfaction up to layer $j$ with $-1 < j < k$. If $j$ is odd, i.e. $\scond{j}{c}$ is an existential condition, we can conclude that $p \models_j c$ as shown in the next lemma.
If $j$ is even, i.e. $\scond{j}{c}$ is universally bound, we can only make statements that depend on $\kmax$. If $\kmax < \nlvl(c) -1$, then $p \not \models_j c$. Otherwise, Corollary \ref{corol:satisfaction} implies  that $p \models c$ and therefore $\kmax = \nlvl(c)-1$.
If $\kmax = \nlvl(c)-1$, we can say that there is at least one even $j \leq \kmax$ with $p \models_j c$ if $c$ ends with $\forall(C_{\nlvl(c)}, \false)$.
An overview of these relations is given in Table \ref{table_abh_kmax}.

\begin{figure}
\centering
\begin{tabular}{c||c|c||c}



$p \models_k c$ & \multicolumn{2}{c||}{$\kmax < \nlvl(c)-1$}  &$ \kmax = \nlvl(c) -1$ \\
%\hline
 & $k \leq \kmax$ & $k > \kmax$ & $k < \kmax$  \\

\hline
$k$ even & \xmark & \xmark &? \\
$k$ odd & \cmark & \xmark & \cmark \\


\end{tabular}
\begin{TableCaption}
\caption{Overview of the satisfaction of layer $k$ with respect to $\kmax$, where \enquote{\cmark} indicates that $p \models_k$, \enquote{\xmark} indicates that $p \not \models_k$ and \enquote{?} indicates that it cannot be concluded from $p \models_k c$ whether $p \models_j c$ or $p \not \models_j c$.  }\label{table_abh_kmax}
\end{TableCaption}
\end{figure}

\begin{lemma}\label{lem_ex_lower}
Given a graph $G$, a morphism $p: C_0 \inj G$ and a constraint $c$ in UANF.
Then for all odd $-1 \leq k \leq \kmax$, i.e. $\scond{k}{c}$ is an existential condition, we have
$$p \models_k c.$$

\end{lemma}
\begin{proof}
If there is an even $0 \leq j < \kmax$, i.e. $\scond{j}{c}$ is universal, with $p \models_j c$, let $j'$ be the smallest of these.
Lemma \ref{lemma_help_lay_sat} implies that $p \models_{\ell} c$ for all $j' \leq \ell < \nlvl(c)$. Otherwise we set $j' = \kmax$.

Let $\ell < j'$, such that $\scond{\ell}{c}$ is a existential condition and let $d = \scond{\ell}{\cut{j'}{c}} = \exists(a_{\ell}: C_{\ell} \inj C_{\ell+1}, e)$ be the subcondition at layer $\ell $ of the truncated condition after layer $j'$ of $c$. 
	Since $\ell < j'$, there must be a morphism $q: C_{\ell} \inj G$ with $q \models d$ and therefore there must be a morphism $q': C_{\ell +1}\inj G$ with $q = q' \circ a_{\ell}$ and $q' \models e$.
	It follows that $q \models \exists(a_{\ell}: C_{\ell} \inj C_{\ell+1}, \true)$ and thus $p \models_{\ell} c$.
\end{proof}

\begin{example}
	We will show counterexamples for all \enquote{?} in Table 
	\ref{table_satisfaction_at_layer} and Table \ref{table_abh_kmax}.
	Consider constraint $c_2 = \forall(C_1^1, \exists(C_2^2, \forall (C_3^2, 
	\exists(C_4^2, \true))))$ given in Figure \ref{fig:constraints}.
	We begin with Table \ref{table_satisfaction_at_layer}.
	\begin{enumerate}
		\item $j < k$, $j$ and $k$ are even. We set $k = 2$ and $j = 0$. 
			 It follows that $\cut{k}{c} = \forall(C_1^1, \exists(C_2^2, \forall (C_3^2, \false)))$ and $\cut{j}{c} = \forall(C_1^1,\false)$.  Then,
			 \begin{enumerate}
			 \item $C_2^2 \models_2 c_2$ and $C_2^2 \not \models_0 c_2$.
			  
			  \item $\emptyset \models_2 c_2$ and $\emptyset \models_0 c_2$.
			  
\end{enumerate}
		\item $j <k$, $k$ is odd and $j$ is even. We set $k = 3$ and $j = 0$. 
		It follows that $\cut{k}{c_2} = c_2$ and $\cut{j}{c} = \forall(C_1^1,\false)$.
		Then,
		\begin{enumerate}
			\item $C_4^2 \models_3 c_2$ and $C_4^2 \not \models_0 c_2$.
			\item $\emptyset \models_3 c_2$ and $\emptyset \models_0 c_2$.
\end{enumerate}				
		 
		
		\item $j > k$, $k$ is odd and $j$ is even. We set $k = 1$ and $j = 2$. It follows that 
		$\cut{k}{c} = \forall(C_1^1, \exists(C_2^2,\true))$ and $\cut{j}{c} = \forall(C_1^1, \exists(C_2^2, \forall (C_3^2, \false)))$. Then,
		\begin{enumerate}
			\item $C_2^2 \models_1 c_2$ and $C_2^2 \models_2 c_2$.
			\item $C_3^2 \models_1 c_2$ and $C_3^2 \not \models_2 c_2$.	
\end{enumerate}				
		  
		\item $j >k$, $k$ and $j$ are odd. We set $k = 1$ and $j = 3$. It follows that $\cut{k}{c} = \forall(C_1^1, \exists(C_2^2,\true))$ and $\cut{j}{c} = c_2$. Then,
		\begin{enumerate}
		 \item $C_2^2 \models_1 c_2$ and $C_2^2 \models_3 c_2$.
		 \item $C_3^2 \models_1 c_2$ and $C_3^2 \not \models_3 c_2$.
\end{enumerate}	 Since  $\cut{j}{c} = c_2$ this is also a counterexample for the \enquote{?} in the column denoted by \enquote{$p \models c$}.
		
	\end{enumerate}
	For the \enquote{?} in Table \ref{table_abh_kmax}, the graph $C_2^2$ satisfies $c_2$ and therefore $\kmax = 3 = \nlvl(c_2)-1$. 
	But, $C_2^2 \models_2 c_2$ and $C_2^2 \not \models_0 c_2$.
\end{example}

