\subsection{Comparison with other concepts of Consistency}\label{comp_general}
In this chapter, the notions of (direct) consistency increase- and maintainment are compared to the already known concepts of consistency guaranteeing, consistency preserving \cite{habel2009correctness}, (direct) consistency increasing and sustaining \cite{kosiol2022sustaining} in order to reveal relationships between them and ensure that (direct) consistency increase- and 
maintainment are indeed a new notions of consistency. 
These relations are displayed in Figure \ref{fig:consistency_relations}.

\begin{figure}
\center
	\begin{tikzpicture}
	
		
		\node(guaran) at (4.5,2) {$c$-guaranteeing};
		\node(pres) at (0,0) {$c$-preserving};
		\node (incr) at (10,0) {increasing w.r.t. $c$};
		\node (direct) at (10,-2) {direct increasing w.r.t. $c$};
		\node (maintaining) at (4.5,0) {maintaining w.r.t. $c$};
		\node (dir_maintaining) at (4.5,-2) {direct maintaining w.r.t. $c$};
		
		
		\draw[-implies,double equal sign distance] (guaran) --  (maintaining); 
		\draw[-implies,double equal sign distance] (guaran) --  (pres);
		\draw[-implies,double equal sign distance] (incr) --  (maintaining);
		\draw[-implies,double equal sign distance] (maintaining) --  (pres);
		\draw[-implies,double equal sign distance] (direct) --  (incr);   
		\draw[-implies,double equal sign distance] (dir_maintaining) --  (maintaining);
		\draw[-implies,double equal sign distance] (direct) --  (dir_maintaining); 
		    
	\end{tikzpicture}
	\caption{Relations of consistency notions.}\label{fig:consistency_relations}
\end{figure}



First, we compare (direct) consistency increase- and maintainment with the notions of consistency-guaranteeing and -preserving.
We start by showing that consistency maintaining implies preserving but the backwards implication does not hold.

\begin{lemma}\label{comparision:maintaining_preserving}
	Let a constraint $c$ in UANF, graphs $G$ and $H$ and a transformation 
	$t: G\Longrightarrow H$ be given. Then, 
	\begin{alignat*}{3}
			&t \text{ is maintaining w.r.t. $c$} && \implies \text{$t$ is 
			$c$-preserving} & \text{ and}\\
			&t \text{ $c$-preserving} && \centernot\implies \text{$t$ is 
			maintaining 	w.r.t. $c$} & \text{ }\\	
	\end{alignat*}
\end{lemma}
\begin{proof}
	\begin{enumerate}
		\item 
			Let $t$ be a consistency maintaining transformation w.r.t. $c$. 
			If $G \not \models c$, $t$ is a $c$-preserving transformation.
			If $G \models c$, it holds that $\nv{j}{G} = 0$ for all 
			$0 \leq j < \nlvl(c)$. Since $t$ is consistency maintaining
			it follows that $\nv{j}{H} = 0$ for all 
			$0 \leq j < \nlvl(c)$ and therefore $H \models c$. 
			It follows that $t$ is a $c$-preserving transformation.
			
		\item	
			Consider graphs $C_1^1$, $C_2^2$ and constraint $c_1$ given in 
			Figure \ref{fig:constraints}. Then, the transformation
			$t: C_2^2 \Longrightarrow C_1^1$ is $c$-preserving, since 
			$C_2^2 \not \models c_1$, but not consistency maintaining
			w.r.t. $c$ since $\nv{0}{C_2^2} = 2$ and $\nv{0}{C_1^1} = 5$.  
	\end{enumerate}
\end{proof}

Obviously, guaranteeing implies consistency maintaining, since this property is embedded in the definition of consistency maintainment. 
The inversion of this implication does not holds, since maintaining is a way stricter notion, in the sense, that the number of removed violations has to be greater or equal than the number of introduced violations.
For guaranteeing transformations this is not the case, an arbitrary number of violations can be inserted, as long as the derived graph satisfies the constraint and therefore guaranteeing does not imply direct increasing, since a direct increasing transformations is not allowed to introduce any new violations.

\begin{lemma}\label{lemma:guaranteeing_maintaining}
	Let a constraint $c$ in UANF, graphs $G$ and $H$ and a transformation 
	$t: G \Longrightarrow H$ be given. Then,
	\begin{alignat*}{3}
		&t \text{ is $c$-guaranteeing} &&\implies t \text{ is maintaining w.r.t 
		$c$} &\text{ and} \\
		&t \text{ is $c$-guaranteeing} &&\centernot \implies t \text{ is 
		direct maintaining w.r.t $c$} &\text{ and} \\
		&t \text{ is maintaining w.r.t. $c$} &&\centernot \implies t \text{ is 
		$c$-guaranteeing} &\text{ } 
	\end{alignat*}
\end{lemma}
\begin{proof}
	\begin{enumerate}
		\item 
			Let $t$ be a c-guaranteeing transformation, then $H \models c$.
			It holds that $\nv{j}{H} = 0$ for all $0 \leq j < \nlvl(c)$ and 
			since the number of violations cannot be negative, $t$ is a 
			maintaining transformation w.r.t. $c$.
		
		\item 
			Let $t$ be a $c$-guaranteeing transformation. Then, $t$ is also
			a $c$-preserving transformation and since each direct maintaining
			transformation is also a maintaining one the statement follows 
			directly with Lemma \ref{comparision:maintaining_preserving}.
			
		\item 
			Consider the transformation $t: G \Longrightarrow H$ shown in Figure 
			\ref{fig:counterexamples_guaranteeing_maintaining} and constraint
			$c_1$ shown in Figure \ref{fig:constraints}.
			Then, $t$ is a maintaining transformation w.r.t $c$, in particular,
			$c$ is a consistency increasing transformation w.r.t. $c$ since 
			$\nv{0}{G} = 10$ and $\nv{0}{H} = 7$.
			But, $t$ is not $c$-guaranteeing since both occurrences 
			of nodes of type \texttt{Class} do not satisfy $\exists 
			C_2^1$.
			
	\end{enumerate}
\end{proof}

\begin{figure}
\includegraphics[scale=0.8]{figures/images/counterexample_direct_guaranteeing}
\caption{example}\label{fig:counterexamples_guaranteeing_maintaining}
\end{figure}

Let $t:G \Longrightarrow H$ be a $c$-guaranteeing transformation. If $G \models c$, 
the transformation is, by definition, not consistency increasing. 
If $G \not \models c$, $t$ is always also a consistency increasing transformation w.r.t. $c$.

\begin{lemma}\label{comp_guaranteeing_minimal}
	Let graphs $G,H$, a constraint $c$ with $G \not \models c$ and a 
	transformation $t: G \Longrightarrow H$ be given. Then
	\begin{equation*}
		t \text{ is $c$-guaranteeing } \implies t \text{ is increasing w.r.t $c$.} 
	\end{equation*}
\end{lemma}

\begin{proof}
	Let $t$ be a c-guaranteeing transformation, then $H \models c$. 
	Since $G \not \models c$,  it holds that $\nv{\maxk{c}{G} + 1} {G} > 0$ and 
	$\nv{\maxk{c}{G} + 1} {H} = 0$. Therefore, $t$ is consistency increasing. 
\end{proof}

The definition of consistency improvement only differs from guaranteeing if the corresponding constraint is universally bound and these notions are identical for existentially bound constraint. 
Therefore, with Lemmas \ref{lemma:guaranteeing_maintaining} and \ref{comp_guaranteeing_minimal}, we can state the following.

\begin{corollary}\label{cor:comp_existentially}
	Let an existentially bound constraint $c$ in ANF and a transformation $t: G 
	\Longrightarrow H$ be given. 
	Then, 
	\begin{alignat*}{2}
		&t \text{ is consistency improving w.r.t $c$ } &&\implies t \text{ is 
		consistency maintaining w.r.t $c$ } \text{ and}\\
		&t \text{ is consistency maintaining w.r.t $c$} &&\centernot \implies t 
		\text{ is consistency improving w.r.t $c$.} 
	\end{alignat*}
	If $G \not \models c$, we can also state that 
	\begin{equation*}
		t \text{ is consistency improving w.r.t $c$ } \implies t \text{ is 
		consistency increasing w.r.t $c$ } 
	\end{equation*}
\end{corollary}

The notions of increae- and improvement are equivalent for universally bound constraints with nesting level $1$. 
Note that, with corollary \ref{cor:comp_existentially}, this equivalence does not hold for existentially bound constraints with nesting level $1$.


\begin{lemma}
	Let a universally bound constraint $c$ in UANF with $\nlvl(c) = 1$, a graph $G$  with $G \not \models c$ and a transformation $t: G \Longrightarrow H$ be given. 
	Then,
	$$t \text{ is consistency improving w.r.t $c$ } \iff t \text{ is consistency increasing w.r.t $c$}$$ 
	
\end{lemma}

\begin{proof}
Let $c = \forall(a: \emptyset \inj C, \false)$. 
Since $\scond{1}{c} = \false$, $\nv{0}{G}$ is the number of occurrences of $C$ in $G$. 
This is exactly the definition of the number of violations for consistency improving transformations and the statement follows immediately.
\end{proof}

For universally bound constraints $c$ with $\nlvl(c) \geq 2$, the notions of (direct) consistency increase-  and maintainment are not related to (direct) consistency improve- and sustainment. 
By definition, (direct) consistency improvement implies (direct) consistency sustainment \cite{kosiol2022sustaining}. 
Therefore, it is sufficient to show that direct improvement does not imply maintainment and that direct increasement does not imply consistency sustainment. 

\begin{figure}
\centering
\includegraphics[scale=1]{figures/sources/transformations_comparison}
\caption{example}\label{fig:counterexamples_increase_improving}
\end{figure}

\begin{lemma}
Let a universally bound constraint $c$ in UANF with $\nlvl(c) \geq 2$, a graph $G$ with $G\not \models c$ and a transformation $t: G \Longrightarrow H$ be given. Then,
\begin{alignat*}{2}
&t \text{ is direct consistency improving w.r.t $c$ } &&\centernot \implies t \text{ is consistency maintaining w.r.t $c$ } \text{ and} \\
&t \text{ is direct increasing w.r.t $c$ } &&\centernot \implies t \text{ is consistency sustaining w.r.t $c$ }
\end{alignat*}
\end{lemma}


\begin{proof}
	\begin{enumerate}
		\item	
			Consider transformation $t_1$ given in Figure 
			\ref{fig:counterexamples_increase_improving} and constraint
			$c_1$ given in Figure \ref{fig:constraints}. The transformation
			$t_1$ is direct consistency improving but not maintaining since
			$\nv{0}{G} = 4$ and $\nv{0}{H} = 5$.
		\item
			Consider the constraint
			$c = \forall (C_1^1, \exists (C_2^1, \forall (C_4^2,d)))$ with 
			$d$ being an existentially bound constraint in ANF with $d \neq 
			\false$ 	composed of the graphs given in Figure 
			\ref{fig:constraints} and
			transformation $t_2$ given in Figure 
			\ref{fig:counterexamples_increase_improving}.
			Then, $t$ is direct consistency increasing, (
			\ref{direct_improving_1}), (\ref{direct_improving_1_2}), 
			(\ref{direct_improving_3}) and (\ref{direct_improving_4}) are 
			trivially satisfied and (\ref{direct_improving_2}) is satisfied 
			since one occurrence of $C_1^1$ that did not satisfy 
			$\exists C_2^1$ in
			$G$ satisfies $\exists C_2^1$ in $H$,  but not consistency 
			sustaining since the number of occurrences of $C_1^1$ that 
			not satisfying $\exists (C_2^1, \forall (C_4^2,d))$ in $H$ is 
			greater than the number of occurrences of $C_1^1$ in $G$ not 
			satisfying $\exists (C_2^1, \forall (C_4^2,d))$.
	\end{enumerate}
\end{proof}

