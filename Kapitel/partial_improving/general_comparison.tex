\subsection{Comparison with other concepts of Consistency}\label{comp_general}
In this chapter, the notions of (direct) consistency increase and maintainment are compared to the already known notions of consistency-guaranteeing, consistency-preserving \cite{habel2009correctness}, (direct) consistency-increasing and sustaining \cite{kosiol2022sustaining}, in order to reveal relations between them and to ensure that (direct) consistency-increase and 
maintainment are indeed new notions of consistency. 
These relationships are summarised in Figure \ref{fig:consistency_relations}.

\begin{figure}
\center
	\begin{tikzpicture}
	
		
		\node(guaran) at (4.5,2) {$c$-guaranteeing};
		\node(pres) at (0,0) {$c$-preserving};
		\node (incr) at (10,0) {increasing w.r.t. $c$};
		\node (direct) at (10,-2) {direct increasing w.r.t. $c$};
		\node (maintaining) at (4.5,0) {maintaining w.r.t. $c$};
		\node (dir_maintaining) at (4.5,-2) {direct maintaining w.r.t. $c$};
		
		
		\draw[-implies,double equal sign distance] (guaran) --  (maintaining); 
		\draw[-implies,double equal sign distance] (guaran) --  (pres);
		\draw[-implies,double equal sign distance] (incr) --  (maintaining);
		\draw[-implies,double equal sign distance] (maintaining) --  (pres);
		\draw[-implies,double equal sign distance] (direct) --  (incr);   
		\draw[-implies,double equal sign distance] (dir_maintaining) --  (maintaining);
		\draw[-implies,double equal sign distance] (direct) --  (dir_maintaining); 
		    
	\end{tikzpicture}
	\caption{Relations of consistency notions.}\label{fig:consistency_relations}
\end{figure}



First, we compare (direct) consistency increase and maintainment with the notions of consistency-guaranteeing, preserving, sustaining and improving in the general case and later on, for some  special cases.
We begin by examining the implications that can be drawn about a consistency-maintaining or consistency-increasing transformation.


\begin{theorem}[\textbf{Implications of a consistency-maintaining or consistency-increasing transformation}]
	Given a condition $c$ in UANF and a transformation $t: G \Longrightarrow H$.
	Then,
	\begin{alignat*}{3}
			&t \text{ is consistency-maintaining w.r.t. $c$} && \implies \text{$t$ is 
			$c$-preserving} & \text{ and}\\
			&t \text{ is consistency-maintaining w.r.t. $c$} &&\centernot \implies t \text{ 
			is $c$-guaranteeing} &\text{ and} \\
		&t \text{ is direct consistency-increasing w.r.t. $c$ } &&\centernot \implies t \text{ is consistency sustaining w.r.t. $c$ }
	\end{alignat*}
\end{theorem}

\begin{proof}
	\begin{enumerate}
		\item 
		$t$ is consistency-maintaining w.r.t. $c$ $\implies$ $t$ is $c$-preserving: Let $t$ be a consistency-maintaining transformation w.r.t. $c$. 
			If $G \not \models c$, then $t$ is a $c$-preserving transformation.
			If $G \models c$, then $\nv{j}{G} = 0$ for all 
			$0 \leq j < \nlvl(c)$. Since $t$ is consistency maintaining
			it follows that $\nv{j}{H} = 0$ for all 
			$0 \leq j < \nlvl(c)$ and hence $H \models c$. 
			It follows that $t$ is a $c$-preserving transformation.
		
		\item
		$t$ is consistency maintaining w.r.t. $c$ $\centernot \implies$ $t$ is $c$-guaranteeing:
		Consider the transformation $t_2: G \Longrightarrow H$ shown in Figure 
			\ref{fig:example_direct} and constraint
			$c_1$ shown in Figure \ref{fig:constraints}.
			As discussed in Example \ref{ex_direct}, $t_2$ is consistency-increasing and thus consistency-maintaining w.r.t. $c_1$. 
			But $t_2$ is not a $c$-guaranteeing transformation, since all 
			occurrences of nodes of type \texttt{Class} do not satisfy 
			$\exists(C_2^1, \true)$.
		\item 
		$t$ is direct consistency maintaining w.r.t. $c$ $\centernot \implies$ $t$ is consistency sustaining w.r.t. $c$:
		Consider the constraint $c = \forall (C_1^1, \exists (C_2^1, \forall (C_4^2,d)))$, where $d$ is an existentially bound constraint in ANF with $d \neq  \false$ 	composed of the graphs given in Figure \ref{fig:constraints}. And consider  transformation $t_2$ given in Figure \ref{fig:counterexamples_increase_improving}. Then, $t$ is direct consistency increasing; the \emph{no new violation by deletion}, \emph{no new violation by insertion}, \emph{no satisfied layer reduction by insertion} and \emph{no satisfied layer reduction by deletion}  formulas are satisfied and the  \emph{general increasing} formula is  satisfied because an occurrence of $C_1^1$ that did not satisfy $\exists (C_2^1,\true)$ in $G$ satisfies $\exists (C_2^1, \true)$ in $H$.  But this transformation is not consistency-sustaining since the number of occurrences of $C_1^1$ not satisfying $\exists (C_2^1, \forall (C_4^2,d))$ in $H$ is greater than the number of occurrences of $C_1^1$ in $G$ not satisfying $\exists (C_2^1, \forall (C_4^2,d))$.
	\end{enumerate} \qedhere
\end{proof}

These results are not surprising, since consistency-maintaining and con\-sis\-ten\-cy-in\-crea\-sing are much stricter notions than guaranteeing and sustaining, in the sense that the notion of violation is more fine-grained.
For example, for guaranteeing transformations, an arbitrary number of violations can be introduced as long as the derived graph satisfies the constraint, and thus guaranteeing does not imply direct increasing, since a direct increasing transformation is not allowed to introduce new violations.
Let us now examine whether a concept of consistency implies the notions of 
consistency-maintaining and increasing.


\begin{theorem}[\textbf{Implications of preserving, guaranteeing, sustaining and improving transformations}]
 	Given a condition $c$ in UANF and a transformation $t:G \Longrightarrow H$.
 	Then,
 	\begin{alignat*}{3}
		&t \text{ is $c$-guaranteeing} &&\implies t \text{ is consistency-maintaining w.r.t. 
		$c$} &\text{ and} \\
		&t \text{ is $c$-guaranteeing} &&\centernot \implies t \text{ is consistency-increasing w.r.t. 
		$c$} &\text{ and} \\
		&t \text{ is $c$-preserving} &&\centernot \implies t \text{ is 
		consistency-maintaining w.r.t. $c$} &\text{ and} \\
		&t \text{ is direct consistency improving w.r.t $c$ } &&\centernot \implies t \text{ is consistency-maintaining w.r.t. $c$ } 
	\end{alignat*}
\end{theorem}
\begin{proof}
	\begin{enumerate}
		\item
		$t$ is $c$-guaranteeing $\implies$ $t$ is consistency-maintaining w.r.t. $c$:
		Let $t$ be a $c$-guaranteeing transformation. Then, $t$ is also a 
		consistency-maintaining transformation w.r.t. $c$ since 
		$H \models c$ and therefore $\nv{j}{H} = 0$ for all $-1 \leq j < \nlvl(c)$.
		\item
		$t$ is $c$-guaranteeing $\centernot \implies$ $t$ is consistency-increasing w.r.t. $c$:
		Assume that $G \models c$ and $H \models c$. Then, $t$ is a $c$-guaranteeing transformation, but not a consistency-increasing one.
		\item
		$t$ is $c$-preserving $\centernot \implies$ $t$ is consistency-maintaining w.r.t. $c$:
		Consider graphs $C_1^1$, $C_2^2$ and constraint $c_1$ given in 
			Figure \ref{fig:constraints}. The transformation
			$t: C_2^2 \Longrightarrow C_1^1$ is $c_1$-preserving, since 
			$C_2^2 \not \models c_1$, but not consistency maintaining
			w.r.t. $c_1$ since $\nvio{0}{c_1}{C_2^2} = 4$ and $\nvio{0}{c_1}{C_1^1} = 6$.
		\item
		$t$ is direct consistency-improving w.r.t. $c$ $\centernot \implies$ $t$ is consistency-maintaining w.r.t. $c$:
		Consider transformation $t_1$ given in Figure 
			\ref{fig:counterexamples_increase_improving} and constraint
			$c_1$ given in Figure \ref{fig:constraints}. The transformation
			$t_1$ is direct consistency-improving since no occurrence of $C_1^1$ is inserted, no occurrence of $C_1^1$ satisfying $\exists(C_2^1, \true)$ is deleted, and one occurrence of $C_1$ satisfies $\exists(C_2^1, \true)$. But, this 
			transformation is not consistency-maintaining w.r.t. $c$ since the 
			number of violations in the first graph is $2$ and the number of 
			violations in the second graph is $3$.
	\end{enumerate}
\end{proof}
\begin{figure}
\centering
\includegraphics[scale=1]{figures/sources/transformations_comparison}
\caption{example}\label{fig:counterexamples_increase_improving}
\end{figure}
This shows, that in general the notions of (direct) consistency increase and maintainment are not related to (direct) consistency improvement and sustainment.
We have shown only some of these relationships. Since (direct) consistency improvement (direct) consistency sustainment, consistency-preserving and consistency-guaranteeing are related, we can conclude results for all pairs of consistency types \cite{kosiol2022sustaining}.
An overview of these is given in Table \ref{table_cons_relations}.

For some special cases, we can infer other types of relationships.

\begin{theorem}[\textbf{Relations of consistency concepts in special cases}]
	Given a constraint $c$ in UANF and a transformation $t: G \Longrightarrow H$.
	\begin{enumerate}
	\item 
		If $G \not \models c$, then 
		$$ \text{$t$ is $c$-guaranteeing $\implies$ $t$ is consistency-increasing w.r.t. $c$.}$$
	\item If $\nlvl(c) =1$, then
	$$t \text{ is consistency improving w.r.t $c$ } \iff t \text{ is consistency increasing w.r.t $c$}$$ 
		
	\end{enumerate}

\end{theorem}

\begin{proof}
	\begin{enumerate}
		\item Let $t$ be a $c$-guaranteeing transformation with $G \not \models c$. Then, $t$ is a consistency-increasing transformation w.r.t. $c$ since 
		$0 < \nv{\maxk{c}{G}+1}{G} < \infty$ and $\nv{j}{H} = 0$ for all 
		$-1 \leq j \leq \nlvl(c)$.
		\item Let $\nlvl(c) = 1$. 
		Since $c$ is in UANF, $\scond{1}{c} = \false$ and $\nv{0}{G}$ is the number of occurrences of $C$ in $G$. 
This is exactly the definition of the number of violations for consistency-improving transformations, and the statement immediately follows.
	\end{enumerate}
\end{proof}
\begin{figure}
\centering
\begin{tabular}{c|c|c|c|c|c|c|c|c|c|c}



$\implies$ &(1) & (2) & (3) &(4) & (5) & (6) & (7) & (8) & (9) & (10) \\
%\hline


\hline
maintaining (1) & \cmark & \xmark & \xmark & \xmark & \xmark &\xmark & \xmark &\xmark & \xmark &\cmark \\
increasing (2)& \cmark & \cmark & \xmark & \xmark & \xmark &\xmark & \xmark &\xmark & \xmark &\cmark\\
direct maintaining(3)& \cmark & \xmark & \cmark & \xmark & \xmark &\xmark & \xmark &\xmark & \xmark &\cmark \\
direct increasing (4) & \cmark & \cmark & \cmark & \cmark & \xmark &\xmark & \xmark &\xmark & \xmark &\cmark \\
improving (5) & \xmark & \xmark & \xmark & \xmark & \cmark * &\cmark * & \xmark * &\xmark * & \xmark *&\cmark * \\
sustaining(6)  & \xmark & \xmark & \xmark & \xmark & \xmark * &\cmark * & \xmark * &\xmark * & \xmark *&\cmark *\\
direct improving (7) & \xmark & \xmark & \xmark & \xmark & \cmark * &\cmark * & \cmark * &\cmark * & \xmark *&\cmark *\\
direct sustaining (8) & \xmark & \xmark & \xmark & \xmark & \xmark * &\cmark * & \xmark * &\cmark * & \xmark *&\cmark *\\
guaranteeing(9)& \cmark & \xmark & \xmark & \xmark & \cmark * &\cmark * & \cmark * &\cmark * & \cmark **&\cmark ** \\
preserving (10) & \xmark & \xmark & \xmark & \xmark & \xmark * &\xmark * & \xmark * &\xmark * & \xmark **&\cmark **\\

\end{tabular}
\begin{TableCaption}
\caption{Overview of the relationships between consistency concepts, \enquote{\cmark}
indicates that the notion in this row implies the notion in the column, and 
\enquote{\xmark} indicates that this implication does not hold. 
All results marked with \enquote{*} are from \cite{kosiol2022sustaining} and those marked with \enquote{**} are from \cite{habel2009correctness}.}\label{table_cons_relations}
\end{TableCaption}
\end{figure}







