\subsection{minimal consistency improving}



\begin{definition}[\textbf{number of violations}]
Let $G$ be a graph and $c$ a constraint in EANF. 
The \emph{number of violations $\nvc(j,G)$ at layer $j$ in $G$} is defined as:
\begin{enumerate}
\item If $j < c_{\max}$:
	$$\nvc(j,G) := 0$$
\item If $j = c_{\max}$, let $d = \forall(a_k: C_{j} \inj C_{j+1},e)$ be the subcondition of $c$ at layer $j+1$.


$$\nvc(j,G) := \sum_{C \in \mathcal{G}_{c}} \sum_{C' \in \mathcal{U}(C,{C_{j+1}})} |\{q \mid q:C_{j+1} \inj G \wedge q \not \models \parcond(1,e,C')\}| $$

\item If $j > c_{\max}$:
	$$\nvc(j,G) := \infty$$
\end{enumerate}

\end{definition}


\begin{definition}[\textbf{minimal consistency improving}]
	Let a graph $G$, a rule $r$ and a constraint $c$ in ANF be given. 

	
	A transformation $G \Longrightarrow_{r,m} H$ is called \emph{minimal consistency improving}, if  $$\nvc(k, H) < \nvc(k, G) $$
	for any $0 \leq k \leq \nl(c)$.
	A rule $r$ is called \emph{minimal consistency improving}, if all of its applications to graphs $G$ with $G \not \models c$ are. 
	
\end{definition}

\begin{lemma}\label{lem_ex_lower}
Let a graph $G$, a morphism $p: C_0 \to G$ and a constraint $c$ in ANF over $C_0$ with $p \models_k c$ be given.
Then, $p \models_j c$ for all $j < k$ such that the subcondition of $c$ at layer $j$ is existentially bound. 

\end{lemma}
\begin{proof}
\begin{enumerate}
	\item The subcondition of $c$ at layer $k$ is existentially bound:	
	If an $j < k$ with $p \models_j c$ exists such that the subcondition of $c$ at layer $j$ is universally bound, let $j_1$ be the smallest of these. With lemma \ref{lemma_help_lay_sat} follows that $p \models_{j_2} c $ for all $j_1 < j_2$.
	Let $\ell < j_1$, such that the subcondition of $c$ at layer $\ell$ is existentially bound and let $d = \exists(a_{\ell}: C_{\ell} \inj C_{\ell+1}, e)$ be the condition up to layer $j_1 - \ell$ of the subcondition of $c$ at layer $\ell$. 
	Since $\ell < j_1$, a morphism $q:  C_{\ell} \to G$ with $q \models d$ must exists and therefore a morphism $q': C_{\ell +1}\to G$ with $q = q' \circ a_k$ must exists. 
	If follows that $q \models \exists(a_{\ell}: C_{\ell} \inj C_{\ell+1}, \true)$ and with that $p \models_{\ell} c$.
	
	\item The subcondition of $c$ at layer $k$ is universally bound:
	With lemma \ref{lemma_help_lay_sat} follows that $p \models_{k+1} c$. 
	Since $c$ is in ANF,  $1.$ can be applied  to $k+1$. 
\end{enumerate}
\end{proof}

\begin{theorem}
	Let a graph $G$, a rule $r$ and a constraint $c$ in ANF be given.
	Let $k < \nl(c)$ be the biggest number, such that $G \models_k c$.
	A transformation $G \Longrightarrow_{r,m} H$ is minimal consistency improving if $G \models_j c$ and  $k < j$.
	
\end{theorem}

%\begin{proof}
%Since $k < \nl(c)$, it follows that $G \not \models c$. 
%With lemma \ref{lemma_lay_sat} follows that the subcondition of $c$ at layer $k$ has to be existentially bound and since $k$ is the biggest number such that $G\models_k c$ it follows that $\nvc(k+1,G) > 0$.
%If the subcondition of $c$ at layer $j$ is universally bound, $H \models c$ follows with lemma  \ref{lemma_lay_sat} and with lemma \ref{lem_ex_lower} $H \models_{k+2} c$ follows. Therefore, $\nvc(k+2,H) = 0$. 
%Otherwise, if the subcondition of $c$ at layer $j$ is existentially bound and therefore $G \models_{k+2} c$, $\nvc(k+2,H) = 0$ follows immediately.
%\end{proof}
\begin{proof}
No $\ell >k$ with $G\models_{\ell} c$ exists and $G \models_k c$. Hence, $\nvc(k,G) > 0$ and $\nvc(k,G) \neq \infty$. 
Since $j >k$, $\nvc(k,H) = 0$ and it follows immediately that the transformation is minimal consistency improving. 
\end{proof}

\begin{definition}[\textbf{direct minimal consistency improving}]
Let $G$ be a graph, $r$ a plain rule and $c$ a constraint in EANF.
Let $d = \forall(a_k: C_{k-1} \inj C_{k}, e)$ be the condition at layer $k = \cmax + 1 \leq \nl(c)$ of $c$ and $$\mathbf{G} := \bigcup_{C' \in \mathcal{G}_c^G} \mathcal{U}(C',C_{k+1})$$ be the set of all minimal upper-graphs of all biggest partially satisfying graphs. 
A transformation $t: G \Longrightarrow_{r,m} H$ is called \emph{direct minimal consistency improving} if $G \models_{k-1}c$ and equations (\ref{direct_improving_1}), (\ref{direct_improving_1_2}) and (\ref{direct_improving_2}) hold. 

Every occurrence of $C_k$ in $G$ that satisfies $\parcond(1,e,C')$ for any $C' \in \mathbf{G}$ still satisfies $\parcond(1,e,C')$ in $H$. 
\begin{equation}\label{direct_improving_1}
\begin{split}
	\forall p: C_{k} \inj G\Big( \bigwedge_{C' \in \mathbf{G}}\big(& p \models \parcond(1,e,C') \wedge \track_t \circ p \text{ is total}\big) \\ &\implies  \track_t \circ p \models \parcond(1,e,C') \Big) 
\end{split}
\end{equation}
Every new inserted occurrence of $C_k$ by $t$ satisfies $\parcond(1,e,C')$ for all $C' \in \mathbf{G}$.
\begin{equation}\label{direct_improving_1_2}
\begin{split}
\forall p': C_{k} \inj H \Big(\neg \exists p : C_k \inj G(p' = \track_t \circ p) \implies \big(\bigwedge_{C' \in \mathbf{G}} p' \models \parcond(1,e,C')\big)\Big) 
\end{split}
\end{equation}
At least one occurrence of $C_k$ in $G$ that does not satisfy $\parcond(1,e,C')$, for any $C' \in \mathbf{G}$, either has been destroyed by $t$ or satisfies $\parcond(1,e,C')$ in $H$. 
\begin{equation}\label{direct_improving_2}
\begin{split}
\exists p: C_k \inj G \Big(\bigvee_{C' \in \mathbf{G}}\big( p \not \models \parcond(1,e,C') &\wedge \track_t \circ p \text{ is not total } \\&\vee(\track_t \circ p \text{is total } \wedge \track_t \circ p  \models \parcond(1,e,C'))\big) \Big) 
\end{split}
\end{equation}

\end{definition}

\begin{lemma}
Let a graph $G$, a constraint $c$ and a direct minimal improving transformation $t: G \Rightarrow_{r,m} H$ w.r.t. $c$ be given. 
Then, $t$ is also a minimal improving transformation. 
\end{lemma}

\begin{proof}

Let $G$ be a graph with $k = \cmax$ and $G \models \parcond(k,c,C)$ with $\parcond(k,c,C)\in \mathcal{P}_c^G$
Let $d$ be the subcondition of $c$ at layer $k+1$.

\begin{enumerate}
\item \label{proof_direct_minimal_4} We show that equations (\ref{direct_improving_1}) and (\ref{direct_improving_1_2}) imply that $\nvc(k,H) \leq \nvc(k,G)$. 
Assume that $\nvc(k,H) > \nvc(k,G)$. Therefore, a morphism $p: C_k \inj G$ with $p \not \models \parcond(1,d,C')$ for any $C' \in \mathcal{U}(C,C_{k+1})$ exists, such that either \ref{proof_direct_minimal_1} or \ref{proof_direct_minimal_2} is satisfied.
\begin{enumerate}
	\item \label{proof_direct_minimal_1}
		There does exist a morphism $q': C_k \inj G$ with $q' \models \parcond(1,d,C')$ and $p = \track_t \circ q'$. 

	\item \label{proof_direct_minimal_2}
		There does not exist a morphism $q : C_k \inj G$, such that $p = \track_t \circ q$. 		
\end{enumerate}
This is a contradiction, if \ref{proof_direct_minimal_1} is satisfied, $q'$ does not satisfy equation (\ref{direct_improving_1}) and if \ref{proof_direct_minimal_2} is satisfied $q$ does not satisfy equation (\ref{direct_improving_1_2}).

\item \label{proof_direct_minimal_3} Since (\ref{direct_improving_2}) is satisfied, a morphism $p:C_k \inj G$ with $p \not \models \parcond(1,d,C')$, such that either $\track \circ p$ is total and $p  \models \parcond(1,d,C')$ or $\track \circ p$ is not total exists, for any $C' \in \mathbf{G}$.
In both cases the following holds 
\begin{equation*}
\begin{split}
	p &\in \{q \mid q:C_k \inj G \wedge q \not \models \parcond(1,e,C')\} \wedge \\
	\track \circ p &\notin \{q \mid q:C_{k+2} \inj H \wedge q \not \models \parcond(1,e,C')\}.
\end{split}
\end{equation*}
With that and \ref{proof_direct_minimal_4} it follows that
$$|\{q \mid q:C_k \inj G \wedge q \not \models \parcond(1,e,C')\}| < |\{q \mid q:C_{k+2} \inj H \wedge q \not \models \parcond(1,e,C')\}|.$$


\end{enumerate}
With \ref{proof_direct_minimal_4} and \ref{proof_direct_minimal_3} follows that $\nvc(k,G) < \nvc(k,G)$ and therefore $t$ is a minimal improving transformation. 
\end{proof}
%\begin{proof}
%\begin{enumerate}
%	Since $G \not \models c$, with lemma \ref{lemma_lay_sat} follows that the subcondition of $c$ at layer $k$ has to be existentially bound. 
%	\item The subcondition of $c$ at layer $j$ is existentially bound:
%	With lemma \ref{lem_ex_lower} follows that $H \models_{k+2} c$.
%	Let $d = \exists(a_{k+2}: C_{k+1} \inj C_{k+2}, e)$ be the subcondition of $c$ with layer $k+2$.
%	The set $\mathcal{C}_{k+2,H,\true}$ contains every graph $C'$ with $C_{k+1} \subseteq C' \subseteq C_{k+2}$. 
%	Since $G \not \models_{k+2} c$, $C_{k+2} \not \in \mathcal{C}_{k+2,G,\true}$ and for every $C'' \in \mathcal{B}_{k+2,G, \true}$ it holds that $C'' \in \mathcal{C}_{k+2,H,\true}$ and $C'' \subset C_{k+2} \in \mathcal{C}_{k+2,H, \true}$.
%	
%	\item The subcondition of $c$ at layer $j$ is universally bound:
%	With lemmas \ref{lemma_help_lay_sat} and \ref{lemma_lay_sat}, $H \models_{k+2} c$ follows and $1.$ can be applied. 
%\end{enumerate}
%\end{proof}
%\begin{definition}[\textbf{minimal consistency improving}]
%Given a rule $r$, a constraint $c$ and a graph $G$ with $G \models_k c$.
%A transformation $G \xRightarrow{r,m} H$ is called \emph{minimal consistency improving} if $H \models_j c$ and $k < j$. 
%The rule $r$ is called \emph{minimal consistency improving} if all of its applications are. 
%
%\end{definition}

