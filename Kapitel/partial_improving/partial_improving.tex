\subsection{Consistency Increasing and Maintaining Transformations and Rules}

With the results above, we are now ready to define the notions of \emph{consistency increasement} and \emph{maintainment}, with increasement being a
special case of maintainment. A transformation $t$ is considered as consistency maintaining if the consistency is not decreased, whereas $t$ is considered as 
consistency increasing if the consistency has been increased.

These notions are designed to only detect transformations that maintain 
(or increase) the consistency of the first two unsatisfied layer of a constraint $c$. 
That means, given a graph $G$ and a constraint $c$, let $k = \kmax +1$ and  $\scond{k}{c} := \forall(a_{k}: C_{k} \inj C_{k+1}, \exists(a_{k+1}: C_{k+1} \inj C_{k+2},d))$.  
A transformation $ t:G \Longrightarrow H$ is considered as  consistency maintaining  if $\maxk{c}{G} \leq \maxk{c}{H}$, i.e. the satisfaction up to layer is not decreased, and at least the same amount of increasing insertions or deletions have been performed than decreasing ones. 
An increasing deletion is the deletion of an occurrence of $C_{k+1}$ that does not satisfy $\exists(a_{k+1}: C_{k+1} \inj C_{k+2},\true)$, an increasing insertion is the insertion of elements of $C_{k+2}$, such that for at least one occurrence $p$ of $C_{k+1}$ it holds that $p \not \models \exists(a^r_{k+1}: C_{k+1} \inj C',\true)$ and $\track_t \circ p \models \exists(a^r_{k+1}: C_{k+1} \inj C',\true)$ for a graph $C' \in \ig{C_{k+1}}{ C_{k+2}}$. 
A decreasing insertion is the creation of an occurrence of $C_{k+1}$ not satisfying $\exists(a_{k+1}: C_{k+1} \inj C_{k+2},\true)$ and a decreasing deletion is the deletion of elements of $C_{k+2}$ such that for an occurrence $p$ of $C_{k+1}$ with $p \models \exists(a^r_{k+1}: C_{k+1} \inj C',\true)$ it holds that $\track_t \circ p \not \models \exists(a^r_{k+1}: C_{k+1} \inj C',\true)$ for a graph $C' \in \ig{C_{k+1}}{ C_{k+2}}$.
If $\maxk{c}{G} < \maxk{c}{H}$ or the number of increasing insertions and 
deletions is greater than the number of decreasing ones, $t$ is considered as 
consistency increasing.

To evaluate this, we define the \emph{number of violations}.
Intuitively, for all occurrences $p$ of $C_{k+1}$ the number of graphs $C' \in \ig{C_{k+1}}{C_{k+2}}$ with $p \not \models \exists C'$ is add up and it can be determined whether more increasing or decreasing actions have been performed by a transformation. 

Note, that the number of violations is defined for each layer of the constraint, but only for the first unsatisfied layer the sum is calculated as described above.
For all layers $k$ with $k \leq \kmax$ it is set to $0$ and for all  layers $k$ with $k > \kmax + 1$ it is set to $\infty$. 
Through this, a transformation $t: G \Longrightarrow H$ that increases the satisfaction up to layer can easily detected since the number of violations in $H$ at layer $\maxc{G} + 1$ will be set to $0$. 

\begin{definition}[\textbf{number of violations}]\label{def:num_violations}
Let a graph $G$ and a constraint $c$ in UANF be given. Let $e = \scond{\kmax + 2}{c}$.
The \emph{number of violations $\nv{j}{G}$ at layer $0 \leq j < \nlvl(c)$ in $G$} is defined as:

\begin{equation*}
	\nv{j}{G} := \begin{cases}
					0 & \text{if $j < \kmax +1$} \\
					\sum_{C' \in \ig{C_{j+1}}{C_{j+2}}} |\{q \mid q:C_{j+1} \inj 
					G \wedge q \not \models \ic{0}{e}{C'}\}| &  \text{if $e \neq 
					\false$ and $j = \kmax+1$} \\
					|\{q \mid q:C_{j+1} \inj G\}| & \text{if $e = \false$ and 
					$j = \kmax + 1$} \\
					\infty &\text{if $j > \kmax + 1$}
				 \end{cases}
\end{equation*}
\end{definition}
Note that the second and third case of Definition \ref{def:num_violations} 
only apply if $G \not \models c$ and $\scond{\kmax}{c}$ is existentially bound.
Therefore $e$ is also existentially bound or equal to $\false$, if $c$ ends with
$\forall(a_{\nlvl(c) -1}: C_{\nlvl(c) -1} \inj C_{\nlvl(c)}, \false)$.
Via the number of violations, we now define \emph{consistency maintaining} and 
\emph{increasing} transformations and rules, by checking whether the number of violations has not been increased, or in case of consistency increasing, has not been decreased for any layer of the constraint.

\begin{definition}[\textbf{consistency maintaining and increasing transformations and rules}]
	Let a graph $G$, a rule $\rho$ and a constraint $c$ in UANF be given. 
	A transformation $t: G \Longrightarrow_{\rho,m} H$ is called \emph{consistency 
	maintaining w.r.t. $c$}, if  $$\nv{k}{ H} \leq \nv{k}{ G} $$
	for any $0 \leq k < \nlvl(c)$.
	The transformation $t$ is called \emph{consistency increasing w.r.t. $c$}
	this inequality is strict.
	A rule $\rho$ is called \emph{consistency maintaining/increasing  w.r.t $c$}, 
	if all of its transformations are. 
	
\end{definition}

Note that if $G \models c$ there does not exist a consistency increasing transformation $G \Longrightarrow H$ w.r.t $c$, since $\nv{j}{G} = 0$ for all $0 \leq j < \nlvl(c)$.
Also, no plain rule $\rho$ is  consistency increasing w.r.t $c$, since a graph $G$ satisfying $c$, such that a transformation $t: G \Longrightarrow_{\rho,m} H$ exists can always be constructed. 
Therefore, each consistency increasing rule has to be equipped with at least one application condition.

As mentioned above, a transformation should be detected as consistency increasing if it increases the satisfaction up to layer, which is shown by the following theorem. 

\begin{theorem}
	Let a graph $G$, a rule $\rho$ and a constraint $c$ in UANF with $G \not 
	\models c$ be given.	A transformation $t: G \Longrightarrow_{\rho,m} H$ is 
	consistency increasing w.r.t. $c$ if $$\maxk{c}{G} < \maxk{c}{H}$$.	
\end{theorem}

\begin{proof}
No $\ell >\maxk{c}{G}$ with $G\models_{\ell} c$ exists. Hence, $\nv{\maxk{c}{G} + 1}{G} > 0$ and $\nv{\maxk{c}{G} + 1}{G} \neq \infty$. 
Since $\maxk{c}{H} > \maxk{c}{G}$, $\nv{\maxk{c}{G}+1}{H} = 0$ and it follows immediately that $t$ is consistency increasing w.r.t. $c$. 
\end{proof}

Since no consistency increasing transformation originating in consistent graphs exist, there do not exist infinite long sequences of consistency increasing transformations.

\begin{theorem}
Let a constraint $c$ in UANF be given. 
Every sequence of consistency increasing transformations w.r.t $c$ is finite.

\end{theorem}

\begin{proof}
Let $G_0$ be a graph and 
$$G_0 \Longrightarrow_{\rho0,m_0} G_1 \Longrightarrow_{\rho_1,m_1} G_2 \Longrightarrow_{\rho_3} \ldots$$
be a sequence of consistency increasing transformations w.r.t $c$.
We assume that $\maxk{c}{G_0} < \nlvl(c)$, otherwise $\nv{j}{G_0} = 0$ for all $0 \leq j < \nlvl(c)$ and no consistency increasing transformation $G_0 \Longrightarrow H$ w.r.t. $c$ exists. 

We show that after at most $j := \nv{\maxk{c}{G_0}+1}{G_0}$ transformations $G_j \models_{\maxk{c}{G_0} + 2} c$. 
Note that $j$ has to be finite, since $G_0$ contains only a finite number of occurrences of $C_{j+1}$. Since each transformation is consistency increasing 
w.r.t. $c$ it holds that $\nv{\maxk{c}{G_{i}}+1}{G_{i+1}} \leq \nv{\maxk{c}{G_i}+1}{G_{i}}-1$ after each transformation.
Therefore, after at most $j$ transformations, $\nv{\maxk{c}{G_0}+1}{G_{j}} \leq \nv{\maxk{c}{G_0}+1}{G_{0}}-j = 0$ and $G_j \models_{\maxk{c}{G_0}+ 2} c$.
By iteratively applying this, it follows that after a finite number of transformations a graph $G_k$ with $G_k \models c$ has to exist. 
Since no consistency increasing transformation $G_k \Longrightarrow_{\rho_k, m_k} G_{k+1}$ exists, the sequence has to be finite.  
\end{proof}

\subsection{Direct Consistency Maintaining and Increasing Transformations}
Let a constraint $c$ in UANF and graphs $G$ with $G \not \models c$ and 
$H$ with $H \models c$ be given. The transformation $t:G \Longrightarrow_{\rho, \id_G} H$ via the rule $\rho = \rle{G}{l}{\emptyset}{r}{H}$ is a consistency 
increasing transformation. 
Therefore, the notions of consistency increase- and maintainment, a similar 
example for a consistency maintaining transformation can easily be constructed, does allow insertions or deletions that are unnecessary in order to increase or 
maintain consistency.
That is, the deletion of occurrences of existentially bound graphs, the deletions of occurrences $p: C_k \inj G$ of universally bound graphs $C_k$ such that 
$p \models \exists(a_k: C_k \inj C_{k+1}, \true)$ or the insertion of occurrences of universally bound graphs and the insertion of occurrences $p$ of intermediate graphs $C'\in \ig{C_{k-1}}{C_{k}}$ with $C_k$ being existentially bound, such that each occurrence $q$ of $C_{k-1}$ with $q = p \circ a^r_{k-1}$ already satisfied $\exists(a^r_{k-1}: C_{k-1} \inj C_k, \true)$. 

\emph{Direct consistency increasing} and \emph{maintaining} transformations are more restricted, in the sense that these unnecessary deletions and insertions are leading to a transformation not being direct consistency increasing or maintaining, respectively. 
The presence of these unnecessary actions can be checked via second-order logic formulas.
Additionally, it is secured that no new violations are introduced since these can always be considered as a unnecessary insertion or deletion.
With that, the removal of one violation is sufficient to state that the transformation is (direct) consistency increasing, which can also be checked via a second order logic formula.
We start by introducing \emph{direct consistency maintaining} transformations.
Intuitively, it is checked by the first two formulas, that no new violations are introduced. 
The third formula secures that at least one violation has been removed and the last formulas secures that the satisfaction up to layer is not decreased. 

\begin{definition}[\textbf{direct consistency maintaining transformations}]
	Let $G$ be a graph $\rho$ a rule and $c$ a constraint in UANF.
	A transformation $t: G\Longleftarrow_{\rho,m}H$ is called \emph{direct
	consistency maintaining w.r.t. $c$} if the following equations hold.
\end{definition}
 
\begin{definition}[\textbf{direct consistency increasing}]\label{def_direct_improving}
Let $G$ be a graph, $\rho$ a plain rule, $c$ a constraint in UANF and  $\scond{\kmax + 1}{c} = \forall(a_{k-1}: C_{k-1} \inj C_{k}, e)$. 
A transformation $t: G \Longrightarrow_{\rho,m} H$ is called \emph{direct consistency increasing w.r.t $c$} if the following equations hold.
Let $$ \mathbf{G} = \begin{cases}
					\ig{C_{k}}{C_{k+1}} &\text{if $e \neq  \false$} \\
					\{C_{k}\} &\text{otherwise}
					\end{cases}$$ 

Every occurrence of $C_k$ in $G$ that satisfies $\ic{0}{e}{C'}$ for any $C' \in \mathbf{G}$ still satisfies $\ic{0}{e}{C'}$ in $H$. 
\begin{equation}\label{direct_improving_1}
\begin{split}
	\forall p: C_{k} \inj G\Big( \bigwedge_{C' \in \mathbf{G}}\big( p \models \ic{0}{e}{C'} \wedge \track_t \circ p \text{ is total}\big) \implies  \track_t \circ p \models \ic{0}{e}{C'} \Big) 
\end{split}
\end{equation}
Let $C' = C_{k+1}$ if $e \neq \false$ and $C' = C_k$ otherwise.
Every new inserted occurrence of $C_k$ by $t$ satisfies $\ic{0}{e}{C'}$.
\begin{equation}\label{direct_improving_1_2}
\begin{split}
\forall p': C_{k} \inj H \big(\neg \exists p : C_k \inj G(p' = \track_t \circ p) \implies \ p' \models \ic{0}{e}{C'}\big)
\end{split}
\end{equation}
At least one occurrence of $C_k$ in $G$ that does not satisfy $\ic{0}{e}{C'}$, for any $C' \in \mathbf{G}$, either has been removed or satisfies $\ic{0}{e}{C'}$ in $H$.
\begin{equation}\label{direct_improving_2}
\begin{split}
\exists p: C_k \inj G \Big(\bigvee_{C' \in \mathbf{G}}\big( p \not \models \ic{0}{e}{C'} \wedge (\track_t \circ p \text{ is not total } \vee \track_t \circ p  \models \ic{0}{e}{C'})\big) \Big) 
\end{split}
\end{equation}

No occurrence of a universally bound graph $C_j$ with $j < \kmax$ gets inserted. 
\begin{equation}\label{direct_improving_3}
\bigwedge_{\substack{i < \kmax \\ i \textit{ even}}} \forall p: C_i \inj H ( \exists p': C_i \inj G (p = \track_t \circ p'))
\end{equation}
No occurrence of an existentially bound graph $C_j$ with $j \leq \kmax$ gets deleted. 
\begin{equation}\label{direct_improving_4}
\bigwedge_{\substack{i \leq \kmax \\ i \textit{ odd}}} \forall p: C_i \inj G( \track_t \circ p \textit{ is total})
\end{equation}

\end{definition}

Note that (\ref{direct_improving_3}) and (\ref{direct_improving_4}) not only secure that the satisfaction up to layer does not decrease, as shown in the following lemma, but also prevent further unnecessary insertions and deletions, since the insertion of a universally and the deletion of a existentially bound graph will never lead to an increase of consistency. 

\begin{lemma}\label{lemma_consistent}
	Let a transformation $t: G \Longrightarrow H$ and a constraint $c$ in UANF be given, such that (\ref{direct_improving_3}) and (\ref{direct_improving_4}) of definition \ref{def_direct_improving} are satisfied. Then,
	$$H \models_{\maxk{c}{G}} c.$$
\end{lemma}
\begin{proof}
Assume that $H \not \models_{\maxk{c}{G}}c$. 
Then, either a new occurrence of an universally bound graph $C_k$ with $k < \maxk{c}{G}$ has been inserted or an occurrence of an existentially bound graph $C_k$ with $k \leq \maxk{c}{G}$ has been destroyed. 
Therefore, the following holds:
$$ \exists p: C_i \inj H(\neg \exists p': C_i \inj G(p = \track_t \circ p')) \vee \exists p:C_j \inj G(\track_t \circ p \textit{ is not total})$$
with $i, j \leq \maxk{c}{G}$, $i$ being even and $j$ being odd. 
It follows immediately that either (\ref{direct_improving_3}) or (\ref{direct_improving_4}) is not satisfied. 
This is a contradiction. 
\end{proof}

Now, we will show the already indicated relationship between direct consistency increasing and consistency increasing, namely that a direct consistency increasing transformation is also consistency increasing. 
Counterexamples, that the inversion of the implication does not hold can easily be constructed, showing that these notions are not identical, but related. 

\begin{lemma}\label{lemma:direct_implies_normal}
Let a graph $G$, a constraint $c$ in UANF, a rule $\rho$ and a direct consistency increasing transformation $t: G \Longrightarrow_{\rho,m} H$ w.r.t. $c$ be given. 
Then, $t$ is also a consistency increasing transformation. 
\end{lemma}

\begin{proof}

Let $k = \maxk{c}{G} +2$ and $d = \scond{k}{c}$ with $d \neq \false$.
With lemma \ref{lemma_consistent} follows that $\maxk{c}{G} \leq \maxk{c}{H}$ and with that $\nv{k}{H} \neq \infty$. Let 


\begin{enumerate}
\item \label{proof_direct_minimal_4} We show that equations (\ref{direct_improving_1}) and (\ref{direct_improving_1_2}) imply that $\nv{k}{H} \leq \nv{k}{G}$. 
Assume that $\nv{k}{H} > \nv{k}{G}$. Therefore, a morphism $p: C_k \inj H$ with $p \not \models \ic{0}{d}{C'}$ for any $C' \in \ig{C_k}{C_{k+1}}$ exists, such that either \ref{proof_direct_minimal_1} or \ref{proof_direct_minimal_2} is satisfied.
\begin{enumerate}
	\item \label{proof_direct_minimal_1}
		There does exist a morphism $q': C_k \inj G$ with $q' \models \ic{0}{d}{C'}$ and $p = \track_t \circ q'$. 

	\item \label{proof_direct_minimal_2}
		There does not exist a morphism $q : C_k \inj G$ with $p = \track_t \circ q$. 		
\end{enumerate}
This is a contradiction, if \ref{proof_direct_minimal_1} is satisfied, $q'$ does not satisfy equation (\ref{direct_improving_1}) and if \ref{proof_direct_minimal_2} is satisfied $q$ does not satisfy equation (\ref{direct_improving_1_2}).
It follows that 
$$|\{q \mid q:C_k \inj G \wedge q \not \models \ic{0}{d}{C'}\}| \leq |\{q \mid q:C_{k} \inj H \wedge q \not \models \ic{0}{d}{C'}\}|$$
for all $C' \in \ig{C_k}{C_{k+1}}$.

\item \label{proof_direct_minimal_3} Since (\ref{direct_improving_2}) is satisfied,  a morphism $p:C_k \inj G$ with $p \not \models \ic{0}{d}{C'}$, such that either $\track \circ p$ is total and $\track_t \circ p  \models \ic{0}{d}{C'}$ or $\track \circ p$ is not total exists, for a graph $C' \in \ig{C_k}{C_{k+1}}$.
In both cases the following holds: 
\begin{equation*}
\begin{split}
	p &\in \{q \mid q:C_k \inj G \wedge q \not \models \ic{0}{d}{C'}\} \wedge \\
	\track \circ p &\notin \{q \mid q:C_{k} \inj H \wedge q \not \models \ic{0}{d}{C'}\}
\end{split}
\end{equation*}
With that and \ref{proof_direct_minimal_4}. it follows that
$$|\{q \mid q:C_k \inj G \wedge q \not \models \ic{0}{d}{C'}\}| < |\{q \mid q:C_{k} \inj H \wedge q \not \models \ic{0}{d}{C'}\}|.$$


\end{enumerate}
In total, $\nv{k}{G} < \nv{k}{G}$ and  $t$ is a consistency increasing transformation.
 
For the special case that $k = \nlvl(c)$ and $c$ ends with $\forall(a_k:C_{k-1} \inj C_k,\false)$ it can be shown in a similar way that  (\ref{direct_improving_1}) and (\ref{direct_improving_1_2}) imply that 
$$|\{q \mid q:C_k \inj G\}| \leq |\{q \mid q:C_k \inj H\}|$$
and that (\ref{direct_improving_2}) implies 
$$|\{q \mid q:C_k \inj G\}| < |\{q \mid q:C_k \inj H\}|.$$
It follows that $t$ is a consistency increasing transformation. 
\end{proof}


