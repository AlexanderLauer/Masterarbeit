\subsection{Consistency Increasing and Maintaining Transformations and Rules}

With the results above, we are now ready to define the notions of \emph{consistency increasement} and \emph{maintainment}, where increasement is a special case of maintainment. A transformation $t$ is considered as consistency maintaining if it does not decrease consistency, while $t$ is considered as 
consistency increasing if it increases the consistency.

These notions are designed to detect only transformations that maintain (or increase) the consistency of the first two unsatisfied layer of a constraint $c$. 
That means, given a graph $G$ and a constraint $c$,  a transformation $ t:G \Longrightarrow H$ is considered as  consistency maintaining if the largest satisfied layer has not decreased, i.e. if $\maxk{c}{G} \leq \maxk{c}{H}$, and at least as many increasing insertions or deletions have been made as decreasing ones.
An increasing deletion is the deletion of an occurrence of $C_{\maxk{c}{G} +2}$ that does not satisfy $\exists(C_{\maxk{c}{G}+3},\true)$, an increasing insertion is the insertion of elements, such that for at least one occurrence $p$ of $C_{\maxk{c}{G}+2}$ it holds that $p \not \models \exists(C',\true)$ and $\track_t \circ p \models \exists(C',\true)$ for an intermediate graph $C' \in \ig{C_{\maxk{c}{G}+2}}{ C_{\maxk{c}{G}+3}}$.
Decreasing insertions and deletions are the opposite of increasing ones. 
That is, a decreasing insertion is the insertion of an occurrence of $C_{\maxk{c}{G}+2}$ that does not satisfy $\exists(C_{\maxk{c}{G}+3}, \true)$ and a decreasing deletion is the deletion of elements such that for one occurrence $p$ of 
$C_{\maxk{c}{G}+2}$ with $p \models \exists(C', \true)$ it holds that 
$\track_t \circ p \not \models \exists(C', \true)$ for an intermediate graph
$C'\ig{C_{\maxk{c}{G}+2}}{ C_{\maxk{c}{G}+3}}$.
If $\maxk{c}{G} < \maxk{c}{H}$ or the number of increasing insertions and deletions is greater than the number of decreasing ones, $t$ is considered as consistency increasing.

To evaluate this, we define the \emph{number of violations}.
Intuitively, for all occurrences $p$ of $C_{\kmax+2}$ the number of graphs $C' \in \ig{C_{\kmax+2}}{C_{\kmax+3}}$ with $p \not \models \exists C'$ is added up, and by comparing these numbers for $G$ and $H$ it can be determined whether there have been more increasing insertions and deletions than decreasing ones.

The number of violations is defined for each layer of the constraint, but only for the first unsatisfied layer is the sum calculated as described above.
For all layers $k$ with $k \leq \kmax$ it is set to $0$ and for all layers $k$ with $k > \kmax + 1$ it is set to $\infty$. 
In this way, a transformation $t: G \Longrightarrow H$ that increases the largest satisfied layer can be easily detected, since the number of violations in $H$ at the layer $\kmax + 1$ will be set to $0$.

\begin{definition}[\textbf{number of violations}]\label{def:num_violations}
Given a graph $G$ and a constraint $c$ in UANF, and let $e = \scond{\kmax + 2}{c}$.
The \emph{number of violations $\nv{j}{G}$ at layer $-1 \leq j < \nlvl(c)$ in $G$} is defined as:

\begin{equation*}
	\nv{j}{G} := \begin{cases}
					0 & \text{if $j < \kmax +1$} \\
					\sum_{C' \in \ig{C_{j+1}}{C_{j+2}}} |\{q \mid q:C_{j+1} \inj 
					G \wedge q \not \models \ic{0}{e}{C'}\}| &  \text{if $e \neq 
					\false$ and $j = \kmax+1$} \\
					|\{q \mid q:C_{j+1} \inj G\}| & \text{if $e = \false$ and 
					$j = \kmax + 1$} \\
					\infty &\text{if $j > \kmax + 1$}
				 \end{cases}
\end{equation*}
\end{definition}
Note that the second and third cases of Definition \ref{def:num_violations} only occur if $G \not \models c$ and $\scond{\kmax}{c}$ is existentially bound. So $e$ is also existentially bound or equal to $\false$ if $c$ ends with $\forall(C_{\nlvl(c)}, \false)$ and $\kmax = \nlvl(c) -2$. Using the number of violations, we now define \emph{consistency maintaining} and \emph{increasing} transformations and rules by checking that the number of violations has not increased or, in the case of consistency increasing, has decreased.

\begin{definition}[\textbf{consistency maintaining and increasing transformations and rules}]
	Given a graph $G$, a constraint $c$ in UANF and a rule $\rho$.
	A transformation $t: G \Longrightarrow_{\rho,m} H$ is called \emph{consistency-maintaining w.r.t. $c$}, if  $$\nv{k}{ H} \leq \nv{k}{ G} $$
	for all $-1 \leq k < \nlvl(c)$.
	The transformation is called \emph{consistency-increasing w.r.t. $c$}
	if 
	$$\nv{k}{ H} < \nv{k}{ G} $$
	for all $-1 \leq k < \nlvl(c)$.
	A rule $\rho$ is called \emph{consistency maintaining or increasing  w.r.t $c$}, 
	if all of its transformations are. 
	
\end{definition}

Note that if $G \models c$, there is no consistency-increasing transformation $G \Longrightarrow H$ w.r.t $c$, since $\nv{j}{G} = 0$ for all $0 \leq j < \nlvl(c)$.
Also, no plain rule $\rho$ is consistency-increasing w.r.t $c$, since a graph $G$ satisfying $c$ such that a transformation $t: G \Longrightarrow_{\rho,m} H$ can always be constructed. 
Therefore, every consistency-increasing rule must have at least one application condition.

As mentioned above, a transformation is considered to be consistency-increasing if the largest satisfied layer is increasing. This property is already indirectly embedded in the definition of consistency-increasing transformations. 

\begin{theorem}
	Given a rule $\rho$ a constraint $c$ in UANF and a graph $G$ with $G \not 
	\models c$.
	A transformation $t: G \Longrightarrow_{\rho,m} H$ is 
	consistency-increasing w.r.t. $c$ if $$\maxk{c}{G} < \maxk{c}{H}.$$.	
\end{theorem}

\begin{proof}
No $\ell >\maxk{c}{G}$ with $G\models_{\ell} c$ exists. Hence, $\nv{\maxk{c}{G} + 1}{G} > 0$ and $\nv{\maxk{c}{G} + 1}{G} \neq \infty$. 
Since $\maxk{c}{H} > \maxk{c}{G}$, $\nv{\maxk{c}{G}+1}{H} = 0$, which immediately implies that $t$ is consistency increasing w.r.t. $c$.
\end{proof}

Since there are no consistency-increasing transformations starting from consistent graphs, there are no infinitely long sequences of consistency-increasing transformations.

\begin{theorem}
Let $c$ be a constraint in UANF. 
Every sequence of consistency-increasing transformations w.r.t. $c$ is finite.

\end{theorem}

\begin{proof}
Let 
$$G_0 \Longrightarrow_{\rho_0,m_0} G_1 \Longrightarrow_{\rho_1,m_1} G_2 \Longrightarrow_{\rho_2,m_2} \ldots$$
be a sequence of consistency-increasing transformations w.r.t. $c$.
We assume that $\maxk{c}{G_0} < \nlvl(c)-1$, otherwise $\nv{j}{G_0} = 0$ for all $0 \leq j < \nlvl(c)$ and there is no consistency-increasing transformation $G_0 \Longrightarrow H$ with respect to $c$.

We show that $G_j \models_{\maxk{c}{G_0} + 2} c$ holds after a maximum of $j := \nv{\maxk{c}{G_0}+1}{G_0}$ transformations. 
Note that $j$ must be finite, since $G_0$ contains only a finite number of occurrences of $C_{j+1}$. Since every transformation is consistency-increasing 
w.r.t. $C$, it follows that $\nv{\maxk{c}{G_{i}}+1}{G_{i+1}} \leq \nv{\maxk{c}{G_i}+1}{G_{i}}-1$ after each transformation.
Therefore, after at most $j$ transformations, $\nv{\maxk{c}{G_0}+1}{G_{j}} \leq \nv{\maxk{c}{G_0}+1}{G_{0}}-j = 0$ and thus $G_j \models_{\maxk{c}{G_0}+ 2} c$.
If this is applied iteratively, it follows that after a finite number of transformations there must exist a graph $G_k$ with $G_k \models c$. 
Since there is no consistency increasing transformation $G_k \Longrightarrow_{\rho_k, m_k} G_{k+1}$, the sequence must be finite.
\end{proof}

\subsection{Direct Consistency Maintaining and Increasing Transformations}
We will now introduce stricter versions of consistency-increasing and consistency-~maintaining transformations, called \emph{direct consistency-maintaining} and \emph{direct consistency-increasing} transformations.
These can also be considered as consistency-maintaining and consistency-increasing  transformations, which do not perform any unnecessary insertions and deletions.
For example, given a constraint $c$ in UANF and graphs $G$ with $G \not \models c$ and $H$ with $H \models c$, the transformation $t:G \Longrightarrow_{\rho, \id_G} H$ via the rule $\rho = \rle{G}{l}{\emptyset}{r}{H}$ is a consistency-increasing transformation. 
Therefore, the notions of consistency-increasing and consistency-maintaining - a similar example of a consistency-maintaining transformation can be easily constructed - allow insertions or deletions that are unnecessary to increase or maintain consistency. 
That is, deleting occurrences of existentially bound graphs, deleting occurrences $p: C_k \inj G$ of universally bound graphs $C_k$ satisfying $\exists(C_{k+1}, \true)$ or inserting occurrences of universally bound graphs and inserting occurrences $p$ of intermediate graphs $C'\in \ig{C_{k-1}}{C_{k}}$ with $C_k$ being existentially bound, such that each occurrence $q$ of $C_{k-1}$ with $q = p \circ a^r_{k-1}$ already satisfies $\exists(C', \true)$.

\emph{Direct consistency-increasing} and \emph{maintaining} transformations are more restricted, in the sense that these unnecessary deletions and insertions cause a transformation not to be direct consistency-increasing or direct consistency maintaining, respectively. 
In addition, we can use second-order logic formulas to check whether a transformation is direct consistency-maintaining.
Furthermore, it is ensured that no new violations are introduced, since these can always be considered as unnecessary insertions or deletions.
Thus, the removal of one violation is sufficient to state that the transformation is (direct) consistency-increasing, which can also be checked using a second-order logic formula.
We start by introducing \emph{direct consistency-maintaining}. Its definition consists of the following conditions

\begin{enumerate}
	\item 
		\emph{Deleting condition:} 
			This condition ensures that no new violations are introduced 
by deleting intermediate graphs $C' \in \ig{C_{\kmax +2}}{C_{\kmax +3}}$. This leads to the insertion of new violations only if an occurrence of $C_{\kmax +2}$ which satisfies $\exists(C',\true)$ in the originating graph does not satisfy $\exists(C',\true)$ in the derived graph of the transformation. 
Therefore, this condition checks that this case does not occur.
			 
	\item	
		\emph{Inserting condition:}
		This condition ensures that no new violations are introduced by inserting an occurrence of $C_{\kmax +2}$. Again, this only causes a new violation to be inserted if that occurrence does not satisfy $\exists(C_{\kmax +3}, \true)$. The condition checks that this is not the case.
	\item
		\emph{Universally condition:}
			This condition ensures that the largest satisfied layer is not reduced by inserting a universally bound graph $C_j$. This can only happen if $j \leq \kmax$, and the condition checks that no occurrences of such universally bound graphs are inserted.
			 
	\item
		\emph{Existentially condition:}
			This condition ensures that the largest satisfied layer is not reduced by deleting an existentially bound graph $C_j$. Again, this can only happen if $j \leq \kmax$. The condition checks that no occurrences of such existentially bound graphs are deleted.
\end{enumerate}

The deleting and inserting conditions ensure that the number of violations is not increased, and the universally and existentially conditions ensure that the largest satisfied layer is not reduced.
Of course, the violation of the universally and existentially conditions does not necessarily lead to a decrease of the largest satisfied layer, but it can also be considered as an unnecessary insertion or deletion. 

Since a condition $c$ in UANF is also allowed to end with $\forall(C_{\nlvl(c)},\false)$ the deleting and inserting conditions contain case discrimination. 
If the constraint $c$ ends with $\forall(C_{\nlvl(c)},\false)$ and $\kmax = \nlvl(c) -2$, there is no graph $C_{\kmax +3}$ and thus no intermediate graphs. Therefore, the deleting condition is always satisfied and 
we will set it to $\true$.
The inserting condition will checks that no new occurrences of  $C_{\kmax +3}$ are introduced at all.
 

\begin{definition}[\textbf{direct consistency maintaining transformations}]
	Given a graph $G$, a rule $\rho$ and a constraint $c$ in UANF.
	If $G \models c$, a transformation $t: G\Longrightarrow_{\rho,m}H$ is
	called \emph{direct consistency-maintaining w.r.t. $c$} if $H \models c$.
	Otherwise, if $G \not \models c$, let $e = \scond{k}{c}$ and $\kmax = \maxk{c}{G}$.
	A transformation $t: G\Longrightarrow_{\rho,m}H$ is called \emph{direct
	consistency maintaining w.r.t. $c$} if the following conditions are 
	satisfied.
	
	\begin{enumerate}
		\item
			\emph{Deleting condition:}
			If $e \neq \false$, then each occurrence of $C_{\kmax+2}$ in $G$ 
			which satisfies $\ic{0}{e}{C'}$ for any 	$C' \in \ig{C_{\kmax+2}}
			{C_{\kmax+3}}$ still satisfies $\ic{0}{e}{C'}$ in $H$:
			\begin{equation}\label{direct_improving_1}
				\begin{split}
					\forall p: C_{k} \inj G\Big( \bigwedge_{C' \in \ig{C_{\kmax+2}}
					{C_{\kmax+3}}}
					\big( &p \models \ic{0}{e}{C'} \wedge \track_t \circ p \text{ 	
					is total}\big)\\&\implies  \track_t \circ p \models \ic{0}{e}
					{C'} \Big) 
				\end{split}
			\end{equation}
			Otherwise, if $e = \false$, the deleting condition is satisfied.
		\item
			\emph{Inserting condition:}
			Let $d = \ic{0}{e}{C_{\kmax+3}}$ if $e \neq \false$ and $d = \false$ 		
			otherwise. Each newly inserted occurrence of $C_{\kmax+2}$ satisfies 
			$d$.
			\begin{equation}\label{direct_improving_1_2}
				\begin{split}
					\forall p': C_{\kmax+2} \inj H \big(\neg \exists p : C_{\kmax+2} \inj G(p' 
					= \track_t \circ p) \implies \ p' \models d\big)
				\end{split}
			\end{equation}
		\item
		\emph{Universally condition:}
			No occurrence of a universally bound graph $C_j$ with $j \leq \kmax$ 
			is inserted. 
			\begin{equation}\label{direct_improving_3}
				\bigwedge_{\substack{i < \kmax \\ C_i \textit{ universally}}} 
				\forall p: 
				C_i \inj H ( \exists p': C_i \inj G (p = \track_t \circ p'))
			\end{equation}
		\item
			\emph{Existentially condition:}
			No occurrence of an existentially bound graph $C_j$ with $j \leq 
			\kmax+1$ is deleted. 
			\begin{equation}\label{direct_improving_4}
				\bigwedge_{\substack{i \leq \kmax \\ C_i \textit{ 
				existentially}}} \forall 
				p: C_i \inj G( \track_t \circ p \textit{ is total})
			\end{equation}
	\end{enumerate}
\end{definition}
 
Before continuing with the definition of direct consistency increasing, let us first show that every direct consistency maintaining transformation is indeed consistency maintaining.  To do this, we first show that satisfying the universally and existentially conditions guarantees that the satisfaction at layer has not been decreased.

\begin{lemma}\label{lemma_consistent}
	Given a transformation $t: G \Longrightarrow H$ and a constraint $c$ in UANF 
	such that the universally and existentially condition is satisfied. Then
	$$H \models_{\maxk{c}{G}} c.$$
\end{lemma}
\begin{proof}
Let us assume that $H \not \models_{\maxk{c}{G}}c$. 
Then either a new occurrence of a universally bound graph $C_i$ with $i < \maxk{c}{G}$ has been inserted, or an occurrence of an existentially bound graph $C_j$ with $j \leq \maxk{c}{G}$ has been destroyed.
Therefore, the following applies:
$$ \exists p: C_i \inj H(\neg \exists p': C_i \inj G(p = \track_t \circ p')) \vee \exists p:C_j \inj G(\track_t \circ p \textit{ is not total})$$
where $i, j \leq \maxk{c}{G}$, $i$ is even and $j$ is odd, i.e. $C_i$ is universally and $C_j$ is existentially bound.
It follows immediately that either the universally  or the existentially condition is not satisfied. 
This is a contradiction. 
\end{proof}

With this we are now going to show that a direct consistency-maintaining transformation is also a consistency-maintaining transformation.

\begin{theorem}\label{thm:direct_maintaining_to_maintaining}
	Given a graph $G$, a constraint $c$ in UANF, a rule $\rho$ and a direct consistency-maintaining transformation $t: G \Longrightarrow_{\rho,m} H$ w.r.t. $c$. 
Then, $t$ is also a consistency-maintaining transformation.
\end{theorem}
\begin{proof}
	With Lemma \ref{lemma_consistent} follows that $\maxk{c}{G} \leq \maxk{c}{H}$ 
	and thus $\nv{\maxk{c}{G}+1}{H} \neq \infty$.
	It remains to show that $\nv{k}{H} \leq \nv{k}{H}$ for all $0 \leq k < 
	\nlvl(c)$.
	In particular, we only need to show that $\nv{\maxk{c}{G}+1}{H} \leq 
	\nv{\maxk{c}{G}+1}{G}$ since for all $-1 \leq j < \maxk{c}{G}+1$ it holds 
	that $\nv{j}{H} = \nv{j}{G} = 0$. Also, since $\nv{j}{G} = \infty$ for all $\maxk{c}{G} + 1 < j < \nlvl(c)$, it follows that $\nv{j}{H} \leq \nv{j}{G}$.
	
	Let  $\kmax = \maxk{c}{G}$ and $d = \scond{\kmax +2}{c}$.
	We show that the deleting and insertion conditions  imply 
	that $\nv{\kmax +1}{H} \leq \nv{\kmax +1}{G}$. 
	
	Let us assume that $\nv{\kmax +1}{H} > \nv{\kmax +1}{G}$. 
	Therefore, there is a morphism $p: C_{\kmax +2} \inj H$ 
	with $p \not \models \ic{0}{d}{C'}$ for any $C' \in \ig{C_{\kmax +2}}
	{C_{\kmax +3}}$ 
	such that either \ref{proof_direct_minimal_1} or 
	\ref{proof_direct_minimal_2} is satisfied. Note that this is only the case if 
	$d \neq \false$. Otherwise there must be a morphism $p$ which satisfies 
	\ref{proof_direct_minimal_2}.
	\begin{enumerate}
		\item \label{proof_direct_minimal_1}
			There is a morphism $q': C_{\kmax +2} \inj G$ with $q' \models \ic{0}
			{d}{C'}$ and $p = \track_t \circ q'$. 

		\item \label{proof_direct_minimal_2}
			There is a morphism $q : C_{\kmax +2} \inj G$ with $p = \track_t 
			\circ q$. 		
	\end{enumerate}
	This is a contradiction, if \ref{proof_direct_minimal_1} is satisfied, $q'$ 
	does not satisfy equation (\ref{direct_improving_1}) and the deletion condition is not satisfied. If 
	(\ref{proof_direct_minimal_2}) is satisfied $p$ does not satisfy equation 
	(\ref{direct_improving_1_2}) since $p$ only satisfies  $\ic{0}{d}{C_{k+2}}$ 
	if $p$ satisfies $\ic{0}{d}{C'}$ for all $C' \in \ig{C_{k+1}}{C_{k+2}}$.
	In this case, the inserting condition is not satisfied. It follows that 
	$$\nv{k}{H} \leq \nv{k}{G}$$ holds and $t$ is a consistency-maintaining transformation.
	
\end{proof}

Let us now introduce the notion of \emph{direct consistency-increasing} transformations. Similar to the definition of consistency-maintaining and consistency-increasing transformations, again this notion is based on the notion of direct consistency-maintaining transformations, in the sense that a direct consistency-increasing transformation is also a direct consistency-maintaining transformation. Since a direct consistency-maintaining transformation $t$ does not introduce any new violations, it is sufficient that $t$ deletes at least one violation to say that $t$ is direct consistency-increasing.

We need to check that at least one violation is removed. 
Again, we need case discrimination if the constraint ends with $\forall(C_{\nlvl(c)}, \false)$ and $\kmax = \nlvl(c) -2$. 
So we introduce two new conditions, one for the general case and one for this special case.

\begin{enumerate}
	\item	
		\emph{General increasing condition}:
			This condition is satisfied if either an occurrence of $C_{\kmax +2}$ that does not satisfy $\exists(\kmax+3, \true)$ is deleted, or an occurrence of $C_{\kmax +2}$ which does not satisfy $\exists(C', \true)$ in the first graph satisfies $\exists(C', \true)$ in the second graph of the transformation with
$C' \in \ig{C_{\kmax +2}}{C_{\kmax +3}}$. Both cases result in the removal of a violation.
	\item 
		\emph{Special increasing condition}:
		This condition is satisfied if an occurrence of $C_{\kmax+2}$ is 
		removed. 
		In the special case that this is the only way to remove a violation.
\end{enumerate}

\begin{definition}[\textbf{direct consistency increasing}]\label{def_direct_improving}
	Given a constraint $c$ in UANF, a rule $\rho$ and a graph $G$ with 
	$G \not \models c$ and let $e = \scond{\kmax +2}{c}$.  
	
	A transformation $t: G \Longrightarrow_{\rho,m} H$ is called \emph{direct consistency-increasing w.r.t. $c$} if it is direct consistency-maintaining w.r.t. $c$ and either the special increasing condition is satisfied if  $\scond{\nlvl(c)-1}{c} = \forall(C_{\nlvl(c)},\false)$ and $\kmax = \nlvl(c) -2$ or the general increasing condition is satisfied otherwise.
	
	\begin{enumerate}
		\item
			General increasing condition:
			\begin{equation}\label{direct_improving_2}
			\begin{split}
				\exists p: C_{\kmax+2} \inj G&\Big(\bigvee_{C' \in \ig{\kmax+2}{\kmax+3}}\big( p \not \models \ic{0}{e}{C'} \wedge 
		\\&(\track_t \circ p \text{ is not total } \vee \track_t \circ p  \models 
		\ic{0}{e}{C'})\big)\Big)
			\end{split}
			\end{equation}
		\item 
			Special increasing condition:
			\begin{equation}
				\exists p: C_{\kmax+2} \inj G(\track_t \circ p \text{ is not total})
			\end{equation}
	\end{enumerate}
	
\end{definition}

Note that the universally and existentially bound conditions not only ensure that the largest satisfied layer does not decrease, as shown in Lemma \ref{lemma_consistent}, but also prevent further unnecessary insertions and deletions, since inserting a universally bound graph and deleting an existentially bound graph will never lead to an increase in consistency.


Now, we will show the already indicated relation between direct consistency increasing and consistency increasing, namely that a direct consistency-increasing transformation is also consistency-increasing transformation.
Counterexamples in which the inversion of the implication does not hold can be easily constructed, showing that these notions are not identical but related. 

\begin{theorem}\label{lemma:direct_implies_normal}
	Given a constraint $c$ in UANF, a rule $\rho$, a graph $G$ with 
	$G \not \models c$ and a direct consistency-increasing transformation
	$t: G \Longrightarrow_{\rho,m} H$ w.r.t. $c$. 
	Then, $t$ is also a consistency-increasing transformation. 
\end{theorem}

\begin{proof}
	By Theorem \ref{thm:direct_maintaining_to_maintaining} it follows that 
	$t$ is a consistency maintaining transformation. Therefore, it is 
	sufficient to show that $\nv{\maxk{c}{G} + 1}{H} < \nv{\maxk{c}{G} + 1}{G}$.
	Let $\kmax = \maxk{c}{G}$ and $d = \scond{\kmax+2}{c}$ with $d \neq \false$.

	Then, the general increasing condition is satisfied, so there exists  a morphism $p:C_{\kmax+2} \inj G$ with $p \not \models \ic{0}{d}{C'}$, such that either $\track \circ p$ is total and $\track_t \circ p  \models \ic{0}{d}{C'}$ or $\track \circ p$ is not total, for a graph $C' \in \ig{C_{\kmax+2}}{C_{\kmax+3}}$.
	In both cases the following applies: 
	\begin{equation*}
		\begin{split}
			p &\in \{q \mid q:C_{\kmax+2} \inj G \wedge q \not \models \ic{0}{d}{C'}\} 
			\wedge \\
			\track \circ p &\notin \{q \mid q:C_{\kmax+2} \inj H \wedge q \not \models 
			\ic{0}{d}{C'}\}
		\end{split}
	\end{equation*}
	Since $t$ is direct consistency maintaining, it follows that 
	$$|\{q \mid q:C_{\kmax+2} \inj G \wedge q \not \models \ic{0}{d}{C}\}| \leq |\{q 
	\mid q:C_{\kmax+2} \inj H \wedge q \not \models \ic{0}{d}{C}\}|.$$
	for all $C \in \ig{C_{\kmax+2}}{C_{\kmax+3}}$. Furthermore, this inequality 
	is strictly satisfied if $C = C'$.
	It immediately follows that $\nv{k}{G} < \nv{k}{G}$ and $t$ is a consistency-
	increasing transformation.

	If $d = \false$, i.e. $\scond{\nlvl(c)-1}{c} = \forall(C_{\nlvl(c)},\true)$ and $\kmax = \nlvl(c)-2$, the special increasing condition is satisfied.
	It holds that $$|\{q \mid q:C_k \inj G\}| \leq |\{q \mid q:C_k \inj H\}|$$, and since $t$ is a direct consistency-maintaining transformation, it can be shown in a similar way as above that satisfying the special increasing condition implies that  
	$$|\{q \mid q:C_k \inj G\}| < |\{q \mid q:C_k \inj H\}|.$$
	It follows that $t$ is a consistency-increasing transformation.
\end{proof}

\begin{figure}
\includegraphics[scale=0.8]{figures/images/direct_example}
\caption{example}\label{fig:example_direct}
\end{figure}

\begin{example} \label{ex_direct}
	Consider constraint $c_1$ given in Figure \ref{fig:constraints}, 
	the transformations $t_1$, $t_2$ and the set $\ig{C_1^1}{C_2^1}$ given 
	in Figure \ref{fig:example_direct}.  
 	Then, $t_1$ is a consistency maintaining transformation.
	The number of violations in both graphs is  $7$. 
	In the first graph, the occurrence $C_1$ does not satisfy $\exists(\mathcal{I}_4, \true)$, $\exists(\mathcal{I}_5, \true)$ and $\exists(\mathcal{I}_6, \true)$ and the occurrence $C_2$ does not satisfy $\exists(\mathcal{I}_3, \true)$, $\exists(\mathcal{I}_4, \true)$, 
	$\exists(\mathcal{I}_5, \true)$ and $\exists(\mathcal{I}_6, \true)$.
	In the second graph, these roles are swapped, i.e. $C_2$ satisfies exactly the intermediate conditions that $C_1$ satisfied in the first graph, and vice versa.
	But, $t_1$ is not a direct consistency-maintaining transformation, since the occurrence $C_1$ satisfies $\exists(\mathcal{I}_3,\true)$ in the first but not in the second graph. Therefore, the deleting condition is not satisfied.
	
	The transformation $t_2$ is consistency increasing w.r.t. $c_1$. The number of violations in the first graph is equal to $11$. The occurrence $C_1$ does not satisfy $\exists(\mathcal{I}_4, \true)$, $\exists(\mathcal{I}_5, \true)$ and $\exists(\mathcal{I}_6, \true)$. Both occurrences $C_2$ and $C_3$ do not satisfy $\exists(\mathcal{I}_3, \true)$, $\exists(\mathcal{I}_4, \true)$,  $\exists(\mathcal{I}_5, \true)$ and $\exists(\mathcal{I}_6, \true)$.
	
	In the second graph, $C_1$ does not satisfy $\exists(\mathcal{I}_3,\true)$, $\exists(\mathcal{I}_5,\true)$ and $\exists(\mathcal{I}_6,\true)$ and both $C_2$ and $C_3$ do not satisfy $\exists(\mathcal{I}_6,\true)$. Therefore, the number of violations in the second graph is $5$.

	But $t_2$ is not a direct consistency increasing transformation, since $C_1$ satisfies $\exists(\mathcal{I}_3,\true)$ in the first graph but not in the second, and the deleting condition is not satisfied.
\end{example}