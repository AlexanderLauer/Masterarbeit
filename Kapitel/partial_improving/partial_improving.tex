\subsection{Consistency increasing transformations and rules}

With the results above, we are now ready to define the notions of \emph{consistency increasement} and \emph{direct consistency increasement}, with direct increasing being a more restrictive version of increasing, yielding the advantage that second-order logic formulas can be used in order to determine whether a transformation is (direct) consistency increasing or not. 

These notions are designed to only detect transformations that increase the consistency of the first two unsatisfied layer of a constraint $c$. 
That means, given a graph $G$ and a constraint $c$, let $k = \kmax +1$ and  $\scond{k}{c} := \forall(a_{k}: C_{k} \inj C_{k+1}, \exists(a_{k+1}: C_{k+1} \inj C_{k+2},d))$.  
A transformation $ t:G \Longrightarrow H$ is considered as (direct) consistency increasing if $\maxk{c}{G} \leq \maxk{c}{H}$, i.e. the satisfaction up to layer is not decreased, and more increasing insertions or deletions have been performed than decreasing ones. 
An increasing deletion is the deletion of an occurrence of $C_{k+1}$ that does not satisfy $\exists(a_{k+1}: C_{k+1} \inj C_{k+2},\true)$, an increasing insertion is the insertion of elements of $C_{k+2}$, such that for at least one occurrence $p$ of $C_{k+1}$ it holds that $p \not \models \exists(a^r_{k+1}: C_{k+1} \inj C',\true)$ and $\track_t \circ p \models \exists(a^r_{k+1}: C_{k+1} \inj C',\true)$ for a graph $C' \in \ig{C_{k+1}}{ C_{k+2}}$. 
A decreasing insertion is the creation of an occurrence of $C_{k+1}$ not satisfying $\exists(a_{k+1}: C_{k+1} \inj C_{k+2},\true)$ and a decreasing deletion is the deletion of elements of $C_{k+2}$ such that for an occurrence $p$ of $C_{k+1}$ with $p \models \exists(a^r_{k+1}: C_{k+1} \inj C',\true)$ it holds that $\track_t \circ p \not \models \exists(a^r_{k+1}: C_{k+1} \inj C',\true)$ for a graph $C' \in \ig{C_{k+1}}{ C_{k+2}}$.

To evaluate this, we define the \emph{number of violations}.
Intuitively, for all occurrences $p$ of $C_{k+1}$ the number of graphs $C' \in \ig{C_{k+1}}{C_{k+2}}$ with $p \not \models \exists C'$ is add up and it can be determined whether more increasing or decreasing actions have been performed by a transformation. 

Note, that the number of violations is defined for each layer of the constraint, but only for the first unsatisfied layer the sum is calculated as described above.
For all layer $k$ with $k \leq \kmax$ it is set to $0$ and for all  layer $k$ with $k > \kmax + 1$ it is set to $\infty$. 
Through this, a transformation $t G \Longrightarrow H$ that increases the satisfaction up to layer can easily detected since the number of violations in $H$ at layer $\maxc{G} + 1$ will be set to $0$. 

\begin{definition}[\textbf{number of violations}]
Let a graph $G$ and a constraint $c$ in UANF be given. 
The \emph{number of violations $\nv{j}{G}$ at layer $j$ in $G$} is defined as:
\begin{enumerate}
\item If $j < \kmax+1$:
	$$\nv{j}{G} := 0$$
\item If $j = \kmax+1$, let $d = \forall(a_k: C_{j+1} \inj C_{j+2},e)$ be the subcondition of $c$ at layer $j+1$.


$$\nv{j}{G} := \begin{cases}
					\sum_{C' \in \ig{C_{j+2}}{C_{j+3}}} |\{q \mid q:C_{j+1} \inj G \wedge q \not \models \ic{0}{e}{C'}\}| &  \text{if $e \neq \false$}\\
					|\{q \mid q:C_{j+1} \inj G\}| & \text{if $e = \false$} 
				\end{cases}$$

\item If $j > \kmax+1$:
	$$\nv{j}{G} := \infty$$
\end{enumerate}

\end{definition}

Via the number of violations, we now define \emph{consistency increasing} transformations and rules, by checking whether the number of violations has decreased for any layer of the constraint.

\begin{definition}[\textbf{consistency increasing}]
	Let a graph $G$, a rule $\rho$ and a constraint $c$ in EANF be given. 

	
	A transformation $G \Longrightarrow_{\rho,m} H$ is called \emph{consistency increasing w.r.t $c$}, if  $$\nvc(k, H) < \nvc(k, G) $$
	for any $0 \leq k \leq \nlvl(c)$.
	A rule $r$ is called \emph{consistency increasing w.r.t $c$}, if all of its applications are. 
	
\end{definition}

Note that if $G \models c$ there does not exist a consistency increasing transformation w.r.t $c$, since $\nvc(j,G) = 0$ for all $1 \leq j \leq \nlvl(c)$.
Also, no plain rule $\rho$ is  consistency increasing w.r.t $c$, since a graph $G$ satisfying $c$, such that a transformation $t: G \Longrightarrow_{\rho,m} H$ exists can always be constructed. 
Therefore, each consistency increasing rule has to be equipped with at least one application condition.

As mentioned above, a transformation should be detected as consistency increasing if it increases the partial consistency, which is shown by the following theorem. 

\begin{theorem}
	Let a graph $G$, a rule $\rho$ and a constraint $c$ in EANF be given.
	A transformation $t: G \Longrightarrow_{\rho,m} H$ is consistency increasing if $\maxc{G} < \maxc{H}$.	
\end{theorem}

\begin{proof}
No $\ell >\maxc{G}$ with $G\models_{\ell} c$ exists. Hence, $\nvc(\maxc{G} + 1,G) > 0$ and $\nvc(\maxc{G} + 1,G) \neq \infty$. 
Since $\maxc{H} > \maxc{G}$, $\nvc(k+1,H) = 0$ and it follows immediately that $t$ is consistency increasing. 
\end{proof}

Since no consistency increasing transformation originating in consistent graphs exist, there do not exist infinite long sequences of consistency increasing transformations.
Additionally, if a set of rules $\mathcal{R}$ is given, and a sequence of consistency increasing transformations with rules of $\mathcal{R}$ ends with a graph $G$, then either $G$ satisfies the constraint or there do not exist any consistency increasing transformations $G \Longrightarrow_{\rho, m} H$ with $\rho \in \mathcal{R}$.

\begin{theorem}
Let a constraint $c$ in EANF and a set of rules $\mathcal{R}$ be given. 
Every sequence of minimal consistency improving transformation w.r.t $c$ with rules in $\mathcal{R}$ is finite.

\end{theorem}

\begin{proof}
Let $G_0$ be a graph and 
$$G_0 \Longrightarrow_{\rho_1} G_1 \Longrightarrow_{\rho_2} G_2 \Longrightarrow_{\rho_3} \ldots$$
be a sequence of minimal consistency improving transformations w.r.t $c$ with $\rho_i \in \mathcal{R}$.
We assume that $\maxc{G_0} < \nlvl(c)$, otherwise $\nvc(j,G_0) = 0$ for all $j \in \{0, \ldots, \nlvl(c)\}$ and no transformation $G_0 \Longrightarrow H$ is consistency increasing. 

We show that after at most $j := \nvc(\cmax_{G_0}+1,G_0)$ transformations $G_j \models_{\maxc{G_0} + 2} c$ holds. 
Note that $j$ has to be finite, since $G_0$ contains only a finite number of occurrences of $C_{j+1}$.
After each transformation it holds that $\nvc(\maxc{G_{i+1}}+1,G_{i+1}) \leq \nvc(\maxc{G_i}+1,G_{i})-1$.
Therefore, after $j$ transformations, $\nvc(\maxc{G_0}+1,G_{j}) \leq \nvc(\maxc{G_0}+1,G_{0})-j = 0$ holds and with that $G_j \models_{\maxc{G_0}+ 2} c$.
By iteratively applying this, it follows that after a finite number of transformations a graph $G_k$ with $G_k \models c$ has to exist. 
Since no consistency increasing transformation $G_k \Longrightarrow G_{k+1}$ exists, the sequence has to be finite.
Also, for some $G_{k'}$ there may not exist a consistency increasing transformation $G_{k'} \Longrightarrow_{\rho} H$ with $\rho \in \mathcal{R}$ and therefore the sequence is also finite, but does not end with a consistent graph.  
\end{proof}

The replacement of a graph not satisfying a constraint $c$ by a $c$ satisfying graph via a transformation is a consistency increasing transformation. 
Therefore, a consistency increasing transformation can perform insertions or deletions that are unnecessary in order to increase the consistency.
That is, the deletion of occurrences of existentially bound graphs, the deletions of occurrences of graphs $C_k$ satisfying $\exists C_{k+1}$ or the insertion of occurrences of universally bound graphs and the insertion of existentially bound graphs $C_k$, such that the corresponding occurrence of $C_{k-1}$ already satisfied $\exists C_k$. 

\emph{Direct consistency increasing} transformations are more restricted, in the sense, that these unnecessary deletions and insertions are leading a transformation to not being direct increasing. 
The presence of these unnecessary actions can be checked via second-order logic formulas.
Additionally, it is secured that no new violations are introduced since these can always be considered as a unnecessary insertion or deletion.
With that, the removal of one violation is sufficient to state that the transformation is (direct) consistency increasing, which can be checked via an additional second order logic formula.
Intuitively, it is checked by the first two formulas, that no new violations are introduced. 
The third formula secures that at least one violation has been removed and the last formulas secure that the partial consistency is not decreased. 


\begin{definition}[\textbf{direct consistency increasing}]\label{def_direct_improving}
Let $G$ be a graph, $\rho$ a plain rule, $c$ a constraint in EANF and  $\scond{c}{\maxc{G} + 1} = \forall(a_k: C_{k-1} \inj C_{k}, e)$. 
A transformation $t: G \Longrightarrow_{\rho,m} H$ is called \emph{direct consistency increasing} if the following equations hold.
Let $$ \mathbf{G} = \begin{cases}
					\mathcal{U}(C_k, C_{k+1}) &\text{if $e \neq  \false$} \\
					\{C_k\} &\text{otherwise}
					\end{cases}$$ 

Every occurrence of $C_k$ in $G$ that satisfies $\parcond(1,e,C')$ for any $C' \in \mathbf{G}$ still satisfies $\parcond(1,e,C')$ in $H$. 
\begin{equation}\label{direct_improving_1}
\begin{split}
	\forall p: C_{k} \inj G\Big( \bigwedge_{C' \in \mathbf{G}}\big(& p \models \parcond(1,e,C') \wedge \track_t \circ p \text{ is total}\big) \\ &\implies  \track_t \circ p \models \parcond(1,e,C') \Big) 
\end{split}
\end{equation}
Let $C' = C_{k+1}$ if $e \neq \false$ and $C' = C_k$ otherwise.
Every new inserted occurrence of $C_k$ by $t$ satisfies $\parcond(1,e,C')$.
\begin{equation}\label{direct_improving_1_2}
\begin{split}
\forall p': C_{k} \inj H \big(\neg \exists p : C_k \inj G(p' = \track_t \circ p) \implies \ p' \models \parcond(1,e,C')\big)
\end{split}
\end{equation}
At least one occurrence of $C_k$ in $G$ that does not satisfy $\parcond(1,e,C')$, for any $C' \in \mathbf{G}$, either has been removed or satisfies $\parcond(1,e,C')$ in $H$.
\begin{equation}\label{direct_improving_2}
\begin{split}
\exists p: C_k \inj G \Big(\bigvee_{C' \in \mathbf{G}}\big( &p \not \models \parcond(1,e,C') \wedge (\track_t \circ p \text{ is not total } \\&\vee(\track_t \circ p \text{ is total } \wedge \track_t \circ p  \models \parcond(1,e,C')))\big) \Big) 
\end{split}
\end{equation}

No occurrence of a universally bound graph $C_j$ with $j < k$ gets inserted. 
\begin{equation}\label{direct_improving_3}
\bigwedge_{\substack{i < k \\ i \textit{ odd}}} \forall p: C_i \inj H ( \exists p': C_i \inj G (p = \track_t \circ p'))
\end{equation}
No occurrence of an existentially bound graph $C_j$ with $j < k$ gets deleted. 
\begin{equation}\label{direct_improving_4}
\bigwedge_{\substack{i < k \\ i \textit{ even}}} \forall p: C_i \inj G( \track_t \circ p \textit{ is total})
\end{equation}

\end{definition}

Note that (\ref{direct_improving_3}) and (\ref{direct_improving_4}) not only secure that the partial consistency does not decrease, as shown in the following lemma, but also prevent further unnecessary insertions and deletions, since the insertion of a universally and the deletion of a existentially bound graph will never lead to an increase of consistency. 

\begin{lemma}\label{lemma_consistent}
	Let a transformation $t: G \Longrightarrow H$ and a constraint $c$ in EANF be given, such that (\ref{direct_improving_3}) and (\ref{direct_improving_4}) of definition \ref{def_direct_improving} are satisfied. Then,
	$$H \models_{\maxc{G}} c.$$
\end{lemma}
\begin{proof}
Assume that $G \not \models_{\maxc{G}}c$. 
Then, either a new occurrence of an universally bound graph $C_k$, with $k \leq \maxc{G}$, of $c$ has been inserted or an occurrence of an existentially bound graph $C_k$, with $k \leq \maxc{G}$,  of $c$ has been destroyed. 
Therefore, the following holds:
$$ \exists p: C_i \inj G(\neg \exists p': C_i \inj (p' = \track_t \circ p)) \vee \exists p:C_j \inj G(\track_t \circ p \textit{ is not total})$$
with $i, j \leq \maxc{G}$, $i$ being odd and $j$ being even. 
It follows immediately that either (\ref{direct_improving_3}) or (\ref{direct_improving_4}) is not satisfied. 
This is a contradiction. 
\end{proof}

Now, we will show the already indicated relationship between direct increasing and increasing, namely that a direct increasing transformation is also increasing. 
Counterexamples showing that the inversion of the implication does not hold can easily be constructed, showing that these notions are not identical, but related. 

\begin{lemma}
Let a graph $G$, a constraint $c$ in EANF and a direct increasing transformation $t: G \Longrightarrow_{\rho,m} H$ w.r.t. $c$ be given. 
Then, $t$ is also a increasing transformation. 
\end{lemma}

\begin{proof}

Let $G$ be a graph, $k = \maxc{G}+1 \leq \nlvl(c)$ and $d = \scond{c}{k}$.
With lemma \ref{lemma_consistent} follows that $\maxc{H} \geq \maxc{G}$ and with that $\nvc(k,H) \neq \infty$.

\begin{enumerate}
\item \label{proof_direct_minimal_4} We show that equations (\ref{direct_improving_1}) and (\ref{direct_improving_1_2}) imply that $\nvc(k,H) \leq \nvc(k,G)$. 
Assume that $\nvc(k,H) > \nvc(k,G)$. Therefore, a morphism $p: C_k \inj H$ with $p \not \models \parcond(1,d,C')$ for any $C' \in \mathcal{U}(C,C_{k+1})$ exists, such that either \ref{proof_direct_minimal_1} or \ref{proof_direct_minimal_2} is satisfied.
\begin{enumerate}
	\item \label{proof_direct_minimal_1}
		There does exist a morphism $q': C_k \inj G$ with $q' \models \parcond(1,d,C')$ and $p = \track_t \circ q'$. 

	\item \label{proof_direct_minimal_2}
		There does not exist a morphism $q : C_k \inj G$, such that $p = \track_t \circ q$. 		
\end{enumerate}
This is a contradiction, if \ref{proof_direct_minimal_1} is satisfied, $q'$ does not satisfy equation (\ref{direct_improving_1}) and if \ref{proof_direct_minimal_2} is satisfied $q$ does not satisfy equation (\ref{direct_improving_1_2}).
It follows that 
$$|\{q \mid q:C_k \inj G \wedge q \not \models \parcond(1,e,C')\}| \leq |\{q \mid q:C_{k} \inj H \wedge q \not \models \parcond(1,e,C')\}|$$
for all $C' \in \mathcal{U}(C_k, C_{k+1})$.

\item \label{proof_direct_minimal_3} Since (\ref{direct_improving_2}) is satisfied,  a morphism $p:C_k \inj G$ with $p \not \models \parcond(1,d,C')$, such that either $\track \circ p$ is total and $\track_t \circ p  \models \parcond(1,d,C')$ or $\track \circ p$ is not total exists, for any $C' \in \mathcal{U}(C_k, C_{k+1})$.
In both cases the following holds: 
\begin{equation*}
\begin{split}
	p &\in \{q \mid q:C_k \inj G \wedge q \not \models \parcond(1,e,C')\} \wedge \\
	\track \circ p &\notin \{q \mid q:C_{k} \inj H \wedge q \not \models \parcond(1,e,C')\}
\end{split}
\end{equation*}
With that and \ref{proof_direct_minimal_4}. it follows that
$$|\{q \mid q:C_k \inj G \wedge q \not \models \parcond(1,e,C')\}| < |\{q \mid q:C_{k} \inj H \wedge q \not \models \parcond(1,e,C')\}|.$$


\end{enumerate}
In total, $\nvc(k,G) < \nvc(k,G)$ and  $t$ is an increasing transformation.
 
For the special case that $k = \nlvl(c) -1$ and $c$ ends with $\forall(a_k:C_{k-1} \inj C_k,\false)$ it can be shown in a similar way that  (\ref{direct_improving_1}) and (\ref{direct_improving_1_2}) imply that 
$$|\{q \mid q:C_k \inj G\}| \leq |\{q \mid q:C_k \inj H\}|$$
and that (\ref{direct_improving_2}) implies 
$$|\{q \mid q:C_k \inj G\}| < |\{q \mid q:C_k \inj H\}|.$$
It follows that $t$ is an increasing transformation. 
\end{proof}


