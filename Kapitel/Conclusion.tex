\section{Conclusion}\label{conclusion}

In summary, we have introduced a rule-based graph repair approach for circular conflict-free constraints and, in particular, for a circular conflict-free set of constraints, and have shown its correctness and termination. 
Our approach can also be used for constraints and sets of constraints that are not circular conflict-free. However, there is no guarantee that the algorithm will terminate and thus return a consistent graph. 

We have introduced new notions of consistency, called (direct) consistency-maintaining and (direct) consistency-increasing transformations and rules, which are finer-grained than previously known concepts. To do this, we introduced the notion of \emph{satisfaction up to layer}, which is a notion of partial consistency, and showed what can be deduced from it. For example, if $k$ is even and a graph satisfies a constraint up to layer $k$, then the graph also satisfies $c$.

We have used the notions of direct consistency-maintaining and direct consistency-increasing transformations to characterise the circumstances under which a given rule set $\mathcal{R}$ is able to repair a constraint or set of constraints. 
To do this, we first introduced the notions of \emph{conflicts within conditions} and \emph{conflicts between conditions}. 
Intuitively, given a constraint $c$, a universally bound graph $C_k$ causes a conflict with an existentially bound graph $C_j$ if an occurrence of $C_j$ can be destroyed by destroying an occurrence of $C_k$. An existentially bound graph $C_j$ causes a conflict for a universally bound graph $C_k$ if an occurrence of $C_k$ can be inserted by inserting elements of $C_j \setminus C_{j-1}$. 
A set of rules $\mathcal{R}$ is capable of repairing a circular conflict-free constraint if there exists a sequence of transformations that repairs an occurrence of a universally bound graph $C_k$ so that it satisfies $\exists(C_{k+1}, \true)$ or destroys its occurrence. Using the notions of direct consistency-maintaining rules and transformations, we ensure that these sequences cannot create or delete occurrences of graphs such that either $C_k$ does not cause a conflict for them or $C_{k+1}$ does not cause a conflict for them.

We have compared the notions of (direct) consistency-maintaining and (direct) con\-sis\-ten\-cy-increasing transformations and rules with the notions of 
consistency-gua\-ran\-tee\-ing, con\-sis\-ten\-cy-preserving, (direct) consistency-improving and (direct) consistency sustaining transformations and rules \cite{kosiol2022sustaining, habel2009correctness}. In general, the notions of (direct) consistency-sustaining and  (direct) consistency-increasing transformations and rules are not related at all to our notions of (direct) consistency-maintaining and (direct) consistency-increasing transformations and rules. 
Only if $\nlvl(c) = 1$, our notions are identical to those of (direct) consistency-sustaining and (direct) consistency-improving transformations. 
But, our notions are related to those of consistency-preserving and consistency-increasing rules and transformations. 
If the constraint is not satisfied, a $c-$guaranteeing transformation is consistency-increasing w.r.t. $c$.
In general, consistency-guaranteeing implies consistency-maintaining and consistency-maintaining implies consistency-pre\-ser\-ving.


Furthermore, we have introduced the weaker notions of \emph{(direct) consistency-main\-tain\-ing rules at layer} and \emph{(direct) consistency-increasing rules at layer}, which allow the construction of less complex application conditions. A rule $\rho$ is direct consistency-maintaining or direct consistency-increasing at layer $k$ if all of its applications at graphs $G$ with $\kmax = k$ are (direct) consistency-maintaining or (direct) consistency-increasing. Using these notions, we have introduced four constructions for application conditions and shown that rules equipped with these conditions are direct consistency-maintaining at layer, direct consistency maintaining and direct consistency-increasing at layer. In particular, we have introduced two types of consistency-increasing application conditions at layer.
First, application conditions for general rules, and second, application conditions for a special set of rules, called basic rules.
Since basic rules, by definition, are not able to introduce new violations or decrease the satisfaction up to layer, it is sufficient to check that at least one violation is removed. Since the left-hand side of a basic increasing rule $\rle{L}{l}{K}{r}{R}$  at layer $k$ must contain an occurrence $p$ of $C_k$ such that $r \circ l^{-1} \circ p \models \exists(C', \true)$ and $p \not \models \exists(C, \true)$ for an intermediate graph $C' \in \ig{C_k}{C_{k+1}}$, it is sufficient to check whether $m \circ p$ does not satisfy $\exists(C', \true)$. This leads to less complex application conditions compared to the application conditions of other approaches. Compared to the general ones, these application conditions are less complex and less restrictive.
We have introduced \emph{derived rules} to ensure that the condition that $L$ must contain an occurrence of $C_k$ is not a restriction on the notion of basic rules. The left-hand sides of derived rules contain an occurrence of $C_k$, and we have shown that these rules are only applicable if the rule from which this rule is derived is applicable at a smaller match that leads to the same derived graph.

Future work is to extend the notions of consistency-maintaining and consistency-increasing transformations for all types of nested constraints, and a rule-based repair process for all satisfiable nested constraints, i.e. constraints that also use Boolean operators. 
Although we have presented characterisations for circular conflict-free constraints, it remains unclear for which practical applications our approach is suitable. This may require implementation and further evaluation of the repair process, characterisations and construction of application conditions.
The notion of conflict between constraints is very strict, and we are confident that there exists a repair process that uses the repair process for one constraint to repair a set of constraints in parallel. To do this, the conflict graph for all graphs of all constraints must be acyclic, i.e. there is no circular conflict in the set of all graphs of all constraints. Then, we can repair all constraints using Algorithm \ref{Algo_conflict_free}. The challenge of this approach is to decide which occurrences of which graphs need to be repaired in order to increase consistency.

We also suggest the following more technical optimisations for the general application conditions.
First, not all overlaps in the set $\mathbf{P}_{C'}$ need to be considered when constructing the no violation inserted part of the maintaining application condition since some of them are implied by others. 
Second, the application condition constructed from the violation removed part of the consistency increasing application condition is far too restrictive, since all occurrences of $P'$ that extend $R$ must satisfy $\exists(C', \true)$ after the transformation.
We are confident that, with some additional theory, the $\nex(P,C' )$ and $\rep(P,C' )$ parts of the application condition can be combined to construct less restrictive and less complex application conditions.




 
