\section{Conclusion}\label{conclusion}

In summary, we have introduced a new concept of consistency, which leads to the new notions of the notions of consistency-maintaining and consistency-increasing transformations and rules, which is finer-grained, compared to the already existing notions of consistency preserving, consistency guaranteeing, consistency sustaining and consistency improving. Finer-grained, in the sense that the smallest increases and decreases of consistency can be detected, namely the insertion or removal of one graph element.
We have compared our notions with the already existing notions mentioned above. The notions of consistency-maintaining and consistency-improving are related to the notions of consistency-preserving and consistency-guaranteeing, and are generally not related to the notions of consistency-sustaining and consistency-improving. Only in special cases, for specific constraints, are these notions related or even identical.


Furthermore, we have introduced application conditions such that a rule equipped with this condition is consistency-maintaining or consistency-increasing. In particular, we introduce two types of application conditions.
First, application conditions for general rules, and second, application conditions for a special set of rules, which we call basic-maintaining and basic-increasing rules, respectively, where the application conditions for basic rules are less complex and smaller in size compared to the general application conditions.

We have introduced a rule-based graph repair process based on our new notions of consistency-maintaining and consistency-increasing for a particular type of constraints in ANF, which we call circular conflict-free constraints, using a given rule set.
We present characterisations of these circular conflict free constraints and a characterisation of a rule set that is able to repair a constraint using our repair process. 
In addition, we extend the notion of conflicts in constraints to the notion of conflicts between constraints, and present a rule-based graph repair process for a satisfiable circular conflict free set of constraints, which is based on the repair process for one constraint.

Future work is to extend the notions of consistency-maintaining and consistency-increasing transformations for all types of nested constraints, and a rule-based repair process for all satisfiable nested constraints. 
Although we have presented characterisations for circular conflict-free constraints, it remains unclear for which practical applications our approach is suitable. This may require implementation and further evaluation of the repair process, characterisations and construction of application conditions.

 
