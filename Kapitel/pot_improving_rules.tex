\subsection{Conflicts within and between conditions}

\begin{definition}[\textbf{Conflicts within condition}]
Let a condition $c$ in UANF be given. 
Then, an existentially bound graph $C_k$ has a \emph{conflict} with an universally bound graph $C_j$ if a transformation $t: G \Longrightarrow_{\rho} H$ with $\rho = C_{k-1} \xhookleftarrow{\id} C_{k-1} \xhookrightarrow{a_{k-1}} C_{k}$ exists such that the following holds 
$$\exists p: C_j \inj H (\neg \exists p': C_j \inj G(\track_t \circ p' = p)).$$
An universally bound graph $C_k$ is has a \emph{conflict} with an existentially bound graph $C_j$ if a transformation $t: G \Longrightarrow_{\rho} H$ with $\rho = C_k \xhookleftarrow{i_{C'}} C' \xhookrightarrow{\id} C'$ and  $C' \in \ig{C_{k-1}} {C_k}$ exists such that the following holds. 
$$\exists p: C_j \inj G (\track_t \circ p \text{ is not total}).$$
A condition $c$ in UANF is called \emph{conflict free} if no graph $C_k$ in $c$ has a conflict with a graph $C_j$ with $0 \leq j < k \leq \nlvl(c)$. 
A graph $C_k$ has a \emph{transitive conflict} with $C_{j'}$ a graph $C_j$ exists such that $C_k$ has a conflict with $C_j$ and $C_j$ has a (transitive) conflict with $C_{j'}$.
A condition $c$ does contain a \emph{circular conflict} if a graph $C_k$ exists such that $C_k$ has a transitive conflict with itself. 
Otherwise, $c$ is called \emph{circular conflict free}.
\end{definition}

\begin{lemma}
Let a constraint $c$ in UANF be given.
\begin{enumerate}
\item Let $C_k$ be an existentially bound graph of $c$. 
Then, $C_k$ has a conflict with $C_j$ if and only if an overlap $P$ of $C_k$ and $C_j$ exists such that 
$$ i_{C_k}^P(C_k\setminus C_{k-1}) \cap i_{C_j}^P(C_j) \neq \emptyset$$ 
and the rule $\rho = C_{k} \xhookleftarrow{l} C_{k-1} \xhookrightarrow{r} C_{k-1}$ is applicable at match $i_{C_k}^P$.
\item Let $C_k$ be an universally bound graph of $c$. 
	Then $C_k$ has a conflict with $C_j$ if and only if an overlap $P$ of $C_k$ and $C_j$ exists such that
	$$i_{C_k}^P(C_k \setminus C_{k-1}) \cap i_{C_j}^P(C_j\setminus C_{j-1}) \neq \emptyset$$
and each rule $\rho = 	C_{k} \xhookleftarrow{i_{C'}} C' \xhookrightarrow{\id} C'$ with $C' \in \ig{C_{k-1}}{C_k}$ is applicable at match $i_{C_k}^P$.

\end{enumerate}
\end{lemma}

\begin{proof}
Let a condition $c$ in UANF be given.
\begin{enumerate}
\item 
\enquote{$\Longrightarrow$}: Let $C_k$ be an existentially bound graph that has a conflict with an universally bound graph $C_j$. 
Then, there does exists a transformation $t: G \Longrightarrow_{\rho} H$ with $\rho = C_{k-1} \xhookleftarrow{l} C_{k-1} \xhookrightarrow{r} C_{k}$ such that a new occurrence $p$ of $C_j$ has been inserted. 
Since only elements of $C_k \setminus C_{k-1}$ have been inserted it holds that $p(C_j) \cap n(C_k\setminus C_{k-1})$ with $n$ being the comatch of $t$. 
The graph $p(C_j) \cup n(C_k)$ is the searched for overlap and the rule  $\rho^{-1} = C_{k} \xhookleftarrow{l} C_{k} \xhookrightarrow{r} C_{k-1}$ has to be applicable at $n$. 
\\
\enquote{$\Longleftarrow$}:
Let $C_k$ be an existentially and $C_j$ an universally bound graph such that an overlap $P$ of $C_k$ and $C_j$ with $i_{C_k}^P(C_k\setminus C_{k-1}) \cap i_{C_j}^P(C_j) \neq \emptyset$ exists such that the rule $\rho = C_{k} \xhookleftarrow{l} C_{k} \xhookrightarrow{r} C_{k-1}$ is applicable at match $i_{C_k}^P$.
Then, the inverse transformation of $t: P \Longrightarrow_{\rho,i_{C_k}^P} H$ is the searched for transformation and $C_k$ has a conflict with $C_j$. 

\item
\enquote{$\Longrightarrow$}: Let $C_k$ be an universally bound graph that has a conflict with an existentially bound graph $C_j$. 
Then, a transformation $t: G \Longrightarrow_{\rho} H$ with $\rho = C_k \xhookleftarrow{l} C' \xhookrightarrow{r} C'$ and $C'$ being a subgraph of $C_k$ exists such that a occurrence $p$ of $C_j$ has been deleted. 
Then, the graph $p(C_j) \cup m(C_k)$ is the searched for overlap and $i_{C_k}^P(C_k \setminus C_{k-1}) \cap i_{C_j}^P(C_j\setminus C_{j-1}) \neq \emptyset$ has to hold since $\rho$ only deletes elements of $C_k\setminus C_{k-1}$.
\\
\enquote{$\Longleftarrow$}:
Let $C_k$ be universally and $C_j$ existentially bound such that an overlap $P$ of $C_k$ and $C_j$ with $i_{C_k}^P(C_k \setminus C_{k-1}) \cap i_{C_j}^P(C_j\setminus C_{j-1}) \neq \emptyset$ exists such that each rule $\rho = 	C_{k} \xhookleftarrow{i_{C'}} C' \xhookrightarrow{\id} C'$ with $C' \in \ig{C_{k-1}}{C_k}$ is applicable at match $i_{C_k}^P$.
Then, the inverse transformation of $t: P \Longrightarrow_{\rho, i_{C_k}^P} H$ is the searched for transformations and $C_k$ has a conflict with $C_j$.
\end{enumerate}
\end{proof}

\begin{definition}[\textbf{Conflicts between conditions}]
	Let conditions $c_1$ and $c_2$ be given. Then, $c_1$ has a conflict with $c_2$ if 
	graphs $C_k$ and $C_j$ of $c_1$ and $c_2$exist such that $C_k$ has
	a conflict with $C_j$ or vice versa. 
	A set of conditions is called \emph{conflict free} if no conditions $c_1$ and $c_2$ 
	exist such that $c_1$ has a conflict with $c_2$ or vice versa.
	A condition $c_1$ has a \emph{transitive conflict with} $c_2$ if a condition $c'$
	exists such that $c_1$ has a conflict with $c'$ and $c'$ has a (transitive) conflict
	with $c_2$.
	A set of conditions does contain a \emph{circular conflict} if a condition $c$ has 
	a transitive conflict with itself. 
	Otherwise, the set is called \emph{circular conflict free}.
\end{definition}

\begin{definition}[\textbf{conflict free graphs}]
	Let a graph $G$ and a constraint $c$ in UANF be given.
	Then, $G$ is called \emph{conflict free} w.r.t $c$ if for
	each transformation
	$t: G \Longrightarrow_{\rho,m} H$ with $$\rho =  C_{\kmax}
	\overset{\id}{\hookleftarrow} C_{\kmax} 
	\overset{i_{C_{\kmax}}}{\hookrightarrow} C'$$
	and $C' \in \ig{C_{\kmax+1}}{C_{\kmax+2}}$ or $$\rho = 
	 C_{\kmax+1}
	\overset{\i_{C'}}{\hookleftarrow} C' 
	\overset{\id}{\hookrightarrow} C'$$  and $C' \in 
	\ig{C_{\kmax}}{C_{\kmax+1 }}$ and (\ref{direct_improving_3})
	 and (\ref{direct_improving_4}) hold. 
\end{definition}

\begin{lemma}
	Let a graph $G$ and a constraint $c$ in UANF be given.
	Let $P$ be the set of all 
	Then, $G$ is conflict free w.r.t $c$ if and only if either 
	$c$ is
	conflict free or
	$$G \models \bigwedge_{P \in P} \forall(a: \emptyset \inj P, \false)$$
\end{lemma}
\begin{proof}

\end{proof}



\subsection{Basic increasing rules}

As described above, the application conditions for general rules are very restrictive. 
This is because the application condition does not allow a transformation to delete occurrences of existentially bound or insert occurrences of  universally bound graphs even if this would not lead to a decrease of satisfaction up to layer. 
Now, we will consider a special set of rules called \emph{basic increasing rules}. 
The main idea is that these rules are not able to delete occurrences of existentially or insert occurrences of universally bound graphs and additionally are able to increase consistency. 
That means, given a basic increasing rule $\rho$, there does exist a transformation $t: G \Longrightarrow_{\rho} H$ such that $t$ is a consistency increasing transformation w.r.t to a constraint $c$ in UANF.
For \emph{basic increasing rules}, consistency increasing application conditions can be constructed that are less restrictive and smaller in size compared to application conditions for general rules. 

First, we define \emph{non-consistency decreasing rules at layer} and \emph{non-consistency decreasing rules up to layer} where a non-consistency decreasing rule at layer $k$ is not able to delete occurrences of $C_{k+1}$, if $\scond{k}{c}$ is existentially bound and not able to create occurrences of $C_{k+1}$ if $\scond{k}{c}$ is universally bound.
This can be checked via second order formulas by, in the first case, checking whether all occurrences of $C_{k+1}$ still exist in $H$ and, in the second case, checking whether all occurrences of $C_{k+1}$ in $H$ already existed in $G$. 
The notion of non-consistency decreasing rules up to layer $k$ is a abbreviation, meaning that a rule $\rho$ is non-consistency decreasing up to layer $k$ if it is non-consistency decreasing at layer $j$ for all $0 \leq j \leq k$. 

 

\begin{definition}[\textbf{non-consistency decreasing rule}]
Let a plain rule $\rho = L \xhookleftarrow{l} K \xhookrightarrow{r} R$ and a constraint $c$ in UANF be given. 
Then, $\rho$ is called a \emph{non-consistency decreasing rule  w.r.t $c$ at layer $k$} if 
\begin{enumerate}
\item If $\scond{k}{c}$ is universally bound and for all transformations $t: G \Longrightarrow_{\rho} H$ it holds that 
\begin{equation}\label{non_consistency_universally}
	\forall p: C_{k+1} \inj H( \exists p':C_{k+1} \inj G( \track_t \circ p' = p)).
\end{equation}
\item If $\scond{k}{c}$ is existentially bound for all transformations $t: G \Longrightarrow_{\rho} H$ it holds that 
\begin{equation}\label{non_consistency_existentially}
	\forall p: C_{k+1} \inj G( \track_t \circ \text{ p is total}).
\end{equation}

\end{enumerate}
The rule $\rho$ is called \emph{non-consistency decreasing w.r.t $c$ up to layer $k$} if $\rho$ is non-consistency decreasing w.r.t $c$ at layer $j$ for all $0 \leq j \leq k$ and it holds that
\begin{equation}\label{non_consistency_extra}
\bigwedge_{C' \in \ig{C_k}{C_{k+1}}} \forall p: C' \inj G(\track_t \circ p \text{ is total})
\end{equation}
if $scond{k}{c}$ is existentially bound.
\end{definition}


Given a graph $G$ and a constraint $c$, we will show that non-decreasing rules w.r.t $c$ up to layer $\kmax$ cannot decrease the satisfaction up to layer.  

\begin{lemma}
Let a graph $G$, a constraint $c$ in UANF and a non-consistency decreasing rule w.r.t $c$ up to layer $\kmax$ be given. 
Then, for each transformation $t: G \Longrightarrow_{\rho} H$ it holds that 
$$\maxk{c}{G} \leq \maxk{c}{H}.$$
\end{lemma}

\begin{proof}
Assume that $\maxk{c}{G} > \maxk{c}{H}$. 
Then, either \ref{pr:non-dec1}. or \ref{pr:non-dec2}. applies. 

\begin{enumerate}
\item \label{pr:non-dec1}
There does exists an occurrence $p:C_j \inj H$ with $0 \leq j \leq \maxk{c}{G}$ of a universally bound graph $C_j$ such that there does not exist a occurrence $q: C_j \inj G$ of $C_j$ with $p = \track_t  \circ q$.
Then $\rho$ is not a non-consistency decreasing rule up to layer $\maxk{c}{G}$ since (\ref{non_consistency_universally}) is not satisfied.  
\item \label{pr:non-dec2}
There does exists an occurrence $p:C_j \inj G$ with $0 \leq j \leq \maxk{c}{G}$ of an existentially bound graph $C_j$ such that $\track \circ p$ is not total. 
Then, $\rho$ is not a non-consistency decreasing rule up to layer $\maxk{c}{G}$ since (\ref{non_consistency_existentially}) is not satisfied.
\end{enumerate}
\end{proof}

Since it is very time consuming to check whether a rule $\rho$ is non-consistency decreasing based on the definition above, we present a characterisation for non-consistency decreasing rules which only relies on $\rho$ itself. 
First, let us assume that $\rho$ is able to create occurrences of a universally bound graph $C_j$. 
This is possible if (a) $\rho$ does insert an edge of $C_j \setminus C_{j-1}$ which connects already existing nodes of $C_j$, since it is unclear whether this would create a new occurrence of $C_j$. 
Or (b) if $\rho$ does insert a node $v$ of $C_j$ such that all edges $e \in E_{C_j}$ with $\src(e) = v$ or $\tar(e) = v$ are also inserted. 
If at least one of these edges is not inserted, it is guaranteed that this insertion does not create an occurrence of $C_j$ since $v$ is only connected to edges that have also been inserted by $\rho$.

Second, let us assume that $\rho$ is able to delete occurrences of a existentially bound graph $C_j$.
This is possible if (a) $\rho$ does delete an edge of $C_j \setminus C_{j-1}$ or (b) $\rho$ does delete a node $v$ of $C_j \setminus C_{j-1}$ such that all edges $e \in E_{C_j}$ with $\src(e) = v$ or $\tar(e) = v$ are also deleted. 
If $\rho$ deletes a node $c$ of $C_{j} \setminus C_{j-1}$ without all its connected edges in $C_j$ there does not exist a transformation such that $\rho$ deletes an occurrence of $C_j$ by deleting this node since the dangling edge condition is not satisfied. 

Then, we are able to determine whether a rule $\rho$ is non-consistency decreasing by checking whether $\rho$ does not satisfy the just mentioned properties.




\begin{lemma}\label{def:non-decreasing}
Let a plain rule $\rho = L \xhookleftarrow{l} K \xhookrightarrow{r} R$ and a constraint $c$ in EANF be given. 
Then, $\rho$ is a non-consistency decreasing rule at layer $k$ w.r.t $c$ if either \ref{category:non-decreasing_1}. or \ref{category:non-decreasing_2}. applies.  


\begin{enumerate}
\item \label{category:non-decreasing_1}
If $C_k$ is universally bound, let $I$ be the set of all isolated nodes of $C_k$, $P$ be the set of all partial morphisms $p: C_k \inj R$, 
$$E = \bigcup_{p \in P} p(E_{C_k}) \cap (E_R \setminus E_K) \text{ and }  V = \bigcup_{p \in P} p(V_{C_k}) \cap (V_R \setminus V_K).$$
Then, $\rho$ is called a \emph{non-consistency decreasing rule at layer $k$ w.r.t $c$} if 
$I \cap V = \emptyset$ and
\begin{alignat*}{2}
	&\big(\neg \exists e \in E: \exists v,v' \in V_K: \src(e) = r(v) \wedge \tar(e) = r(v')\big) & \wedge \\
	&\big(\forall v \in V: \exists e \in E_{C_k}: \exists p \in P: \src(e) = p^{-1}(v) \vee \tar(e) = p^{-1}(v) \\ &\wedge \neg \exists e' \in E: \exists p \in P: p^{-1}(e')=e\big)
\end{alignat*}
holds.


\item \label{category:non-decreasing_2}
If $C_k$ is existentially bound, let $I$ be the set of all isolated nodes of $C_k$, $P$ be the set of al morphisms $p:C_k \inj L$, 
$$E = \bigcup_{p \in P} p(E_{C_k}) \cap (E_L \setminus E_K) \text{ and }V = \bigcup_{p \in P} p(V_{C_k}) \cap (V_L \setminus V_K).$$  

\end{enumerate}
Let $I$ be the set of all isolated nodes of $C_k$. Then, $\rho$ is called a \emph{non-consistency decreasing rule at layer $k$ w.r.t $c$} if $I \cap V = \emptyset$ and 
\begin{alignat*}{2}
	&\big(\neg \exists e \in E_{C_k}\setminus E_{C_{k-1}}: \exists p \in P: \exists e' \in E: p(e) = e'\big) &\wedge  \\
	&\big(\forall v \in V: \exists e \in E_{C_k}: \exists p \in P: \src(e) = p^{-1}(v) \vee \tar(e) = p^{-1}(v) \\ &\wedge \neg \exists e' \in E: \exists p \in P: p^{-1}(e')=e\big)
\end{alignat*}
\end{lemma}
\begin{proof}

\end{proof}

Now we are ready to define \emph{basic increasing rules at layer $k$} with the set of basic increasing rules being a subset of the set of non-decreasing rules.
With the property that each basic increasing rule $\rho$ is also a non-decreasing rule up to layer $k$ it is satisfied that the satisfaction up to layer is not decreased. 
Additionally, if $\scond{k}{c}$ is existentially bound it has to be checked that the consistency for an occurrence of $C_{k}$ is not decreased and that $\rho$ removes an violation. 
This can be done by either deleting an element of $C_{k+1}\setminus C_k$ if $\scond{k}{c}$ is universally bound or by inserting elements of $C_{k+1}$ such that for one occurrence $p$ of $C_{k}$ the consistency in increased if $\scond{k}{c}$ is existentially bound. 
Additionally, the left-hand side of each basic increasing rule has to contain $C_{k+1}$ if $\scond{k}{c}$ is universally bound such that either $C_{k+1}$ gets deleted and the left-hand side has to contain $C_k$ if $\scond{k}{c}$ is existentially bound such that the consistency of $C_{k}$ is increased.
This property yields the advantage that application conditions for basic increasing rules are less complex, smaller in size and during a repair process matches of these rules are easier to handle.

This, on first sight seems like a restriction of the set of basic increasing rules but the context of each rule $\rho$ that does satisfy all properties of a basic increasing rule excluding that $C_k$ or $C_{k+1}$ is a subgraph of the left-hand side can be expanded such that $\rho$ is a basic increasing rule and the semantic of $\rho$ is not increased. 
Later on, a method to derive this rules will be presented.

Basic increasing rules at layer $k$ are called \emph{deleting basic increasing rules} if $\scond{k}{c}$ is universally bound and \emph{inserting basic increasing rules} if $\scond{k}{c}$ is existentially bound.
For deleting basic increasing rules we introduce the restriction that these rules are only allowed to delete edges but no nodes of $C_{k+1}$ since otherwise it is not possible, given a rule set and a constraint to decide whether this rules set is able to repair an arbitrary graph based only on deleting basic increasing rules. 
For example, consider a rule that deletes a node of $C_{k+1}$.
Then, it is unknown whether this node is connected by edges not belonging to $C_{k+1}$ and it is unclear whether all occurrence of $C_{k+1}$ could be destroyed by $\rho$ since the dangling edge condition could be unsatisfied. 


\begin{definition}[\textbf{basic increasing rule}]\label{def_basicIncreasing}
	Let a constraint $c$ in UANF and a plain rule 
	$\rho = L \xhookleftarrow{l} K \xhookrightarrow{r} R$ 
	be given. Then, $\rho$ is called  \emph{basic increasing w.r.t $c$ at layer $k$} if 

	\begin{enumerate}
		\item If $\scond{k}{c}$ is universally bound: $\rho$ is a non-consistency decreasing rule
		up to layer $k+1$ and there does exist a morphism 
		$i: C_{k+1} \inj L$ such that
		\begin{equation*}
			\begin{split}
				&\exists e \in E_{C_{k+1}} \setminus E_{C_k} \big(\neg \exists e' \in 
				E_K(i(e) = l(e')\big)  \wedge \\
				&\forall v \in V_{C_{k+1}} \setminus V_{C_k}\big(\exists v' \in V_K(i(c)
				 = l(v')\big)
			\end{split}	
		\end{equation*}
		holds. Then, $\rho$ is called a \emph{deleting basic increasing rule}.
\item If $\scond{k}{c}$ is existentially bound: $\rho$ is a non-consistency decreasing rule up to layer $k$ and there does exist a morphism $i: C_{k} \inj L$ such that 
	$$i \not \models \exists (a_k^r: C_{k} \inj C', \true) \wedge 	r \circ l^{-1} \circ i \models \exists(a_k^r:C_{k} \inj C', \true)$$ 
	for any $C'\in \ig{C_{k}}{ C_{k+1}}$.
	Then, $\rho$ is called a \emph{inserting basic increasing rule with $C'$}.
\end{enumerate}
\end{definition}

Again, \ref{def_basicIncreasing} relies on every transformation of a rule $\rho$. 
Therefore, we present an alternative method to determine whether $\rho$ satisfies \ref{def_basicIncreasing} or not by checking that $\rho$ does not delete any edges or nodes of $C_{k+1} \setminus C_k$. 

\begin{lemma}
	let the sets $E$ and $V$ be constructed as introduced in lemma \ref{def:non-decreasing} it holds that
	\begin{alignat*}{2}
		\forall v \in V: \exists v' \in V_K: l(v') = v &\wedge \\
		\forall e \in E: \exists e' \in E_K: l(e') = e
	\end{alignat*}
	Then, $\rho$ is called an \emph{inserting basic increasing rule} and $i$ is called the increasing morphism of $\rho$.
\end{lemma}
\begin{proof}

\end{proof}

\begin{lemma}\label{lemma:equiv_direct_normal}
Let a constraint $c$ in UANF, a basic increasing rule $\rho$ at layer $\kmax +1$ and a transformation $t: G \Longrightarrow_{\rho} H$ be given. 
Then, 
$$t \text{ is consistency increasing w.r.t $c$} \iff t \text{ is direct consistency increasing w.r.t $c$}$$
\end{lemma}
\begin{proof}
The \enquote{$\Longleftarrow$} side of the equivalence holds with lemma \ref{lemma:direct_implies_normal}. 
We need to show that the \enquote{$\Longrightarrow$} side of the equivalence holds. 
Let $t: G \Longrightarrow_{\rho} H$ be a consistency increasing transformation and $\rho$ a basic increasing rule. 
Then, $t$ satisfies (\ref{def_basicIncreasing}) and the satisfaction of (\ref{direct_improving_1}) follows immediately.
Since $\rho$ is also a non-consistency decreasing rule up to layer $\kmax$, $t$ satisfies (\ref{non_consistency_universally}) and (\ref{non_consistency_existentially}), then the satisfaction (\ref{direct_improving_1_2}), (\ref{direct_improving_3}) and (\ref{direct_improving_4}) also follows immediately.
\begin{enumerate}
\item If $\rho$ is a deleting basic increasing rule. 
Since $t$ is a consistency increasing rule there has to exist a morphism $p:C_{\kmax+2} \inj G$ such that $\track_t \circ p$ is not total and with that (\ref{direct_improving_2}) is satisfied. 
\item If $\rho$ is a inserting basic increasing rule there exists an occurrence $p: C_{\kmax + 2} \inj G$ and a graph $C \in \ig{C_{\kmax +2}}{C_{\kmax+3}}$ such that $p \not \models \ic{\kmax +2}{c}{C}$ and $\track_t \circ p \not \models \ic{\kmax +2}{c}{C}$. 
Therefore (\ref{direct_improving_2}) is satisfied. 
\end{enumerate}
In total follows that $t$ is a direct consistency increasing transformation w.r.t $c$.
\end{proof}

For basic increasing rules at layer $j < \kmax+1$ this equality does not hold since these rules are allowed to destroy occurrences of universally bound graphs $C_{j'}$  or to create occurrences of existentially bound graphs $C_{j'}$ with $j < j'< \kmax +1$.
With this results follow that if $t: G\Longrightarrow_{\rho} H$ is a $c$-guaranteeing transformation and $\rho$ is a basic increasing rule at layer $\kmax +1$, $t$ is also a direct increasing rule w.r.t $c$. 

%\subsection{Comparison with other consistency concepts and basic increasing rules}

\begin{figure}
\center
	\begin{tikzpicture}
	
		
		\node(guaran) at (3,4) {$c$-guaranteeing};
		\node(pres) at (6,0) {sustaining w.r.t $c$};
		\node (incr) at (6,2) {direct increasing w.r.t $c$};
		\node (direct) at (9,4) {direct improving w.r.t $c$};
 
		
		\draw[-implies,double equal sign distance] (guaran) --  (incr); 
		\draw[-implies,double equal sign distance] (incr) --  (pres);
		\draw[-implies,double equal sign distance] (direct) --  (incr);
		\draw[-implies,double equal sign distance] (guaran) --  (direct);    
		    
	\end{tikzpicture}
	\caption{consistency relations for basic rules.}\label{fig:consistency_relations_basic}
\end{figure}


Before we continue with the construction of application conditions for basic increasing rules, let us once more compare the notion of direct consistency increasing with the notions of (direct) consistency improving and sustaining and consistency guaranteeing and preserving with the restriction that only basic increasing rules are used. 
The results of these differ from the results of chapter \ref{comp_general} and are displayed in figure \ref{fig:consistency_relations_basic}.
It can be seen that in the case of basic increasing rules  the notions of direct consistency increasing and direct consistency improving are related in the sense that improving implies direct increasing and therefore direct increasing is a finer notion than improving. 

Additionally, with lemma \ref{lemma:equiv_direct_normal} follows immediately that guaranteeing implies direct increasing. 
We show that improving implies (direct) increasing and that (direct) increasing does not imply sustaining which also implies that increasing does not imply improving since each improving transformation is also a sustaining one. 
\begin{lemma}
Let a constraint $c$ in UANF, a basic increasing rule $\rho$ and a transformation $t: G \Longrightarrow_{\rho} H$ be given. 
Then, 
\begin{equation*}
	\begin{split}
		&t \text{ is improving w.r.t $c$ } \implies t \text{ is direct consistency increasing w.r.t $c$ } \wedge \\
		&t \text{ is direct increasing w.r.t $c$ } \centernot \implies t \text{ is consistency sustaining w.r.t $c$}
	\end{split}
\end{equation*}
\end{lemma}

\begin{proof}
\begin{enumerate}
\item Let $t$ be a consistency improving transformation w.r.t $c$. 

\end{enumerate}
\end{proof}

\begin{definition}[\textbf{application conditions for basic increasing rules}]
Let a constraint $c$ in UANF and a basic increasing rule  $\rho = L \xhookleftarrow{l} K \xhookrightarrow{r} R$ w.r.t $c$ at layer $k$ be given. We define the application condition
for $\rho$ as:
 
\begin{enumerate}
\item If $\rho$ is a deleting increasing rule:
$$\apb{j}{\rho} := \begin{cases}
				\bigvee_{P \in \overlay(L,C_k)} \bigwedge_{P' \in \eol(P, a_k)} \exists(i_L^P: L \inj P, \neg \exists(i_P^{P'}: P \inj P', \true)) &\text{if $j = k$} \\
				\false &\text{if $j \neq k$} .
				\end{cases}$$
					
\item If $\rho$ is an inserting increasing rule. Let $a^p:C_k \inj C$ be a partial morphism of $a_k$ , $e: C_k \inj L$ be the increasing morphism of $c$ and $Q$ be set of all overlaps $P$ of $L$ and $C$ such that $i_L^P \circ e \models \exists(a^p:C_k \inj C, \true)$:				
$$\apb{j}{\rho} := \begin{cases}
				\bigvee_{P \in Q} \exists(i_L^P: L \inj P, \true) &\text{, if j = k}\\
				\false	&\text{, if } j \neq k
				\end{cases}$$
\end{enumerate}
\end{definition}

\begin{theorem}\label{api_direct_proof}
	Let a graph $G$, a constraint $c$ and a basic increasing rule $\rho$ at layer
	$\kmax < k \leq \kmax +2$ be given. Then, each transformation 
	$$t: G \Longrightarrow_{\rho',m} H$$ with the rule $\rho' = (\rho, \apb{k}{\rho})$ is
	a direct consistency increasing transformation.
\end{theorem}
\begin{proof}
	Since $\rho$ is a basic increasing rule it is also a non-consistency decreasing rule 
	up to layer $\kmax +2$. Therefore, (\ref{direct_improving_1}) is
	trivially satisfied since $\rho$ satisfies (\ref{non_consistency_extra}), also 
	(\ref{direct_improving_1_2}) is satisfied since $\rho$ satisfies
	(\ref{non_consistency_universally}) also (\ref{direct_improving_3}) and 
	(\ref{direct_improving_4}) are satisfied because $\rho$ satisfies 
	(\ref{non_consistency_universally}) and (\ref{non_consistency_existentially}) for all
	$0 \leq j \leq k+2$.
	It remains to show that (\ref{direct_improving_2}) is satisfied.
	\begin{enumerate}
		\item 
			Let $\rho$ be a deleting basic increasing rule.
			Then, $k = \kmax +1$ and $\apb{k}{\rho} = \bigvee_{P \in \overlay(L,C_k)}
			\bigwedge_{P' \in \eol(P, a_k)} \exists(i_L^P: L \inj P, \neg \exists(i_P^{P'}:
			P \inj P', \true))$.
			If $m \models \apb{j}{\rho}$ there does exist a occurrence $p :C_{k} \inj G$ with 
			$p \not \models \exists(a_{k+1}: C_k \inj C_{k+1}, \true)$ and $p(C_k \setminus 
			C_{k-1}) \cap m(L \setminus K) \neq \emptyset$. If follows that $\track_t \circ p$ 
			is not total and (\ref{direct_improving_2}) is satisfied.
		\item
			Let $\rho$ be a inserting basic increasing rule with $C \in \ig{C_k}{C_{k+1}}$. 
			Then, $k = \kmax +1$ and $\apb{j}{\rho} = \bigvee_{P \in Q} 
			\exists(i_L^P: L \inj P, \true)$. 
			If $m \models \apb{j}{\rho}$, $m \circ e \not \models \exists(a_k^r: C_k \inj 
			C, \true)$ with $e$ being the increasing morphism of $\rho$. 
			Therefore $\track_t \circ m \circ i \models \exists(a_k^r: C_k \inj 
			C, \true)$ and (\ref{direct_improving_2}) is satisfied.
	\end{enumerate}
	In total follows that $t$ is a direct consistency increasing transformation.
\end{proof}












\begin{theorem}\label{caracter_rep_rule_set_at_layer}
Let a constraint $c$ in EANF and a set of rules $\mathcal{R}$   be given. 
Then, $\mathcal{R}$ is a repairing set of $c$ at layer $k \leq \nlvl(c)$ if either \ref{cat_rep_set_1} or \ref{cat_rep_set_2} applies. 

\begin{enumerate}
	\item \label{cat_rep_set_1}
	 For any universally bound graph $C_j$ at layer $j \leq k$ of $c$, $(\rho, \api(j, C_{j+1})) \in \mathcal{R}$ and $\rho$ is a deleting potentially minimal improving rule at layer $j$ with $C_{j+1}$, such that $\rho$ only deletes edges of $C_j$.
	\item \label{cat_rep_set_2}
	A decomposition $\mathbf{P}$ of $C_k$ with $C_{k-1}$ exists, such that for each $P \in \mathbf{P}$ a rule $(\rho, \api(k,P)) \in \mathcal{R}$ exists, such that $\rho$ is an inserting basic improving rule at layer $k$ with $P$.
	
	\item \label{cat_rep_set_3}
		There does exist a sequence of graphs 
			$$C_k = C'_0 \inj C'_1 \inj \ldots \inj C'_n = C_{k+1}$$
		such that a inserting basic increasing rule $\rho_i  = L_i \xhookleftarrow{l_i} K_i \xhookrightarrow{r_i} R_i$ at layer $k$ with $C_i$ exists such that $L_i \inj R_{i-1}$ and $e_i = i_{C{i-1}}^{C_i} \circ e_{i-1}$ with $e_i$ being the increasing morphism of $\rho_i$ for all $i \in \{1, \ldots n\}$.
\end{enumerate}
\end{theorem}

\begin{proof}
Let a constraint $c$ in EANF, a rule set $\mathcal{R}$ and a graph $G$ with $k = \cmax$ and $\cmax < \nlvl(c)$ be given. 
We show that a sequence $G = C_0' \Rightarrow \ldots \Rightarrow C_n' = H$ with rules of $\mathcal{R}$ exists, such that $H \models_{k+2} c$ if \ref{cat_rep_set_1}. or \ref{cat_rep_set_2}.  of theorem \ref{caracter_rep_rule_set_at_layer} is satisfied.
\begin{enumerate}
\item
 Assume that \ref{cat_rep_set_1}. of theorem \ref{caracter_rep_rule_set_at_layer} holds. Let $(\rho, \api(j, C_{j+1})) \in \mathcal{R}$, such that $\rho = L \xhookleftarrow{l} K \xhookrightarrow{r} R$ is a deleting potentially minimal improving rule at layer $j \leq k$ with $C_{j+1}$ and $C_j$ is a universally bound graph of $c$.
Then, $\api(j, C_{j+1}) = \exists\bigl(b: L \inj C_{j}, \neg \exists(a_j:C_{j} \inj C_{j+1}, \true)\bigr)$.
Let $q: C_j \inj G$ be a morphism such that  $q \not \models \exists(a_j: C_j \inj C_{j+1}, \true)$.
Since $L \subseteq C_j$, we can construct a morphism $m_1: L \inj G$ with $m_1 = q \circ b$ and therefore $m_1 \models \api(j,C_{j+1})$.
Since $r$ only deletes edges, a transformation $t: G = G_0 \Rightarrow_{r,m_1} G_1$ exists and $\track_t \circ p$ is not total. 
Because $r$ does not insert any elements of $C_j$: 
$$|\{q: C_k \inj G_0 \mid  q \not \models  d\}| < |\{q: C_k \inj G_1 \mid  q \not \models  d\}|$$
with $d = \exists(a_j: C_j \inj C_{j+1}, \true)$.
By iteratively applying this construction, we can generate a finite sequence of transformations
$$G = G_0 \Rightarrow_{r, m_1} G_1 \Rightarrow_{r, m_2} \ldots \Rightarrow_{r,m_n} G_n = H$$
such that $|\{q: C_k \inj G_n \mid  q \not \models  d\}| = 0$ and therefore $H \models_j c$. 
With lemma \ref{lemma_lay_sat}, $H \models_{k+2} c$ and $H \models c$ follows. 

\item 
Assume that \ref{cat_rep_set_2}. of theorem \ref{caracter_rep_rule_set_at_layer} holds. 
Let $\rho_0 = (\rho, \api(k,P)) \in \mathcal{R}$, such that $\rho$ is an inserting basic improving rule of $c$ at layer $k$ with $P \in \mathbf{P}$. 
Then, $\api(k, P) = \neg \exists(b: L \inj P, \true)$.
Let $q_0 : C_k \inj G$ be a morphism, such that $q_0 \not \models \exists(a'_k: C_k \inj P, \true)$ with $a'_k$ being a partial morphism of $a_k$.
Since $L = C_k$, we set $m_0: C_k \inj G$ with $m_0 = q_0$. 
It follows that $m_0 \models \neg \exists(a'_k: C_k \inj P, \true) = \api(k,P)$.
Because $r$ does not delete any elements, a transformation $t_0: G \Longrightarrow_{r_0,m_0} G_1$ exists and $\track_t \circ q \models \exists(a'_k: C_k \inj P, \true)$.  
We set $q_1 = \track_{t_0} \circ q_0$ and apply the same method to $q_1$. 

By iteratively applying this, we can construct a finite sequence of transformations 
$$G \Longrightarrow_{r_0, m_0} G_0 \Longrightarrow_{r_1,m_1} \ldots \Longrightarrow_{r_n,m_n} G_n$$ 
such that $m_i = \track_{t_{i-1}} \circ \ldots \circ \track_{t_0} \circ m_0$ and $q\models \exists(b_i: C_k \inj P_i, \true)$ for all $P_i \in \mathbf{P}$ with $q = \track_{t_n} \circ q_n$.
Let $p_i: P_i \inj G_n$ be the morphism, such that  $q = p_i \circ b_i$. 

Now, we can construct a morphism $p: C_{k+1} \inj G$ with 
$$ p(e) := \begin{cases}
			p_1(e) &\text{,if } e \in P_1\\
			&\vdots\\
			p_j(e) &\text{,if } e \in P_j.
		\end{cases}$$
		
Let $e \in C_k$, because $q(e) = p_i \circ b_i(e)$ and $q(e) = p_{\ell} \circ b_{\ell}(e)$ and $b_i$ and $b_{\ell}$ are both partial morphisms of $a_k$, it follows that $b_i(e) = b_{\ell} (e)$ and therfore $p_i (e) = p_{\ell} (e)$. 
Because $(P_i \cap P_{\ell})\setminus C_k = \emptyset$ for all $i \neq \ell$, $p$ is a morphism and by the definition of $p$ it follows that 
$q = p \circ a_k$ and therefore $q \models \exists(a_k: C_k \inj C_{k+1}, \true)$.

By iteratively applying this whole construction to all occurrences of $C_k$ that do not satisfy $\exists(a_k:C_k \inj C_{k+1},\true)$ the derived graph graph $H$ does not contain any occurrences of $C_k$ not satisfying $\exists(a_k: C_k \inj C_{k+1},\true)$ and therefore  $H \models_{k+2} c$. 

\end{enumerate}
\end{proof}
\begin{corollary}
If a set of rule $\mathcal{R}$ is a repairing set of $c$ at layer $k \leq \nlvl(c)$ and \ref{cat_rep_set_1}. of theorem \ref{caracter_rep_rule_set_at_layer} applies, then $\mathcal{R}$ is a repairing set of $c$ at layer $j$ for all $k \leq j \leq \nlvl(c)$. 
\end{corollary}

\begin{lemma}
Let a graph $G_0$, a constraint $c$ in EANF and a repairing set $\mathcal{R}$ at layer $\maxc{G_0}  +1$ be given, such that each rule in $ \mathcal{R}$ is a basic increasing.
Then, for every sequence
$$G_0 \Longrightarrow_{\rho_0, m_0} G_1 \Longrightarrow_{\rho, m_0} \ldots \Longrightarrow_{\rho_n, m_n} G_n$$
such that $\rho_i = (\rho, \api(\maxc{G_0} + 1, P)$ with $\rho \in \mathcal{R}$ being a consistency increasing rule at layer $\maxc{G_0} + 1$ with $P$, it holds that 

\end{lemma}

\begin{figure}
\centering
\begin{algorithm}[H]
	%\DontPrintSemicolon
	\KwData{A graph $G$, a constraint $c$ in EANF  and a set of rules $\mathcal{R}$}
	\KwResult{A graph $H$ with $H \models c$}
	extract basic rules of $\mathcal{R}$\;
	\uIf{$\mathcal{R}$ is a repairing set}{
		repair
	}
	\Else{
		repair with basic rules + application condition\;
		repair with normal rules + application condition\;
		generate a basic increasing rule set $\mathcal{R}'$\;
		repair with rules of $\mathcal{R}'$ \;
	}


\caption{ Floyd-Warshall Algorithmus}\label{Algo_floyd}

\end{algorithm}
\end{figure}



