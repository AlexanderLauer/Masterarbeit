\subsection{potentially minimal improving rules}
%
%\begin{definition}[\textbf{potentially minimal improving rule}]
%Let a plain rule $r = L \xhookleftarrow{l} K \xhookrightarrow{r} R$ be given.
%
%Let $E$ be the set of all existentially bound graphs and $U$ be the set of all universally bound graphs $C_j$ of $c$ with $j \leq k$. 
%The rule $r$ is called \emph{potentially minimal improving} at layer $k$, if
%$$(L \setminus K) \cap \bigcup_{C_j \in E} C_j\setminus C_{j-1} = \emptyset$$
%and $$(R \setminus K) \cap \bigcup_{C_j \in U} C_j\setminus C_{j-1} = \emptyset$$ 
%and either
%	$$(L \setminus K) \cap (C_{k+1} \setminus C_k) \neq \emptyset$$
%or 
%$$(R \setminus K) \cap (C_{k+2} \setminus C') \neq \emptyset$$
%\end{definition}
%
%
%\begin{definition}[\textbf{appl. conditions for potentially minimal improving rules}]
%Let a graph $G$, a constraint $c$, with $G \models \parcond(k,c,C')$ and $G \not \models_j c$ for all $j >k$, and a potentially minimal improving rule  $r = L \xhookleftarrow{l} K \xhookrightarrow{r} R$ be given.
%
%$$\ap(k, C) := \bigvee_{P \in \overlay(L, C_{k+1})} \exists(L \inj P, \neg \exists(P \inj Q, \true))$$
%
%$$\ap(k,C) :=  \bigvee_{P \in \overlay(L, C_{k+1})} \exists(L \inj P, \neg \exists(P \inj Q, \true)) \wedge \shift(\exists(R \inj Q', \true),r)$$
%
%
%
%\end{definition}

\begin{definition}[\textbf{potentially minimal improving rule}]
Let a constraint $c$ and a plain rule $r = L \xhookleftarrow{} K \xhookrightarrow{} R$ be given.
The rule $r$ is called \emph{potentially minimal improving} w.r.t $c$ at layer $k$ with $C_k \subseteq C \subseteq C_{k+1}$, if
\begin{equation}\label{pot_1}
	(L \setminus K) \cap C_{k+1} = \emptyset
\end{equation}

and 
\begin{equation}\label{pot_2}
	(R \setminus K) \cap C_k = \emptyset 
\end{equation}

and either \ref{pot_min_impr_1}. or \ref{pot_min_impr_2}. applies. 

\begin{enumerate}
\item \label{pot_min_impr_1} The rule deletes elements of $C_k \setminus C_{k-1}$:
	\begin{equation} \label{pot_del}
		L \subseteq C_{k} \quad \text{with} \quad (L \setminus K)\cap (C_{k} \setminus C_{k-1}) \neq \emptyset
	\end{equation}

\item \label{pot_min_impr_2} The rule creates an instance of an upper-graph of $C$:
	\begin{equation}\label{pot_ins}
		C' \subseteq R \quad \text{with} \quad (R \setminus K) \cap (C_{k+1} \setminus C') \neq \emptyset \quad \text{for any} \quad C' \in \mathcal{U}(C, C_{k+1})
	\end{equation}
If \ref{pot_min_impr_1} applies, $r$ is called a \emph{deleting potentially improving rule}.
If \ref{pot_min_impr_2} applies, $r$ is called an \emph{inserting potentially improving rule}.
\end{enumerate}


\end{definition}



\begin{definition}[\textbf{appl. conditions for potentially minimal improving rules}]
Let a constraint $c$ in EANF and a potentially minimal improving rule  $r = L \xhookleftarrow{} K \xhookrightarrow{} R$ w.r.t $c$ at layer $k$  with $C_k \subseteq C \subseteq C_{k+1}$ be given. 
We define the application condition for $r$ as:

$$\api(j, C) := \begin{cases}
				\exists\bigl(L \inj C_k, \neg \exists(C_k \inj C, \true)\bigr) &\text{, if j = k} \\
				\false &\text{, if } j \neq k.
				\end{cases}$$

\end{definition}

\begin{theorem}
Let a graph $G$, a constraint $c$ in EANF, with $G \models \parcond(\cmax, c, C)$ and $\parcond(\cmax, c, C) \in \mathcal{P}_c^G$, and a potentially minimal improving rule $r = L \xhookleftarrow{} K \xhookrightarrow{} R$ at layer $\cmax$ with $C_{\cmax+1}\subseteq C \subseteq C_{\cmax+2}$ be given. 
Then, the rule $r' = (r, \api(k,C))$ is a direct minimal consistency improving rule. 
\end{theorem}
\begin{proof}
Let $t:G \Longrightarrow_{r',m} H$ be a transformation, $k = \cmax+1$ and $e$ be the subcondition of $c$ at layer $k+1$. 
We show that $t$ is a direct minimal consistency improving transformation and with that, the statement follows.
Firstly, we show that equation (\ref{direct_improving_1}) is satisfied.
Let $p: C_{k} \inj G$ be a morphism. 
If $r$ is a deleting minimal improving rule, either \ref{proof_api_1}. or  \ref{proof_api_2}. applies and if $r$ is a inserting and not  an deleting minimal improving rule, only \ref{proof_api_2}. applies, because $c$ cannot destroy any occurrence of $C_k$. 
\begin{enumerate}
	\item\label{proof_api_1} If $p(C_k) \cap m(L\setminus K) \neq \emptyset$, $\track_t \circ p$ is not total, since at least one element of $p(C_k)$ has been deleted by $t$ and $p$ does satisfy $\bigwedge_{C' \in \mathbf{G}}(p \models \parcond(1,e,C') \wedge \track_t \circ p \text{ is total}) \implies \track_t \circ p \models \parcond(1,e,C'))$. 
	\item\label{proof_api_2} If $p(C_k \cap L) \cap m(L\setminus K) = \emptyset$, $\track_t \circ p$ is total. 
	Because (\ref{pot_1}) holds, $t$ does not delete any elements of $C_{k+1}$ and therefore $p \models \parcond(1,e,C') \implies \track_t \circ p \models \parcond(1,e,C')$ for all $C_k \subseteq C' \subseteq C_{k+1}$.
\end{enumerate} 
With \ref{proof_api_1}. and  \ref{proof_api_2}. follows that (\ref{direct_improving_1}) is satisfied. 

Secondly, we show that equation (\ref{direct_improving_1_2}) is satisfied.
Let $p':C_k \inj H$ be a morphism. 
Because (\ref{pot_2}) is satisfied, $t$ does not create any elements of $C_k$ and therefore, there must exist an morphism $p : C_k \inj G$ with $\track_t \circ p = p'$. If follows that (\ref{direct_improving_1_2}) is satisfied. 

Lastly, we show that equation (\ref{direct_improving_2}) is satisfied. 
Since $\api(k,C)$ is satisfied, a morphism $p: C_k \inj G$ with $p \not \models \exists(C_k \inj C, \true) = \parcond(1,e,C)$ and $p(C_k) \cap m(L) \neq \emptyset$ must exist. 
If $r$ is an deleting improving rule, $t$ deletes at least one element of $C_k$, it follows that $\track_k \circ p$ is not total and therefore (\ref{direct_improving_2}) is satisfied. 
If $r$ is an inserting improving rule and not an deleting one, no element of $C_k$ is deleted by $t$ and therefore $\track_t \circ p$ is total. Because (\ref{pot_ins}) holds, $\track_t \circ p  \models \exists(C_k  \inj C', \true) = \parcond(1,e,C')$ for an $C' \in \mathcal{U}(C, C_{k+1})$ and with that (\ref{direct_improving_2}) is satisfied.   
\end{proof}

\begin{definition}[\textbf{repairing rule set}]
Let a constraint $c$ in EANF and a set of rules $\mathcal{R}$ be given. 
Then, $\mathcal{R}$ is called a \emph{repairing rule set} if for all graphs $G$ a sequence 
$$G = G_0 \Rightarrow_{r_0} \ldots \Rightarrow_{r_{n-1}} G_n = H$$
exists, such that $r_j \in \mathcal{R}$ for all $j \in \{0,\ldots, n-1\}$ and $H \models c$. 

\end{definition}

\begin{lemma}
Let a constraint $c$ in EANF w and a set of rules $\mathcal{R}$ be given. 
Then $\mathcal{R}$ is a repairing set for $c$ if either \ref{categories_repair_set_1}. applies for every layer of $c$ or \ref{categories_repair_set_2}. applies.
\begin{enumerate}
\item \label{categories_repair_set_1}
Let $C_j$ be a existentially bound graph of $c$.
A set of graphs $$C_{k-1} \subset C_0' \subset \ldots \subset C_n' = C_k$$ exists, such that $\mathcal{R}$ contains a inserting potentially minimal improving rule at layer $j$ with $C_{\ell}$ with $\ell = 0, \ldots n$.
\item \label{categories_repair_set_2}For one universally bound graph $C_j$ of $c$, $\mathcal{R}$ contains a deleting potentially minimal improving rule and for every existentially bound graph $C_{\ell}$ with $\ell < k$ \ref{categories_repair_set_1} applies. 
\end{enumerate}
\end{lemma}