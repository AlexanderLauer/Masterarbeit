\subsection{potentially minimal improving rules}
%
%\begin{definition}[\textbf{potentially minimal improving rule}]
%Let a plain rule $r = L \xhookleftarrow{l} K \xhookrightarrow{r} R$ be given.
%
%Let $E$ be the set of all existentially bound graphs and $U$ be the set of all universally bound graphs $C_j$ of $c$ with $j \leq k$. 
%The rule $r$ is called \emph{potentially minimal improving} at layer $k$, if
%$$(L \setminus K) \cap \bigcup_{C_j \in E} C_j\setminus C_{j-1} = \emptyset$$
%and $$(R \setminus K) \cap \bigcup_{C_j \in U} C_j\setminus C_{j-1} = \emptyset$$ 
%and either
%	$$(L \setminus K) \cap (C_{k+1} \setminus C_k) \neq \emptyset$$
%or 
%$$(R \setminus K) \cap (C_{k+2} \setminus C') \neq \emptyset$$
%\end{definition}
%
%
%\begin{definition}[\textbf{appl. conditions for potentially minimal improving rules}]
%Let a graph $G$, a constraint $c$, with $G \models \parcond(k,c,C')$ and $G \not \models_j c$ for all $j >k$, and a potentially minimal improving rule  $r = L \xhookleftarrow{l} K \xhookrightarrow{r} R$ be given.
%
%$$\ap(k, C) := \bigvee_{P \in \overlay(L, C_{k+1})} \exists(L \inj P, \neg \exists(P \inj Q, \true))$$
%
%$$\ap(k,C) :=  \bigvee_{P \in \overlay(L, C_{k+1})} \exists(L \inj P, \neg \exists(P \inj Q, \true)) \wedge \shift(\exists(R \inj Q', \true),r)$$
%
%
%
%\end{definition}

\begin{definition}[\textbf{potentially minimal improving rule}]
Let a constraint $c$ and a plain rule $r = L \xhookleftarrow{} K \xhookrightarrow{} R$ be given.
The rule $r$ is called \emph{potentially minimal improving} w.r.t $c$ at layer $k$ with $C_k \subset P \subseteq C_{k+1}$ and $k \in \{1,3, \ldots , \nl(c)\}$, if
\begin{equation}\label{pot_1}
	(L \setminus K) \cap (C_{k-1} \cup (C_{k+1} \setminus C_k)) = \emptyset
\end{equation}
	
and 
\begin{equation}\label{pot_2}
	(R \setminus K) \cap C_k = \emptyset 
\end{equation}

and either \ref{pot_min_impr_1}. or \ref{pot_min_impr_2}. applies. 

\begin{enumerate}
\item \label{pot_min_impr_1} The rule $r$ deletes  elements of $C_k \setminus C_{k-1}$:
	\begin{equation}
				 L \subseteq C_k \wedge L \setminus K \neq \emptyset
	\end{equation}
	Then, $r$ is called a \emph{deleting potentially minimal improving rule}.

\item \label{pot_min_impr_2} The rule $r$ creates an instance of $P$:
	\begin{equation}\label{pot_ins}
		L = C_k \wedge P \subseteq R 
	\end{equation}
	Then, $r$ is is called an \emph{inserting potentially minimal improving rule}.
\end{enumerate}


\end{definition}



\begin{definition}[\textbf{appl. conditions for potentially minimal improving rules}]
Let a constraint $c$ in EANF and a potentially minimal improving rule  $r = L \xhookleftarrow{} K \xhookrightarrow{} R$ w.r.t $c$ at layer $k$  with $C_{k} \subseteq P \subseteq C_{k+1}$ be given. 
We define the application condition for $r$ as:
 
\begin{enumerate}
\item If $r$ is a deleting potentially minimal improving rule:
$$\api(j, P) := \begin{cases}
				\exists\bigl(a_0: L \inj C_{k}, \neg \exists(a_1:C_{k} \inj C_{k+1}, \true)\bigr) &\text{, if j = k} \\
				\false &\text{, if } j \neq k.
				\end{cases}$$

\item If $r$ is an inserting potentially minimal improving rule:				
$$\api(j, P) := \begin{cases}
				\neg \exists(b: L \inj P, \true) &\text{, if j = k}\\
				\false	&\text{, if } j \neq k
				\end{cases}$$

\end{enumerate}



\end{definition}

\begin{theorem}\label{api_direct_proof}
Let a graph $G$, a constraint $c$ in EANF, with $G \models \parcond(\cmax, c, C)$ and $\parcond(\cmax, c, C) \in \mathcal{P}_c^G$, and a potentially minimal improving rule $r = L \xhookleftarrow{} K \xhookrightarrow{} R$ at layer $\cmax$ with $C_{\cmax+1}\subseteq P \subseteq C_{\cmax+2}$ be given. 
Then, $r' = (r, \api(\cmax+1,P))$ is a direct minimal consistency improving rule. 
\end{theorem}
\begin{proof}
Let $t:G \Longrightarrow_{r',m} H$ be a transformation, $k = \cmax+1$ and $e$ be the subcondition of $c$ at layer $k+1$. 
We show that $t$ is a direct minimal consistency improving transformation.
Firstly, we show that equation (\ref{direct_improving_1}) is satisfied.
Let $p: C_{k} \inj G$ be a morphism. 
If $r$ is a deleting potentially minimal improving rule, either \ref{proof_api_1}. or  \ref{proof_api_2}. applies, if $r$ is an inserting and not  a deleting potentially  minimal improving rule, only \ref{proof_api_2}. applies, because $r$ cannot destroy any occurrences of $C_k$ in $G$. 
\begin{enumerate}
	\item\label{proof_api_1} If $p(C_k) \cap m(L\setminus K) \neq \emptyset$, $\track_t \circ p$ is not total, since at least one element of $p(C_k)$ has been deleted by $t$ and $p$ does satisfy $\bigwedge_{C' \in \mathbf{G}}(p \models \parcond(1,e,C') \wedge \track_t \circ p \text{ is total}) \implies \track_t \circ p \models \parcond(1,e,C'))$. 
	\item\label{proof_api_2} If $p(C_k) \cap m(L\setminus K) = \emptyset$, $\track_t \circ p$ is total, since no element of $p(C_k)$ has been deleted by $t$.  
	Because (\ref{pot_1}) holds, $t$ does not delete any elements of $C_{k+1}\setminus C_k$ and therefore $p \models \parcond(1,e,C') \implies \track_t \circ p \models \parcond(1,e,C')$ for all $C_k \subseteq C' \subseteq C_{k+1}$.
\end{enumerate} 
With \ref{proof_api_1}. and  \ref{proof_api_2}. follows that (\ref{direct_improving_1}) is satisfied. 

Secondly, we show that equation (\ref{direct_improving_1_2}) is satisfied.
Let $p':C_k \inj H$ be a morphism. 
Because (\ref{pot_2}) is satisfied, $t$ does not create any elements of $C_k$ and therefore, there must exist an morphism $p : C_k \inj G$ with $\track_t \circ p = p'$. If follows that (\ref{direct_improving_1_2}) is satisfied. 

Lasty, we show that (\ref{direct_improving_2}) is satisfied. 
We consider the cases that $r$ is a deleting minimal potentially improving rule and that $r$ is an inserting minimal potentially improving rule. 

\begin{enumerate}
\item If $r$ is a deleting potentially minimal improving rule, the condition $\api(k,P) = \exists\bigl(a_0: L \inj C_{k}, \neg \exists(a_1:C_{k} \inj C_{k+1}, \true)\bigr)$ is satisfied by $m$. 
Therefore a morphism $p :C_k \inj G$ with $p \not \models \neg \exists(a_1:C_k \inj C_{k+1}, \true) = \parcond(1,e,C_{k+1})$ and $m = p \circ a_0$ must exist. 
Since $r$ is a deleting rule, at least one element of $p(C_k)$ has been deleted by $t$ and therefore $\track_t \circ p$ is not total. 
It follows that (\ref{direct_improving_2}) is satisfied. 

\item 
If $r$ is a inserting potentially minimal improving rule, $\api(k,P) = \neg \exists(b: L \inj P, \true)$ is satisfied by $m$. 
Because $L = C_k$, $m \models \neg \exists(b: C_k \inj P, \true) = \parcond(1,e,P)$.
Since (\ref{pot_ins}) is satisfied, $\track \circ p$ is total and $\track_t \circ p \models \parcond(1,e,P)$. Therefore, (\ref{direct_improving_2}) is satisfied. 
In total follows that $t$ is a direct minimal consistency improving transformation and therefore $r'$ is direct minimal consistency improving rule. 
\end{enumerate}
\end{proof}

\begin{definition}[\textbf{repairing rule set}]
Let a constraint $c$ in EANF and a set of rules $\mathcal{R}$ be given. 
Then, $\mathcal{R}$ is called a \emph{repairing rule set for $c$ at layer $k$}  if for all graphs $G$ with $k = \cmax$ a sequence 
$$G = G_0 \Rightarrow_{r_0} \ldots \Rightarrow_{r_{n-1}} G_n = H$$
exists, such that $r_j \in \mathcal{R}$ for all $j \in \{0,\ldots, n-1\}$ and $H \models_{k+2} c$. 

\end{definition}

\begin{corollary}
	Let a constraint $c$ in EANF and a set of rules $\mathcal{R}$ be given. 
	If $\mathcal{R}$ is a repairing rule set for $c$ at layer $k$, $\mathcal{R}$ is a repairing rule set w.r.t $c$ at layer $j$ for all $k < j \leq \nl(c)$. 
\end{corollary}
\begin{corollary}
	Let a constraint $c$ in EANF and a repairing rule set $\mathcal{R}$ for $c$ at layer $k$, for all $k\in \{1,3,\ldots, \nl(c)\}$, be given. 
	Then, for all graphs $G$, a sequence
	$$G = G_0 \Rightarrow_{r_0} \ldots \Rightarrow_{r_{n-1}} G_n = H$$
	exists, such that $r_j \in \mathcal{R}$ for all $j \in \{0, \ldots, n-1\}$ and $H \models c$. 
\end{corollary}

\begin{theorem}\label{carachter_rep_rule_set_at_layer}
	Let a constraint $c$ in EANF and a set of rules $\mathcal{R}$ be given. 
	Then, $\mathcal{R}$ is a repairing set of $c$ at layer $k \leq \nl(c)$ if either \ref{category_rep_set_at_layer_1}. or \ref{category_rep_set_at_layer_2}. holds.
	\begin{enumerate}
		\item\label{category_rep_set_at_layer_1} For any universally bound graph $C_j$ at layer $j \leq k$ of $c$, $(r, \api(j, C_{j+1})) \in \mathcal{R}$ and $r= L \xhookleftarrow{} K \xhookrightarrow{} R$ is a deleting potentially minimal improving rule at layer $j$ with $C_{j+1}$, such that $r$ only deletes edges of $C_j$:
			$$(L \setminus K) \cap V_{C_k} = \emptyset$$ 
		\item\label{category_rep_set_at_layer_2}  Let $C_{k+1}$ be the existentially bound graph of $c$ at layer $k+1$. A set of graphs 
		$$ \mathbf{P} \subseteq \{ P \mid C_k \subseteq P \subseteq C_{k+1}\},$$
		such that the inclusions $i_{\ell} : P_{\ell} \inj C_{k+1}$ for all $P_{\ell} \in \mathbf{P}$ are jointly surjective, 
		exists, and for all $P \in \mathbf{P}$, a rule $(r, \api(k, P)) \in \mathcal{R}$ exists, such that $r = L \xhookleftarrow{} K \xhookrightarrow{} R$ is an inserting potentially minimal improving rule at layer $k$ with $P$ and $L = K$. 
	\end{enumerate}
\end{theorem}


\begin{proof}
Let a constraint $c$ in EANF, a rule set $\mathcal{R}$ and a graph $G$ with $k = \cmax$ and $\cmax < \nl(c)$ be given. 
We show that a sequence $G = C_0' \Rightarrow \ldots \Rightarrow C_n' = H$ with rules of $\mathcal{R}$ exists, such that $H \models_{k+2} c$ if \ref{category_rep_set_at_layer_1}. or \ref{category_rep_set_at_layer_2}.  of theorem \ref{carachter_rep_rule_set_at_layer} is satisfied.
\begin{enumerate}
\item Assume that \ref{category_rep_set_at_layer_1}. of theorem \ref{carachter_rep_rule_set_at_layer} holds. Let $(r, \api(j, C_{j+1})) \in \mathcal{R}$, such that $r = L \xhookleftarrow{} K \xhookrightarrow{} R$ is a deleting potentially minimal improving rule at layer $j \leq k$ with $C_{j+1}$ and $C_j$ is a universally bound graph of $c$.
Then, $\api(j, C_{j+1}) = \exists\bigl(a_0: L \inj C_{j}, \neg \exists(a_1:C_{j} \inj C_{j+1}, \true)\bigr)$.
Let $q: C_j \inj G$ be a morphism such that  $q \not \models \exists(C_j \inj C_{j+1}, \true)$.
Since $L \subseteq C_j$, we can construct a morphism $m_1: L \inj G$ with $m_1(L) = q(L)$.
It holds that $m_1 = q \circ a_0$, therefore $m_1 \models \api(j,C_{j+1})$.
Since $r$ only deletes edges, a transformation $t: G = G_0 \Rightarrow_{r,m_1} G_1$ exists and $\track_t \circ p$ is not total. 
Because $r$ does not insert any elements of $C_j$: 
$$|\{q: C_k \inj G_0 \mid  q \not \models  d\}| < |\{q: C_k \inj G_1 \mid  q \not \models  d\}|$$
with $d = \exists(C_j \inj C_{j+1}, \true)$.
By iteratively applying this construction, we generate a finite sequence of transformations
$$G = G_0 \Rightarrow_{r, m_1} G_1 \Rightarrow_{r, m_2} \ldots \Rightarrow_{r,m_n} G_n = H$$
such that $|\{q: C_k \inj G_n \mid  q \not \models  d\}| = 0$ and therefore $H \models_j c$. 
With lemma \ref{lemma_lay_sat}, $H \models_{k+2} c$ and $H \models c$ follows. 

\item 
Assume that \ref{category_rep_set_at_layer_2}. of theorem \ref{carachter_rep_rule_set_at_layer} holds. 
Let $r_0 = (r, \api(k,P)) \in \mathcal{R}$ be an inserting potentially minimal improving rule of $c$ at layer $k$ with $P \in \mathbf{P}$. 
Then, $\api(k, P) = \neg \exists(b: L \inj P, \true)$.
Let $q : C_k \inj G$ be a morphism, such that $q \not \models \exists(C_k \inj P, \true)$.
Since $L = C_k$, we set $m: C_k \inj G$ with $m_0 = q$. 
It follows that $m_0 \models \neg \exists(C_k \inj P, \true) = \api(k,P)$.
Because $r$ does not delete any elements, a transformation $t_0: G \Longrightarrow_{r_0,m_0} G_1$ exists and $\track_t \circ q \models \exists(C_k \inj P, \true)$.  
By iteratively applying this construction with rules $(r, \api(k,P)) \in \mathcal{R}$ such that $r $ is a inserting potentially minimal improving rule of $c$ at layer $k$ with $P \in \mathbf{P}$ we can construct a finite sequence of transformations
$$G \Longrightarrow_{r_0, m_0} G_0 \Longrightarrow_{r_1,m_1} \ldots \Longrightarrow_{r_n,m_n} G_n$$ 
such that $m_i = \track_{t_{i-1}} \circ \ldots \circ \track_{t_0} \circ m_0$ and $G_n \models \exists(C_k \inj P_j, \true)$ for all $P_j \in \mathbf{P}$. 
Because the inclusions $t_{\ell}: P_{\ell} \inj C_{k+1}$ for all $P_{\ell} \in \mathbf{P}$ are jointly surjective, $\track_{t_n} \circ \ldots \circ q \models \exists(C_k \inj C_{k+1}$ follows. 
Therefore, $G_n \models_{k+2} c$. 
\end{enumerate}
\end{proof}


