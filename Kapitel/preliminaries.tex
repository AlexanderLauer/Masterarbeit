\section{Preliminaries}

\begin{definition}[\textbf{subgraph}]
Let $G_1$ and $G_2$ be graphs. 
The graph $G_2$ is called a \emph{subgraph} of $G_1$ if an injective morphism $f: G_2 \to G_1$ exists. 
We use the notation $G_2 \subseteq G_1$ if $G_2$ is a subgraph of $G_1$ and $G_2 \subset G_1$ if $f$ is not bijective. 

\end{definition}

\begin{definition}[\textbf{uppergraph}]
Let $G_1$ and $G_2$ be graphs with $G_1 \subseteq G_2$. 
A graph $C$ is called an \emph{upper-graph} of $G_1$ w.r.t $G_2$, if $G_1 \subset C \subseteq G_2$ 
The set of uppergraphs of $G_1$ w.r.t. $G_2$ is denoted by $\mathcal{U}(G_1,G_2)$. If $G_1 = G_2$, we set $\mathcal{U}(G_1,G_2) = \{G_1\}$.

\end{definition}

\begin{definition}[\textbf{overlap}]
Let $G$ and $G'$ be graphs. 
A graph $H$ is called an \emph{overlap of $G$ and $G'$} if morphisms $p: G \inj H$ and $p' :G' \inj H$ such that $p$ and $p'$ are jointly surjective. 
The set of all overlaps of $G$ and $G'$ is denoted by $\overlay(G,G')$.
\end{definition}
\begin{definition}[\textbf{overlap at morphism}]
Let $C$,$G$ and $C'$ with $C \subset C'$ be graphs and $p: C \inj G$ a morphism. 
A graph $H$ is called an \emph{overlap of $G$ and $C'$ at $p$} if a morphism $p': C' \inj H$ with $p'|_C = p$ exists.
The set of all overlaps of $G$ and $C'$ at $p$ is denoted by $\overlay_p(G, C')$.

\end{definition}


\begin{definition}[\textbf{partial morphism}]
Let $f : G_1 \to G_2$ and $g: G_3 \to G_4$ be morphisms. 
The morphism $g$ is called a \emph{partial morphism} of $f$ if $G_3 \subseteq G_1$, $G_4 \subseteq G_2$ and $f|_{G_3} = g$.

\end{definition}



\begin{definition}[\textbf{nested graph condition}]
A \emph{graph condition} over a graph $C_0$ is inductively defined as follows:

\begin{itemize}
	\item \textsf{true} is a graph condition over every graph.
	\item $\exists(a:C_0 \xhookrightarrow{} C_1,d)$ is a graph condition over $C_0$ if $a$ is a injective graph morphism and $d$ is a graph condition over $C_1$.
	\item $\neg d$ is a graph condition over $C_0$ if $d$ is a graph condition over $C_0$.
	\item $d_1 \wedge d_2$ and $d_1 \vee d_2$ are graph conditions over $C_0$ if $d_1$ and $d_2$ are graph conditions over $C_0$.

\end{itemize}
	Conditions over the empty graph $\emptyset$ are called \emph{constraints}.
	Every injective morphism $p :C_0 \xhookrightarrow{} G$ satisfies \textsf{true}. 
	An injective morphism $p$ satisfies $\exists(a:C_0 \xhookrightarrow{} C_1,d)$ if there exists an injective morphism  $q : C_1 \xhookrightarrow{} G$ such that $q \circ a = p$ and $q$ satisfies $c$. 
	An injective morphism satisfies $\neg d$ if it does not satisfy $d$, it satisfies $d_1 \wedge d_2$ if it satisfies $d_1$ and $d_2$ and it satisfies $d_1 \wedge d_2$ if it satisfies $d_1$ or $d_2$. 
	A graph $G$ satisfies a constraint $c$, $G \models c$, if $p : \emptyset \xhookrightarrow{} G$ satisfies $c$.
We use the abbreviation $\forall(a:C_0 \xhookrightarrow{} C_1,d) := \neg \exists(a:C_0 \xhookrightarrow{} C_1,\neg d)$.

The \emph{nesting level} $\nl$ of a condition is defined as $\nl(\textsf{true} = 0$ and  $\nl(\exists(a: P \to Q, d)) := \nl(d) +1$.
\end{definition}

\begin{definition}[\textbf{alternating quantifier normal form (ANF)}\cite{sandmann2019rule}]
A graph condition $c$ is in \emph{alternating normal form} (ANF) if it is of the form 
$$c = Q(a_1 : C_0 \xhookrightarrow{} C_1, \overline{Q}(a_2:C_1 \xhookrightarrow{} C_2, Q(a_3 : C_2 \xhookrightarrow{} C_3, \overline{Q}(a_4 : C_3 \xhookrightarrow{} C_4, \ldots))))$$
with $Q \in \{ \exists, \forall\}$ and $\overline{Q} = \exists$ if $Q = \forall$, $\overline{Q} = \forall$ if $Q = \exists$.
\end{definition}

