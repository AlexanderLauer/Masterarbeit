\subsection{Basic Increasing and Maintaining Rules}
 
The construction of the application conditions introduced in the previous section, aswell as the constructed application conditions itself, are very complex.
For a certain set of rules, which we will call \emph{basic consistency increasing rules},
for which, we are able to construct application conditions with the same property,
namely that a rule equipped with this application condition is consistency increasing at layer, in a less complex manner.  
The main idea is, that these rules are (a) not able to delete occurrences of existentially or insert occurrences of universally bound graphs and (b) are able to increase consistency at a certain layer. 
That means, given a basic increasing rule $\rho$, there does exist a transformation $t: G \Longrightarrow_{\rho} H$ such that $t$ is a consistency increasing transformation w.r.t to a constraint $c$.

To ensure that (a) is met, we firstly introduce \emph{basic consistency maintaining rules up to layer}, which means that, given a constraint, a plain rule is not able 
to delete existentially bound and insert universally bound graphs up to a certain layer. 
For the definition, we use the notion of consistency maintaining rules up to layer. 
The set of basic consistency maintaining rules up to layer is indeed a subset of the 
set of consistency maintaining rules up to layer since these rules have to be plain rules, whereas consistency maintaining rules up to layer are allowed to be 
equipped with application conditions, i.e. $\maintaining(\cdot,\cdot)$.


\begin{definition}[\textbf{basic consistency maintaining rule up to layer}]
	Let a plain rule $\rho$ and a constraint $c$ in UANF be given.
	Then, $\rho$ is called \emph{basic consistency maintaining up to layer
	$-1 \leq k < \nlvl(c)$ w.r.t. $c$} if $\rho$ is a consistency maintaining 
	rule at layer $k$.
\end{definition}

\begin{example}
	Consider the rules \emph{\texttt{moveFeature}} and 
	\emph{\texttt{assignFeature}} given in Figure \ref{fig:rules} and constraint
	$c_1$ given in Figure \ref{fig:constraints}.
	The rule \emph{\texttt{assignFeature}} is a basic consistency maintaining 
	rule up to layer $1$ w.r.t. $c$, whereas \emph{\texttt{moveFeature}} is 
	not a basic consistency maintaining rule.
\end{example}

Since infinite many transformations via a plain rule $\rho$ exist, it is impossible to check whether $\rho$ is a basic consistency maintaining rule up to layer based on the definition above. Therefore, we present a characterisation of basic consistency maintaining rules, which only relies on $\rho$ itself. 
First, let us assume that $\rho$ is able to create occurrences of a universally bound graph $C_j$. 
This is possible if (a) $\rho$ does insert an edge of $C_j \setminus C_{j-1}$ which connects already existing nodes of $C_j$, since it is unclear whether this would create a new occurrence of $C_j$, or (b) if $\rho$ does insert a node $v$ of $C_j$ such that all edges $e \in E_{C_j}$ with $\src(e) = v$ or $\tar(e) = v$ are also inserted. 
If at least one of these edges is not inserted, it is guaranteed that this insertion does not create an occurrence of $C_j$ since $v$ is only connected to edges that have also been inserted by $\rho$.

Second, let us assume that $\rho$ is able to delete occurrences of a existentially bound graph $C_j$.
This is possible if (a) $\rho$ does delete an edge of $C_j \setminus C_{j-1}$ or (b) $\rho$ deletes a node $v$ of $C_j \setminus C_{j-1}$ such that all edges $e \in E_{C_j}$ with $\src(e) = v$ or $\tar(e) = v$ are also deleted. 
If $\rho$ deletes a node $c$ of $C_{j} \setminus C_{j-1}$ without all its connected edges in $C_j$ there does not exist a transformation via $\rho$ such 
that an occurrence of $C_j$ is deleted by deleting this node since the dangling edge condition would not be satisfied. 
A rule satisfying this properties does not decrease the satisfaction at layer. 
Additionally, we have to ensure that the number of violations will not be 
increased. For this, we have to check that $\rho$ is not able to insert 
occurrences of the corresponding universally bound, as described above, and
that $\rho$ is not able to remove occurrence of any intermediate graph. 
This is only ensured, if $\rho$ does not remove any elements of $C'$ if the 
set of intermediate graphs is given by $\ig{C}{C'}$.

To check that a plain rule satisfies these properties, we make use of the dangling edge condition, or in other words, we check that the rule is not applicable at 
certain overlaps of $L$ and an existentially bound graph or that the inverse
rule is not applicable at certain overlaps of $R$ and an universally bound graph.

\begin{lemma}\label{def:non-decreasing}
	Let a plain rule $\rho = L \xhookleftarrow{l} K \xhookrightarrow{r} R$ and a 
	constraint $c$ in EANF be given. Then, $\rho$ is a basic consistency 
	maintaining rule up to layer $-1 \leq k < \nlvl(c)$ w.r.t $c$ if 
	\ref{category:non-decreasing_1} and \ref{category:non-decreasing_2} apply 
	for all $k$  and \ref{category:non-decreasing_3} applies if $k \leq \nlvl(c) 
	-3$ and $\scond{k}{c}$ is existentially bound, i.e. $k$ is odd.  
	\begin{enumerate}
		\item \label{category:non-decreasing_1}
			Let $E$ be the set of all existentially bound graphs $C_j$ with
			$0 \leq j \leq k+1$.
			For each graph $C \in E$ and each overlap $P \in \overlay(L,C)$ 
			with $i_L^P(L \setminus K) \cap i_C^P(C) \neq \emptyset$, 
			the transformation
			$$t: P \Longrightarrow_{\rho, i_L^P} H$$ does not exist.


		\item \label{category:non-decreasing_2}
			Let $U$ be the set of all universally bound graphs $C_j$ with
			$0 \leq j \leq k+2$.
			For each graph $C \in U$ and each overlap $P \in \overlay(L,C)$ 
			with $i_R^P(R \setminus K) \cap i_C^P(C) \neq \emptyset$ 
			the transformation
			$$t: P \Longrightarrow_{\rho^{-1}, i_R^P} H$$ does not exist.		  

		\item \label{category:non-decreasing_3}
			For all graphs $P \in \overlay(L, C_{k+3})$ it holds that
			$$i_L^P(L \setminus K) \cap i_{C_{k+3}}^P(C_{k+3} \setminus C_{k+1})
			\neq \emptyset
			$$
\end{enumerate}

\end{lemma}
\begin{proof}
	Let $\rho = L \xhookleftarrow{l} K \xhookrightarrow{r} R$ be a rule 
	that satisfies the characterisations listed in Lemma 
	\ref{def:non-decreasing} with $-1 \leq k < \nlvl(c)$. Let us assume that 
	$\rho$ is not a consistency maintaining rule up to layer $k$ w.r.t. $c$.
	Then, there exists a transformation $t: G \Longrightarrow_{\rho} H$ with 
	$\kmax{c}{G} = k$ such that $t$ is not consistency maintaining w.r.t. $c$.
	Therefore, either $\kmax{c}{G}> \kmax{c}{H}$ or $\nv{k+1}{G} > \nv{k+1}{H}$.
	\begin{enumerate}
		\item 
			If $\maxk{c}{G}> \maxk{c}{H}$, either an universally bound graph
			$C_j$ with $j \leq k$ has been inserted or an existentially bound
			graph $C_{j'}$ with $j' \leq k+1$ has been removed.
			
			If the first case applies, there does exist an overlap $P \in 
			\overlay(R, C_j)$ with $i_R^P(R \setminus K) \cap i_{C_j}^P({C_j}) 
			\neq \emptyset$ such that $n \models \exists(i_R^P: R \inj P, \true)$
			Since $\rho$ is a plain rule, the transformation
			$t: P \Longrightarrow_{\rho^{-1}, i_R^P} H'$ must exist. 
			
			If the second case applies, there does exist an overlap $P \in 
			\overlay(L, C_{j'})$ with $i_L^P(L \setminus K) \cap i_{C_{j'}}^P(C) 
			\neq \emptyset$ such that $m \models \exists(i_L^P: L \inj P, \true)$
			Since $\rho$ is a plain rule, the transformation
			$t: P \Longrightarrow_{\rho, i_L^P} H'$ must exist. 
			
			In both cases, this is a contradiction.
		\item
			If $\nv{k+1}{G} < \nv{k+1}{H}$, this case only 
			occurs if $\scond{k}{c}$ is existentially bound since otherwise 		
			$\kmax{c}{G} = \nlvl(c)-1$ and a non-maintaining transformation would 
			lead to a decrease of the satisfaction at layer. Then, either an 
			occurrence of $C_{k+2}$ has been inserted or an occurrence $p: 
			C_{k+2} \inj G$ exists such that $p \models \ic{\scond{k+2}{c}}{C}$
			and $\track_t \circ p \not \models \ic{\scond{k+2}{c}}{C}$ for any
			$C \in \ic{C_{k+2}}{C_{k+3}}$.
			
			In the first case we can show, similar to above, that an overlap
			$P \in \overlay(R,C_{k+2})$ with $i_R^P(R\setminus K) \cap 
			i_{C_{k+2}}^P(C_{k+2}) \neq \emptyset$ such that the transformation
			$t: P \Longrightarrow_{\rho^{-1}, i_R^P} H'$ must exist.
			
			In the second case, an overlap $P \in \overlay(L, C_{k+3}$ with 
			$i_L^P(L \setminus K) \cap i_{C_{k+2}}^P(C_{k+2} \setminus C_{k+1})
			\neq \emptyset$ must exist, since otherwise, the occurrence of $C$ 
			can not be removed by $\rho$.
			
			Again, in both cases, this is a contradiction. 
	\end{enumerate}	    
	In total follows that $\rho$ is a basic consistency maintaining rule up to 
	layer $k$.
\end{proof}

Now, we are ready to introduce \emph{basic increasing rules at layer $k$} with $k$ being odd. The set of basic increasing rules is a subset of the set of maintaining rules at layer $k$ which ensures that the satisfaction at layer as well as the number of violation will not be decreased.
Additionally, the left-hand side of these rule do contain an occurrence $p$ of the  universally bound graph $C_{k+2}$, such that this occurrence either will be 
removed,  i.e. elements of $C_{k+2}\setminus C_{k+1}$ will be deleted, or an intermediate graph $C \in \ig{C_{k+2}}{C_{k+3}}$ will be inserted.
Of course, this second case only occurs if $k < \nlvl(c) -3$ with $c$ being the respective constraint. 
This property yields the advantage that application conditions for basic increasing rules are less complex and smaller, since it can be exactly determined how this rule removes a violation  and therefore, no overlaps have to be considered.

This, on first sight seems like a restriction of the set of basic increasing rules but the context of each rule $\rho$ that does satisfy all properties of a basic increasing rule excluding that $C_{k+2}$ is a sub-graph of the left-hand side can be expanded such that $\rho$ is a basic increasing rule and the semantic of $\rho$ is not increased. 
Later on, a method to derive this rules will be presented.

Basic increasing rules at layer $k$ are called \emph{deleting basic increasing rules} if $p$ will be removed  and \emph{inserting basic increasing rules} if $\scond{k}{c}$ an intermediate graph will be inserted.
For our repairing process, we will introduce the restriction that deleting basic increasing rules are only allowed to delete edges but no nodes of $C_{k+2}$ since otherwise it is not possible, given a rule set and a constraint to decide whether this rules set is able to repair an arbitrary graph based only on deleting basic increasing rules. 
For example, consider a rule that deletes a node of $C_{k+2}$.
Then, it is unknown whether this node is connected to nodes not belonging to $C_{k+2}$ and it is unclear whether all occurrence of $C_{k+2}$ could be destroyed by $\rho$ since the dangling edge condition might be unsatisfied. 


\begin{definition}[\textbf{basic increasing rule}]\label{def_basicIncreasing}
	Let a constraint $c$ in UANF and a consistency maintaining rule 
	$\rho = (\ac,  L \xhookleftarrow{l} K \xhookrightarrow{r} R)$ up to layer 
	$-1 \leq k < \nlvl(v) -1$, with $k$ odd, be given. Then, $\rho$ is called  
	\emph{basic increasing w.r.t $c$ at layer 
	$k$} if a morphism $p: C_{k+2} 
	\inj L$, called the \emph{increasing morphism}, exists such that either 
	\ref{incr_1} or \ref{incr_2} applies.
	
	\begin{enumerate}
		\item \label{incr_1}
			$r \circ l^{-1} \circ p$ is not total. Then, $\rho$ is called
			a \emph{deleting basic increasing rule}.
		\item \label{incr_2}	
			If $k < \nlvl(c)-2$, there does exist an intermediate graph
			$C \in \ig{C_{k+2}}{C_{k+3}}$ such that $p \not \models \exists(
			a_{k+2}^r:C_{k+2} \inj C, \true)$, $r \circ l^{-1} \circ p$ is 
			total and 
			$r \circ l^{-1} \circ p \models \exists(a_{k+2}^r:C_{k+2} \inj C,
			\true)$. Then, $\rho$ is called a \emph{inserting basic increasing 
			rule with $C$}. 
	\end{enumerate} 
\end{definition}

\begin{example}
	Consider the rule \emph{\texttt{assignFeature}} given in Figure 
	\ref{fig:rules} and constraint $c_1$ given in Figure \ref{fig:constraints}.
	Then, assign Feature is a inserting basic rule with $C_2^2 \in \ig{C_1^1}
	{C_2^1}$ w.r.t. $c_1$ but not a inserting basic rule with respect to the 
	constraint $\forall(C_2^2, \exists C_2^1)$ since the left-hand side 
	of  \emph{\texttt{assignFeature}} does not contain an occurrence 
	of $C_2^2$.
\end{example}
Again, \ref{def_basicIncreasing} relies on every transformation of a rule $\rho$. 
Therefore, we present an alternative method to determine whether $\rho$ satisfies \ref{def_basicIncreasing} or not by checking that $\rho$ does not delete any edges or nodes of $C_{k+1} \setminus C_k$. 

As mentioned above, given a consistency maintaining rule $\rho$ we can derive 
basic increasing rules that are only applicable if at a match and graph if 
$\rho$ is applicable.

\begin{definition}[\textbf{derived basic increasing rules}]
	Let a constraint $c$ in UANF and a consistency maintaining rule $\rho = (\ac,  
	L \xhookleftarrow{l} K \xhookrightarrow{r} R)$ up to layer
	$-1 \leq k < \nlvl(c)-1$,with $k$ odd, such that $\rho$ satisfies 
	\ref{incr_1} or \ref{incr_2} of Definition \ref{def_basicIncreasing}, 
	be given. 
	The set of \emph{derived basic increasing rules of $\rho$} can be 
	constructed in the following way:
	Let 
		$$\mathbf{G} :=
				\begin{cases}
					\ig{C_{k+2}}{C_{k+3}} &\text{if $k < \nlvl(c)-2$} \\
					\{C_{k+2}\} &\text{otherwise.}
				\end{cases}
		$$	
	For each $P \in \mathbf{G}$ and $L' \in \overlay(L,P)$: If the diagram shown 
	in Figure 
	\ref{fig_dpo_construction} is a transformation, i.e. (1) and (2) are 
	pushouts and $$\rho' = (\shiftm(\ac, i_L^{L'}), \rle{L'}{l'}{K'}{r'}{R'})$$
	is a basic increasing rule at layer $k$, then, $\rho'$ is a derived basic increasing rule.  
	
\end{definition}
\begin{figure}
\center
	\begin{tikzpicture}
		\node (L) at (0,2) {$L$};
		\node (K) at (2,2) {$K$};
		\node (R) at (4,2) {$R$};
		\node (G) at (0,0) {$L'$};
		\node (D) at (2,0) {$K'$};
		\node (H) at (4,0) {$R'$};
		\node (1) at (1,1) {(1)};
		\node (2) at (3,1) {(2)};
		
		\draw[left hook-stealth] (K) edge node [above] {$l$}  (L); 
		\draw[right hook-stealth] (K) edge node [above] {$r$}  (R); 
		\draw[left hook-stealth] (D)   edge node[above]{$l'$}(G); 
		\draw[right hook-stealth] (D)  edge node [above]{$r'$}(H);
		\draw[left hook-stealth] (K) edge node[fill = white] {$k$} (D);  
		\draw[left hook-stealth] (L) -- node[left]{$i_L^{L'}$} (G);
		\draw[left hook-stealth] (R) edge node [right] {$i_R^{R'}$} (H);    
	\end{tikzpicture}
	\caption{Pushout diagram for Lemma \ref{derived_appl}.}\label{fig_dpo_construction}
\end{figure}

Via this construction, we can now show that a each consistency increasing transformation via a consistency maintaining rule up to layer can be replaced 
by a transformation via a rule of the derived basic increasing rules of this rule. 

\begin{lemma}
	Let a constraint $c$ and a consistency maintaining rule $\rho = (\ac, \rho')$ 
	up to layer $-1 \leq k < \nlvl(c)$ be given. 
	Then, for each consistency increasing transformation 
	$$t: G \Longrightarrow_{\rho,m} H$$ there does exist a transformation 
	$$t': G \Longrightarrow_{\rho'',m''} H$$ such that $\rho''$ is a derived
	basic increasing rule of $\rho$.
\end{lemma}

\begin{proof}
	If $t$ is a consistency increasing transformation w.r.t. $c$, there
	does exist an occurrence $p:C_k \inj G$ of a universally bound graph $C_k$ 
	such that either $\track \circ p$ is not total or there does exist a 
	graph $C \in \ig{C_k}{C_{k+1}}$ such that $q \not \models \exists(a_k^r:C_k 
	\inj C, \true)$ and  $\track_t \circ q \models \exists(a_k^r:C-k \inj C, 
	\true)$.
	Also, there does exist an overlap $P \in \overlay(C_k,L)$ such that an 
	occurrence $m': P \inj G$ such that $m = m' \circ i_L^P$ exists. It follows 
	that there does exist a
	transformation $t: P \Longrightarrow_{\rho'} Q$ and therefore, a derived
	basic increasing rule $\rho'' =(\ac', \rho''')$ of $\rho$ with $P$ as 
	left-hand side and $Q$ as right-hand side must exists.
	For the morphism $m'$ follows that $m' \models \ac'$ and since 
	$\rho''$ deletes and inserts exactly the same elements as $\rho$, the 
	statement follows immediately. 
\end{proof}
\begin{lemma}\label{lemma:equiv_direct_normal}
Let a constraint $c$ in UANF, a basic increasing rule $\rho$ at layer $\kmax +1$ and a transformation $t: G \Longrightarrow_{\rho} H$ be given. 
Then, 
$$t \text{ is consistency increasing w.r.t $c$} \iff t \text{ is direct consistency increasing w.r.t $c$}$$
\end{lemma}
\begin{proof}
The \enquote{$\Longleftarrow$} side of the equivalence holds with lemma \ref{lemma:direct_implies_normal}. 
We need to show that the \enquote{$\Longrightarrow$} side of the equivalence holds. 
Let $t: G \Longrightarrow_{\rho} H$ be a consistency increasing transformation and $\rho$ a basic increasing rule. 
Then, $t$ satisfies (\ref{def_basicIncreasing}) and the satisfaction of (\ref{direct_improving_1}) follows immediately.
Since $\rho$ is also a non-consistency decreasing rule up to layer $\kmax$, $t$ satisfies (\ref{non_consistency_universally}) and (\ref{non_consistency_existentially}), then the satisfaction (\ref{direct_improving_1_2}), (\ref{direct_improving_3}) and (\ref{direct_improving_4}) also follows immediately.
\begin{enumerate}
\item If $\rho$ is a deleting basic increasing rule. 
Since $t$ is a consistency increasing rule there has to exist a morphism $p:C_{\kmax+2} \inj G$ such that $\track_t \circ p$ is not total and with that (\ref{direct_improving_2}) is satisfied. 
\item If $\rho$ is a inserting basic increasing rule there exists an occurrence $p: C_{\kmax + 2} \inj G$ and a graph $C \in \ig{C_{\kmax +2}}{C_{\kmax+3}}$ such that $p \not \models \ic{\kmax +2}{c}{C}$ and $\track_t \circ p \not \models \ic{\kmax +2}{c}{C}$. 
Therefore (\ref{direct_improving_2}) is satisfied. 
\end{enumerate}
In total follows that $t$ is a direct consistency increasing transformation w.r.t $c$.
\end{proof}

For basic increasing rules at layer $j < \kmax+1$ this equality does not hold since these rules are allowed to destroy occurrences of universally bound graphs $C_{j'}$  or to create occurrences of existentially bound graphs $C_{j'}$ with $j < j'< \kmax +1$.
With this results follow that if $t: G\Longrightarrow_{\rho} H$ is a $c$-guaranteeing transformation and $\rho$ is a basic increasing rule at layer $\kmax +1$, $t$ is also a direct increasing rule w.r.t $c$. 

%\subsection{Comparison with other consistency concepts and basic increasing rules}

\begin{figure}
\center
	\begin{tikzpicture}
	
		
		\node(guaran) at (3,4) {$c$-guaranteeing};
		\node(pres) at (6,0) {sustaining w.r.t $c$};
		\node (incr) at (6,2) {direct increasing w.r.t $c$};
		\node (direct) at (9,4) {direct improving w.r.t $c$};
 
		
		\draw[-implies,double equal sign distance] (guaran) --  (incr); 
		\draw[-implies,double equal sign distance] (incr) --  (pres);
		\draw[-implies,double equal sign distance] (direct) --  (incr);
		\draw[-implies,double equal sign distance] (guaran) --  (direct);    
		    
	\end{tikzpicture}
	\caption{consistency relations for basic rules.}\label{fig:consistency_relations_basic}
\end{figure}


Before we continue with the construction of application conditions for basic increasing rules, let us once more compare the notion of direct consistency increasing with the notions of (direct) consistency improving and sustaining and consistency guaranteeing and preserving with the restriction that only basic increasing rules are used. 
The results of these differ from the results of chapter \ref{comp_general} and are displayed in figure \ref{fig:consistency_relations_basic}.
It can be seen that in the case of basic increasing rules  the notions of direct consistency increasing and direct consistency improving are related in the sense that improving implies direct increasing and therefore direct increasing is a finer notion than improving. 

Additionally, with lemma \ref{lemma:equiv_direct_normal} follows immediately that guaranteeing implies direct increasing. 
We show that improving implies (direct) increasing and that (direct) increasing does not imply sustaining which also implies that increasing does not imply improving since each improving transformation is also a sustaining one. 
\begin{lemma}
Let a constraint $c$ in UANF, a basic increasing rule $\rho$ and a transformation $t: G \Longrightarrow_{\rho} H$ be given. 
Then, 
\begin{equation*}
	\begin{split}
		&t \text{ is improving w.r.t $c$ } \implies t \text{ is direct consistency increasing w.r.t $c$ } \wedge \\
		&t \text{ is direct increasing w.r.t $c$ } \centernot \implies t \text{ is consistency sustaining w.r.t $c$}
	\end{split}
\end{equation*}
\end{lemma}

\begin{proof}
\begin{enumerate}
\item Let $t$ be a consistency improving transformation w.r.t $c$. 

\end{enumerate}
\end{proof}



