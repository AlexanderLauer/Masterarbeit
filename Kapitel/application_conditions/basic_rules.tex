\subsection{Basic Consistency-increasing and Consistency-maintaining Rules}
 
The construction of the application conditions introduced in the previous section, as well as the constructed application conditions themselves, are very complex.
For a certain set of rules, which we will call \emph{basic consistency-increasing rules}, we are able to construct application conditions with the same property, namely that a rule equipped with this application condition is consistency-increasing at layer, in a less complex way.  
The main idea is that these rules (a) are not able to delete occurrences of existentially bound graphs or insert occurrences of universally bound graphs and (b) are able to increase consistency at a certain layer.
That is, given a basic increasing rule $\rho$, there exists a transformation $t: G \Longrightarrow_{\rho} H$ such that $t$ is a consistency increasing transformation with respect to a constraint $c$.

To ensure that (a) is satisfied, we first introduce \emph{basic consistency-maintaining rules up to layer}, which means that, given a constraint, a plain rule is not able to delete existentially bound and insert universally bound graphs up to a certain layer.  For the definition, we use the notion of consistency maintaining rules up to layer.  The set of basic consistency-maintaining rules up to layer is actually a subset of the set of consistency-maintaining rules up to layer, since these rules must be plain rules, whereas consistency-maintaining rules up to layer are allowed to have application conditions, i.e. $\maintaining(\cdot,\cdot)$.


\begin{definition}[\textbf{basic consistency maintaining rule up to layer}]
	Given a plain rule $\rho$ and a constraint $c$ in UANF.
	Then, $\rho$ is called \emph{basic consistency maintaining rule up to layer
	$-1 \leq k < \nlvl(c)$ w.r.t. $c$} if it is a direct consistency-maintaining rule at layer $k$ w.r.t. $c$.
\end{definition}

\begin{example}
	Consider the rules \emph{\texttt{moveFeature}},
	\emph{\texttt{assignFeature}} and \emph{\texttt{addDependency}}  given in Figure \ref{fig:rules} and constraints
	$c_1$ and $c_2$ given in Figure \ref{fig:constraints}.
	The rule \emph{\texttt{assignFeature}} is a basic consistency maintaining 
	rule up to layer $1$ w.r.t. $c_1$, whereas \emph{\texttt{moveFeature}} is 
	not a basic consistency maintaining rule w.r.t. $c_1$.
	The rule \emph{\texttt{addDependency}} is a basic consistency-maintaining rule up to layer $-1$ w.r.t. $c_2$, but is not a basic consistency-maintaining rule up to layer $1$ w.r.t. $c_2$ since it can insert occurrences of $C_3^2$.
\end{example}

Since there are infinitely many transformations via a plain rule $\rho$, it is impossible to check whether $\rho$ is a basic consistency maintaining rule up to a level based on the definition above. Therefore, we present a characterisation of basic consistency-maintaining rules that relies only on $\rho$ itself. 

First, let us assume that $\rho$ is able to create occurrences of a universally bound graph $C_j$. This is possible if (a) $\rho$ inserts an edge of $C_j \setminus C_{j-1}$ connecting pre-existing nodes of $C_j$, since it is unclear whether this would create a new occurrence of $C_j$, or (b) if $\rho$ inserts a node $v$ of $C_j$, so that all edges $e \in E_{C_j}$ with $\src(e) = v$ or $\tar(e) = v$ are also inserted.
If at least one of these edges is not inserted, it is guaranteed that this insertion will not create an occurrence of $C_j$, since $v$ is only connected to edges that have also been inserted by $\rho$.

Second, suppose $\rho$ is able to delete occurrences of an existentially bound graph $C_j$.
This is possible if (a) $\rho$ deletes an edge of $C_j \setminus C_{j-1}$ or (b) $\rho$ deletes a node $v$ of $C_j \setminus C_{j-1}$ such that all edges $e \in E_{C_j}$ with $\src(e) = v$ or $\tar(e) = v$ are also deleted.
If $\rho$ deletes a node $c$ of $C_{j} \setminus C_{j-1}$ without all its connected edges in $C_j$, there is no transformation via $\rho$ such that an occurrence of $C_j$ is deleted by deleting that node, since the dangling edge condition would not be satisfied. 
A rule that satisfies these properties does not reduce the largest satisfied layer.

We also need to ensure that the number of violations is not increased. To do this, we have to check that $\rho$ is not able to insert occurrences of the corresponding universally bound graph, as described above, and that $\rho$ is not able to remove occurrences of any intermediate graph. This is only ensured if $\rho$ does not remove any elements of $C{k+1}\setminus C_k$ when the set of intermediate graphs is given by $\ig{C_k}{C_{k+1}}$.

To check that a plain rule satisfies these properties, we use the dangling edge condition, or in other words, we check that the rule does not apply to certain overlaps of $L$ and an existentially bound graph, or that the inverse rule does not apply to certain overlaps of $R$ and a universally bound graph.

\begin{lemma}\label{def:non-decreasing}
	Given a plain rule $\rho = \rle{L}{l}{K}{r}{R}$ and a constraint $c$ in UANF.
	Let $-1 \leq k < \nlvl(c)$ be odd, then $\rho$ is a basic consistency-maintaining rule up to layer $k$ w.r.t. $c$ if \ref{category:non-decreasing_1} and \ref{category:non-decreasing_2} hold for all $k$, and \ref{category:non-decreasing_3} holds if $k < \nlvl(c) -2$.  
	\begin{enumerate}
		\item \label{category:non-decreasing_1}
			For each existentially bound graph $C_j$ with $2 \leq j \leq k+1$ and each overlap $P \in \overlay(L,C_j)$ 
			with $i_L^P(L \setminus K) \cap i_{C_j}^P(C_j \setminus C_{j-1} ) \neq \emptyset$, 
			the rule $\rho$ is not applicable at match $i_L^P$.
		\item \label{category:non-decreasing_2}
			
			For each universally bound graph $C_j$ with $1 \leq j \leq k+2$ and each overlap $P \in \overlay(R,C_j)$ 
			with $i_R^P(R \setminus K) \cap i_{C_j}^P(C_j) \neq \emptyset$, the rule $\rho^{-1}$ is not applicable at match $i_r^P$.
		\item \label{category:non-decreasing_3}
			For all graphs $P \in \overlay(L, C_{k+3})$ it holds that
			$$i_L^P(L \setminus K) \cap i_{C_{k+3}}^P(C_{k+3} \setminus C_{k+2})
			= \emptyset
			.$$
\end{enumerate}

\end{lemma}
\begin{proof}
	Let $\rho = L \xhookleftarrow{l} K \xhookrightarrow{r} R$ be a rule 
	that satisfies the characterisations listed in Lemma 
	\ref{def:non-decreasing} with $-1 \leq k < \nlvl(c)$ odd. Suppose $\rho$ is not a direct consistency-maintaining rule up to layer $k$ w.r.t. $c$.
	Then, there is a transformation $t: G \Longrightarrow_{\rho,m} H$ with $\maxk{c}{G} = k$ such that $t$ is not direct consistency-maintaining w.r.t. $c$.
	Therefore, either the deleting, inserting, universally or existentially condition is not satisfied.
	\begin{enumerate}
		\item 
			If the deleting condition is not satisfied, then $k < \nlvl(c) 
			-2$. There is an occurrence $p:C_{k+2} \inj G$ such that 
			$p \models \ic{0}{\scond{k+2}{c}}{C'}$ and $\track_t \circ p \models 
			\ic{0}{\scond{\kmax+2}{c}}{C'}$ with $C' \in \ig{C_{k+2}}{C_{k+3}}$.
			So an overlap $P \in \overlay(L, C_{k+3})$ with $i_L^P(L \setminus K) \cap i_{C_{k+3}}^P(C_{k+3}) \setminus 
			C_{k+2})= \emptyset$ must exist. This is a contradiction.
			
		\item 
			If  the inserting condition is not satisfied, there is
			an occurrence $p:C_{k+2} \inj H$ such that no morphism 
			$q: C_{k+2} \inj G$ with $p = \track_t \circ q$ exists and 
			and $p \not \models \false$ if $\scond{{k+2}}{c} = \false$ and 
			$p \not \models \ic{0}{\scond{{k+2}}{c}}{C_{j+3}}$ otherwise.
			So there is an overlap $P \in \overlay(R, C_{k+2})$ with 
			$i_R^P(R\setminus K)\cap i_{C_{k+2}}^P(C_{k+2}) \neq \emptyset$
			such that $\rho^{-1}$ is applicable at match $i_R^P$.
			This is a contradiction.
			
			
		\item 
			If the universally condition is not satisfied, there is
			an occurrence $p : C_j \inj H$ of an universally bound graph $C_j$ 
			with $1 \leq j \leq k+2$ such that no morphism $q:C_j \inj G$ with 
			$\track_t \circ q = p$ exists.
			So there is an overlap $P \in \overlay(R, C_j)$ with 
			$i_R^P(R\setminus K)\cap i_{C_j}^P(C_j) \neq \emptyset$ 
			such that $\rho^{-1}$ is applicable at match $i_R^P$.
			This is a contradiction.
		\item
			If the universally condition is not satisfied, there is
			an occurrence $p : C_j \inj H$ of an existentially bound graph $C_j$ 
			with $2 \leq j \leq k+1$ such that $\track_t \circ p$ is not total.
			So there is an overlap $P \in \overlay(L, C_j)$ with 
			$i_L^P(L \setminus K) \cap i_{C_j}^P(C_j\setminus C_{j-1}) \neq \emptyset$ such that the rule $\rho$ is applicable at match $i_L^P$.
			This is a contradiction.
	\end{enumerate}	    
	In summary, $\rho$ is a basic consistency-maintaining rule up to layer $k$.
\end{proof}

Now we are ready to introduce \emph{basic increasing rules at layer $k$}, where $k$ is odd. The set of basic increasing rules is a subset of the set of consistency-maintaining rules at layer $k$, which ensures that the largest satisfied layer as well as the number of violations will not increase.
In addition, the left-hand side of this rule contains an occurrence $p$ of the universally bound graph $C_{k+2}$, so either this occurrence is removed, i.e. elements of $C_{k+2}\setminus C_{k+1}$ are deleted, or an intermediate graph $C \in \ig{C_{k+2}}{C_{k+3}}$ is inserted.
Of course, this second case only occurs if $k < \nlvl(c)-2$, where $c$ is the corresponding constraint.
This property has the advantage that the application conditions for basic increasing rules are less complex and smaller, since it can be determined exactly how this rule removes a violation, and therefore no overlaps need to be considered.

This, at first sight, seems to be a restriction of the set of basic increasing rules, but the context of any rule $\rho$ that satisfies all the properties of a basic increasing rule except that $C_{k+2}$ is a subgraph of the left-hand side can be extended so that $\rho$ is a basic increasing rule and the semantic of $\rho$ is not increased. 
A method for deriving these rules will be presented later.

Basic increasing rules at layer $k$ are called \emph{deleting basic increasing rules} when $p$ is removed and \emph{inserting basic increasing rules} when an intermediate graph is inserted.
For our repair process, we will introduce the restriction that deleting basic increasing rules may only delete edges but not nodes of $C_{k+2}$, since otherwise it is not possible to decide, given a rule set and a constraint, whether this rule set is able to repair an arbitrary graph based only on deleting basic increasing rules.
For example, consider a rule that deletes a node from $C_{k+2}$.
Then it is unknown whether this node is connected to nodes that do not belong to $C_{k+2}$, and it is unclear whether all occurrences of $C_{k+2}$ could be destroyed by $\rho$, since the dangling edge condition might not be satisfied.


\begin{definition}[\textbf{basic increasing rule}]\label{def_basicIncreasing}
	Given a constraint $c$ in UANF and a  direct consistency-maintaining rule $\rho = (\ac,  L \xhookleftarrow{l} K \xhookrightarrow{r} R)$ up to layer $-1 \leq k \leq \nlvl(c) -2$, where $k$ is odd. Then, $\rho$ is a  
	\emph{basic increasing rule w.r.t $c$ at layer 
	$k$} if a morphism $p: C_{k+2} 
	\inj L$, called the \emph{increasing morphism}, exists such that either 
	\ref{incr_1} or \ref{incr_2} holds.
	
	\begin{enumerate}
		\item \label{incr_1}
			\emph{Universally deleting}:
			$r \circ l^{-1} \circ p$ is not total. Then, $\rho$ is called
			a \emph{deleting basic increasing rule}.
		\item \label{incr_2}	
			\emph{Intermediate inserting}:
			If $k < \nlvl(c)-2$, there is an intermediate graph
			$C' \in \ig{C_{k+2}}{C_{k+3}}$ such that $p \not \models \exists(
			a_{k+2}^r:C_{k+2} \inj C', \true)$, $r \circ l^{-1} \circ p$ is 
			total and 
			$r \circ l^{-1} \circ p \models \exists(a_{k+2}^r:C_{k+2} \inj C',
			\true)$. Then, $\rho$ is called a \emph{inserting basic increasing 
			rule with $C$}. 
	\end{enumerate} 
\end{definition}

\begin{example}
	Consider the rule \emph{\texttt{assignFeature}} given in Figure \ref{fig:rules} and constraint $c_1$ given in Figure \ref{fig:constraints}. Then, \emph{\texttt{assignFeature}} is an inserting basic rule with $C_2^2 \in \ig{C_1^1} {C_2^1}$ w.r.t. $c_1$ but is not an inserting basic rule with respect to the constraint $\forall(C_2^2, \exists( C_2^1, \true))$ since the left-hand side of  \emph{\texttt{assignFeature}} does not contain an occurrence of $C_2^2$.
\end{example}


As mentioned above, given a direct consistency-maintaining rule $\rho$, we can derive basic increasing rules that are applicable when $\rho$ is applicable by extending the context of that rule so that it contains an occurrence of the graph $C_{k+2}$.


\begin{definition}[\textbf{derived rules}]
	Given a constraint $c$ in UANF and a rule $\rho = (\ac, \rle{L}{l}{K}{r}{R})$. The set of \emph{derived rules from $\rho$} at level $0 \leq k \leq \nlvl(c)-2$, where $k$ is odd, contains rules characterised as follows:
	Let 
	$$\mathbf{G} := \begin{cases}
						 \{C_{k+2}\} & \text{if $k = \nlvl(c)-2$ is 
						existentially bound} \\
						\ig{C_{k+2}}{C_{k+3}} &\text{otherwise.}
					\end{cases}
	$$
	For $P \in \mathbf{G}$ and $L' \in \overlay(L,P)$: If the diagram shown 
	in Figure 
	\ref{fig_dpo_construction} is a transformation, i.e. (1) and (2) are 
	pushouts, and for the characterisations of Definition 
	\ref{def_basicIncreasing} holds that
	
	$$\rle{L'}{l'}{K'}{r'}{R'} \text{  is universally deleting or intermediate inserting }$$
%	\begin{alignat*}{3}
%		&\rho \text{ is universally deleting } &&\implies \rle{L'}{l'}{K'}{r'}{R'} 
%			\text{  is universally deleting  } &\wedge \\
%			&\rho \text{ is intermediate inserting } &&\implies \rle{L'}{l'}{K'}{r'}{R'} 
%			\text{ is intermediate inserting}
%	\end{alignat*}
	the rule $$\rho' = (\shiftm(\ac, i_L^{L'}), \rle{L'}{l'}{K'}{r'}{R'})$$
	is a derived rule of $\rho$ at layer $k$.
\end{definition}
\begin{figure}
\center
	\begin{tikzpicture}
		\node (L) at (0,2) {$L$};
		\node (K) at (2,2) {$K$};
		\node (R) at (4,2) {$R$};
		\node (G) at (0,0) {$L'$};
		\node (D) at (2,0) {$K'$};
		\node (H) at (4,0) {$R'$};
		\node (1) at (1,1) {(1)};
		\node (2) at (3,1) {(2)};
		
		\draw[left hook-stealth] (K) edge node [above] {$l$}  (L); 
		\draw[right hook-stealth] (K) edge node [above] {$r$}  (R); 
		\draw[left hook-stealth] (D)   edge node[above]{$l'$}(G); 
		\draw[right hook-stealth] (D)  edge node [above]{$r'$}(H);
		\draw[left hook-stealth] (K) edge node[fill = white] {$k$} (D);  
		\draw[left hook-stealth] (L) -- node[left]{$i_L^{L'}$} (G);
		\draw[left hook-stealth] (R) edge node [right] {$i_R^{R'}$} (H);    
	\end{tikzpicture}
	\caption{Pushout diagram for Lemma \ref{derived_appl}.}\label{fig_dpo_construction}
\end{figure}

\begin{example}
	Consider the rule \emph{\texttt{assignFeature}} given in Figure 
	\ref{fig:rules} and constraint $c_1$ given in Figure \ref{fig:constraints}.
	The set of derived rules from $\rho$ at layer $1$ is given in Figure 
	\ref{fig:derived}.
\end{example}

\begin{figure}

	\includegraphics[scale=0.8]{figures/images/example_derived_rules}

	\caption{Derived rules of \texttt{assignFeature} and $c_1$.}\label{fig:derived}
\end{figure}

Obviously, a rule $\rho'$ contained in the set of rules derived from a rule $\rho$ is only applicable to a match $m'$ if $\rho$ is applicable to a match $m$ with $m = m' \circ i$, where $i$ is the inclusion of the left side of $\rho$ in the left side of $\rho'$.
Therefore, given a set of rules $\mathcal{R}$, extending $\mathcal{R}$ by the set of all derived rules for each rule of $\mathcal{R}$ does not extend the expressiveness of $\mathcal{R}$.
The main idea of the concept of derived rules is to extend a given set of rules by as many basic increasing rules as possible without extending the expressiveness of that set.

\begin{lemma}
Given a constraint $c$ in UANF and a rule 
	 $\rho = (\ac, \rle{L}{l}{K}{r}{R})$ 
	be given, such that $\rho$ is 
	a direct maintaining rule up to layer $-1 \leq k \leq \nlvl(c)-2$, where $k$ is odd, which is universally deleting and intermediate inserting according to Definition 
	\ref{def_basicIncreasing}. Then every rule contained in the set of derived rules of $\rho$ at layer $k$ is a basic increasing rule.
\end{lemma}

\begin{proof}
	Let $\rho' = (\ac', \rle{L'}{l'}{K'}{r'}{R'})$ be one of these derived rules. Since $\rho'$ deletes and inserts exactly the same elements as $\rho$ and $m', \models \ac' \iff m' \circ i_L^{L'} \models \ac$, $\rho'$ is a direct consistency maintaining rule up to layer $k$ and is universally deleting or intermediate inserting according definition \ref{def_basicIncreasing}.
	It follows that $\rho'$ is a basic increasing rule at layer $k$.
\end{proof}

In transformations via a rule $\rho$ such that the match intersects an occurrence of a universally bound graph $C_{k+2}$, $\rho$ can be replaced by a derived rule of $\rho$ at level $k$. 

\begin{lemma}
Given a constraint $c$ in UANF and a rule $\rho = (\ac, \rle{L}{l}{K}{r}{R})$.
	Then, for each transformation $$t: G \Longrightarrow_{\rho,m}H$$ such that an occurrence $p:C_{k+2} \inj G$ of a universally bound graph $C_{k+2}$ with $p(C_{k+2}) \cap m(L) \neq \emptyset$ exists, there exists a transformation $$t': G \Longrightarrow_{\rho', m'}H$$ where $\rho'$ is a derived rule of $\rho$ at layer $k$.
\end{lemma}

\begin{proof}
	Since $p(C_{k+2}) \cap m(L) \neq \emptyset$ there is an overlap $P \in \overlay(C_{k+2},L)$ such that there exists a morphism $q: P \inj G$ with $m = q \circ i_L^P$ and $p = q \circ i_{C_{k+2}}^P$. 
	Since $t$ exists, there is a derived rule $\rho' = (\ac', 
	\rle{L'}{l'}{K'}{r'}{R})$ of $\rho$ at layer $k$, where $L' = P$.
	We set $m' = q$, since $m = m' \circ i_L^{L'} \models \ac$, it follows that $m' = \ac'$, and since $\rho$ removes and inserts the same elements as $\rho$, there is the transformation $t': G \Longrightarrow_{\rho',m'} H$. 
\end{proof}

This allows us to replace consistency-increasing transformations via a direct consistency-maintaining rule $\rho$ at layer $k$ by a rule derived from $\rho$ at layer $k$, i.e. a basic increasing rule at layer $k$.

%\begin{lemma}\label{lemma:equiv_direct_normal}
%Let a constraint $c$ in UANF, a basic increasing rule $\rho$ at layer $\kmax +1$ and a transformation $t: G \Longrightarrow_{\rho} H$ be given. 
%Then, 
%$$t \text{ is consistency increasing w.r.t $c$} \iff t \text{ is direct consistency increasing w.r.t $c$}$$
%\end{lemma}
%\begin{proof}
%The \enquote{$\Longleftarrow$} side of the equivalence holds with lemma \ref{lemma:direct_implies_normal}. 
%We need to show that the \enquote{$\Longrightarrow$} side of the equivalence holds. 
%Let $t: G \Longrightarrow_{\rho} H$ be a consistency increasing transformation and $\rho$ a basic increasing rule. 
%Then, $t$ satisfies (\ref{def_basicIncreasing}) and the satisfaction of (\ref{direct_improving_1}) follows immediately.
%Since $\rho$ is also a non-consistency decreasing rule up to layer $\kmax$, $t$ satisfies (\ref{non_consistency_universally}) and (\ref{non_consistency_existentially}), then the satisfaction (\ref{direct_improving_1_2}), (\ref{direct_improving_3}) and (\ref{direct_improving_4}) also follows immediately.
%\begin{enumerate}
%\item If $\rho$ is a deleting basic increasing rule. 
%Since $t$ is a consistency increasing rule there has to exist a morphism $p:C_{\kmax+2} \inj G$ such that $\track_t \circ p$ is not total and with that (\ref{direct_improving_2}) is satisfied. 
%\item If $\rho$ is a inserting basic increasing rule there exists an occurrence $p: C_{\kmax + 2} \inj G$ and a graph $C \in \ig{C_{\kmax +2}}{C_{\kmax+3}}$ such that $p \not \models \ic{\kmax +2}{c}{C}$ and $\track_t \circ p \not \models \ic{\kmax +2}{c}{C}$. 
%Therefore (\ref{direct_improving_2}) is satisfied. 
%\end{enumerate}
%In total follows that $t$ is a direct consistency increasing transformation w.r.t $c$.
%\end{proof}


%\subsection{Comparison with other consistency concepts and basic increasing rules}

\begin{figure}
\center
	\begin{tikzpicture}
	
		
		\node(guaran) at (3,4) {$c$-guaranteeing};
		\node(pres) at (6,0) {sustaining w.r.t $c$};
		\node (incr) at (6,2) {direct increasing w.r.t $c$};
		\node (direct) at (9,4) {direct improving w.r.t $c$};
 
		
		\draw[-implies,double equal sign distance] (guaran) --  (incr); 
		\draw[-implies,double equal sign distance] (incr) --  (pres);
		\draw[-implies,double equal sign distance] (direct) --  (incr);
		\draw[-implies,double equal sign distance] (guaran) --  (direct);    
		    
	\end{tikzpicture}
	\caption{consistency relations for basic rules.}\label{fig:consistency_relations_basic}
\end{figure}


Before we continue with the construction of application conditions for basic increasing rules, let us once more compare the notion of direct consistency increasing with the notions of (direct) consistency improving and sustaining and consistency guaranteeing and preserving with the restriction that only basic increasing rules are used. 
The results of these differ from the results of chapter \ref{comp_general} and are displayed in figure \ref{fig:consistency_relations_basic}.
It can be seen that in the case of basic increasing rules  the notions of direct consistency increasing and direct consistency improving are related in the sense that improving implies direct increasing and therefore direct increasing is a finer notion than improving. 

Additionally, with lemma \ref{lemma:equiv_direct_normal} follows immediately that guaranteeing implies direct increasing. 
We show that improving implies (direct) increasing and that (direct) increasing does not imply sustaining which also implies that increasing does not imply improving since each improving transformation is also a sustaining one. 
\begin{lemma}
Let a constraint $c$ in UANF, a basic increasing rule $\rho$ and a transformation $t: G \Longrightarrow_{\rho} H$ be given. 
Then, 
\begin{equation*}
	\begin{split}
		&t \text{ is improving w.r.t $c$ } \implies t \text{ is direct consistency increasing w.r.t $c$ } \wedge \\
		&t \text{ is direct increasing w.r.t $c$ } \centernot \implies t \text{ is consistency sustaining w.r.t $c$}
	\end{split}
\end{equation*}
\end{lemma}

\begin{proof}
\begin{enumerate}
\item Let $t$ be a consistency improving transformation w.r.t $c$. 

\end{enumerate}
\end{proof}



