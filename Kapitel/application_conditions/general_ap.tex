\subsection{General Application Conditions}
We start by introducing consistency maintaining application conditions, i.e.
a rule equipped with this application condition will only be applicable if 
the corresponding transformation is consistency maintaining. 
In particular, we will show that this transformation is even direct consistency maintaining.

The maintaining application condition consists of the three parts $\remain{k}{\rho'}$, $\ins{k}{\rho'}$ and  $\wors{k}	{\rho'}$ that are connected via the boolean $\wedge-$operator with $\rho'$ being a plain rule and $0 \leq k < \nlvl(c)$ for a given constraint $c$. 
As already discussed, the satisfaction at layer is decreased if and only if occurrences of existentially bound graphs $C_j$ have been deleted or 
occurrences of universally bound graphs $C_{j}$ have been inserted, with $0 \leq j \leq k$.
To check that no existentially bound graph $C_j$, $0 \leq j \leq k$, will be deleted, we use the condition constructed by $\remain{k}{\rho'}$ that check that 
no overlap $P$ of the left-hand-side of $\rho'$ with $C_j$, such that the application of $\rho'$ at $P$ leads to a deletion of $C_j$, exists in the originating graph. 
To ensure that no universally bound graph $C_j$, $0 \leq j \leq k$, will be inserted, the condition constructed by $\ins{k}{\rho'}$ checks that no overlap $P$ of the right-hand-side of $\rho'$ and $C_j$, such that the application of $\rho'^{-1}$ at $P$ would destroy the occurrence of $C_j$, exists in the derived graph. Note, that a condition constructed in this way is a 
right application condition. Therefore, we use the shift over rule operator to construct an equivalent left application condition. 
To verify that the number of violations is not decreased, we use the condition 
constructed by $\wors{k}	{\rho'}$, that checks, in the same manner as $\ins{k}{\rho'}$, that no occurrences of graphs $C' \in \ig{C_{k+2}}{C_{k+3}}$ will be deleted if $k < \nlvl(c) -2$. Otherwise, this condition is set to $\true$.



\begin{definition}[\textbf{consistency maintaining application condition}]
	Let a rule $\rho = (\ac, \rho')$ with $\rho' = \rle{L}{}{K}{}{R}$ and a 
	constraint
	$c$ in UANF be given.  
	The \emph{maintaining application condition} of $c$ for $\rho$ at layer 
	$-1 \leq k < \nlvl(c)$  
	is defined  as $\ac \wedge \main{k}{\rho'}$ with
	$$\main{k}{\rho'} :=  \remain{k}{\rho'} \wedge \ins{k}{\rho'} \wedge \wors{k}
	{\rho'} $$ 
	and
		 
	
	\begin{enumerate}
		\item 
			Let $E$ be the set of all existentially bound graphs graphs $C_j$ 
			with $j \leq k+1$ and  $\mathbf{P}_{C_j}$ be the set all overlaps 
			$P'$ 
			of $L$ and $C_j$ with $i_L^{P'}(L \setminus K) \cap i_{C_j}^{P'}(C_j)
			\neq \emptyset$: 
			$$\remain{k}{\rho'} := \bigwedge_{C \in E} \bigwedge_{P' \in 
			\mathbf{P}_{C_j}} \neg \exists(i_L^{P'}: L \inj P', \true)$$	
	
		\item 
			Let $U$ be the set of all universally bound graphs $C_j$ with $j \leq
			k+2$, and $\mathbf{P}_{C_j} $ be the set of all overlaps $P'$ of $R$ 
			and $C_j$ with $i_R^{P'}(R \setminus K) \cap i_{C_j}^{P'}(C_j) \neq 
			\emptyset$:
			$$\ins{k}{\rho'} := \bigwedge_{C \in U} \bigwedge_{P' \in \mathbf{P}
			_{C_j}} \shift(\neg \exists(i_R^{P'}: R \inj P', \true), \rho')$$

		\item 
			If $k > \nlvl(c) -3$ or $\scond{k}{c}$ is universally bound,
			$ \wors{k}{\rho'} = \true$.
			Otherwise, Let $E$ be the set of all overlaps of $L$ and $C'$ with 
			 $i_L^{P'}(L \setminus K) \cap i_{C'}^{P'}(C')\neq \emptyset$ for 
			 all $C' \in \ig{C_{k+2}}{C_{k+3}}$: 
			$$\wors{k}{\rho'} := \bigwedge_{P \in E} \neg \exists(i_L^{P}: L 
			\inj P', \true) $$
	\end{enumerate}
\end{definition}

\begin{example}

\end{example}

Let us now show that each rule equipped with the according application condition is a consistency maintaining rule at layer. 

\begin{theorem}\label{thm_maintaining_ac}
	Let a constraint $c$ in UANF be given.
	Each rule $ \rho = (\ac', \rho') $ with $\ac' = \ac \wedge \main{k}{\rho'}$ 
	and $-1 \leq k < \nlvl(c)$ is a consistency maintaining rule at layer $k$ 
	w.r.t. $c$.
	  
\end{theorem} 
\begin{proof}
	Let a graph $G$, such that $\kmax = k$,  and a transformation $t : G
	\Longrightarrow_{\rho} H$ be given. 
	It is sufficient to show that $t$ is a direct consistency maintaining
	transformation. 
	If $k < \nlvl(c) -1$, it follows that $G \not \models c$ and that 
	$k$ is odd. We show that $t$ satisfies (\ref{direct_improving_1}), 
	(\ref{direct_improving_1_2}), (\ref{direct_improving_3}) and 
	(\ref{direct_improving_4}).
	
	\begin{enumerate}
		\item 
			 Assume that (\ref{direct_improving_1}) does not hold. Then, 
			 $e = \scond{k+2}{c} \neq \false$ and a morphism $p: C_{k+2} \inj G$ 
			 exists, such that $p \models \ic{0}{e}{C'}$, $\track_t \circ p$ 
			 is total and $\track_t \circ p \not \models \ic{0}{e}{C'}$ for a 
			 graph $C' \in \ig{C_{k+2}}{C_{k+3}}$. 
			 Therefore, an overlap $P$ of $L$ and $C'$ such that $i_{C_{k+2}}^P 
			 \models \exists(a^r_{k+2}: C_{k+2} \inj C', \true)$ with $i_L^P(L 
			 \setminus K) \cap i_{C'}^P(C' \setminus C_{k+2}) \neq \emptyset$ 
			 must exist and $m \models \exists(i_L^P: L \inj P, \true)$ holds. 
			 Thus, $\wors{k}{\rho'} $ and consequently also $ \main{k}{\rho'}$ 
			 cannot be satisfied. 

		\item 
			Assume that  (\ref{direct_improving_1_2}) does not hold and let

			$$ d := \begin{cases} 
				\ic{0}{\scond{k+2}{c}}{C_{k+3}} & \text{if $\scond{k+2}{c}
				\neq \false$} \\
				\false &\text{otherwise.}
	
				\end{cases}	
			$$

  			Then, a morphism $p': C_{k+2} \inj H$ with $p' \not \models d$ 
  			exists, such that no morphism $p :C_{k+2} \inj G$ with $\track_t 
  			\circ p = p'$ exists. Therefore, an overlap $P$ of $R$ and $C_{k+2}$ 
  			with $i_R^P(R\setminus K) \cap i_{C_{k+2}}^P(C_{k+2}) \neq \emptyset$ 
  			exists, such that $m \models \shift(\exists(i_R^P: R \inj P, \true), 
  			\rho')$. Hence, $m$ does not satisfy $\ins{k}{\rho'}$. 

		\item 
			Assume that (\ref{direct_improving_3}) does not hold. Then, a 
			morphism $p: C_j \inj H$ with $C_j$ being universally bound and  $j < 
			k$ exists, such that no morphism $p': C_j \inj G$ with $\track_t 
			\circ p' = p$ exists. Then, an overlap $P$ of $C_j$ and $R$ with 
			$i_R^P(R \setminus K) \cap i_{C_j}^P(C_j) \neq \emptyset$ exists, 
			such that $m \models \shift(\exists(i_R^P: R \inj P, \true), \rho)$. 
			Hence, $m \not \models  \ins{k}{\rho'}$. 

		\item 
			Assume that (\ref{direct_improving_4}) does not hold. Then, a 
			morphism $p :C_j \inj G$ with $C_j$ being existentially bound and $j 
			\leq k$ exists, such that $\track_t \circ p$ is not total. Then, an 
			overlap $P$ of $C_j$ and $L$ with $i_L^P(L \setminus K) \cap i_{C_j}
			^P(C_j) \neq \emptyset$ exists, such that $m \models \exists(i_L^P:L 
			\inj P, \true)$. Hence, $m \not \models \remain{k}{\rho'}$.
	\end{enumerate}
	If $k = \nlvl(c)-1$, it follows immediately that $G \models c$. 
	Therefore, we need to show that $H \models c$.
	It follows that $\wors{k}{\rho'} = \true$. Assume that $H \not \models c$.
	Therefore, either an occurrence $p : C_j \inj H$ of an universally bound 
	graph $C_j$ exists such that no $q : C_j \inj G$ with $p = \track_t \circ q$ 
	exists or an occurrence $p': C_{j'} \inj G$ of an existentially bound graph
	$C_{j'}$ exists, such that $\track_t \circ p'$ is not total.
	If the first case applies, an overlap $P$ of $C_j$ and $R$ with 
	$i_{C_j}^P(C_j) \cap i_{R}^P(R\setminus K) \neq \emptyset$ exists, such that
	$m \models \shift(\neg \exists(i_{R}^P: R \inj P, \true), \rho')$ and 
	therefore $m \not \models \ins{k}{\rho'}$.
	If the second case applies, an overlap $P$ of $C_{j'}$ and $L$ with 
	$i_{C_{j'}}^P(C_{j'}) \cap i_{L}^P(L\setminus K) \neq \emptyset$ exists, 
	such that
	$m \models \neg \exists(i_{L}^P L \inj P, \true)$ and 
	therefore $m \not \models \remain{k}{\rho'}$.
	By contradiction follows that $H \models c$.
	
	In total follows that $\rho$ is a consistency maintaining rule at layer $k$
	w.r.t. $c$.
\end{proof}

For an application condition, such that a rule equipped with it is consistency 
increasing at layer, we have additionally to ensure that at least one violation will be removed.
To check this via an application condition, it is necessary (a) to check that an occurrence $p$ of the universally bound graph $C_{\kmax +2}$ exists, such that $p$ and the match $m$ do overlap, i.e $p(C_{\kmax +2}) \cap m(L) \neq \emptyset$, and, if the sub-condition at layer $\kmax +2$ is not equal to $\false$, (b) that $p$ does not satisfy $c' := \exists C_{\kmax +3}$. 
Only in this case, it is possible that the transformation does remove a violation. 
To ensure that $p$ does not satisfy $c'$, the non-existence of all possible overlaps $P$ of $L$ and $C_{\kmax +3}$ such that $p \models c'$ has to be checked. 
For this, we introduce \emph{extended overlaps}.
Intuitively, given an overlap $C$ of $L$ and $C_{\kmax +2}$ with $p: C_{\kmax + 2} \inj C$, the overlap is extended with elements of $C_{\kmax +3}$ such that 
$p \models C_{\kmax +3}$.

\begin{definition}[\textbf{extended overlaps}]
	Let graphs $C_0$, $C_1$, $G$ and morphisms $i_{C_0}^G: C_0 \inj G$ and 
	$i_{C_0}^{C_1}:C_0 \inj C_1$ be given.
	The set of \emph{extended overlaps of $G$ with $i_{C_0}^G$ and 
	$i_{C_0}^{C_1}$}, denoted by $\eol(G,i_{C_0}^G, i_{C_0}^{C_1})$ is 
	defined as:
	
	$$\eol(G,i_{C_0}^G, i_{C_0}^{C_1}) := \{P \in \overlay(G,C_1) \mid
		i_G^P \circ i_{C_0}^G \models \exists(i_{C_0}^{C_1}:C_0 \inj C_1, \true)
		\}$$

\end{definition}
%\begin{definition}[\textbf{extended overlaps}]
%Let $G$, $C_0$ and $C_1$ with $i_{C_0}^{C_1}:C_0 \inj C_1$ be graphs. 
%Let $P$ be an overlap of $C_0$ and $G$ with the inclusion $i_{C_0}^P : C_0 \inj P$. 
%The set of \emph{extended overlaps of $P$ with $i_{C_0}^{C_1}$}, denoted by $\eol(C,i_{C_0}^{C_1})$, is the set of all overlaps $Q$ of $G$ and $C_1$, such that an injective morphism $i_P^{Q}: P \inj Q$ with $i_P^{Q} \circ i_{C_0}^P \models \exists(i_{C_0}^{C_1}: C_0 \inj C_1, \true)$ exists. 
%\end{definition}

Via the notion of extended overlaps we are now able to check that a violation exists, to decide whether a transformation is able to remove a violation at all. 
It remains to check whether such a violation will be removed.

In the definition below, $\vFound{}{\cdot}{\cdot}$ and $\vRepaired{}{\cdot}{\cdot}$ ensure that a violation will be removed, with $\vFound{}{\cdot}{\cdot}$ ensuring that a violation is present and $\vRepaired{}{\cdot}{\cdot}$ ensuring that this violation will be removed. Note, that the construction of these is divided in two cases. 
Firstly, either $k \leq \nlvl(c)-3$  and secondly, $k = \nlvl(c)-2$ and the constraint ends with a condition of the form $\forall(a: C_0 \inj C_1, \false)$.

For $\vFound{}{\cdot}{\cdot}$, the first case will be checked via extended overlaps as already described above. 
In the second case, it is sufficient to check whether an occurrence $p$ of $C_{\nlvl(c)}$ with $m(L) \cap p(C_{\nlvl(c)}) \neq \emptyset$ exists.

For $\vRepaired{}{\cdot}{\cdot}$, in the first case, a violation can be removed by either deleting an occurrence $p$ of $C_{k+2}$ or inserting elements of $C_{k+3}$, such that $p \not \models \exists C'$ and $\track_t \circ p \models \exists C'$ for a graph $C' \in \ig{C_k´{k+2}}{C_{k+3}}$.
In the second case, a violation can only be removed by deleting an occurrence $p$ of $C_1$. 
This will only occur if $m(L \setminus K) \cap p(C_1) \neq \emptyset$. 

\begin{definition}[\textbf{consistency increasing application condition}]\label{def_appl_cond}
	Let a rule $\rho = (\ac, \rho'$ with $\rho' = L \xhookleftarrow{} K 
	\xhookrightarrow{} R$ and a constraint $c$ in UANF be given. Let $0 \leq k < 
	\nlvl(c)$ be even, i.e. $\scond{k}{c}$ is universally bound, and $C \in 
	\ig{C_{k+1}}{C_{k+2}}$ if $\scond{k+1}{c} \neq \false$ and $C = C_{k+1}$ 
	otherwise.
	The \emph{increasing application condition of $c$ for $\rho$ at layer $k$ 
	with $C$} is defined as
	\begin{equation}
		\incr{k}{C}{\rho} := \ac \wedge \main{k-1}{\rho} \wedge \big(\bigvee_{P
		\in \overlay(L,C_{k+1})} \vFound{}{P}{C} \wedge \vRepaired{}{P}{C}\big) 
	\end{equation}
	with
\begin{enumerate}
\item Let $a^r: C_{k+1} \inj C$ be the restricted morphism of $a_{k+1}$ and $i_L^P$ and $i_P^Q$ the inclusions of $L$ in $P$ and $P$ in $Q$, respectively:
	$$\nex(P,C') := 
			\begin{cases}
			\exists (i_L^P: L \inj P, \true) &\textit{if } \scond{k+1}{c} = 
			\false \\
			 \exists(i_L^P: L \inj P, \bigwedge_{Q \in \eol(P,i_{C_{k+1}}^P,a^r)} \neg 
			 \exists(i_P^Q: P \inj Q, \true)) &\textit{otherwise}
			\end{cases}			
			$$
			
\item If $i_L^P(L \setminus K) \cap i_{C_{k+1}}^P(C_{k+1}) \neq 
		\emptyset$, we set
$$\rep(P,C') := \true $$

Otherwise, let $P'$ be the graph derived by the transformation $P \Longrightarrow_{\rho,m} P'$. Then, $P'$ is an overlap of $R$ and $C_{k+1}$.
If this transformation does not exist, we set 	$\rep(P,C') := \false$.
Let $a^r:C_{k+1} \inj C$ be the restricted morphism of $a_{k+1}$, then 

$$\rep(P,C') :=  \begin{cases}
				\false &\text{if $\scond{k+2}{c} = \false$} \\
				\bigvee_{Q \in \eol(P',i_{C_{k+1}}^{P'}, a^r)} \shift(\forall (i_R^{Q}: R \inj P', 
				\exists(i_{P'}^Q:P' \inj Q,\true)), \rho) & \text{otherwise.}
				\end{cases}$$
 

\end{enumerate}
\end{definition}



Note, that, in case that no occurrence of $C_{k+1}$ not satisfying $\exists C_{k+2}$ will be removed, $\incr{k}{C'}{\rho}$ for any $C' \in \ig{C_{k+1}} {C_{k+2}}$ will only be evaluated with $\true$ if an occurrence $p$ of $C_{k+1}$ with $p \not \models \exists(a_{k+1}^r:C_{k+1} \inj C', \true)$ and $\track_t \circ p \models \exists(a_{k+1}^r:C_{k+1} \inj C', \true)$ exists. 
For any smaller improvements, i.e a similar improvement for a sub-graph $C'' \in \ig{C_{k+1}}{C_{k+2}}$ of $C'$, $\incr{k}{C'}{\rho}$ would be evaluated with $\false$.
For any bigger improvements, i.e the same improvement for a super-graph $C'' \in \ig{C_{k+1}}{C_{k+2}}$ of $C'$, $\incr{k}{C'}{\rho}$ would also be evaluated with $\false$, if $p \models \exists(a_{k+1}^r:C_{k+1} \inj C', \true)$.
In both cases, the application condition  would prohibit the transformation, even if it would be consistency increasing. 
To resolve this problem, multiple application conditions could be combined by 
$$\bigvee_{C' \in \ig{C_{k+1}}{C_{k+2}}} \incr{k}{C'}{\rho}.$$ 
This application condition will be evaluated with $\true$ if the cases described above do appear, with the drawback that this leads to a huge condition, even if duplicate conditions are removed.
At least all duplicates of $\maintaining()$ can be removed, since they are identical for each $\incr{k}{C'}{\rho}$ and only need to be constructed once.

In general, these application conditions are a trade-off between conditions-size and restrictiveness. 
They are very restrictive, since they do not allow any deletions of occurrences of existentially bound and insertions of universally bound graphs. 
For example, any of these application conditions with the rule \texttt{moveFeature} and constraint $c_1$ will be equivalent to $\false$; the maintaining part of the condition will always be evaluated with $\false$ since \texttt{moveFeature} does remove elements of the existentially bound graph $C_2^1$.
A change of the conditions constructed by $\maintaining()$ such that it is checked whether two nodes of the type \texttt{Feature} are  connected to a node \texttt{Class} will yield application conditions that are satisfiable with \texttt{moveFeature}, but for a similar rule moving two nodes of type \texttt{Feature}, this newly constructed $\nwo()$ would still be evaluated with $\false$. 
Therefore, this only leads to a slight decrease of restrictiveness. 

The conditions constructed by $\rem()$ and $\nin()$ could be changed in a similar fashion.
For $\rem()$ and the universally bound graph $C_j$, by checking whether there does exist an additional occurrence $p$ of $C_{j+1}$ such that $p \models \scond{j+2}{\cut{\kmax}{c}}$ and for $\nin()$, by checking whether an introduced occurrence $p$ of $C_j$ does satisfy $\scond{j+1}{\cut{\kmax}{c}}$.
The construction of these is similar to the construction of consistency guaranteeing application conditions as introduced by Habel and Pennemann \cite{habel2009correctness}, which is known to construct huge application conditions. 
Also, they do get more and more restrictive for increasing $k$, since the number of conditions constructed by $\rem()$ and $\nin()$ also increases.


\begin{example}
Consider the rule \emph{\texttt{assignFeature}} of figure \ref{fig:rules} and constraint $c_1$ of figure \ref{fig:constraints}. 
The application condition of $c_1$ at layer $1$ with $C_2^1$ for \emph{\texttt{assignFeature}} constructed by definition \ref{def_appl_cond} is shown in figure \ref{fig:appl_cond}.
The first two rows are conditions constructed by $\nex(\cdot, \cdot)$ and the third row is the condition constructed by $\rep(\cdot, \cdot)$.
Note that $\rem()$, $\nin()$ and $nwo()$ did not construct any conditions since \emph{\texttt{assignFeature}}  does not create elements of $C_1^1$ and does not delete elements of $C_2^1$. 

Additionally, this application condition is also a consistency improving application condition w.r.t $c_2$.
\end{example}

Let us now show that the construction above generates consistency increasing application conditions.


\begin{theorem}
	Let a constraint $c$ in UANF be given. Each rule $\rho = (\ac', \rho')$ with
	$\ac' = \ac \wedge \incr{k}{C}{\rho}$ with $0 \leq k < \nlvl(c)$ being even
	and $C = C{k+1}$ if $\scond{k+1}{c} = \false$ and $C \in \ig{C_{k+1}}
	{C_{k+2}}$ otherwise, Then, $\rho$ 
	is a consistency increasing rule at layer $k-1$ w.r.t. $c$.

\end{theorem}
\begin{proof}
	Let a transformation $t: G \Longrightarrow_{\rho} H$ with $\maxk{c}{G} = k-1$
	be given. 
	Since $\main{k-1}{\rho}$ is contained in $\incr{k}{C}{\rho}$, $t$ is a 
	consistency maintaining transformation at layer $k-1$ with Theorem 
	\ref{thm_maintaining_ac}. It remains to show that $t$ satisfies 
	(\ref{direct_improving_2}).
	
	Let $\scond{k}{c} = \forall(a_k: C_k \inj C_{k+1},e)$ be the sub-condition of 
	$c$ at layer $k$. 
	
	\begin{enumerate}
		\item	
			If $e = \false$, assume that (\ref{direct_improving_2}) does not 		
			hold, then, no 
			morphism $p: C_{k+1} \inj G$ exists, such that $\track_t \circ p$ is
			not total.
			Therefore, no overlap $P$ of $L$ and $C_{k+1}$ with
			$i_L^P(L\setminus K) \cap i_{C_{k+1}}^P(C_{k+1}) \neq \emptyset$
			exists. 
			It follows that  $\rep(P,C') = \false$ and $m \not \models \incr{k}	
			{C}{\rho}$
		
		\item 
			Otherwise, let $P \in \overlay(L, C_{k+1})$. We show that 
			$m \models \vFound{}{P}{C} \wedge \vRepaired{}{P}{C}$ implies 
			that (\ref{direct_improving_2}) holds.
			If $m \models \vFound{}{P}{C}$, there does exist a morphism
			$p: P \inj G$ with $m = p \circ i_L^P$ and $p \models \neg \exists(
			i_P^Q:P \inj Q, \true)$ for all $Q \in \eol(P,a^r)$.
			Therefore, $q \not \models 
			\exists(a_{k+1}:C_{k+1} \inj C_{k+2}, \true)$ with $q = p \circ 
			i_{C_{k+1}}^P$. 
			
			If $i_L^P(L \setminus K) \cap i_{C_{k+1}}^P(C_{k+1}) \neq \emptyset$,
			$\track_t \circ q$ is not total, since all occurrence $i$
			of $C_{k+1}$ with $i = p' \circ i_{C_{k+1}}^P$, $p': P \inj G$ and 
			$m = p' \circ i_L^P$ in $G$ will be removed by $t$.
			Otherwise,  $\track_t \circ q$ is total and there does 
			exist a morphism $p: P' \inj H$ such that 
			$\track_t\circ q = p \circ i_{C_{k+1}}^{P'}$.
			Since $m \models \vRepaired{}{P}{C}$,  all morphisms $p \circ 
			i_{C_{k+1}}^{P'}$ with $n = p \circ i_R^Q$ satisfy 
			$\ic{0}{e}{C}$. Therefore, $\track_t \circ q \models 
			\ic{0}{e}{C}$.
			It follows that   (\ref{direct_improving_2}) is satisfied.
	\end{enumerate}	 
	Therefore, $\rho$ is a consistency increasing rule at layer $k-1$.
\end{proof}