\subsection{Application Condition for general Rules}
We will now introduce consistency-maintaining and consistency-increasing application conditions at layer for general rules. 
This means that a rule equipped with these application conditions is consistency-maintaining at layer or consistency-increasing at layer. We will also introduce an application condition such that any rule equipped with it is indeed a consistency-maintaining rule.

Let us start with the consistency-maintaining application condition at layer. 
This  application condition has an odd parameter $-1 \leq k < \nlvl(c)$ which specifies which graphs of a constraint are to be considered. 
In particular, only the graphs $C_j$ with $0 \leq j \leq k+3$ are considered.
Note that there is no graph with $\kmax$ even and $\kmax \neq \nlvl(c)-1$. In this case, it can be shown that the application condition with $k = \nlvl(c)-2$ produces a consistency-preserving application condition.  
So it is not a restriction that $k$ must be odd. 
The maintaining application condition consists of the following three parts, which also use the $k$ parameter:
\begin{enumerate}

	\item \emph{No violation inserted ($\wors{k}{} $):} 
		This application condition checks that no new violations are introduced by removing occurrences of intermediate graphs from the set $\ig{C_{k+2}}{C_{k+3}}$.  
		It corresponds to the \emph{no new violation by deletion} formula in the sense that a rule that satisfies this application condition also satisfies the \emph{no new violation by deletion} formula when applied to a graph with $\kmax = k$.
		There are several cases for this application condition. If $k = \nlvl(c)-2$, the constraint ends with $\forall(C_{\nlvl(c)}, \false)$. Therefore no violations can be introduced by removing occurrences of intermediate graphs. In particular, there is no graph $C_{k+3}$.
If $k = \nlvl(c) -1$, every transformation $t: G \Longrightarrow H$ with $G \models c$ is direct consistency maintaining w.r.t. $c$ if $H \models c$. This is ensured by $\remain{k}{}$ and $\ins{k}{}$.
  Therefore, if $k \geq \nlvl(c)-2$, we set the application condition to $\true$.
  
  \item \emph{No universally inserted ($\ins{k}{}$):}
		This application condition checks that no occurrences of universally bound graphs $C_j$ with $1 \leq j \leq k+2$ are inserted. It corresponds to the \emph{no satisfied layer reduction by insertion} and \emph{no new violations by insertion} formulas, in the sense that a rule that satisfies this application condition also satisfies the \emph{no satisfied layer reduction by insertion} and \emph{no new violations by insertion} formulas when applied to a graph with $\kmax = k$.

	\item \emph{No existentially destroyed ($\remain{k}{}$):}
		 This application condition checks that no occurrences of existentially bound graphs $C_j$ with $2 \leq j \leq k+1$ are removed. It corresponds to the \emph{no satisfied layer reduction by deletion} formula since a rule  that satisfies this application condition also satisfies the \emph{no satisfied layer reduction by deletion} formula when applied to a graph with $\kmax = k$.
	
	
		\end{enumerate}

Recall that given a constraint $c$ in UANF, each subcondition $\scond{k}{c}$
is a condition over the graph $C_k$ and the morphism with domain $C_k$  is denoted by $a_k$.

\begin{definition}[\textbf{maintaining application condition at layer}]\label{def_main}
	Given a  rule $\rho = (\ac, \rho')$ with $\rho' = \rle{L}{}{K}{}{R}$, 
	a constraint $c$ in UANF and an odd $-1 \leq k < \nlvl(c)$.
	The \emph{maintaining application condition} of $c$ for $\rho$ at layer 
	$k$  is defined  as $\ac \wedge \main{k}{\rho'}$ with
	$$\main{k}{\rho'} :=  \remain{k}{\rho'} \wedge \ins{k}{\rho'} \wedge \wors{k}
	{\rho'} $$ 
	where $\remain{k}{\rho'}$, $ \ins{k}{\rho'}$ and $\wors{k}$ are defined as
		 
	
	\begin{enumerate}
	\item No violation inserted:
			 Let $\mathbf{P}_{C'}$ be the set of all overlaps $P$ of $L$ and $C'$ 
			 with 
			 $i_L^{P}(L \setminus K) \cap i_{C'}^{P}(C'\setminus C_{k+2})\neq \emptyset$ for 
			 $C' \in \ig{C_{k+2}}{C_{k+3}}$: 
			 \begin{equation*}
				\wors{k}{\rho'} :=			 	
			 	\begin{cases}
			 	 	\true &\text{if $k \geq \nlvl(c)-2$} \\
			 	 	\bigwedge_{C' \in \ig{C_{k+2}}{C_{k+3}}}\bigwedge_{P \in \mathbf{P}_{C'}} \neg \exists(i_L^{P}: L 
			\inj P', \true) &\text{otherwise}
			 	\end{cases}
			 \end{equation*}
			 
		\item No universally inserted:
		
			Let $\mathbf{U}$ be the set of all universally bound graphs $C_j$ with $1 \leq j \leq
			k+2$, and $\mathbf{P}_{C_j} $ be the set of all overlaps $P'$ of $R$ 
			and $C_j$ with $i_R^{P'}(R \setminus K) \cap i_{C_j}^{P'}(C_j\setminus C_{j-1}) \neq 
			\emptyset$:
			$$\ins{k}{\rho'} := \bigwedge_{C \in \mathbf{U}} \bigwedge_{P' \in \mathbf{P}
			_{C}} \shift(\neg \exists(i_R^{P'}: R \inj P', \true), \rho')$$

		
		\item No existentially destroyed:
			If $k = -1$, we set $\remain{k}{\rho'} := \true$.
			Otherwise, let $\mathbf{E}$ be the set of all existentially bound graphs $C_j$ 
			with $2 \leq j \leq k+1$ and  $\mathbf{P}_{C_j}$ be the set all overlaps 
			$P'$ 
			of $L$ and $C_j$ with $i_L^{P'}(L \setminus K) \cap i_{C_j}^{P'}(C_j\setminus C_{j-1})
			\neq \emptyset$: 
			$$\remain{k}{\rho'} := \bigwedge_{C \in \mathbf{E}} \bigwedge_{P' \in 
			\mathbf{P}_{C}} \neg \exists(i_L^{P'}: L \inj P', \true)$$	
	
		
			
	\end{enumerate}
\end{definition}

We are aware that there are optimisations for $\ins{k}{} $ and $\remain{k}{\rho'}$, since according to \cite{behr2020efficient} not all overlaps need to be considered. But for the simplicity of our definition, we have not implemented these optimisations.
\begin{example}
	\begin{enumerate}
		\item
			Consider the constraint $c$ given in Figure \ref{fig:appl_cond} and the rule \emph{\texttt{re\-move\-Feature}}. The conditions $\remain{-1}{\emph{\texttt{removeFeature}}}$, $\remain{1}{\emph{\texttt{removeFeature}}}$ and $\remain{3}{\emph{\texttt{removeFeature}}}$ are also given in Figure  \ref{fig:appl_cond}; $\remain{-1}{\emph{\texttt{removeFeature}}}$ is equal to $\true$. $\remain{1}{\emph{\texttt{removeFeature}}}$ checks that no occurrences of $C_2$ are inserted, while $\remain{3}{\emph{\texttt{removeFeature}}}$ checks that no occurrences of $C_2$ and $C_4$ are inserted. Obviously, $\remain{1}{\emph{\texttt{removeFeature}}}$ is contained in $\remain{3}{\emph{\texttt{removeFeature}}}$. Note that there are conditions in $\remain{3}{\emph{\texttt{removeFeature}}}$ that imply each other. For example, the first condition implies the second and third. 
		Using the optimisations according to \cite{behr2020efficient}, these can be removed.
		\item
		Again consider constraint $c$ given in Figure \ref{fig:appl_cond} and 
		the rule \emph{\texttt{assignFeature}}. The condition $\ins{-1}{\emph{\texttt{assignFeature}}}$ is equal to $\true$ since \emph{\texttt{assignFeature}} cannot create any occurrences of the first universally bound graph of the constraint. The condition $\ins{1}{\emph{\texttt{assignFeature}}}$ is given in Figure \ref{fig:nui}.
		
		The condition  $\ins{3}{\emph{\texttt{as\-sign\-Fea\-ture}}}$ is equal to $\ins{1}{\emph{\texttt{assignFeature}}}$ since there are only two universally bound graphs and $\ins{1}{\emph{\texttt{assignFeature}}}$ already considers both.
		\item
		Consider the constraint $c$ given in Figure \ref{fig_vio} and the rule \emph{\texttt{addDependency}}.
		Then, $\wors{-1}{\emph{\texttt{addDependency}}}$ is given in Figure \ref{fig_vio}. The condition $\wors{1}{\emph{\texttt{addDependency}}}$ is 
		equal to $\true$ since $1 > \nlvl(c)-2 = 2-2$.
		
	\end{enumerate}
\end{example}
\begin{figure}
\centering
\includegraphics[scale=0.7]{figures/sources/constraint_nui_example}


\caption{No universally inserted conditions for the constraint given in Figure \ref{fig:appl_cond} and rule \texttt{assignFeature.}}\label{fig:nui}
\end{figure}
\begin{figure}

\includegraphics[scale=0.8]{figures/images/vio}


\caption{No violation inserted conditions for the constraint and rule \texttt{addDependency.}}\label{fig_vio}
\end{figure}
Let us now show that every rule equipped with $\main{k}{}$ is a consistency-maintaining rule at layer $k$.

\begin{theorem}\label{thm_maintaining_ac}
	Given a constraint $c$ in UANF.
	Every rule $\rho = (\ac \wedge \main{k}{\rho'}, \rho')$ where $\rho' = \rle{L}{l}{K}{r}{R}$ and 
	$-1 \leq k < \nlvl(c)$ is odd is a direct consistency maintaining rule w.r.t. $c$ at layer $k$.
	  
\end{theorem} 
\begin{proof}
	Given a graph $G$ with $\kmax = k$ and a transformation $t: G \Longrightarrow_{\rho,m} H$.
	We show that $t$ is a direct consistency maintaining transformation w.r.t. $c$.
	
	 We show that $t$ satisfies the \emph{no new violation by deletion}, \emph{no new violation by insertion}, \emph{no satisfied layer reduction by insertion} and \emph{no satisfied layer reduction by deletion} formulas.
	
	\begin{enumerate}
		\item 
			 Assume that $t$ does not satisfy the \emph{no new violation by deletion} formula.
			 Then $\kmax < \nlvl(c) -1$,
			 $e = \scond{k+2}{c} \neq \false$ and there is a morphism $p: C_{k+2} \inj G$ 
			 such that $p \models \ic{0}{e}{C'}$, $\track_t \circ p$ 
			 is total and $\track_t \circ p \not \models \ic{0}{e}{C'}$ for a 
			 graph $C' \in \ig{C_{k+2}}{C_{k+3}}$. 
			 Therefore, there is an overlap $P$ of $L$ and $C'$ with $i_L^P(L 
			 \setminus K) \cap i_{C'}^P(C' \setminus C_{k+2}) \neq \emptyset$  such that $i_{C'}^P \circ a^r_{k+2} 
			 \models \exists(a^r_{k+2}: C_{k+2} \inj C', \true)$ 
			 and $m \models \exists(i_L^P: L \inj P, \true)$. 
			 Thus, $\wors{k}{\rho'} $ and consequently also $ \main{k}{\rho'}$ 
			 cannot be satisfied. 

		\item 
			Assume that $t$ does not satisfy the \emph{no new violation by insertion} formula. Let

			$$ d := \begin{cases} 
				\ic{0}{\scond{k+2}{c}}{C_{k+3}} & \text{if $\scond{k+2}{c}
				\neq \false$} \\
				\false &\text{otherwise.}
	
				\end{cases}	
			$$

  			Then, there is a morphism $p': C_{k+2} \inj H$ with $p' \not \models d$ 
  			such that no morphism $p :C_{k+2} \inj G$ with $\track_t 
  			\circ p = p'$ exists. Therefore, there is an overlap $P$ of $R$ and $C_{k+2}$ 
  			with $i_R^P(R\setminus K) \cap i_{C_{k+2}}^P(C_{k+2}\setminus C_{k+1}) \neq \emptyset$ 
  			 such that $m \models \shift(\exists(i_R^P: R \inj P, \true), 
  			\rho')$. Hence, $m$ does not satisfy $\ins{k}{\rho'}$. 

		\item 
			Assume that $t$ does not satisfy the \emph{no satisfied layer reduction by insertion} formula. Then, there is a 
			morphism $p: C_j \inj H$ with $0 \leq j <k$ and $C_j$ universally bound such that no morphism $p': C_j \inj G$ with $\track_t 
			\circ p' = p$ exists. Then, there is an overlap $P$ of $C_j$ and $R$ with 
			$i_R^P(R \setminus K) \cap i_{C_j}^P(C_j\setminus C_{j-1}) \neq \emptyset$ 
			such that $m \models \shift(\exists(i_R^P: R \inj P, \true), \rho)$. 
			Hence, $m \not \models  \ins{k}{\rho'}$. 

		\item 
			Assume that $t$ does not satisfy the \emph{no satisfied layer reduction by deletion} formula.  Then, there is a 
			morphism $p :C_j \inj G$ with $j \leq k+1$ and $C_j$  existentially bound and   such that $\track_t \circ p$ is not total. Then, there is an 
			overlap $P$ of $C_j$ and $L$ with $i_L^P(L \setminus K) \cap i_{C_j}
			^P(C_j\setminus C_{j-1}) \neq \emptyset$, such that $m \models \exists(i_L^P:L 
			\inj P, \true)$. Hence, $m \not \models \remain{k}{\rho'}$.
	\end{enumerate}
	It follows that $\rho$ is a direct consistency-maintaining rule at layer $k$
	w.r.t. $c$.
\end{proof}

With the constructions described in Definition \ref{def_main} we are also able to construct direct consistency-maintaining application conditions. 

\begin{theorem}
	Given a constraint $c$ in UANF. Every rule $\rho$ equipped with 
	the application condition 
	$$ \Big(\bigwedge_{\substack{-1 \leq i < \nlvl(c) \\ i \textit{ odd}}} \wors{i}{\rho}\Big) \wedge \ \ins{\nlvl(c)-1}{\rho}$$
	is a direct consistency-maintaining rule w.r.t. $c$.
\end{theorem}

\begin{proof}
	Let $\rho = \rle{R}{r}{K}{l}{L}$ be a rule equipped with this application condition. 
	We show that $\rho$ is a consistency-maintaining rule at layer $k$ w.r.t. $c$ 
	for all $-1 \leq k \leq \nlvl(c) -1$.
	Obviously, $\ins{\nlvl(c)-1}{\rho}$ contains $\ins{j}{\rho}$ for all 
	$-1 < j \leq \nlvl(c)-1$.
	The set of intermediate graphs always contains the second graphs on which this set was built, so $\wors{i}{\rho}$ contains the condition 
	$$ \bigwedge_{P' \in 
			\mathbf{P}_{C_{i+3}}} \neg \exists(i_L^{P'}: L \inj P', \true)$$
	which also checks that no occurrence of $C_{i+3 }$ is deleted. Therefore 
	$$\Big(\bigwedge_{\substack{-1 \leq i < \nlvl(c) \\ i \textit{ odd}}} \wors{i}{\rho}\Big)$$ 
	must contain $\remain{\nlvl(c)-1}{\rho}$ and therefore it contains $\remain{j}{\rho}$ for all $-1 \leq j \leq \nlvl(c)-1$.
	So we can rewrite this application condition into the equivalent condition
	$$\Big(\bigwedge_{\substack{-1 \leq i < \nlvl(c) \\ i \textit{ odd}}} \wors{i}{\rho} \wedge \ins{i}{\rho} \wedge \remain{i}{\rho} \Big) =  \Big(\bigwedge_{\substack{-1 \leq i < \nlvl(c) \\ i \textit{ odd}}} \main{i}{\rho} \Big).$$
It follows that $\rho$ is a direct consistency-maintaining rule at layer $k$ for all 
$-1 \leq k < \nlvl(c)$. Since $-1 \leq \kmax < \nlvl(c)$ for each graph $G$, all transformations of $\rho$ are direct consistency-maintaining w.r.t. $c$. Hence, $\rho$ is a consistency-maintaining rule w.r.t. $c$.
\end{proof}
In the following, we will introduce the direct  consistency increasing application conditions at layer. 
For this, we will introduce the notion of \emph{extended overlap}, which will be useful to detect whether a violation is removed by a transformation. 
Intuitively, given an overlap and a morphism $a$, the overlap is extended such that an overlap morphism satisfies $\exists(a, \true)$.

\begin{definition}[\textbf{extended overlaps}]
	Given an overlap $(G, i_{C_0}^G, i_{C_1}^G)$ of $C_0$ and $C_1$ and a 
	morphism $e: C_0 \inj H$. 
	The set of \emph{extended overlaps} of $G$ at $i_{C_0}^G$ with $e$, 
	denoted by $\eol(G, i_{C_0}^G,e)$, is defined as	
	$$\eol(G,i_{C_0}^G, e) := \{P \in \overlay(G,H) \mid
		i_G^P \circ i_{C_0}^G \models \exists(e:C_0 \inj H, \true)
		\}.$$

\end{definition}
In other words, $\eol(G, i_{C_0}^G,e)$ is the set of all overlaps of $G$ and $H$ such that the square in Figure \ref{fig_eol} is commutative, i.e. $i_H^P \circ e = i_G^P \circ i_{C_0}^G$.  
\begin{figure}
\center
	\begin{tikzpicture}
		\node (L) at (0,2) {$C_0$};
		\node (K) at (2,2) {$H$};
		
		\node (G) at (0,0) {$G$};
		
		\node (H) at (2,0) {$P$};
		
		\draw[right hook-stealth] (L) edge node [above] {$e$}  (K);
		\draw[left hook-stealth] (L) edge node [left] {$i_{C_0}^G$}  (G);
		\draw[right hook-stealth] (G) edge node [above] {$i_G^P$}  (H);
		\draw[right hook-stealth] (K) edge node [right] {$i_H^P$}  (H);
		
	\end{tikzpicture}
	\caption{Diagram for the alternative definition of extended overlaps.}\label{fig_eol}
\end{figure}
Using extended overlaps, we will be able to check whether a violation has been 
removed.

For the consistency-increasing application condition at layer, we will use the maintaining application condition at layer.
All that remains is to ensure that at least one violation is removed. 
To do this, we must first check that there is a violation in the match. This means that the match and a violation are overlapping.
Finally, we need to check that this violation is removed.


Again, the increasing application condition has the odd parameter $-1 \leq k < \nlvl(c)-1$, which specifies which constraint graphs are to be considered. Note that $k$ must not be $\nlvl(c)-1$, since all graphs with $\kmax = \nlvl(c)-1$ satisfy the constraint and no consistency-increasing transformations are originating from those graphs.
We also use a second parameter $C'$, which is an intermediate graph of $C_{k+2}$ and $C_{k+3}$ if $c$ contains a graph $C_{k+3}$, i.e. $k < \nlvl(c)-2$, and $C'$ is set to $C_{k+2}$ otherwise. 
A rule equipped with this application condition is a consistency-increasing rule at layer $k$.
It consists of the following parts:
\begin{enumerate}
	\item The maintaining application condition ($\main{k}{}$):
			As already discussed, this application condition ensures that 
			a rule equipped with this application condition is a consistency-maintaining rule at layer $k$.
	\item \emph{Violation exists} ($\nex()$): 
		This condition checks that there is a violation at the match, i.e. there is an occurrence $p$ of $C_{k+2}$ with $m(L) \cap p(C_{k+2}) \neq \emptyset$ not satisfying $\exists(C', \true)$. If $k = \nlvl(c) -2$, then $c$ ends with $\forall(C_{\nlvl(c)}, \false)$ and thus there is no graph $C_{k+3}$ in $c$. In this case, it is sufficient to check only that $m(L) \cap p(C_{k+2}) \neq \emptyset$.
	\item \emph{Violation removed} ($\rep()$):
		This condition checks that the violation is removed.  This can be done in several ways, either by deleting the occurrence $p$ or by inserting elements such that $ p \models \exists(C', \true)$. This leads to case discrimination.  The first case is easy to check, if $m(L \setminus K) \cap p(C_{k+2}\setminus C_{k+1}) \neq \emptyset$, $p$ is removed and this condition can be set to $\true$. Otherwise, we need to check that the violation has been removed by an additional condition that checks whether $p$ satisfies $\exists(C', \true)$ after the transformation. The last case is the special case where the constraint ends with $\forall(C_{\nlvl(c)}, \true)$ and $k = \nlvl(c) -2$. Then there is only one way to remove a violation, by removing the occurrence $p$. So the condition is set to $\true$ if $m(L \setminus K) \cap p(C_{k+2}\setminus C_{k+1}) \neq \emptyset$ and to $\false$ otherwise. 
\end{enumerate}






\begin{definition}[\textbf{consistency increasing application condition at layer}]\label{def_appl_cond}
	Given a rule $\rho = (\ac, \rho')$ with $\rho' = \rle{L}{l}{K}{r}{R}$ and a constraint $c$ in UANF. Let $0 \leq k < \nlvl(c)-1$ be odd and $C' = C_{k+2}$ if $k = \nlvl(c) -2$
	and $C' \in \ig{C_{k+2}}{C_{k+3}}$ otherwise.  
	The \emph{increasing application condition of $c$ for $\rho$ at layer $k$ 
	with $C'$} is defined as
	\begin{equation}
		\incr{k}{C'}{\rho} := \ac \wedge \main{k}{\rho'} \wedge \big(\bigvee_{P
		\in \overlay(L,C_{k+2})} \vFound{}{P}{C'} \wedge \vRepaired{}{P}{C'}\big) 
	\end{equation}
	with
\begin{enumerate}
\item Violation exists: Let $a_{k+2}^r: C_{k+2} \inj C'$ be the restricted morphism of $a_{k+2}$ and $i_L^P$ and $i_P^Q$ be overlap morphisms of $P$ and $Q$, respectively:
	$$\nex(P,C') := 
			\begin{cases}
			\exists (i_L^P: L \inj P, \true) &\textit{if } \scond{k+2}{c} = 
			\false \\
			 \exists(i_L^P: L \inj P, \bigwedge_{Q \in \eol(P,i_{C_{k+2}}^P,a^r_{k+2})} \neg 
			 \exists(i_P^Q: P \inj Q, \true)) &\textit{otherwise}
			\end{cases}			
			$$
			
\item Violation removed: 

\begin{enumerate}
	\item If $i_L^P(L \setminus K) \cap i_{C_{k+2}}^P(C_{k+2}\setminus C_{k+1}) \neq 
		\emptyset$, $\rep(P,C') := \true $.
	\item If $\scond{k+2}{c} = \false$, i.e. $k = \nlvl(c)-2$,
		$$\rep(P,C') :=
		\begin{cases} 
			\true &\text{if $i_L^P(L \setminus K) \cap i_{C_{k+2}}^P(C_{k+2}\setminus C_{k+1}) \neq \emptyset$}\\
			\false &\text{otherwise}.
		\end{cases}$$
	\item Otherwise, let $P'$ be the overlap of $R$ and $C_{k+2}$ such that there is a transformation $P' \Longrightarrow_{\rho^{-1}, i_R^{P'}} P$. 
If this overlap or transformation does not exist, we set $\rep(P,C') := \false$ and otherwise 
$$\rep(P,C') := \shift(\forall (i_R^{Q}: R \inj P', 
				\bigvee_{Q \in \eol(P',i_{C_{k+2}}^{P'}, a^r_{k+2})}  \exists(i_{P'}^Q:P' \inj Q,\true)), \rho).$$
\end{enumerate}
\end{enumerate}

\end{definition}


\begin{example}
\begin{enumerate}
	\item 
	Consider constraint $c_1$ given in Figure \ref{fig:constraints} and the rule 
\emph{\texttt{as\-sign\-Fea\-ture}}.
There is only one overlap $P$ of $L$ and $C_1^1$ which is shown in Figure \ref{fig_increasing_1}. 
The $\nex(P,C')$ and $\nex(P, C')$ parts of $\incr{-1}{C'}{\emph{\texttt{assignFeature}}}$ with $C' = C_2^2$ and $C' = C_2^1$, respectively, are also given in Figure \ref{fig_increasing_1}.

	\item
	Consider the constraint and rule given in Figure \ref{fig_increasing_2}.
	The consistency-increasing application condition of this rule and at layer $-1$ with $C_1$ of the constraint is also given in this Figure.
\end{enumerate}
\end{example}



\begin{figure}
\centering
\includegraphics[scale=0.8]{figures/images/increasing_1}


\caption{Examples for $\nex(P,C_2^2), \nex(P,C_2^1), \rep(P,C_2^2)$ and  $\rep(P,C_2^1)$ using the rule \texttt{assignFeature} and constraint $c_1$ given in Figure \ref{fig:constraints}.}\label{fig_increasing_1}
\end{figure}
\begin{figure}
\centering
\includegraphics[scale=0.7]{figures/sources/example_incr}


\caption{Example of $\incr{-1}{C_1}{\rho}$ using this constraint and the rule \texttt{removeDependency}.}\label{fig_increasing_2}
\end{figure}

Let us now show that a rule equipped with the consistency-increasing transformation condition at layer $k$ is indeed a consistency-increasing rule at layer $k$.


\begin{theorem}
	Given a constraint $c$ in UANF. 
	Every rule $\rho = (\ac \wedge \incr{k}{C'}{\rho'},\rho')$ with $-1 \leq k < \nlvl(c)-1$ odd and $C' \in \ig{C_{k+2}}{C_{k+3}}$ if $k < \nlvl(c)-2$ and 
	$C' = C_{k+2}$ otherwise is a direct consistency-increasing rule w.r.t. $c$ at layer $k$.
	
\end{theorem}
\begin{proof}
	Let a transformation $t: G \Longrightarrow_{\rho} H$ with $\maxk{c}{G} = k$
	be given. 
	Since  $\incr{k}{C'}{\rho}$ contains $\main{k}{\rho}$, $t$ is a direct consistency maintaining transformation at layer $k$ according to Theorem \ref{thm_maintaining_ac}. It remains to show that $t$ satisfies 
	the general or special increasing formula respectively.
	
	
	\begin{enumerate}
		\item	
			If $k = \nlvl(c) -2$, i.e. $c$ ends with a condition of the form $\forall(C_{\nlvl(c)},\false)$, assume that $t$ does not satisfy the special increasing formula. Then there is no morphism $p: C_{k+2} \inj G$ such that $\track_t \circ p$ is not total.
			So there is no overlap $P$ of $L$ and $C_{k+2}$ with $i_L^P(L\setminus K) \cap i_{C_{k+2}}^P(C_{k+2}\setminus C_{k+1} ) \neq \emptyset$ such that $m \models \exists(i_L^P: L \inj P, \true)$.
			Since, $k = \nlvl(c) -2$ it follows that either $\rep(P',C') = \false$ or $m \not \models \nex(P',C')$ for all $P \in \overlay(L,P)$. Therefore $m \not \models \incr{k}	{C}{\rho}$, this is a contradiction.
		
		\item 
			Otherwise let $P \in \overlay(L, C_{k+2})$. We show that 
			$m \models \vFound{}{P}{C'} \wedge \vRepaired{}{P}{C'}$ implies  that $t$ satisfies the general increasing formula.
			If $m \models \vFound{}{P}{C'}$, there is a morphism
			$p: P \inj G$ with $m = p \circ i_L^P$ and $p \models \neg \exists(
			i_P^Q:P \inj Q, \true)$ for all $Q \in \eol(P,i_{C_{k+2}}^P, a^r_{k+2})$.
			Therefore, $q:= p \circ 
			i_{C_{k+2}}^P\not \models 
			\exists(a_{k+2}^r:C_{k+2} \inj C', \true)$. 
			
			For $\vRepaired{}{P}{C'}$ there are two cases. Either $\vRepaired{}{P}{C'} = \true$ or $\vRepaired{}{P}{C'}$ is the condition described in Definition \ref{def_appl_cond} and $m \models \vRepaired{}{P}{C'}$. 
			In the first case, if $\vRepaired{}{P}{C'} = \true$ it follows that 
			$i_L^P(L \setminus K) \cap i_{C_{k+2}}^P(C_{k+2} \setminus C_{k+1}) \neq \emptyset$. 
			Therefore $\track_t \circ q$ is not total and the general increasing formula is satisfied. 
			
			Otherwise,  if $\vRepaired{}{P}{C'}$ is the condition described in Definition \ref{def_appl_cond} and $m \models \vRepaired{}{P}{C'}$, $\track_t \circ q$ is total and there is 
			a morphism $p': P' \inj H$ with 
			$\track_t\circ q = p' \circ i_{C_{k+2}}^{P'}$.
			Since $m \models \vRepaired{}{P}{C'}$,  all morphisms $p'':P' \inj H$ with $n = p'' \circ i_R^Q$ satisfy $\bigvee_{Q \in \eol(P',i_{C_{k+2}}^{P'}, a^r_{k+2})}  \exists(i_{P'}^Q:P' \inj Q,\true)$. 
			It follows that $p'' \circ i_{C_{k+2}}^P \models \exists(C', \true)$ and in particular that $\track_t\circ q = p' \circ i_{C_{k+2}}^{P'} \models \exists(C', \true)$. Therefore, the general increasing formula is satisfied.
	\end{enumerate}	 
	In summary, $\rho$ is a direct consistency-increasing rule at layer $k$.
\end{proof}


Note that $\incr{k}{C'}{\rho}$ is only evaluated to $\true$ if 
an occurrence $p: C_{k+2} \inj G$ that does not satisfy $\exists(C', \true)$ is either removed or satisfies $\exists(C', \true)$ in the derived graph.
For all smaller improvements, i.e. a similar improvement for a subgraph $C'' \in \ig{C_{k+2}}{C'}$ of $C'$, $\incr{k}{C'}{\rho}$ would be evaluated to $\false$.
For any larger improvements, i.e. the same improvement for a supergraph $C'' \in \ig{C'}{C_{k+3}}$ of $C'$, $\incr{k}{C'}{\rho}$ will also be evaluated to $\false$ if the repaired occurrence of $C_{k+2}$ satisfies $\exists(C', \true)$.
In both cases, the application condition would prohibit the transformation, even though it would be direct consistency-increasing.
To solve this problem, several application conditions could be combined by
$$\bigvee_{C' \in \ig{C_{k+2}}{C_{k+3}}} \incr{k}{C'}{\rho}.$$ 
This application condition will be evaluated to $\true$ if the cases described above occur, with the drawback that this results in a huge condition, even if duplicate conditions are removed.
At least all duplicates of $\main{}{}$ can be removed, since they are identical for each $\incr{k}{C'}{\rho}$ and only need to be constructed once.

In general, these application conditions are a compromise between condition size and restrictiveness. 
They are very restrictive because they do not allow deletions of occurrences of existentially bound graphs and insertions of universally bound graphs.
For example, any of these application conditions for the rule \texttt{moveFeature} and the constraint $c_1$ will be equivalent to $\false$; the maintaining part of the condition will always be evaluated to $\false$, since \texttt{moveFeature} always removes occurrences of the existentially bound graph $C_2^1$.
Changing the conditions constructed by $\maintaining()$ to check whether two nodes of type \texttt{Feature} are connected to a node of type \texttt{Class} will give application conditions that can be satisfied with \texttt{moveFeature}. However, for a similar rule moving two nodes of type \texttt{Feature}, this newly constructed $\maintaining()$ would still be evaluated to $\false$.
So this would only lead to a slight decrease in restrictiveness. 

The conditions constructed by $\rem()$ and $\nin()$ could be modified in a similar way.
For $\rem()$ and an occurrence $p$ of the universally bound graph $C_{j}$, by checking whether there are two occurrences $p_1,p_2$ of $C_{j}$ with $p = p_1 \circ a_j = p_2 \circ a_j$, and for $\nin()$, by checking whether an introduced occurrence $p$ of $C_j$ satisfies $\exists(C_{j+1}, \true)$.
As above, this only leads to a small decrease of restrictiveness. 
Also, the consistency-increasing application condition becomes more and more restrictive as $k$ increases, since the number of conditions and in particular the number of negative application conditions also increases.
