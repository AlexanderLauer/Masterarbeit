\subsection{Repairing rule Sets}

Given a set of rules and a constraint, it is unclear whether or not it is possible to repair a graph using the rules of that set. 
Therefore, we introduce the notion of \emph{repairing rule sets}, which is a characterisation of rule sets that are able to repair a graph w.r.t. a circular conflict free constraint.
First, we introduce the notion of \emph{repairing sequences}. 
A repairing sequence is a sequence of rule applications that either destroys an occurrence of a universal or inserts an occurrence of an existentially bound graph, and is applicable to each occurrence of these graphs. 
To ensure that these sequences are applicable to every occurrence, it is necessary to ensure that no nodes of these occurrences are removed and that the left-hand side of the first rule of the repairing sequence is contained in that occurrence. 
In other words, every repairing sequence of $C_k$ starts with a transformation
originating in $C_k$ if $C_k$ is universally bound and $C_{k-1}$ if 
$C_k$ is existentially bound.

\begin{definition}[\textbf{repairing sequence}]
	Let a constraint $c$ in UANF and a set of rules $\mathcal{R}$ be given. 
	\begin{enumerate}
		\item 
			If $C_k$ is existentially bound, a sequence of transformations 
			$$C_{k-1} = G_0 \overset{t_1}{\Longrightarrow}_{\rho_1,m_1} G_1 
			\overset{t_2}{\Longrightarrow}_{\rho_2,m_2} \ldots 
			\overset{t_n}{\Longrightarrow}_{\rho_n, m_n} 
			G_n$$
			with $\rho_i \in \mathcal{R}$ is called a \emph{repairing sequence 
			of $C_k$} if $G_n \models_{k}c$, $\track_{t_n} \circ \ldots \circ 
			\track_{t_1} \circ \id_{C_{k-1}}$ is total and the concurrent rule 
			of this sequence is a basic consistency maintaining rule w.r.t.
			$\forall(C_j, \false)$ for all universally bound graphs $C_j$ 
			such that $C_k$ has no conflict with $C_j$. 
		\item	
			If $C_k$ is universally bound, a sequence of transformations  
			$$C_k = G_0 \overset{t_1}{\Longrightarrow}_{\rho_1,m_1} G_1 
			\overset{t_2}{\Longrightarrow}_{\rho_2,m_2} \ldots 
			\overset{t_n}{\Longrightarrow}_{\rho_n, m_n} 
			G_n$$
			with $\rho_i \in \mathcal{R}$ is called a \emph{repairing sequence 
			of $C_k$} if $G_n \models_k c$, for each node $v \in V_{G_0}$ there 
			does exist a node $v' \in V_{G_n}$ with $v' = \track_{t_n}(\ldots 
			\track_{t_1}(v))$ and the concurrent rule 
			of this sequence  
			is a basic consistency maintaining 
			rule w.r.t. $\forall(C_j, \true)$ for universally bound graphs 
			$C_j$.
	\end{enumerate}
\end{definition}

In both cases the insertion of additional elements, i.e. $G_n \neq C_{k+1}$ if $C_k$ is existentially bound and $G_n \neq C$ for all $C \in \ig{C_{k-1}}{C_k}$ if $C_k$ is universally bound, could lead to the insertion of universally bound graphs. 
For an existentially bound graph, this can happen if there is an overlap with a universally bound graph in a similar way as shown in figure \ref{fig:example_univ}. 
To ensure that this does not happen, we need the additional condition that the concurrent rule is a basic consistency maintaining rule with respect to certain constraints.
If $G_n = C_{k+1}$ if $C_k$ is existentially or $G_n = C$ with $C \in \ig{C_{k-1}}{C_k}$ if $C_k$ is universally bound, this condition is not needed as  the following Theorem shows.
\begin{figure}
\centering
\includegraphics[scale=0.7]{figures/images/example_overlap}

\caption{Sketch of an overlap of an existentially bound graph $C_k$ and a universally bound graph $C_j$ that could lead to an insertion of $C_j$ via repairing sequences.}\label{fig:example_univ}
\end{figure}

\begin{theorem}\label{special_rep_sequence}
	Let a constraint $c$ in UANF and a set of rules $\mathcal{R}$ be given. 
	\begin{enumerate}
		\item 
			If $C_k$ is existentially bound and there is sequence 
			$$C_{k-1} \Longrightarrow_{\rho_1,m_1} \ldots 
			\Longrightarrow_{\rho_n,
				m_n} C_k$$ 
			with $\rho_i \in \mathcal{R}$ such that $\track_{t_n} \circ 
			\ldots \track_{t_1} \circ 
			 \id_{C_{k-1}}$ is 
			total and $C_k \models_{k}c$. Then, this is a repairing sequence 
			for $C_k$. 
		\item
			If $C_k$ is universally bound and there is  a sequence 
			$$C_k \Longrightarrow_{\rho_1,m_1} \ldots \Longrightarrow_{\rho_n, 	
			m_n} C$$ 
			with $\rho_i \in \mathcal{R}$, $C \in \ig{C_{k-1}}{C_k}$, $C\models_k c$ and for each 
			node $v \in V_{G_0}$ there does exist a node $v' \in V_{G_n}$ with 
			$v' = \track_{t_n}(\ldots \track_{t_1}(v))$.
			Then, this is a repairing sequence for $C_k$.
	\end{enumerate}
\end{theorem} 
\begin{proof}
	\begin{enumerate}
		\item 
			If $C_k$ is existentially bound, the concurrent rule is given by 
			$\rho = \rle{C_{k-1}}{\id}{C_{k-1}}{a_k}{C_k}$.
			Let $C_j$ be a universally bound graph such that $C_k$ has no 
			conflict with $C_j$ and  $\rho$ is not a 
			basic consistency maintaining rule w.r.t. $\forall(C_j, \true)$.
			With Lemma \ref{basic_conflict} follows immediately that 
			$C_k$ has a conflict with $C_j$, this is a contradiction.
		\item
			If $C_k$ is universally bound, the concurrent rule is given  by 
			$\rho = \rle{C_k}{a_{k-1}^r}{C}{\id}{C}$. 
			Then, $\rho$ is a basic consistency maintaining rule w.r.t. 
			$\forall(C_j, \true)$ for all universally bound graphs since 
			$\rho$ does not insert any elements. 
	\end{enumerate}
\end{proof}
 
\begin{definition}[\textbf{repairing rule set}]
	Let a set of rules $\mathcal{R}$ and a circular conflict free constraint $c$ 
	in UANF be given. 
	Then, $\mathcal{R}$ is called a \emph{repairing rule set of 
	$c$} if there does exist a repairing sequence for each existentially bound 
	graph of $c$ and, if $\nlvl(c)$ is odd, i.e. $c$ ends with a condition 
	of the form $\forall(C_{\nlvl(c)}, \false)$, $\mathcal{R}$ contains 
	a repairing sequence for $C_{\nlvl(c)}$.
\end{definition}

Note that there cannot exist a repairing sequence for a universally bound graphs $C_k$ such that $C_k \setminus C_{k-1}$ does not contain any edges. 
Therefore, there is no repairing set for all constraints of the form $\forall(C_1, \false)$ such that $E_{C_1} = \emptyset$.

\begin{theorem}
	Let a circular conflict free constraint $c$ in UANF and a repairing set 
	$\mathcal{R}$ of $c$ be given.
	Then, for each graph $G$ with $G \not \models c$, there is a 
	sequence of transformations 
	$$G = G_0 \Longrightarrow_{\rho_1, m_1} \ldots \Longrightarrow_{\rho_n, m_n} 
	G_n$$  
	with $\rho_i \in \mathcal{R}$ such that $G_n \models c$.
\end{theorem}
We will postpone the proof of this Theorem, as it follows immediately from the termination of our repair process.

\begin{example}
	Consider the constraints $c_1$, $c_4$ and the sequences shown in Figure \ref{fig:rep_set}. The first sequence is not a repairing sequence for the existentially bound graph of $c_4$, since $G_1 \not \models_1 c_4$ and therefore a rule set containing only this rule is not a repairing set w.r.t $c_4$.
	The second sequence is a repairing sequence for the existentially bound 
	graph of $c_4$, since the last graph satisfies $c_4$ and the existentially bound graph has a conflict with the universally bound graph. Therefore, the condition for the concurrent rule is also satisfied, and 
	a rule set containing this rule is a repairing set w.r.t. $c_4$.
	
	
	The third sequence is a repairing sequence for $c_1$ since the last graph 
	satisfies $c_1$ and the sequence satisfies the criteria given in Theorem 
	\ref{special_rep_sequence}. Note that this sequence consists of two applications of the same rule. A set of rules containing this rule is a repairing set w.r.t. $c_1$.
\end{example}
\begin{figure}
\centering
\includegraphics[scale=0.8]{figures/sources/repairingseq}

\caption{Repairing sequences for $c_1$ and $c_4$.}\label{fig:rep_set}
\end{figure}     
%\begin{corollary}
%	Let a constraint $c$ in UANF and a rule set $\mathcal{R}$ be given. If either (a)
%	$\mathcal{R}$ is a repairing rule set at layer $k$ for all odd $0 \leq k \leq \nlvl(c)$
%	or (b) $\mathcal{R}$ is a repairing rule set at layer $k$ with $k$ being even and $\mathcal{R}$
%	is also a repairing rule set for all odd $0 \leq j <k$ then for all graphs $G$ a sequence
%	$$G = G_0 \Longrightarrow_{\rho_0} \ldots \Longrightarrow_{\rho_{n-1}} G_n$$ 
%	with $G_n \models c$ and $\rho_j \in \mathcal{R}$ for all $0 \leq j \leq n-1$ exists.
%\end{corollary}
%
%\begin{theorem}\label{caracter_rep_rule_set_at_layer}
%Let a constraint $c$ in EANF and a set of rules $\mathcal{R}$   be given. 
%Then, $\mathcal{R}$ is a repairing set of $c$ at layer $k \leq \nlvl(c)$ if either \ref{cat_rep_set_1} or \ref{cat_rep_set_2} applies. 
%
%\begin{enumerate}
%	\item \label{cat_rep_set_1}
%	 For any universally bound graph $C_j$ at layer $j \leq k$ of $c$, $(\rho, \api(j, C_{j+1})) \in \mathcal{R}$ and $\rho$ is a deleting potentially minimal improving rule at layer $j$ with $C_{j+1}$, such that $\rho$ only deletes edges of $C_j$.
%	\item \label{cat_rep_set_2}
%	A decomposition $\mathbf{P}$ of $C_k$ with $C_{k-1}$ exists, such that for each $P \in \mathbf{P}$ a rule $(\rho, \api(k,P)) \in \mathcal{R}$ exists, such that $\rho$ is an inserting basic improving rule at layer $k$ with $P$.
%	
%	\item \label{cat_rep_set_3}
%		There does exist a sequence of graphs 
%			$$C_k = C'_0 \inj C'_1 \inj \ldots \inj C'_n = C_{k+1}$$
%		such that a inserting basic increasing rule $\rho_i  = L_i \xhookleftarrow{l_i} K_i \xhookrightarrow{r_i} R_i$ at layer $k$ with $C_i$ exists such that $L_i \inj R_{i-1}$ and $e_i = i_{C{i-1}}^{C_i} \circ e_{i-1}$ with $e_i$ being the increasing morphism of $\rho_i$ for all $i \in \{1, \ldots n\}$.
%\end{enumerate}
%\end{theorem}
%
%\begin{proof}
%Let a constraint $c$ in EANF, a rule set $\mathcal{R}$ and a graph $G$ with $k = \cmax$ and $\cmax < \nlvl(c)$ be given. 
%We show that a sequence $G = C_0' \Rightarrow \ldots \Rightarrow C_n' = H$ with rules of $\mathcal{R}$ exists, such that $H \models_{k+2} c$ if \ref{cat_rep_set_1}. or \ref{cat_rep_set_2}.  of theorem \ref{caracter_rep_rule_set_at_layer} is satisfied.
%\begin{enumerate}
%\item
% Assume that \ref{cat_rep_set_1}. of theorem \ref{caracter_rep_rule_set_at_layer} holds. Let $(\rho, \api(j, C_{j+1})) \in \mathcal{R}$, such that $\rho = L \xhookleftarrow{l} K \xhookrightarrow{r} R$ is a deleting potentially minimal improving rule at layer $j \leq k$ with $C_{j+1}$ and $C_j$ is a universally bound graph of $c$.
%Then, $\api(j, C_{j+1}) = \exists\bigl(b: L \inj C_{j}, \neg \exists(a_j:C_{j} \inj C_{j+1}, \true)\bigr)$.
%Let $q: C_j \inj G$ be a morphism such that  $q \not \models \exists(a_j: C_j \inj C_{j+1}, \true)$.
%Since $L \subseteq C_j$, we can construct a morphism $m_1: L \inj G$ with $m_1 = q \circ b$ and therefore $m_1 \models \api(j,C_{j+1})$.
%Since $r$ only deletes edges, a transformation $t: G = G_0 \Rightarrow_{r,m_1} G_1$ exists and $\track_t \circ p$ is not total. 
%Because $r$ does not insert any elements of $C_j$: 
%$$|\{q: C_k \inj G_0 \mid  q \not \models  d\}| < |\{q: C_k \inj G_1 \mid  q \not \models  d\}|$$
%with $d = \exists(a_j: C_j \inj C_{j+1}, \true)$.
%By iteratively applying this construction, we can generate a finite sequence of transformations
%$$G = G_0 \Rightarrow_{r, m_1} G_1 \Rightarrow_{r, m_2} \ldots \Rightarrow_{r,m_n} G_n = H$$
%such that $|\{q: C_k \inj G_n \mid  q \not \models  d\}| = 0$ and therefore $H \models_j c$. 
%With lemma \ref{lemma_lay_sat}, $H \models_{k+2} c$ and $H \models c$ follows. 
%
%\item 
%Assume that \ref{cat_rep_set_2}. of theorem \ref{caracter_rep_rule_set_at_layer} holds. 
%Let $\rho_0 = (\rho, \api(k,P)) \in \mathcal{R}$, such that $\rho$ is an inserting basic improving rule of $c$ at layer $k$ with $P \in \mathbf{P}$. 
%Then, $\api(k, P) = \neg \exists(b: L \inj P, \true)$.
%Let $q_0 : C_k \inj G$ be a morphism, such that $q_0 \not \models \exists(a'_k: C_k \inj P, \true)$ with $a'_k$ being a partial morphism of $a_k$.
%Since $L = C_k$, we set $m_0: C_k \inj G$ with $m_0 = q_0$. 
%It follows that $m_0 \models \neg \exists(a'_k: C_k \inj P, \true) = \api(k,P)$.
%Because $r$ does not delete any elements, a transformation $t_0: G \Longrightarrow_{r_0,m_0} G_1$ exists and $\track_t \circ q \models \exists(a'_k: C_k \inj P, \true)$.  
%We set $q_1 = \track_{t_0} \circ q_0$ and apply the same method to $q_1$. 
%
%By iteratively applying this, we can construct a finite sequence of transformations 
%$$G \Longrightarrow_{r_0, m_0} G_0 \Longrightarrow_{r_1,m_1} \ldots \Longrightarrow_{r_n,m_n} G_n$$ 
%such that $m_i = \track_{t_{i-1}} \circ \ldots \circ \track_{t_0} \circ m_0$ and $q\models \exists(b_i: C_k \inj P_i, \true)$ for all $P_i \in \mathbf{P}$ with $q = \track_{t_n} \circ q_n$.
%Let $p_i: P_i \inj G_n$ be the morphism, such that  $q = p_i \circ b_i$. 
%
%Now, we can construct a morphism $p: C_{k+1} \inj G$ with 
%$$ p(e) := \begin{cases}
%			p_1(e) &\text{,if } e \in P_1\\
%			&\vdots\\
%			p_j(e) &\text{,if } e \in P_j.
%		\end{cases}$$
%		
%Let $e \in C_k$, because $q(e) = p_i \circ b_i(e)$ and $q(e) = p_{\ell} \circ b_{\ell}(e)$ and $b_i$ and $b_{\ell}$ are both partial morphisms of $a_k$, it follows that $b_i(e) = b_{\ell} (e)$ and therfore $p_i (e) = p_{\ell} (e)$. 
%Because $(P_i \cap P_{\ell})\setminus C_k = \emptyset$ for all $i \neq \ell$, $p$ is a morphism and by the definition of $p$ it follows that 
%$q = p \circ a_k$ and therefore $q \models \exists(a_k: C_k \inj C_{k+1}, \true)$.
%
%By iteratively applying this whole construction to all occurrences of $C_k$ that do not satisfy $\exists(a_k:C_k \inj C_{k+1},\true)$ the derived graph graph $H$ does not contain any occurrences of $C_k$ not satisfying $\exists(a_k: C_k \inj C_{k+1},\true)$ and therefore  $H \models_{k+2} c$. 
%
%\end{enumerate}
%\end{proof}
%\begin{corollary}
%If a set of rule $\mathcal{R}$ is a repairing set of $c$ at layer $k \leq \nlvl(c)$ and \ref{cat_rep_set_1}. of theorem \ref{caracter_rep_rule_set_at_layer} applies, then $\mathcal{R}$ is a repairing set of $c$ at layer $j$ for all $k \leq j \leq \nlvl(c)$. 
%\end{corollary}
%
%\begin{lemma}
%Let a graph $G_0$, a constraint $c$ in EANF and a repairing set $\mathcal{R}$ at layer $\maxc{G_0}  +1$ be given, such that each rule in $ \mathcal{R}$ is a basic increasing.
%Then, for every sequence
%$$G_0 \Longrightarrow_{\rho_0, m_0} G_1 \Longrightarrow_{\rho, m_0} \ldots \Longrightarrow_{\rho_n, m_n} G_n$$
%such that $\rho_i = (\rho, \api(\maxc{G_0} + 1, P)$ with $\rho \in \mathcal{R}$ being a consistency increasing rule at layer $\maxc{G_0} + 1$ with $P$, it holds that 
%
%\end{lemma}
%


%\subsubsection{Construction of  repairing sets}

\begin{definition}[\textbf{decomposition of a graph}]\label{def_composition}
Let graphs $G_0 \inj G_1$ be given. 

A decomposition of $G_1$ with $G_0$ is a minimal set 
 $$\mathbf{D} \subseteq \{D_v \mid v \in V_{C_1 \setminus C_0} \}$$ of subgraphs of $G_1$, such that every element of $G_1$ is contained in at least one $D \in \mathbf{D}$  and  every $D_v$ is constructed in the following way:
$G_0 \inj D_v$, $v \in D_v$ and for all nodes $v' \in D \setminus C_0$ $D$ contains all edges $e \in E_{G_1}$ and all nodes $u \in V_{G_1}$ such that  either $\tar(e) = v' \wedge \src(e) = u$ or  $\src(e) = u \wedge \tar(e) = v'$. 
\end{definition}

\begin{lemma}\label{decomp_sets_lemma}
Let graphs $G_0 \inj G_1$ and a decomposition $\mathbf{D}$ of $G_1$ with $G_0$ be given. 
Then, for each pair $D, D' \in \mathbf{D}$ with $D \neq D'$ the following holds:
$$(D\setminus C_k)\cap (D' \setminus C_k) = \emptyset$$
\end{lemma}

\begin{proof}
 Assume that $(D\setminus C_k)\cap (D' \setminus C_k) \neq \emptyset$, therefore a node $v \in G_1 \setminus G_0$ with $v \in D \cap D'$ exists. 
 By the construction of $D$ and $D'$ it follows that $D = D_v$ and $D' = D_v$ and therefore $D = D'$. 
This is a contradiction.
\end{proof}

\begin{lemma}
Let graphs $G_0 \inj G_1$ and a decomposition $\mathbf{D}$ of $G_1$ with $G_0$ be given. 
Then, $$ G_1 = \bigcup_{P \in \mathbf{P}} P.$$ 

\end{lemma}

\begin{proof}
Let $H := \bigcup_{P \in \mathbf{P}} P$.
Firstly, we show that $H \subseteq G_1$.
Since every $P \in \mathbf{P}$ is a subgraph of $G_1$ it follows that $V_H \subseteq V_{G_1}$ and $E_H \subseteq E_{G_1}$

Secondly, we show that $G_1 \subseteq H$.
Let $u \in V_{G_1}$ be a node, if $u \in V_{G_0}$, then $u$ is contained in every $P \in \mathbf{P}$ and therefore $u \in V_H$. 
Otherwise, if $u \not \in V_{G_0}$, then $u$ has to be, by the definition of $\mathbf{P}$, contained in at least one $P \in \mathbf{P}$ and  $V_{G_1} \subseteq V_H$ follows. 
Let $e \in E_{G_1}$ be an edge. 
If $e \in E_{G_0}$, then $e$ is contained in every $P \in \mathbf{P}$ and $e \in E_H$. 
Otherwise, if $e \not  \in E_{G_0}$, by the definition of $\mathbf{P}$, $e$ has to be contained in at least one $P \in \mathbf{P}$. 
It follows that $e \in E_H$ and with that $E_{G_1} \subseteq E_H$. 
\end{proof}

Now we are ready to provide the construction for 
\begin{construction}\label{construction:rule_set}
	Let a constraint $c$ in UANF be given. For each existentially bound subcondition
	$\scond{k}{c}$  we construct a graph decomposition $\mathbf{D_k}$ as described
	in definition \ref{def_composition}. Then, for each element $D \in \mathbf{D_k}$
	a rule $$\rho =  C_k \xhookleftarrow{\id} C_k \xhookrightarrow{i_{C_k}^D} D$$
	is derived. 
	
	Additionally, for each universally bound subcondition $\scond{k}{c}$ and for each
	edge $e \in E_{C_{k+1} \setminus C_k}$ a rule 
	$$\rho =  C_{k+1} \xhookleftarrow{i_D^{C_{k+1}}} D \xhookrightarrow{\id} D$$
	with $D = C_{k+1} \setminus \{e\}$ is derived.
\end{construction}

\begin{example}

\end{example}

